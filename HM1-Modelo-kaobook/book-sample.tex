%%%%%%%%%%%%%%%%%%%%%%%%%%%%%%%%%%%%%%%%%%%%%%%%%%%%%%%%%%%%%%%%
%% MODELO PARA CREAR CADERNO DE EXERCICIOS CON NOTAS AO MARXE %%
%% Actualizado: Decembro 2022                                 %%
%%%%%%%%%%%%%%%%%%%%%%%%%%%%%%%%%%%%%%%%%%%%%%%%%%%%%%%%%%%%%%%%
% 
% PREÁMBULO
% =========
%
% Tipo de documento: kaobook class
% --------------------------------
\documentclass[fontsize=10pt,twoside=true,secnumdepth=1]{kaobook}
%
% Idioma
% ------
\usepackage[spanish,activeacute,es-tabla]{babel}
%\usepackage[english]{babel} % Load characters and hyphenation
%\usepackage[english=british]{csquotes}	% English quotes
%
% Load packages for testing
% -------------------------
\usepackage{blindtext}
%\usepackage{showframe} % Uncomment to show boxes around the text area, margin, header and footer
%\usepackage{showlabels} % Uncomment to output the content of \label commands to the document where they are used
%
% Caga de paquetes necesarios
% ---------------------------
\usepackage{enumerate} % entornos de listas
\usepackage{multicol}  % varias columnas texto
%\usepackage{fancyhdr}  % encabezado personalizado
\usepackage{fancybox}  % entornos con cajas
\usepackage{pdfpages}  % páginas pdf
%
% Load the bibliography package
% -----------------------------
\usepackage{kaobiblio}
\addbibresource{minimal.bib} % Bibliography file
%
% Load mathematical packages for theorems and related environments
% --------------------------
\usepackage{kaotheorems}
%
% Load the package for hyperreferences
% ------------------------------------
\usepackage{kaorefs}
%
% Directorio de gráficos
% ----------------------
\graphicspath{{images/}{./}} % Paths where images are looked for
%
% Índex
% -----
\makeindex[columns=3, title=Alphabetical Index, intoc] % Make LaTeX produce the files required to compile the index
%
%   ---------   Personalización de textos    ---------
%   Opción en Galego.-
\addto\captionsspanish{%
\renewcommand{\contentsname}{Índice}
\renewcommand{\listfigurename}{Índice de ilustracións}
\renewcommand{\listtablename}{Índice de táboas}
\renewcommand{\bibname}{Bibliografía}
\renewcommand{\indexname}{Indice alfabético}
\renewcommand{\figurename}{Ilustración}
\renewcommand{\tablename}{Táboa}
\renewcommand{\appendixname}{Anexo}
%\renewcommand{\abstractname}{Resumo}
%\renewcommand{\partname}{BLOQUE} % Cambio Parte BLOQUE
%\renewcommand{\chaptername}{TEMA} % Cambio CAPÍTULO por TEMA en mayúsculas %
}
%
% Tipograía:
% ----------
% Fuente HEURÍSTICA (cómoda de leer)
\usepackage{heuristica}
% Fuente LIBERTINE (cómoda para apuntes)
%\usepackage{libertineRoman}
%\usepackage[proportional]{libertine}
% Fuente ROMANDE (estilo antiguo pero no muy cómoda)
%\usepackage{romande} %
%
% ===================
% DOCUMENTO PRINCIPAL
% ===================
%
\begin{document}
%
%
% Metadatos documento
% -------------------
%
\titlehead{Plantilla para apuntamentos de clase}
\title[Modelo cadernos de actividades de Historia da Música
{\normalfont\texttt{kaobook}} Class]
{Modelo para crear cadernos de actividades \\
baseado no {\normalfont\texttt{kaobook}} Class}
\author[RPM]{Roberto Prado Martínez}
\date{\today}
\publishers{RPM}
%
\frontmatter % Denotes the start of the pre-document content
%
% TODO: Crear páxinas de licenzas
% COPYRIGHT PAGE
% --------------

\makeatletter
\uppertitleback{\@titlehead} % Header

\lowertitleback{
	\textbf{Licenza de uso} \\
	Este Caderno de audicións e actividades © 2022 creado por Roberto Prado Martínez, está baixo unha licenza CC BY-NC-SA 4.0. Para ver unha copia da licenza, visite: \\
	http://creativecommons.org/licenses/by-nc-sa/4.0/
	
	\medskip
	
	\textbf{No copyright} \\
	\cczero\ This book is released into the public domain using the CC0 code. To the extent possible under law, I waive all copyright and related or neighbouring rights to this work.
	
	To view a copy of the CC0 code, visit: \\\url{http://creativecommons.org/publicdomain/zero/1.0/}
	
	\medskip
	
	\textbf{Colophon} \\
	This document was typeset with the help of \href{https://sourceforge.net/projects/koma-script/}{\KOMAScript} and \href{https://www.latex-project.org/}{\LaTeX} using the \href{https://github.com/fmarotta/kaobook/}{kaobook} class.
	
	\medskip
	
	\textbf{Publisher} \\
	First printed in May 2019 by \@publishers
}
\makeatother
%
% Adicatoria
% ----------

\dedication{
	The harmony of the world is made manifest in Form and Number, and the heart and soul and all the poetry of Natural Philosophy are embodied in the concept of mathematical beauty.\\
	\flushright -- D'Arcy Wentworth Thompson
}
%
%
% Note that \maketitle outputs the pages before here
\maketitle
%
% Prefacio
% --------
%
\chapter*{Presentación}
\blindtext
%
% ÍNDICE, ÍNDICE DE TÁBOAS E FIGURAS
% ----------------------------------
\begingroup % Local scope for the following commands
%
% Define the style for the TOC, LOF, and LOT
\setstretch{1} % Uncomment to modify line spacing in the ToC
%\hypersetup{linkcolor=blue} % Uncomment to set the colour of links in the ToC
\setlength{\textheight}{230\vscale} % Manually adjust the height of the ToC pages
%
% Turn on compatibility mode for the etoc package
\etocstandarddisplaystyle % "toc display" as if etoc was not loaded
\etocstandardlines % "toc lines as if etoc was not loaded
%
\tableofcontents % Output the table of contents
%
\listoffigures % Output the list of figures
%
% Comment both of the following lines to have the LOF and the LOT on different pages
\let\cleardoublepage\bigskip
\let\clearpage\bigskip
%
\listoftables % Output the list of tables
%
\endgroup

% Fin de índices
% 
% ==================
% CORPO DO DOCUMENTO
% ==================
%
\mainmatter % Denotes the start of the main document content, resets page numbering and uses arabic numbers
%
\setchapterstyle{kao} % Choose the default chapter heading style
%
% Primeira parte (1a Avaliación)
% ------------------------------
\pagelayout{wide} % No margins
\addpart{Parte primeira}

\pagelayout{margin} % Restore margins
\chapter{Tema 1}
\blindtext
\chapter{Tema 2}
\blindtext
\chapter{Tema 3}
\blindtext
\chapter{Tema 4}
%
% Segunda parte (2a Avaliación)
% -----------------------------
\pagelayout{wide}
\addpart{Parte segunda}

\pagelayout{margin} % Restore margins
\chapter{Tema 5}
\blindtext
\chapter{Tema 6}
\blindtext
\chapter{Tema 7}
\blindtext
%
% Terceira Parte (3a Avaliación)
% -----------------------------
\pagelayout{wide}
\addpart{Parte terceira}

\pagelayout{margin} % Restore margins
\chapter{Tema 8}
\blindtext
\chapter{Tema 9}
\blindtext
\chapter{Tema 10}
\blindtext
%
% APÉNDICES
% ---------
\appendix % From here onwards, chapters are numbered with letters, as is the appendix convention

\pagelayout{wide} % No margins
\addpart{Anexos}
\pagelayout{margin} % Restore margins

\chapter{Outro capítulo}
\blindtext

%----------------------------------------------------------------------------------------
%
\backmatter % Denotes the end of the main document content
\setchapterstyle{plain} % Output plain chapters from this point onwards
%
% The bibliography needs to be compiled with biber using your LaTeX editor, or on the command line with 'biber main' from the template directory
%
\defbibnote{bibnote}{Aquí as referencias das citas en orde.\par\bigskip} % Prepend this text to the bibliography
\printbibliography[heading=bibintoc, title=Bibliografía, prenote=bibnote] % Add the bibliography heading to the ToC, set the title of the bibliography and output the bibliography note

% INDEX
% -----
% The index needs to be compiled on the command line with 'makeindex main' from the template directory

\printindex % Output the index

\end{document}
