% Options for packages loaded elsewhere
\PassOptionsToPackage{unicode}{hyperref}
\PassOptionsToPackage{hyphens}{url}
%
\documentclass[
]{article}
\usepackage{amsmath,amssymb}
\usepackage{lmodern}
\usepackage{iftex}
\ifPDFTeX
  \usepackage[T1]{fontenc}
  \usepackage[utf8]{inputenc}
  \usepackage{textcomp} % provide euro and other symbols
\else % if luatex or xetex
  \usepackage{unicode-math}
  \defaultfontfeatures{Scale=MatchLowercase}
  \defaultfontfeatures[\rmfamily]{Ligatures=TeX,Scale=1}
\fi
% Use upquote if available, for straight quotes in verbatim environments
\IfFileExists{upquote.sty}{\usepackage{upquote}}{}
\IfFileExists{microtype.sty}{% use microtype if available
  \usepackage[]{microtype}
  \UseMicrotypeSet[protrusion]{basicmath} % disable protrusion for tt fonts
}{}
\makeatletter
\@ifundefined{KOMAClassName}{% if non-KOMA class
  \IfFileExists{parskip.sty}{%
    \usepackage{parskip}
  }{% else
    \setlength{\parindent}{0pt}
    \setlength{\parskip}{6pt plus 2pt minus 1pt}}
}{% if KOMA class
  \KOMAoptions{parskip=half}}
\makeatother
\usepackage{xcolor}
\IfFileExists{xurl.sty}{\usepackage{xurl}}{} % add URL line breaks if available
\IfFileExists{bookmark.sty}{\usepackage{bookmark}}{\usepackage{hyperref}}
\hypersetup{
  hidelinks,
  pdfcreator={LaTeX via pandoc}}
\urlstyle{same} % disable monospaced font for URLs
\usepackage{graphicx}
\makeatletter
\def\maxwidth{\ifdim\Gin@nat@width>\linewidth\linewidth\else\Gin@nat@width\fi}
\def\maxheight{\ifdim\Gin@nat@height>\textheight\textheight\else\Gin@nat@height\fi}
\makeatother
% Scale images if necessary, so that they will not overflow the page
% margins by default, and it is still possible to overwrite the defaults
% using explicit options in \includegraphics[width, height, ...]{}
\setkeys{Gin}{width=\maxwidth,height=\maxheight,keepaspectratio}
% Set default figure placement to htbp
\makeatletter
\def\fps@figure{htbp}
\makeatother
\usepackage[normalem]{ulem}
% Avoid problems with \sout in headers with hyperref
\pdfstringdefDisableCommands{\renewcommand{\sout}{}}
\setlength{\emergencystretch}{3em} % prevent overfull lines
\providecommand{\tightlist}{%
  \setlength{\itemsep}{0pt}\setlength{\parskip}{0pt}}
\setcounter{secnumdepth}{-\maxdimen} % remove section numbering
\ifLuaTeX
  \usepackage{selnolig}  % disable illegal ligatures
\fi

\author{}
\date{}

\begin{document}

\hypertarget{a-muxfasica-no-renacemento}{%
\subsection{A MÚSICA NO RENACEMENTO}\label{a-muxfasica-no-renacemento}}

\href{http://open.spotify.com/user/javierjuradoluque/playlist/6hkvm3DeOXNPNpkdOMWdzc}{Audicións}

\hypertarget{contexto-histuxf3rico}{%
\subsubsection{CONTEXTO HISTÓRICO}\label{contexto-histuxf3rico}}

O concepto de \href{http://es.wikipedia.org/wiki/Renacimiento}{Renacemento} aparece para a historiografía da arte e da cultura en época tan distante como o século XIX. Este termo refírese á recuperación da cultura da \href{http://es.wikipedia.org/wiki/Antig\%C3\%BCedad_cl\%C3\%A1sica}{Antigüidade} \href{http://es.wikipedia.org/wiki/Antig\%C3\%BCedad_cl\%C3\%A1sica}{clásica}
tras o longo período, supostamente escuro, que suporía a \href{http://es.wikipedia.org/wiki/Edad_Media}{Idade Media.}

Efectivamente, os séculos XV e XVI trouxeron grandes cambios, así como a
formación dalgúns dos principios estruturais que estiveron operativos na
cultura europea até as revolucións burguesas dos séculos XVIII e XIX. De
feito, basta lembrar que a historiografía fai comezar no século XV unha
nova \href{http://es.wikipedia.org/wiki/Edad_Moderna}{Idade Moderna.} En
contrapartida, cabe dicir que en non poucos aspectos a cultura do
Renacemento supuxo unha elaboración de trazos que xa se presentaron na
Idade Media.

Adoita situarse a orixe do Renacemento como fenómeno cultural e
artístico nos primeiros anos do XV, nas cidades italianas do norte e en
Roma, xunto coas de Flandes e Países Baixos e, por tanto, relacionado
con áreas de forte desenvolvemento urbano e comercial. Desde estes focos
iniciais, o Renacemento estenderíase a toda Europa paulatinamente. A
cronoloxía clásica distingue entre
\href{http://es.wikipedia.org/wiki/Quattrocento}{\emph{quattrocento}} e
\href{http://es.wikipedia.org/wiki/Cinquecento}{\emph{cinquecento}.}

Facendo unha síntese, podemos enunciar os trazos xerais da cultura do
Renacemento:

\begin{itemize}
\item
  Crecemento económico e demográfico: o Renacemento ve nacer os
  principios da economía capitalista (bancos, letras, crédito...).
  Paralelamente, obsérvase un crecemento das cidades e das clases que
  lle son propias: a burguesía. O propio concepto moderno de Estado
  alcanza a súa formulación nas obras de
  \href{http://es.wikipedia.org/wiki/Maquiavelo}{Maquiavelo.} A nivel
  social, a expansión das clases burguesas favoreceu a demanda dunha
  arte laica, en detrimento do protagonismo da arte relixioso que
  dominara o período anterior.
\item
  Desenvolvemento científico e tecnolóxico, ilustrado pola revolución
  \href{http://es.wikipedia.org/wiki/Cop\%C3\%A9rnico}{copernicana} e a
  creatividade de personaxes como
  \href{http://es.wikipedia.org/wiki/Leonardo_da_Vinci}{Leonardo dá
  Vinci.} sentan as bases do
  \href{http://es.wikipedia.org/wiki/M\%C3\%A9todo_cient\%C3\%ADfico}{método
  científico} e experimental que se desenvolve con forza e que abre a
  porta á ciencia moderna e á sociedade tecnolóxica.
\item
  A nivel cultural pode falarse dun xiro fundamental coa invención da
  \href{http://es.wikipedia.org/wiki/Imprenta}{imprenta} a mediados do
  XV e as posibilidades de expansión de ideas que iso supón. O vigor das
  universidades e a circulación de información favoreceron a expansión
  do \href{http://es.wikipedia.org/wiki/Humanismo}{Humanismo.} O
  Humanismo comprende toda unha antropoloxía dentro da cal a persoa pasa
  a ocupar un lugar central como punto desde o que se observa e valora a
  realidade. O pensamento individual asume a responsabilidade de
  elaborar unha interpretación correcta do mundo mediante un medio
  crítico baseado na experimentación. O punto de vista crítico do
  Humanismo respecto diso dunha interpretación dogmática do mundo é, en
  boa parte, o desencadenamento da crise relixiosa e dos movementos
  protestantes do XVI.
\item
  A creación artística será un dos aspectos máis rechamantes do período.
  En primeiro lugar polo novo concepto de arte: o artista xa non é un
  artesán ao servizo da inspiración divina, senón un creador que aspira
  ao status de home de ciencia. Poucos períodos da Historia de Occidente
  coñeceron un ritmo de produción de obras de arte tan intenso en
  calidade e cantidade como este, porque posta ao servizo da exaltación
  do poder persoal (príncipes, papas...) a creación artística convértese
  nun elemento de prestixio privilexiado, facendo do mecenado unha
  institución obrigada para calquera poderoso.
\end{itemize}

As principais características da música renacentista son:

\begin{itemize}
\item
  \uline{Aspectos estéticos}: o músico do Renacemento desenvolveu un
  concepto de creación que era esencialmente novo. A obra musical é a
  expresión subxectiva do autor, realizada conforme a unhas regras
  racionais e codificadas. Con todo, dado que os músicos dos séculos XV
  e XVI non contaban con modelos antigos, pode dicirse que aquí a
  continuidade coas formas do XIV foi moito maior, de maneira que (como
  noutras ocasións ao longo da historia da música) non é posible falar
  de ruptura.
\end{itemize}

\begin{itemize}
\item
  \uline{Aspectos sociolóxicos}: a música recupera o lugar que, dentro
  da sociedade, desempeñaba para Platón: un aspecto fundamental na
  formación moral do cidadán e tamén un espazo expresivo que o Estado
  debe controlar moi de preto. Dado o seu valor pedagóxico, ocupa tamén
  un lugar principal como espazo para expandir determinadas ideoloxías,
  a través das súas formas e, especialmente, dos textos que a acompañan.
\end{itemize}

Por outra banda, hai que destacar a revalorización da formación musical
na alta sociedade.

\href{http://es.wikipedia.org/wiki/Castiglione}{Castiglione} en Il
\emph{Cortegiano} expresa así este ideal:

\begin{quote}
Tedes que saber que a min non me gusta o cortesán si non é aínda músico
e si, ademais de entender e estar seguro de ler, non coñece varios
instrumentos; pois si pensámolo ben, non se pode atopar ningún repouso
ao traballo nin medicamento para almas enfermas que sexa máis honesto e
meritorio no lecer que este e, sobre todo, na corte.
\end{quote}

\begin{itemize}
\item
  A nivel de relacións de mercado, no Renacemento reafírmase a figura do
  músico profesional, ao servizo da igrexa, unha corte ou unha
  corporación. Os músicos aproveitarán as mellores ofertas en razón da
  ampliación do abanico de posibles clientes, antes case exclusivamente
  eclesiástico. Ademais da seguridade que proporcionaba o sistema do
  mecenado, a posibilidade de publicar as súas obras proporciona unha
  maior diversificación de ingresos e un rápido coñecemento do autor. A
  especialización da música levará á separación entre intérprete e
  receptor, abrindo así o camiño ao concepto moderno de público.
\item
  A imprenta, que posibilitou a difusión da música impresa coa
  consecuente fixación de modelos tanto en obras como en estilos. A
  invención de
  \href{http://en.wikipedia.org/wiki/Sort_\%28typesetting\%29}{tipos
  móbiles} e a súa adaptación á impresión musical con Petrucci e máis
  tarde con Attaingnant
  \href{http://es.wikipedia.org/wiki/Pierre_Attaignant}{,} máis a
  adaptación ao sistema de gravación sobre ferro a finais do XVI,
  fixeron chegar as composicións aos máis recónditos lugares de Europa.
\end{itemize}

\hypertarget{a-muxfasica-renacentista}{%
\subsubsection{A MÚSICA RENACENTISTA}\label{a-muxfasica-renacentista}}

\hypertarget{tuxe9cnicas-compositivas}{%
\paragraph{Técnicas compositivas}\label{tuxe9cnicas-compositivas}}

As técnicas de composición da época son:

\begin{enumerate}
\def\labelenumi{\arabic{enumi}.}
\item
  \textbf{Composición a través de
  \href{http://es.wikipedia.org/wiki/Cantus_firmus}{\emph{cantus
  firmus}:}} consiste en utilizar unha melodía preexistente como base
  dunha nova composición polifónica. O \emph{cantus firmus} procede de
  música relixiosa ou profana, ou ben é de nova creación. O seu modo
  consérvase intacto ou se modifica e o seu ritmo ou estrutura de frase
  tamén pode variar. Aínda que as melodías preexistentes están presentes
  na maioría das primeiras composicións polifónicas xa na Idade Media,
  como vimos, o termo C.F. utilízase por convención para referirse tan
  só á música posterior, a partir do XV.
\end{enumerate}

Hai varios procedementos para a utilización do C.F.:

\begin{itemize}
\item
  Paráfrasis: a voz superior é a que leva o C.F., elaborándoo
  libremente.
\item
  C.F. de tenor: podería estar no tenor ou moverse entre voces
  (\emph{cantus firmus} migrante ou errante).
\item
  Parodia: cando se utiliza como C.F. unha obra polifónica, en todo ou
  en parte (cada voz da obra polifónica proporciona o C.F. para as voces
  da nova composición).
\end{itemize}

\begin{enumerate}
\def\labelenumi{\arabic{enumi}.}
\item
  \textbf{Contrapunto imitativo}: consiste en expor nunha voz un motivo
  (antecedente) que repite outra voz a unha distancia concreta
  (consecuente). Esta entrada pode producirse a altura variable ou ben
  manterse de forma estrita, estendéndose de igual ou similar maneira ás
  demais entradas. Ademais, o antecedente pode estar encuberto polos
  seguintes procedementos:
\end{enumerate}

\begin{itemize}
\item
  Espello ou \emph{speculum} (inversión), convertendo os intervalos
  ascendentes en descendentes e viceversa.
\item
  Cangrexo ou movemento cancrizante (retrogradación), sendo o
  consecuente igual que o antecedente lido de dereita a esquerda.
\item
  Inversión retrogradada, xuntando ambos os procedementos anteriores.
\item
  Aumentación, a duración das notas do consecuente alárgase en relación
  ás do antecedente.
\item
  Diminución, cando se produce o efecto contrario.
\end{itemize}

\begin{enumerate}
\def\labelenumi{\arabic{enumi}.}
\item
  \textbf{Variación}: consiste en repetir un tema variándoo ou
  cambiándoo cada vez que aparece. É factible variar a melodía, o ritmo,
  a harmonía, etc. Empregouse de forma destacada na música instrumental.
  Neste caso, non adoita aparecer o tema de partida, senón directamente
  as variacións ou "diferenzas".
\item
  \textbf{Textura homofónica}: todas as voces cantan en vertical e co
  mesmo ritmo e texto.
\end{enumerate}

\hypertarget{22-caracteruxedsticas}{%
\paragraph{2.2 Características}\label{22-caracteruxedsticas}}

Podemos destacar as seguintes:

\begin{itemize}
\item
  \textbf{Rítmicamente}, as obras tamén se simplifican, fuxindo das
  complexidades da Idade Media. A isorritmia utilízase unicamente en
  celebracións relixiosas solemnes e mantense só durante o século XV.
  Emprégase a \textbf{indicación de compás} ao comezo dunha partitura,
  aínda que \textbf{non se fai uso das liñas divisorias}. O sistema de
  medida mantén os conceptos de \emph{tempus} e \emph{prolatio} herdados
  desde o XIV, evolucionando ao longo do período. O tipo de
  \textbf{pulsación} preferido é o chamado \emph{tactus}, sucesión de
  pulsos regulares de igual xerarquía. Ademais desta idea de pulsación
  interveñen outros enriquecementos rítmicos, especialmente no primeiro
  renacemento, con numerosas síncopas, hemiolias e outras
  irregularidades. Conforme avanza o período, o \textbf{ritmo armónico}
  cobra importancia, especialmente na Escola Romana.
\item
  As \textbf{melodías son fluídas e sinxelas}, estruturadas segundo a
  respiración natural, e menos artificiais que no Medioevo. Móvense
  maioritariamente de grao e por salto consonante, sendo o seu ámbito
  reducido. Utilizan pequenas células ou motivos que se reelaboran ao
  longo do discurso musical.
\item
  \textbf{Utilízanse os antigos modos}, ampliados a doce a metade de
  século grazas aos teóricos
  \href{http://es.wikipedia.org/wiki/Glareanus}{Glareanus} e
  \href{http://es.wikipedia.org/wiki/Gioseffo_Zarlino}{Zarlino} (ao
  considerar a existencia dun si bemol constante no lido daría lugar á
  aparición do jónico ou modo de do; o mesmo feito dará lugar á
  consideración dun eólico ou modo da diferenciado do dórico). Os
  compositores empregan cada vez con máis liberdade as notas alteradas
  (que non pertencen ao modo no que está escrita a obra) e vaise
  preparando o tránsito da vella arquitectura modal ao sistema bimodal
  moderno. As notas non escritas e que se elevan para dar lugar a
  sensibles dos modos dan lugar a un sistema de
  \href{http://es.wikipedia.org/wiki/M\%C3\%BAsica_ficta}{\emph{musica
  ficta}} (ficticia, xa que está fóra do ámbito creado por Guido
  d'Arezzo), que se basea nunhas regras estritas aínda que non do todo
  coñecidas por nós.
\item
  Ordénase e regula o emprego de \textbf{notas auxiliares á harmonía}
  (floreos, notas de paso, retardos, escapadas e \emph{cambiatas}).
\item
  O ideal sonoro baséase nunha textura a catro ou máis voces mixtas de
  igual importancia, que presentan por igual os motivos musicais nunha
  \textbf{textura de contrapunto imitativo}. En canto ás voces, aumenta
  o seu ámbito en relación ao habitual da Idade Media. As utilizadas son
  \emph{superius} ou \emph{cantus}, \emph{contratenor altus},
  \emph{tenor} e \emph{contratenor bassus}.
\item
  É frecuente a utilización de \textbf{motivos recorrentes} (pequenas
  células ou ideas melódicas que se presentan unha e outra vez na obra,
  suxeitas a pequenas elaboracións).
\item
  A polifonía considérase verticalmente, polo que a concepción sucesiva
  das voces que se utilizaba na Idade Media substitúese por unha
  concepción simultánea, pasando do horizontal ao vertical.
\item
  Non hai unha música totalmente instrumental ou vocal; toda a música
  instrumental pódese cantar e viceversa. De igual forma, é factible
  dobrar ou non as voces con instrumentos, aínda que non existe
  especificidad instrumental aínda (calquera instrumento podería dobrar
  calquera voz, sempre que corresponda á súa tesitura).
\item
  No que se refire á relación texto-música, esta expresa os contidos do
  texto sen dificultar a súa comprensión. Nace o termino
  \href{http://es.wikipedia.org/wiki/Musica_reservata}{\emph{música
  reservata}} (termo quizais acuñado por Josquin deas Prez), para
  referirse a aquela música que trata de traducir o máis fielmente
  posible o sentido do texto. Aplicada ao xénero profano dará logo
  tardíamente aos madrigalismos.
\item
  As \textbf{formas} máis empregadas son:

  \begin{enumerate}
  \def\labelenumi{\arabic{enumi}.}
  \item
    Na música relixiosa:

    \begin{itemize}
    \item
      o \textbf{motete} (no significado de peza relixiosa paralitúrgica
      de composición libre),
    \item
      a \textbf{misa}, máis as \textbf{partes do oficio}.

      Abandónase a isorritmia e a politextualidad no motete.
    \end{itemize}
  \item
    Na música profana:

    serán a\emph{chanson}, \emph{frottola}, villancico e o
    \href{http://es.wikipedia.org/wiki/Madrigal_\%28m\%C3\%BAsica\%29}{madrigal};
    este terá unha estrutura similar ao motete pero máis avanzado
    rítmica e harmónicamente. Ademais, experimenta un enorme auxe a
    música instrumental, con multitude de xéneros, así como a música de
    danza.
  \end{enumerate}
\item
  Cada escola presenta unhas características en relación ao número de
  voces, cadencias empregadas, textura, modos, etc.
\item
  Utilízase \textbf{un mesmo estilo} para a música relixiosa e a
  profana.
\item
  Respecto da notación, non hai ningunha achega, senón unha
  \textbf{simplificación} da herdanza recibida. Prodúcese o baleirado
  das notas negras que conduce á "nota branca", baseada nos principios
  da nota mensural negra. Isto permite usar a nota negra cunha nova
  significación de diminución do valor dunha nota, o que posibilita a
  obtención de particularidades rítmicas como a formación de grupos
  irregulares, hemiolias e cambios de compás.
\item
  Existen dous \uline{tipos de textura}:

  \begin{itemize}
  \item
    \textbf{Polifónica}, contrapuntística ou imitativa: baseada na idea
    da imitación a intervalos armónicos regulares e na independencia
    rítmica das voces.
  \item
    \textbf{Homofónica} (ou, con máis propiedade, homorrítmica): as
    voces marchan xuntas formando acordes; é a textura propia de xéneros
    como \href{http://es.wikipedia.org/wiki/Chanson}{\emph{a chanson},}
    a \href{http://es.wikipedia.org/wiki/Frottola}{\emph{frottola}} ou a
    \href{http://es.wikipedia.org/wiki/Villancico}{panxoliña.} É
    habitual que nestas composicións existan algúns fragmentos
    contrapuntísticos, dependendo do compositor de que se trate.
  \end{itemize}
\end{itemize}

Para ejemplificar isto, observaremos as dúas obras seguintes que son, en
realidade, a mesma. Heinrich Isaak escribiu dúas versións de
\emph{Innsbruck, ich muss dich lassen} (Innsbruck, debo abandonarche),
unha homofónica, coa melodía na voz superior, e outra polifónica
(\emph{Gross} \emph{lei muss ich tragen}), partindo dun canon a dous
entre as voces intermedias. Na
\href{http://open.spotify.com/track/6sFWmiPy6R5EeOC58TXO9Y}{interpretación}
que escoitamos, ambas as posibilidades van seguidas, o que facilita a
comprensión das diferenzas entre os dous tipos de textura empregados no
Renacemento:

\begin{figure}
\centering
\includegraphics{/home/robertopradomartinez/Documentos/GitHub/Historia-I/Referentes-TEMARIO/media/image1.jpeg}
\caption{}
\end{figure}

\begin{figure}
\centering
\includegraphics{/home/robertopradomartinez/Documentos/GitHub/Historia-I/Referentes-TEMARIO/media/image2.jpeg}
\caption{}
\end{figure}

\hypertarget{23-periodizaciuxf3n}{%
\paragraph{\texorpdfstring{\textbf{2.3
Periodización}}{2.3 Periodización}}\label{23-periodizaciuxf3n}}

Até entrado o século XVI prevalece un estilo internacional,
relativamente uniforme e en gran medida ditado por compositores
franceses e franco-flamencos. Pódense diferenciar 5 xeracións:

\begin{itemize}
\item
  1ª (1400-1460):
  \href{http://es.wikipedia.org/wiki/John_Dunstable}{Dunstable,}
  \href{http://es.wikipedia.org/wiki/Guillaume_Dufay}{Dufay} e
  \href{http://es.wikipedia.org/wiki/Gilles_Binchois}{Binchois.}
\item
  2ª (1450-1500): \href{http://en.wikipedia.org/wiki/Busnois}{Busnois} e
  \href{http://es.wikipedia.org/wiki/Ockeghem}{Ockeghem.}
\item
  3ª (1490-1520): Josquin
  \href{http://es.wikipedia.org/wiki/Josquin_des_Pr\%C3\%A9s}{deas
  Prés,} \href{http://es.wikipedia.org/wiki/Jacob_Obrecht}{Obrecht,}
  \href{http://es.wikipedia.org/wiki/Heinrich_Isaac}{Isaak.}
\item
  4ª (1520-1560):
  \href{http://es.wikipedia.org/wiki/Cl\%C3\%A9ment_Janequin}{Jannequin,}
  \href{http://es.wikipedia.org/wiki/Adrian_Willaert}{Willaert,}
  \href{http://es.wikipedia.org/wiki/Nicolas_Gombert}{Gombert,}
  \href{http://es.wikipedia.org/wiki/Clemens_non_Papa}{Clemens non
  Papa.}
\item
  5ª (1560-1600):
  \href{http://es.wikipedia.org/wiki/Orlando_di_Lasso}{Lasso.}
\end{itemize}

Ao redor de 1550 xorden e desenvólvense distintos \textbf{estilos
nacionais}. Pódense distinguir:

\begin{itemize}
\item
  Escola romana:
  \href{http://es.wikipedia.org/wiki/Giovanni_Pierluigi_da_Palestrina}{Palestrina.}
\item
  Escola española:
  \href{http://es.wikipedia.org/wiki/Crist\%C3\%B3bal_de_Morales}{Morais,}
  \href{http://es.wikipedia.org/wiki/Francisco_Guerrero}{Guerreiro,}
  \href{http://es.wikipedia.org/wiki/Tom\%C3\%A1s_Luis_de_Victoria}{Vitoria.}
\item
  Escola veneciana:
  \href{http://es.wikipedia.org/wiki/Andrea_Gabrieli}{Andrea} e
  \href{http://es.wikipedia.org/wiki/Giovanni_Gabrieli}{Giovanni
  Gabrieli.}
\item
  Escola inglesa:
  \href{http://es.wikipedia.org/wiki/William_Byrd}{Byrd,}
  \href{http://es.wikipedia.org/wiki/Thomas_Tallis}{Tallis,}
  \href{http://es.wikipedia.org/wiki/Thomas_Morley}{Morley.}
\end{itemize}

\hypertarget{as-escolas-compositivas}{%
\subsubsection{As escolas compositivas}\label{as-escolas-compositivas}}

\hypertarget{31-a-escola-franco-flamenca}{%
\paragraph{\texorpdfstring{\textbf{3.1 A escola
Franco-flamenca}}{3.1 A escola Franco-flamenca}}\label{31-a-escola-franco-flamenca}}

\begin{quote}
A guerra dos Cen Anos conduciu a unha diminución da importancia musical
de Francia, que fai desviar a hexemonía musical desta nación e de Italia
cara a Inglaterra, Borgoña e, sobre todo, Flandes. Estas cortes serán
escenario de festas relixiosas e profanas nas que a música ocupa un
posto relevante. Nelas fúndanse capelas musicais principescas a
imitación da papal.

Na evolución da música do continente atopamos dúas tendencias: unha,
central, herdeira do \emph{Ars Nova} e do \emph{Ars Subtilior} e outra,
periférica, de influencia inglesa. Foi esencial neste sentido a achega
de Dunstable, que estaba ao servizo do duque de Bedford e, por tanto,
pertencía ás tropas invasoras do continente, e que entrou en contacto
con músicos franco-flamencos e borgoñones. A súa música caracterízase
por:
\end{quote}

\begin{itemize}
\item
  Emprego de ritmos regulares, melodías sinxelas e consonancias de 3ª e
  6ª (fabordon).
\item
  Utilización do mesmo tenor como C.F. para outorgar coherencia á misa
  (misa de tenor).
\item
  Motetes a tres voces (algún a catro); aínda emprega a isorritmia
  nalgún deles (doce do trinta conservados).
\item
  Cancións a tres voces con influencias tanto italianas como da
  \emph{Chandon}.
\end{itemize}

Adóitanse distinguir varias xeracións de compositores franco-flamencos
(ou borgoñones):

\begin{itemize}
\item
  A \textbf{primeira xeración} ten ao seu máximo representante en Dufay,
  que realiza composicións sinxelas, a tres e catro voces, nas que se
  evita a imitación. Nas súas misas estableceuse como práctica regular a
  composición do común como un todo musicalmente unificado mediante a
  utilización dun mesmo \emph{cantus firmus}, ben fose sacro ou profano.
  Abandónase a politextualidad e a isorritmia progresivamente. {[}Para o
  resto das súas características, consulta os apuntamentos da audición
  \href{http://es.wikipedia.org/wiki/Nuper_Rosarum_Flores}{\emph{Nuper
  rosarum flores}}{]}. Na \emph{chanson} emprega fórmelas \emph{fixes}
  herdadas da Idade Media.
\item
  A \textbf{segunda xeración} de franco-flamencos está representada por
  Ockeghem. A súa característica máis importante é a utilización do
  contrapunto imitativo, a miúdo rigoroso, especializándose os
  compositores na técnica do canon. En canto ás técnicas
  contrapuntísticas utilizadas, usábanse catro posibilidades en canto á
  marcha das voces, forma fundamental (na dirección orixinal),
  retrogradación, investimento e retrogradación do investimento. O ritmo
  faise máis complexo e a textura máis densa. Compón misas sobre
  diferentes procedementos: C.F. de tenor (con C.F. por imitación ou ben
  errante), paráfrasis, libre... {[}Para o resto das súas
  características, consulta os apuntamentos da audición \emph{Missa
  Caput}.{]}
\item
  Os compositores da \textbf{terceira xeración} viaxaron a distintas
  cortes e igrexas de Europa, polo que recibiron importantes influencias
  musicais, tanto do sur como do norte de Europa. Obsérvase un
  aligeramiento das texturas, así como ritmo e melodía máis sinxela. A
  figura principal foi Josquin deas Prés. As súas grandes obras están
  desenvolvidas a partir de C.F. tomado do canto chairo ou ben de obras
  profanas, como no caso de
  \href{http://open.spotify.com/track/5SWo3PgUcCLFblqvS2U0sw}{\emph{L'homme
  arme},} tema moi utilizado nesta época e empregado como C.F. por el en
  dúas misas (Misa \emph{L'homme} \emph{arme} \emph{super voces
  musicais} e Misa
  \href{http://open.spotify.com/track/4bx9nePdM3wwj3Z7FZ0owv}{\emph{L'homme
  arme} de sexto ton}). Este autor ten especial preocupación polo
  inteligible do texto. Usa o simbolismo musical, é dicir, símbolos
  expresados musicalmente (tresillos para a trinidad, por exemplo).
  {[}Para o resto das súas características, consulta os apuntamentos da
  audición \emph{Déploration sur lle mort d'Ockeghem}.{]}
\end{itemize}

\begin{itemize}
\item
  Na \textbf{cuarta xeración}, a escritura vólvese máis densa, sendo
  habituais as composicións a cinco e seis voces. As misas máis
  frecuentes son as de parodia, utilizando o material dunha
  \emph{chanson} ou dun motete.
\item
  Na \textbf{quinta xeración}, a textura utilizada é a de cinco ou seis
  voces \emph{a capella} e cóidase a disposición do texto e a relación
  entre a música e os contidos expresados por el. Utilizan texturas
  contrapuntísticas e homofónicas alternadas. A forma principal é o
  motete. Un dos principais compositores foi Orlando di Lasso.
\end{itemize}

\hypertarget{32-a-muxfasica-alemuxe1-lutero-e-a-reforma}{%
\paragraph{\texorpdfstring{**3.2 A música alemá: Lutero e a Reforma
}{**3.2 A música alemá: Lutero e a Reforma }}\label{32-a-muxfasica-alemuxe1-lutero-e-a-reforma}}

En 1517, coa fixación das
\href{http://es.wikipedia.org/wiki/Las_95_tesis}{95 teses de Wittemberg}
por Lutero \href{http://es.wikipedia.org/wiki/Mart\%C3\%ADn_Lutero}{,}
iníciase o proceso que habería de desembocar no cisma protestante. A
partir de 1521, o luteranismo difúndese por toda Europa central. No caso
de Inglaterra o contencioso político-diplomático co papado levará á
ruptura en 1534.

Desde o primeiro momento, Lutero mostrouse partidario de aproveitar a
música dentro da praxe litúrxica e social. Concibiu unha dobre praxe:
unha máis popular, en lingua alemá, que permitise a participación activa
da comunidade, e outra máis elaborada, na tradición do estilo
franco-flamenco.

A coral é en orixe un canto relixioso monódico, en alemán, con ritmo
sinxelo, melodía de ámbito curto, por graos conxuntos e estilo silábico.
A súa sinxeleza, e o feito de que moitas destas melodías fosen
populares, garantiron unha ampla difusión deste estilo vocal. A
execución da coral ía desde os máis sinxelos monódicos (asemblea ao
unísono) até as grandes presentacións das corais harmonizados, onde á
melodía cantada pola asemblea súmase o acompañamento de órgano ou
instrumentos e dun coro especializado.

Ao longo da evolución da coral, distinguimos:

\begin{itemize}
\item
  Coral de tenor (como C.F.), propio do século XVI, tanto polifónico
  como homofónico. De estilo sinxelo, é cadrado, sobre baixo armónico e
  con cadencias claras.
\item
  Coral de soprano (como C.F., é dicir, paráfrasis), a catro partes con
  acompañamento de órgano; aparece a finais do XVI.
\item
  Motete-coral, da mesma época, tomando a coral como C.F. en estilo
  motete (contrapunto imitativo).
\end{itemize}

\hypertarget{33-a-muxfasica-da-contrareforma-a-escola-romana}{%
\paragraph{\texorpdfstring{\textbf{3.3 A música da Contrareforma: a
Escola
Romana}}{3.3 A música da Contrareforma: a Escola Romana}}\label{33-a-muxfasica-da-contrareforma-a-escola-romana}}

A resposta da igrexa católica ás reformas protestantes foi a
convocatoria do
\href{http://es.wikipedia.org/wiki/Concilio_de_Trento}{Concilio de}
\href{http://es.wikipedia.org/wiki/Concilio_de_Trento}{Trento,} que se
celebrou na cidade do mesmo nome entre os anos 1545-1563. É nos dous
últimos anos cando se abordan as cuestións musicais e se debate sobre a
música a empregar na liturxia.

As sesións do Concilio de Trento puxeron de manifesto a necesidade dunha
reforma en profundidade da práctica musical no seo da liturxia. Os
procedementos das misas de parodia, os excesos dos instrumentos, o
artificio das voces... chegaban a facer incomprensible o texto sacro,
polo que conduciron a un afastamento perigoso para a dogma e a
edificación de os fieis que se debía buscar na liturxia. Estas
orientacións de estilo deron lugar ao que se coñece como Escola romana.

Por Escola Romana enténdese ao grupo de compositores que actuaron
durante o século XVI na Capela Papal en Roma, cuxo máximo representante
é G. P. dá Palestrina. A súa música caracterízase por unha melodía
sinxela e \emph{cantabile}, ritmo fluído e tranquilo, texturas
imitativas pero non excesivamente densas, harmonías triádicas, textos
comprensibles con preferencia ao emprego de tenores gregorianos como
C.F. (aínda que emprega tamén paráfrasis, parodia e composición libre).
O seu estilo é sobrio austero e equilibrado. A súa música foi
considerada a expresión máis perfecta do estilo eclesiástico. A súa arte
resume todo o século que lle precede, englobando todas as técnicas da
composición polifónica e os seus xéneros. {[}Para o resto das súas
características, consulta os apuntamentos da audición *Misa Ascendo ad
atrem .{]}

A obra de Palestrina é case integramente relixiosa, aínda que tamén
compuxo madrigales sacros e profanos. Foi admirado en toda Europa, aínda
que a súa calidade non reside tanto na novidade dos seus métodos como na
intelixencia coa que os utilizou.

\hypertarget{34-a-escola-veneciana}{%
\paragraph{\texorpdfstring{\textbf{3.4 A Escola
Veneciana}}{3.4 A Escola Veneciana}}\label{34-a-escola-veneciana}}

Si a Escola Romana significou o culmen e perfeccionamento dentro dunha
evolución, máis que innovación, a Escola Veneciana presenta
características singulares que a diferencian claramente do resto das
escolas polifónicas.

Aparecen por primeira vez música para dúas ou máis coros de voces
(\href{http://es.wikipedia.org/wiki/Estilo_policoral_veneciano}{coro
spezzato}), creando con iso uns efectos antifonales de coros que
dialogan. O escenario no que se dispuña a música, a catedral de estilo
bizantino de San Marcos, cuberta de cúpulas e de magníficos mosaicos,
proporcionaba un marco incomparable para a experimentación da técnica
policoral; a estes coros podían agregárselles instrumentos para
reforzalos. O efecto policoral, tal e como se practicou en Venecia en
tempos dos
\href{http://open.spotify.com/track/6r9zLucRqaSSyV4XS4t7HL}{Gabrieli,}
está na base da técnica do \emph{concertato} que se impón como o ideal
estético do Barroco a partir dos inicios do XVII. O aumento no número de
voces e o emprego do dobre coro proporcionoulle gran brillo e esplendor.

Todo isto é a representación do ambiente festivo-relixioso, cívico e
social que se desenvolve ao redor da
\href{http://es.wikipedia.org/wiki/Bas\%C3\%ADlica_de_San_Marcos}{Basílica
de San Marcos,} o Palacio do Duce e a nobreza veneciana. O promotor da
nova escola veneciana é o flamenco
\href{http://es.wikipedia.org/wiki/Adrian_Willaert}{Willaert,} mestre de
capela de San Marcos. Ademais do novo estilo do dobre coro, os seus
progresos foron considerables no campo da música instrumental. Outros
autores, á parte dos Gabrieli, foron
\href{http://es.wikipedia.org/wiki/Gioseffo_Zarlino}{Zarlino} e
\href{http://es.wikipedia.org/wiki/Cipriano_de_Rore}{Cipriano de Rore.}

\hypertarget{35-inglaterra}{%
\paragraph{\texorpdfstring{\textbf{3.5
Inglaterra}}{3.5 Inglaterra}}\label{35-inglaterra}}

A igrexa anglicana separouse da católica en 1534, durante o reinado de
\href{http://es.wikipedia.org/wiki/Enrique_VIII_de_Inglaterra}{Enrique
VIII.} A misa non sufriu grandes cambios, salvo a substitución do latín
polo inglés. Ademais, a separación de Roma non implicará que se deixe de
compor música para a liturxia católica (como fixo William Byrd). Dentro
dos cantos propiamente anglicanos destacamos o \emph{Sevice} (servizo) e
o \emph{Anthem} (que poderiamos traducir como Himno). Ou \emph{Anthem}
constitúe unha das formas máis peculiares da liturxia anglicana,
ocupando un lugar similar ao da coral protestante en canto á súa función
e trazos formais. Dentro do \emph{Service} inclúense os cantos de
oficios de maitines e vésperas, así como a misa anglicana.

\hypertarget{36-o-renacemento-en-espauxf1a}{%
\paragraph{\texorpdfstring{\textbf{3.6 O Renacemento en
España}}{3.6 O Renacemento en España}}\label{36-o-renacemento-en-espauxf1a}}

Este período é considerado como a Idade de Ouro da música española, que
alcanza fama internacional. A obra dos grandes mestres polifonistas do
Renacemento español é concentrada, austera e impregnada dun profundo
misticismo. Este misticismo, que é a súa calidade máis determinante,
conséguese a través dunha expresividade profunda. As necesidades
expresivas levan aos nosos músicos a compor nunha linguaxe moderna, con
disonancias, emprego artístico do silencio e uso persoal do contrapunto.

Estas calidades emparentan aos nosos músicos con outros artistas da
época que tratan de dicir o mesmo con outros medios artísticos, por
exemplo \href{http://es.wikipedia.org/wiki/El_Greco}{O Greco} (en cuxa
pintura existe o mesmo misticismo),
\href{http://es.wikipedia.org/wiki/Alonso_Berruguete}{Berruguete}
(escultura) ou san
\href{http://es.wikipedia.org/wiki/Juan_de_la_Cruz}{Juan da Cruz} e
santa \href{http://es.wikipedia.org/wiki/Teresa_de_\%C3\%81vila}{Teresa
de Ávila} (literatura). A maior parte da nosa música vocal é relixiosa,
aínda que tamén existan obras profanas de Juan Vázquez e Mateo Frecha.

A) A época dos Reis Católicos

Cos Reis Católicos dá comezo un período de esplendor cultural. A música
española, tanto durante este reinado como cos posteriores da casa dos
Austrias, foi moito máis rica e importante do que se pensou até o século
XIX, cando se descubriu o \emph{Cancioneiro musical de} \emph{Palacio}
na biblioteca do Palacio Real de Madrid.

Debido ás relacións comerciais e familiares dos monarcas españois con
Anjou, Borgoña, Flandes e Nápoles, a música renacentista da Península
asimilou a técnica contrapuntística franco-flamenca e a linguaxe
homofónico presente nas formas semipopulares italianas, caracterizándose
por posuír unha das tradicións máis ricas dentro de Europa. Así, xunto
ao estilo internacional xorde en España unha escola polifónica nacional,
de tradición moito máis popular. A isto contribúe tamén o feito de que
os Reis Católicos non nutren a súa Capela Real só de músicos
estranxeiros, como os seus antecesores, senón que os integrantes das
súas capelas vocal e instrumental son músicos dos seus reinos. O estilo
nacional componse fundamentalmente de música profana, mentres que o
internacional era empregado sobre todo para música sacra.

A principal fonte a través da que se coñece o repertorio e os usos dos
instrumentos renacentistas é o
\href{http://es.wikipedia.org/wiki/Cancionero_Musical_de_Palacio}{\emph{Cancionero
de Palacio},} que contén cancións baseadas en cantos anteriores e no que
se atopa o repertorio musical da corte española da época. A maior parte
pertencen ao xénero da panxoliña (que toma a estrutura dos virelais ou
das ballatas), ademais de haber cancións e romances. Moitas composicións
tratan de amor cortés e outras teñen temática popular e cortesá.

Aínda que a música máis importante foi a relixiosa, os nobres cultivaron
a música profana e instrumental, seguindo unha tradición que comezara no
Medioevo e que estaba moi influída polas músicas que cantaba o pobo. A
música profana defínese pola influencia que recibe da música popular, o
apego ao texto en castelán, un forte carácter rítmico e certa tendencia
á homorritmia, con pouco uso do contrapunto. Esta música profana
expresouse a través de tres formas importantes: o romance, a ensalada e
a panxoliña.

\begin{itemize}
\item
  O \href{http://open.spotify.com/track/6ltOfn1evFQ2ZtXesbKAEM}{romance}
  é unha forma polifónica de carácter popular sobre temas do antigo
  romancero. Está integrado por catro frases musicais que corresponden
  ao catro versos de cada estrofa (con esta música cántanse todas as
  demais cuartetas). Consérvanse moitos no \emph{O} \emph{cancionero de
  Palacio}.
\item
  A
  \href{http://open.spotify.com/track/4dLxTrRRCpqV59G9S2GqvC}{ensalada}
  é un xénero polifónico profano no que se mesturan os diferentes
  estilos do madrigal, canción popular, panxoliña, romance e danza (de
  aí o seu nome).
\item
  A panxoliña (villancico) é unha forma musical profana de orixe popular
  que consta de tres partes: refrán, copla e refrán, levando a melodía
  na voz superior, cunha harmonía sinxela, unha textura homorrítmica e
  un estilo silábico. O nome fai referencia ás cancións "de viláns", de
  estilo popular nos temas. Formalmente ten as seguintes
  características:

  \begin{itemize}
  \item
    Composicións para tres ou catro voces, que poden ser dobradas ou
    suplidas por instrumentos.
  \item
    A harmonía é xeralmente sinxela, con uso dun baixo armónico.
  \item
    A melodía adoita ir na voz superior, é silábica, móvese
    preferentemente por graos conxuntos e é facilmente cantable.
  \item
    O ritmo é flexible, utilizándose con frecuencia ritmos de danzas.
  \end{itemize}
\item
  A textura predominante é de tipo homofónico.
\item
  O texto pode tratar temas amatorios ou galantes, pero tamén de tipo
  político; na segunda metade do século XVI fai a súa aparición a
  panxoliña relixiosa.
\item
  A forma é similar á do virelai (AbbaA), onde A sería a parte do refrán
  que se repite sempre tal cal, bb correspondería aos cambios (partes
  que mudan en cada copla) e a corresponde á volta (música de A, pero
  cambiando o texto).
\end{itemize}

Un tipo concreto de panxoliña é o relixioso, dotado dun texto sacro,
xeralmente en castelán, composto para unha celebración eclesiástica
concreta (como o Nadal ou a Semana Santa).

Exemplo:
\href{http://open.spotify.com/track/0l6KJwUYfv2GuKJ2HpOq1L}{\emph{Todos
os bens do mundo},} \emph{panxoliña} de
\href{http://es.wikipedia.org/wiki/Juan_del_Enzina}{Juan do Enzina,}
\emph{ilustra} este xénero.

\begin{quote}
Todos os bens do mundo
\end{quote}

\begin{quote}
pasan presto e a súa memoria,
\end{quote}

\begin{quote}
salvo a fama e a gloria.
\end{quote}

\begin{quote}
O tempo leva os uns,
\end{quote}

\begin{quote}
a outra fortuna e sorte.
\end{quote}

\begin{quote}
e ao cabo vén a morte,
\end{quote}

\begin{quote}
que non nos dexa ningúns.
\end{quote}

\begin{quote}
Todos son bens fortunos
\end{quote}

\begin{quote}
E de moi pouca memoria
\end{quote}

\begin{quote}
salvo a fama e a gloria.
\end{quote}

\begin{quote}
Procuremos boa fama,
\end{quote}

\begin{quote}
que xamais nunca se perde,
\end{quote}

\begin{quote}
árbore que sempre está verde
\end{quote}

\begin{quote}
e co froito na rama.
\end{quote}

\begin{quote}
Todo ben que ben se chama
\end{quote}

\begin{quote}
pasa presto e a súa memoria
\end{quote}

\begin{quote}
salvo a fama a gloria.
\end{quote}

Esta panxoliña está presente no \emph{Cancionero Musical de Palacio},
tamén chamado \emph{Cancionero} \emph{Barbieri} (por ser este músico
madrileño o seu descubridor). Polo número de pezas que contén, pode
dicirse que é o mellor compendio do que debeu ser a música de corte do
período.

As composicións son de autor diverso, ademais dalgunhas anónimas e
pertencen a xéneros relacionados coa música popular: \emph{romances,
frottolas, panxoliñas, cancións}... a maioría están escritas en castelán
pero tamén as hai en italiano, portugués, vasco e francés. As formas e
estilos tamén son variados, de modo que xunto a pezas en estilo
polifónico atopamos outras en estilo homofónico. Entre todos os autores
presentes, Juan do Enzina (ou da Aciñeira) é o máis representativo, xa
que ademais de músico, foi un relevante escritor do Renacemento, con
importante obra poética e escénica\emph{.}

\emph{Todos os bens do mundo} é unha composición de carácter
moralizante, cun argumento propio da época: a fugacidad da vida e a
permanencia da memoria por medio da honra e a fama.

Trátase dunha composición estrófica: unha panxoliña a catro voces,
marcadamente homorrítmico, baseado nos acordes nas cadencias da e re,
cun posicionamento armónico característico nos xéneros polifónicos
considerados populares.

Desde o punto de vista formal, do mesmo xeito que os preto de medio
centenar de panxoliñas que escribiu Juan do Enzina no Cancionero Musical
de Palacio, conserva literalmente o esquema dos virelais: un refrán
(neste caso de tres versos) máis estrofa de sete versos asimétricos (4 +
3 semellantes estes últimos aos do refrán). A música tamén conserva o
mesmo esquema: A(refrán) / BBA (estrofa), mostra dun certo arcaísmo,
propio do primeiro Renacemento musical na península.

Como é habitual, a harmonía é sinxela, con utilización dun baixo
armónico (frecuentes saltos de 4ª e 5ª), e con terceiras paralelas entre
as voces. A melodía atópase na voz superior e discorre silábica e por
graos conxuntos. Desde ou punto de vista rítmico segue un patrón
sinxelo.

Como xa se viu, o compositor máis destacado do período é Juan do Enzina
(1468-1530), músico e poeta que compón a parte máis importante da súa
obra cando entra ao servizo do Duque de Alba. Aínda que foi sacerdote,
non compuxo música para a liturxia, probablemente ao estar o seu
traballo vinculado a unha corte nobiliaria e secular. Foi autor de
moitas panxoliñas de tipo pastoril e de música instrumental relacionada
cos madrigales cultos italianos. Conxuga nestas obras o elemento popular
tradicional coas máis vangardistas tendencias do renacemento italiano.

En música relixiosa destaca Francisco de Peñalosa (1470-1528), o
compositor máis prolífico do seu tempo. Aínda que compón panxoliñas e
cancións, case toda a súa obra é de carácter relixioso.

B) A segunda época: Música nos reinados de Carlos I e Felipe II.

Entre 1516 e 1598, coincidindo cos reinados de Carlos I e Felipe II,
alcanza a súa plenitude o renacemento musical español. O elevado nivel
técnico e expresivo dos compositores do momento leva as súas obras a
gozar da máxima consideración entre os seus coetáneos europeos. Os
contactos que sempre existiran entre España e outros países
intensificáronse no século XVI debido en parte ao nomeamento de Carlos I
como Carlos V, emperador de Alemaña, e á súa sucesión por parte de
Felipe II. Tamén contribúe o papel que asume a Coroa Española como
paladín das directrices propugnadas pola Igrexa de Roma.

Os grandes compositores do Renacemento español son Cristóbal de Morais e
Tomas Luís de Vitoria no ámbito relixioso e vocal, e Antonio de Cabezón
(1510-1566) no instrumental. No plano teórico, a figura de Bartolomeu
Ramos de Parella alcanza gran soa internacional.

Cristóbal de Morais (1500-1533) é o máis grande polifonista da escola
andaluza da primeira metade do século XVI. Morais, compositor de gran
prestixio, obtivo recoñecemento alén das nosas fronteiras, sendo as súas
partituras editadas en Italia, Francia, Alemaña e os Países Baixos. As
súas
\href{http://open.spotify.com/track/68KvnM5BJ153VwHjQOMaKN}{composicións}
caracterízanse pola súa austeridade, alternando texturas
contrapuntísticas coa homofonía; trátase do autor español que máis se
deixou influír polas composicións do franco-flamencos. A súa obra é
maioritariamente relixiosa.

As abulense Tomas Luís de Vitoria (1548-1611) estudou en Roma, onde
traballou como mestre de capela do Seminario de Roma. A case totalidade
da súa obra é de carácter relixioso. Con todo, aparecen nela recursos
expresivos característicos da linguaxe do madrigal. Nel apréciase unha
evolución desde un primeiro estilo influenciado por Palestrina,
engadindo a austeridade e misticismo propiamente hispanos, como se pode
observar no seu motete
\href{http://open.spotify.com/track/6WWvS5n5Btte7xchnbPOw1}{Vere
languores.} Destaca na súa obra:

\begin{itemize}
\item
  O coidado pola comprensibilidad do texto seguindo as directrices do
  Concilio de Trento.
\item
  A utilización de disonancias e silencios como elementos expresivos.
\item
  Unha alternancia de pasaxes homofónicos con outros suavemente
  contrapuntísticos.
\item
  O emprego de cromatismos atrevidos ou de cambios rítmicos aproxímano
  xa á expresividade do Barroco, igual que o emprego do policoralismo.
\end{itemize}

Outros dos grandes polifonistas foron Francisco de Guerreiro (1528-1599)
e Juan Vázquez (1500-1560).

En canto á música instrumental, destaca a composición para órgano e
vihuela. Antonio de Cabezón, cego desde neno, chegou a ser músico de
cámara de Carlos V e organista principal do seu fillo Felipe II. É
considerado como o creador da escola organística española, aínda que
compuxo tamén para tecla, arpa e viola dá gamba. Coñecía as técnicas e
estilos europeos, de onde toma a súa concepción harmónica e
contrapuntística. Compuxo unha ampla obra, case toda relixiosa, entre a
que destacan tientos ou "fugas" (especie de motetes instrumentais
baseados na imitación, a catro voces), todos eles creación propia, aínda
que algúns toman pequenos elementos melódicos do canto gregoriano. Os
tientos de Cabezón caracterízanse pola variedade de estrutura e polo uso
da composición utilizando a aumentación e a diminución.

En contraste, as súas diferenzas (ciclos de variacións sobre temas
populares ou cancións francesas) toman sempre un elemento xerador alleo.
Destas destacan as
\href{http://open.spotify.com/track/3KIn5dZJbmc4bJ4aJpyGnl}{\emph{Diferenzas
sobre o canto do}}
\href{http://open.spotify.com/track/3KIn5dZJbmc4bJ4aJpyGnl}{\emph{Caballero}}
pola súa beleza e perfección:

\begin{figure}
\centering
\includegraphics{/home/robertopradomartinez/Documentos/GitHub/Historia-I/Referentes-TEMARIO/media/image3.jpeg}
\caption{}
\end{figure}

A vihuela gozou de gran popularidade no século XVI. Autores como Milán,
Narváez ou Diego Pisador compuxeron fantasías, diferenzas e cancións
que, aínda que influídas polas obras dos laudistas italianos, constitúen
un repertorio imprescindible dentro da música renacentista española.
Obra a destacar son as
\href{http://open.spotify.com/track/6NTeSGJjH5nL7FRHf0L09S}{\emph{Diferenzas
sobre Gárdame as vacas},} musicada por varios autores.

\hypertarget{xuxe9neros-vocais-nacionais}{%
\subsubsection{XÉNEROS VOCAIS
NACIONAIS}\label{xuxe9neros-vocais-nacionais}}

\hypertarget{41-previos}{%
\paragraph{\texorpdfstring{\textbf{4.1
Previos}}{4.1 Previos}}\label{41-previos}}

En xeral, estes xéneros comparten unhas características, aínda que
existen particularidades zonais:

\begin{itemize}
\item
  Ritmo preciso e marcado, con células ou motivos repetidos.
\item
  Melodía cadrada e delimitada por cadencias, sinxelas e
  \emph{cantabiles}.
\item
  Harmonía baseada na tríada, con baixos armónicos e con numerosas
  progresións tonales.
\item
  Textura preferentemente homofónica, con presenza de pasaxes imitativos
  (de forma destacada na \emph{chanson}).
\end{itemize}

A
\href{http://open.spotify.com/track/0d6zn7tm3FYToN3oR0ad9M}{\emph{frottola}}
é un xénero italiano que xorde a finais do XV e inicios do XVI. En canto
á \emph{chanson}, distínguense por períodos a \emph{chanson} en estilo
motete (principios do XVI),
\emph{\href{http://open.spotify.com/track/3Gnfe3ubxrEsquzLAGx9uH}{chanson}}
\href{http://open.spotify.com/track/3Gnfe3ubxrEsquzLAGx9uH}{homofónica}
(descritiva e moi expresiva, de mediados do XVI) e a influída polo
madrigal (máis cromática e contrapuntística, da segunda metade do XVI).

\hypertarget{42-o-madrigal}{%
\paragraph{\texorpdfstring{\textbf{4.2 O
madrigal}}{4.2 O madrigal}}\label{42-o-madrigal}}

O madrigal do XVI non ten nada que ver, formal nin literariamente, co do
século XIV, co que só comparte o nome, en virtude dunha parcial
coincidencia en temas de tipo pastoril e amatorio. O madrigal
renacentista orixínase en Italia no segundo cuarto do XVI, pero
expándese por toda Europa, interpretándose en toda clase de reunións
sociais e cortesás.

O madrigal é unha forma musical polifónica de xénero profano, organizada
en seccións en función do texto, de carácter descritivo que pretende, a
través da unión de letra e música, expresar sentimentos. Na súa
expresión alcanza un nivel moi depurado; adoita ser a catro ou cinco
voces, \emph{a capella} (aínda que se podían introducir instrumentos
dobrando ou substituíndo algunha parte) e foi unha das manifestacións
máis acabadas do ideal de \emph{musica reservata}. É a expresión lírica
do ser humano que canta as súas vivencias, os seus sentimentos, que
trata de demostrar que el é o centro do mundo. Dáse en ambientes
cortesáns como exaltación de sentimentos sensuais e amorosos e naqueles
lugares onde é máis forte o ambiente profano, como Venecia e Inglaterra.

As súas características son:

\begin{itemize}
\item
  Emprego dunha linguaxe musical difícil, culto e para minorías.
\item
  Uso do cromatismo e de linguas vernáculas.
\item
  Gran elaboración contrapuntística combinada con homofonía.
\item
  Forma: libre, rexida polo texto; cada frase textual articula unha
  sección de música.
\item
  Texto: formado por estrofas cun esquema de rima libre e un número
  moderado de versos (xeralmente de sete e once sílabas). O que máis
  caracteriza ao madrigal é o coidado no tratamento do texto, que debe
  ser o máis fielmente posible traducido musicalmente, a miúdo mediante
  sofisticados procedementos que se denominan madrigalismos. Para iso
  son frecuentes os cambios de ritmo e o uso de cromatismos.
\end{itemize}

Podemos falar dunha evolución dentro do madrigal:

\begin{itemize}
\item
  \href{http://open.spotify.com/track/0fmcNVwu3fbEqcoHCx13X0}{Madrigal
  primitivo} (1530-1550). Escritura a catro voces, con predominio da
  textura homofónica, frases marcadas por cadencias claras e harmonía
  sinxela. As voces podían ser dobradas instrumentalmente. Autores
  importantes son Verdelot, Arcadelt e Willaert.
\item
  \href{http://open.spotify.com/track/1pb7MyBUi3pZnbxwoSu2yu}{Madrigal
  clásico} (1550-1580). A cinco ou seis voces, contén efectos
  descritivos e todo tipo de madrigalismos "visuais", aceptándose o
  cromatismo como medio de expresión. Autores destacados son Orlando dei
  Lasso e Cipriano de Rore.
\item
  Madrigal tardío ou manierista (1580-1620). Acentúase todo tipo de
  madrigalismos, cromatismos e irregularidades rítmicas. Destacan
  Gesualdo, Príncipe de Venosa e Claudio Monteverdi, autor co que se dá
  tránsito ao Barroco. O madrigal alcanza o seu máximo esplendor e
  experimenta todas as audacias posibles.
\end{itemize}

Claudio Monteverdi foi ou último gran compositor de madrigales; o seu
oito libros de madrigales marcan o tránsito do madrigal renacentista ao
barroco. Desta colección, o catro primeiros libros son de estilo
renacentista, pero a utilización dun baixo continuo e a preferencia pola
monodia acompañada en lugar da polifonía suporán o inicio do Barroco.

A expansión do madrigal italiano xerou variantes locais que nalgún caso
alcanzaron un desenvolvemento autónomo do modelo orixinal, como é o caso
do madrigal inglés, con figuras da talla de Thomas Morley. Derivado
deste xénero creouse un tipo de canción con acompañamento de laúde e un
tipo de viola dá gamba
(\href{http://en.wikipedia.org/wiki/Lyra_viol}{\emph{lyra viol}})
chamado simplemente \emph{song} ou
\href{http://open.spotify.com/track/2qswQJGVw0CID7luFgs7iA}{\emph{ayre},}
así como cancións para solista ou dúo con acompañamento de
\emph{consort} instrumental.

\hypertarget{5-muxfasica-instrumental}{%
\subsubsection{5. MÚSICA INSTRUMENTAL}\label{5-muxfasica-instrumental}}

Até case o século XVI a historia musical desenvólvese case
exclusivamente sobre o plano vocal. Tan só interviñeron os instrumentos
dentro da música culta en forma limitada e incidental, e como simples
duplicadores da voz. Con todo, o Renacemento espertou xa un interese
pola música instrumental que segue en aumento ata que a mediados do
século XVIII excedeu en importancia á música vocal. Aparecen xa libros
nos que describen os instrumentos e danse instrucións sobre como
interpretalos; ademais, temos numerosas fontes escultóricas e
pictóricas.

A emancipación da música instrumental pode ser atribuída a varios
factores:

\begin{itemize}
\item
  Histórico-sociais: crecemento (en número e posición social) das clases
  burguesas e das cidades, o que supón unha maior demanda de música
  profana e dentro dela da especificamente instrumental.
\item
  Organológicos: durante os séculos XV e XVI prodúcese un notable
  progreso na construción dos instrumentos, o que repercute nun avance
  na afinación, extensión, tesitura e posibilidades dinámicas.
\end{itemize}

Como consecuencia da citada emancipación da música instrumental:

\begin{itemize}
\item
  Xurdiron novas formas
\item
  Interprétanse varios instrumentos xuntos
  (\href{http://es.wikipedia.org/wiki/Consort}{\emph{consort}}).
\item
  Algúns comezan a agruparse, segundo características técnicas e
  sonoras, por familias e por tesituras dentro de cada familia, co que
  de desenvolvemento destas supuxo.
\item
  A nacente imprenta comeza a imprimir libros de música instrumental.
\item
  A música instrumental anótase comunmente ou ben empregando tablaturas.
  Existen varios tipos destas, segundo época e zonas, estendidas
  maioritariamente para instrumentos de corda pulsada e de teclado.
\end{itemize}

A pesar da súa importancia, o Concilio de Trento prohibiu o uso de
instrumentos nas igrexas, excepto o órgano. A prohibición pon de
manifesto que estes instrumentos se usaron amplamente en datas previas.

Entre os compositores italianos dedicados a música instrumental é
especialmente famoso Giovanni Gabrielli (1557-1612), organista da
catedral de San Marcos de Venecia; a súa música posúe un carácter
instrumental, ao outorgar a cada instrumento unha función distinta e
característica.

\hypertarget{51-clasificaciuxf3n}{%
\paragraph{\texorpdfstring{\textbf{5.1
Clasificación}}{5.1 Clasificación}}\label{51-clasificaciuxf3n}}

A música instrumental clasifícase en función dos modelos dos que deriva:

a) Repertorio derivado de modelos vocais

Durante o Renacemento, a música vocal e instrumental estaba
estreitamente relacionada, de forma que as pezas vocais podían ser
interpretadas indistintamente por voces ou por instrumentos (de feito,
unha posibilidade é a mera transcrición de música vocal). Por iso, as
obras instrumentais derivadas de modelos vocais son cancións, madrigales
ou motetes adaptados para instrumentos. As principais son as seguintes:

\begin{itemize}
\item
  Canzona: consisten en principio en transcricións do xénero
  \emph{chanson} feitas para instrumentos, polo que teñen carácter
  homofónico e vertical. A canzona comeza sendo unha forma uniseccional
  monotemática que evoluciona cara á multiseccionalidad e o
  politemático, con temas diferentes en cada sección. A evolución
  posterior deste xénero desembocará na \emph{sonata dá chiesa} do
  Barroco.
\item
  Ricercar: composición instrumental desenvolvida sobre o modelo
  proporcionado polo motete. O seu carácter é imitativo e
  contrapuntístico. Ao principio tiña un carácter improvisatorio: os
  distintos temas preséntanse e son tratados en imitación, ao modo de
  cada frase do motete. Progresivamente asume formas máis
  estandarizadas, con repetición de temas segundo un plan establecido e
  mellor articulación do conxunto, de forma que o ricercar monotemático
  despraza aos modelos anteriores.
\end{itemize}

b) Obras de carácter improvisado

As obras incluídas neste apartado presentan unha terminología confusa,
ás veces coincidentes con outras (preludio, fantasía, ricercar...).
Crese que esta música aparece pola necesidade de exercitar os dedos e
probar a afinación do instrumento antes de comezar a interpretación, por
iso eran obras para solista inicialmente. Atopamos dous tipos:

\begin{itemize}
\item
  Variacións sobre temas populares e melodías de danza.
\item
  Pezas libres, como \emph{a toccata} (para instrumentos de tecla). Os
  preludios son pezas que anteceden a unha execución vocal, mentres que
  as fantasías son semellantes aos \emph{ricercari} pero de tratamento
  máis libre. En España empregábase o tento, composición para órgano a
  catro voces. A fantasía é de tipo libre, con numerosos recursos
  polifónicos e contrapuntísticos
\end{itemize}

c) Música para danza

Xeralmente escribíase en tabulaturas ou libros (non se improvisaba); a
instrumentación aínda non se especificaba. Loxicamente tiña un ritmo moi
marcado e agrupábanse en dúos ou tríos (que máis adiante conformarán
suites de danzas), contrastantes entre si. As principais danzas
renacentistas son a
\href{http://es.wikipedia.org/wiki/Baja_danza}{\emph{bassa danza},}
\href{http://es.wikipedia.org/wiki/Pavana}{pavana,} pasamezzo, gallarda,
saltarello, gavota, \href{http://es.wikipedia.org/wiki/Branle}{branle,}
etc. Varios destes bailes están descritos no
\href{http://memory.loc.gov/cgi-bin/ampage?collId=musdi\&fileName=219//musdi219.db\&recNum=3\&itemLink=r?ammem/musdibib:@field\%28NUMBER+@od1\%28musdi+219\%29\%29\&linkText=0}{\emph{Orchesography}}
de \href{http://es.wikipedia.org/wiki/Thoinot_Arbeau}{Arbeau.}

Como exemplo de pavana,
\href{http://open.spotify.com/track/6lvQfnbXx6sBCYDl92L3wc}{\emph{Belle
qui tiens ma vie}:}

\begin{figure}
\centering
\includegraphics{/home/robertopradomartinez/Documentos/GitHub/Historia-I/Referentes-TEMARIO/media/image4.jpeg}
\caption{}
\end{figure}

Como exemplo de \emph{brandle} veremos o Rondel VIN, de
\href{http://es.wikipedia.org/wiki/Tielman_Susato}{Tielman Susato:}

\begin{figure}
\centering
\includegraphics{/home/robertopradomartinez/Documentos/GitHub/Historia-I/Referentes-TEMARIO/media/image5.jpeg}
\caption{}
\end{figure}

d) Música baseada na variación

As variacións son pezas baseadas nunha melodía ou nun baixo. Utilízanse
no Renacemento non só como forma fixa, senón tamén como forma
improvisada; de aí a súa importante presenza en edicións de toda Europa.
Hai dous tipos de variacións (\emph{diferenzas}):

\begin{itemize}
\item
  As compostas sobre material preexistente ao que aplican diferentes
  procedementos (ornamentación, cambios de ritmo ou metro, diminución,
  etc.).
\item
  As que se engaden a ese material dado. A técnica máis habitual é a de
  engadido a un baixo dado, que funciona como unha especie de C.F. No
  caso de ritmos de danza que actúan como tales teriamos (por exemplo) a
  \href{http://es.wikipedia.org/wiki/Fol\%C3\%ADa}{folía:}

  \begin{itemize}
  \item
    O pasamezzo antigo:
  \item
    O pasamezzo moderno:
  \item
    A romanesca:
  \end{itemize}
\end{itemize}

\begin{figure}
\centering
\includegraphics{/home/robertopradomartinez/Documentos/GitHub/Historia-I/Referentes-TEMARIO/media/image6.jpeg}
\caption{}
\end{figure}

\begin{figure}
\centering
\includegraphics{/home/robertopradomartinez/Documentos/GitHub/Historia-I/Referentes-TEMARIO/media/image9.jpeg}
\caption{}
\end{figure}

Obra fundamental neste aspecto é o
\href{http://es.wikipedia.org/wiki/Tratado_de_glosas}{Tratado de glosas}
para violón escrito por Diego
\href{http://es.wikipedia.org/wiki/Diego_Ortiz}{Ortiz} en 1553.

\hypertarget{52-muxfasica-para-teclado}{%
\paragraph{\texorpdfstring{\textbf{5.2 Música para
teclado}}{5.2 Música para teclado}}\label{52-muxfasica-para-teclado}}

Ademais do exposto con anterioridade, interesa comentar algúns aspectos
referidos a instrumentos concretos. No caso do órgano, instrumento
litúrxico por excelencia, o seu repertorio está composto por obras xa
tratadas (canzonas, ricercares, fantasías, tocatas):

\begin{itemize}
\item
  Transcricións de pezas vocais.
\item
  Obras sobre C.F., sobre todo as que parten dunha coral ao que
  ornamentan.
\item
  Preludios, de forma libre sobre pasaxes escalísticos e pedais.
\item
  Improvisacións sobre aires de danza.
\end{itemize}

No que se refire ao instrumento de corda pinzada con teclado (espineta,
virginal, clave), a súa música experimenta un importante auxe a finais
do XVI, estacando as variacións.

\hypertarget{53-o-consort}{%
\paragraph{\texorpdfstring{\textbf{5.3 O
\emph{consort}}}{5.3 O consort}}\label{53-o-consort}}

Neste apartado entraría a música para:

\begin{itemize}
\item
  \emph{Whole consort}, agrupación integrada por instrumentos da mesma
  familia.
\item
  \emph{Broken consort}, instrumentos de diferente familia.
\end{itemize}

\end{document}
