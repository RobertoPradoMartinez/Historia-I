% COMENTARIOS DE TEXTOS: ARISTOXENO DE TARENTO
% Para incluír en exercicios de clase
% Desenvolve o tema das proporcións matemáticas

% -------------------------------
% Texto de Aristoxeno de Tarento:
% -------------------------------
\paragraph{\texorpdfstring{Aristoxeno de Tarento, \emph{Elementos de
harmonía} (s. IV
aC)}{Aristoxeno de Tarento, Elementos de harmonía (s. IV aC)}}\label{aristoxeno-de-tarento--elementos-de-harmonuxeda--s.-iv-ac}

\begin{quote}
Aristoxeno é o máis orixinal dos teóricos musicais gregos. Foi discípulo de Aristóteles, e nos seus escritos sobre música rexeita as teorías pitagóricas, baseadas no número, para propor un estudo dos sons musicais baseado na percepción. Aínda que a liña maioritaria do pensamento musical grego é a pitagórica, Aristoxeno tivo tamén numerosos seguidores, chegando a súa influencia ata finais da época helenística, no tratado \emph{Sobre a música} de Arístides Quintiliano (s. II DC).
\end{quote}


\begin{multicols}{3}
\setlength{\columnseprule}{1pt}
{\small
\noindent
Falemos agora da harmonía e das súas partes.

\noindent
Hai que sinalar, de maneira xeral, que toda teoría que se refira a un canto calquera debe explicar como a voz, pola tensión e a distensión, forma naturalmente os intervalos, pois pretendemos que a voz se mova cun movemento natural e non forme un intervalo por azar. Para isto tentaremos basear as nosas demostracións na experiencia; non faremos nisto como os nosos predecesores.

\noindent
Uns razoan de maneira moi estraña: rexeitan o xuízo do oído, cuxa exactitude non admiten; buscan razóns puramente abstractas. Ao seu entender, hai certas proporcións numéricas, certas leis de velocidades relativas de vibración das que dependen o agudo e o grave; e, partindo de aí, fan os razoamentos máis extraordinarios e máis afastados dos datos da experiencia.

\noindent
Outros dan as súas opinións como oráculos, sen razoamento nin demostración: nin sequera saben enunciar convenientemente os propios feitos naturais.

\noindent
Pola nosa banda, trataremos de recoller todos os feitos que son evidentes para quen coñecen a música, para despois demostrar as consecuencias que resultan destes feitos fundamentais.
}
\end{multicols}


% Exercicio sobre o texto:
% ------------------------
\begin{ejercicio}[]
Que terorías rexeita?\dotfill \\
Con que concepción ou teoría das vistas no punto \ref{o-pensamento-musical--na-antiguxfcidade-cluxe1sica} da páxina \pageref{o-pensamento-musical--na-antiguxfcidade-cluxe1sica} relacionas este relato? \ldots
%\par
 \vspace*{0.50cm} % espazo vertical
\end{ejercicio}
%

