% COMENTARIOS DE TEXTOS: CENSORINO
% Para incluír en exercicios de clase
% Desenvolve o tema do Ethos

% ------------------------------------------
% Texto Censorino - Sobre o día do nacemento
% ------------------------------------------

\paragraph{\texorpdfstring{Censorino, \emph{Sobre o día do nacemento}
(s. III
dC)}{Censorino, Sobre o día do nacemento (s. III dC)}}\label{censorino--sobre-o-duxeda-do-nacemento--s.-iii-dc}

\begin{quote}
Censorino foi un \emph{grammaticus} (mestre) romano. A súa obra \emph{De die natale} foi escrita como agasallo de aniversarios para o seu patrón. O interese deste texto, pertencente a esta obra, é que nos presenta o punto de vista do non especialista: o autor non é músico, nin matemático, nin filósofo; representa así o pensamento xeral da poboación de cultura media de finais da época imperial romana.
\end{quote}


\begin{multicols}{3}
\setlength{\columnseprule}{1pt}
{\small

\noindent
En apoio disto temos a afirmación de Pitágoras de que o mundo enteiro está feito segundo un plan musical e que o sete astros errantes entre o ceo e a Terra, que afectan o nacemento dos mortais, móvense rítmicamente e en posicións que corresponden a intervalos musicais, e emiten varios sons consoantes coa súa altitude que dan lugar conxuntamente á melodía
máis exquisita. Pero esta é inaudible para nós debido á grandiosidade do son, que nosos limitados oídos son incapaces de aprehender.

\noindent
{[}\ldots{}{]} Pitágoras cría que a distancia da Terra á Lúa era duns 126.000 estadios, e que isto era o intervalo dun ton. Entón, desde a Lúa ao planeta Mercurio {[}\ldots{}{]} hai a metade desa distancia, ou un semitono. De Mercurio a {[}\ldots{}{]} Venus hai aproximadamente o mesmo, é dicir, outro semitono; por tanto, o Sol está ao triplo de
distancia, un total de ton e medio. A estrela do Sol está así a unha distancia de tres tons e medio da Terra, formando unha quinta, e a dous tons e medio da Luna, formando unha cuarta.

\noindent
Desde o Sol ao planeta Marte {[}\ldots{}{]} o intervalo é o mesmo que da Terra á Luna, é dicir, un ton; de Marte ao planeta Júpiter {[}\ldots{}{]} hai a metade diso, un semitono. De Júpiter ao planeta Saturno {[}\ldots{}{]} a distancia é outro semitono e de alí ao ceo máis alto onde están os signos do Zodíaco, de nuevo un semitono. Así pois, do ceo máis alto ao Sol o intervalo é dunha cuarta (dous tons e medio), e do punto máis alto da Terra ao mesmo ceo é de seis tons, que forman a consonancia dunha oitava. {[}\ldots{}{]} Todo este universo é unha harmonía. Esta é a razón de que Dorilao escribise que o mundo é o instrumento de Deus.
}
\end{multicols}


% Exercicio sobre o texto:
% ------------------------
\begin{ejercicio}[]
Que teorías ou teorías segundo o punto \ref{o-pensamento-musical--na-antiguxfcidade-cluxe1sica} da páxina \pageref{o-pensamento-musical--na-antiguxfcidade-cluxe1sica}  identificas neste fragmento de texto?
%\par
 \vspace*{0.50cm} % espazo vertical
\end{ejercicio}
%

