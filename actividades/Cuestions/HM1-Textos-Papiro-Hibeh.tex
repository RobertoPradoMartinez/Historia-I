% COMENTARIOS DE TEXTOS: PAPIRO DE HIBEH
% Para incluír en exercicios de clase
% Desenvolve o tema do Ethos

% ----------------------
% Texto Papiro de Hibeh:
% ----------------------
\paragraph{\texorpdfstring{\emph{Papiro de Hibeh} (s. IV
aC)}{Papiro de Hibeh (s. IV aC)}}\label{papiro-de-hibeh--s.-iv-ac}

\begin{quote}
Trátase dun papiro anónimo atopado na localidade de Hibeh (Exipto) datado no século {\small IV} a.C (época clásica), contemporáneo posiblemente de Platón. Aparentemente é un fragmento dun discurso dun músico ante un auditorio de músicos, en que se expón unha dura crítica aos teóricos musicais con especial referencia á teoría do \emph{ethos}.
\end{quote}


\begin{multicols}{3}
\setlength{\columnseprule}{1pt}
{\small
\noindent
Marabilleime a miúdo, oh cidadáns!, de que vós non vos deades conta do modo equivocado en que algúns consideran as artes que vós mesmos practicades. Estes, definíndose estudiosos das \emph{harmoniai}, examinan e xulgan os cantos, confrontando uns con outros, e algúns sen razón critícanos; outros, tamén sen razón, encómianos. Estes din que non os deben xulgar pola súa habilidade no tocar ou no cantar ---para as execucións musicais eles admiten renderse ante os outros e reivindican, en cambio, como a súa exclusiva ocupación a especulación teórica sobre a música---, mentres que, segundo parece, afánanse moito nestas actividades nas cales se din inferiores aos outros e toman a treo aquelas nas que pretenden ser particularmente versados.\\
Din que algunhas melodías fan aos oíntes temperantes; outras os volven xuiciosos; outras, xustos, valerosos ou viles, e non saben que nin o \emph{genos}\footnote{A palabra deriva d \emph{gene} (pl. \emph{genera}) dependentes de tres tetracordos básicos: diatónico, cromático e enarmónico (a pesar da coincidencia de nomes non hai relación cos conceptos
actuais). Probablemente, o máis antiguo fora o diatónico, sendo os outros dous engadidos por influencia asiática.} cromático podería facer tornar viles nin o enarmónico valerosos a quen os empregan na súa música. Quen non sabe que os etolios, os dólopes e os que se reúnen (na Anfizionia pilaico-délfica) xunto ás Termópilas e que nos seus cantos usan o \emph{genos} diatónico son máis valerosos que os actores tráxicos habituados a cantar no \emph{genos} enarmónico? Non é certo que o \emph{genos} cromático envileza e que o enarmónico infunda valor. A súa é de xeito sinxelo impudicia; dedican gran parte do seu tempo á música, pero tocan peor que os citaristas, cantan peor que os cantantes, expresan os seus xuízos moito peor que calquera xuíz e, en suma, todo o que fan fano moito peor que os outros, tamén no tocante á denominada ciencia harmónica, á cal, con todo, afirman dedicar toda a súa atención: cando ouven a música non logran dicir unha soa palabra, déixanse embargar pola emoción e marcan co pé o ritmo segundo os sons do instrumento que acompañan co canto. [\ldots]
}
\end{multicols}


% Exercicio sobre o texto:
% ------------------------
\begin{ejercicio}[]
Identifica que teoría ou teorías rexeita o texto anterior.\\
Xustifica a túa resposta \ldots
%\par
 \vspace*{2.0cm} % espazo vertical
\end{ejercicio}
%

