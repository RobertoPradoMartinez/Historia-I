% COMENTARIOS DE TEXTOS: PAPIRO DE HIBEH
% Para incluír en exercicios de clase
% Desenvolve o tema do Ethos

% --------------------------
% Texto Platón - A república
% --------------------------

\paragraph{\texorpdfstring{Platón, \emph{A república} (ss. V-IV
aC)}{Platón, A república (ss. V-IV aC)}}\label{platuxf3n--a-repuxfablica--ss.-v-iv-ac}

\begin{quote}
Platón é un dos principais pensadores da antigüidade clásica; o seu pensamento mantívose durante séculos ata o final do helenismo, e de aí pasou ao pensamento cristián; mantense en parte durante a Idade Media e rexorde no Renacemento. Neste texto, Sócrates (que expresa as ideas do autor) conversa co músico Glaucón sobre os diversos estilos musicais da súa época. É unha exposición clara da teoría do \emph{ethos}.
\end{quote}


\begin{multicols}{3}
\setlength{\columnseprule}{1pt}
{\small

\noindent
Entón Glaucón botouse a rir e dixo:

\noindent
---Pola miña banda, Sócrates, temo que non vou acharme incluído nese mundo de que falas; pois polo momento non estou en condicións de conxecturar que é o que imos dicir, aínda que o sospeito.

\noindent
---De todos os xeitos ---contestei---, supoño que isto primeiro si estarás en condicións de afirmalo: que a melodía se compón de tres elementos, que son letra, harmonía e ritmo.

\noindent
---Si ---dixo---. Iso polo menos seino.

\noindent
---Agora ben, teño entendido que as palabras da letra en nada difiren das non acompañadas con música canto á necesidade de que unhas e outras se ateñan á mesma maneira e normas establecidas hai pouco.

\noindent
---É verdade ---dixo.

\noindent
---Polo que toca á harmonía e ritmo, han de acomodarse á letra.

\noindent
---Como non?

\noindent
---Agora ben, dixemos que nas nosas palabras non necesitabamos para nada de trenos e queixumes.

\noindent
---Non, efectivamente.

\noindent
---Cales son, pois, as harmonías lastimeras? Dimas ti, que es músico.

\noindent
---Lídaa mixta ---enumerou---, lídaa tensa e outras semellantes.

\noindent
---Teremos, por tanto, que suprimilas, non? ---dixen---. Porque non son
aptas nin aínda para mulleres de mediana condición, canto menos para
homes.

\noindent
---Exacto.

\noindent
---Tampouco hai nada menos apropiado para os gardiáns que a embriaguez,
molicie e preguiza.

\noindent
---Como vai habelo?

\noindent
---Pois ben, cales das harmonías son peiraos e convivales?

\noindent
---Hai variedades da jonia e lida ---dixo--- que adoitan ser cualificadas de laxas.

\noindent
---E serviríasche algunha vez destas harmonías, querido, ante un público de guerreiros?

\noindent
---De ningún xeito ---negou---. Pero paréceme que omites a doria e a frigia.
\noindent
---É que eu non entendo de harmonías ---dixen---; mais permite aquela
que sexa capaz de imitar debidamente a voz e acentos dun heroe que, en
acción de guerra ou outra esforzada empresa, sofre un revés ou unha
ferida ou a morte ou outro infortunio semellante e, con todo, aínda en
tales circunstancias deféndese firme e valientemente contra a súa mala
fortuna. E outra que imite a alguén que, nunha acción pacífica e non
forzada, senón espontánea, tenta convencer a outro de algo ou lle
suplica, con preces se é un deus ou con advertencias ou amoestacións se
se trata dun home; ou ao contrario, que atende aos rogos, leccións ou
reconvenciones doutro e, logrando, como consecuencia diso, o que
apetecía, non se envanece, así a todo, observa en todo momento sensatez
e moderación e móstrase satisfeito coa súa sorte. Estas dúas harmonías,
violenta e pacífica, que mellor poden imitar as voces de xentes
desdichadas ou felices, prudentes ou valerosas, son as que debes deixar.
\noindent
---Pois ben ---dixo---, as harmonías que desexas conservar non son
outras que as que eu citaba agora mesmo.
\noindent
---Entón ---seguín---, a execución das nosas melodías e cantos non
precisará de moitas cordas nin do panarmónico.
\noindent
---Non creo ---dixo.
\noindent
---Non teremos, pois, que manter construtores de triángulos, péctides e
demais instrumentos policordes e poliarmónicos.
\noindent
---Parece que non.
\noindent
---E que? Admitirás na cidade aos flauteros e flautistas? Non é a frauta
o instrumento que máis sones distintos ofrece, ata o punto de que os
mesmos instrumentos panarmónicos son imitación súa?
\noindent
---En efecto, o é ---dixo.
\noindent
---Non quédanche, pois ---dixen---, máis que a lira e cítara como
instrumentos útiles na cidade; no campo, os pastores poden empregar unha
especie de frauta de pan.
\noindent
}
\end{multicols}


% Exercicio sobre o texto:
% ------------------------
\begin{ejercicio}[]
Que terorías rexeita?\dotfill \\
Xustifica a túa resposta \ldots
%\par
 \vspace*{2.0cm} % espazo vertical
\end{ejercicio}
%

