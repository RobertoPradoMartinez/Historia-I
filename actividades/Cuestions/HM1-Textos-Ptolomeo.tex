% COMENTARIOS DE TEXTOS: CLAUDIO PTOLOMEO
% Para incluír en exercicios de clase
% Desenvolve o tema das proporcións matemáticas

% -------------------------
% Texto de Claudio Ptolomeo
% -------------------------

\paragraph{\texorpdfstring{Claudio Ptolomeo, \emph{Harmónicos} (s. II
dC)}{Claudio Ptolomeo, Harmónicos (s. II dC)}}\label{claudio-ptolomeo--harmuxf3nicos--s.-ii-dc}

\begin{quote}
Aínda que é máis coñecido como astrónomo, Ptolomeo foi tamén matemático, xeógrafo e filósofo. En \emph{Harmónicos} resume as teorías musicais de toda a antigüidade. Neste texto reúnense as dúas liñas fundamentais do pensamento musical grego: a pitagórica --baseada na especulación matemática-- e a aristoxénica, baseada na percepción.
\end{quote}

\begin{multicols}{3}
\setlength{\columnseprule}{1pt}
{\small
\noindent
Creo que demostrei suficientemente que os intervalos harmónicos ata \emph{o emmeleis} están definidos intrinsecamente por certas proporcións fundamentais, e respondín tamén á pregunta de que proporción corresponde a cada un deles. Quen se interesou profundamente pola causa perceptiva dos nosos cálculos, así como pola súa investigación práctica ---é dicir,
polos métodos que examinei para usar o monocordio--- non pode dubidar xa de que en todas as afinacións a corroboración do oído é boa. A consecuencia natural é que calquera que practique estes cálculos, se conserva algunha sensibilidade para a beleza, debe asombrarse ante o poder e a beleza que habita nas harmonías; con todo, isto coincide tamén completamente cos cálculos do intelecto, e coa maior precisión descobre e produce as afinacións no uso práctico. Tamén será presa, por dicilo así, dun sacro anhelo de comprender e entender as verdadeiras relacións desta facultade con outros fenómenos do noso mundo. Por conseguinte, tentaremos tratar esta última parte da nosa tarefa científica da maneira
máis ampla posible, para dar expresión ao carácter sublime desta marabillosa facultade.
}
\end{multicols}


% Claudio Ptolomeo: Teoría do Ethos
% ---------------------------------

\paragraph{\texorpdfstring{Claudio Ptolomeo, \emph{Harmónicos} (s. II
dC)}{Claudio Ptolomeo, Harmónicos (s. II dC)}}

\begin{quote}
Neste segundo texto, Ptolomeo presenta unha visión singular da música e dos intervalos musicais.
\end{quote}

% Neste segundo texto, Ptolomeo presenta unha visión singular da teoría do ethos: os inter- valos musicais corresponden ás facultades da alma.

\begin{multicols}{3}
\setlength{\columnseprule}{1pt}
{\small
\noindent
As facultades orixinais da alma son tres: a facultade do pensamento, a facultade do sentimento e a facultade da vida. Os intervalos orixinais idénticos e consoantes son tamén tres: a identidade da oitava e as consonancias da quinta e a cuarta. Podemos, por tanto, comparar a oitava coa facultade do pensamento ---pois en ambos prevalece a simplicidade, a
igualdade e a equivalencia---, a quinta coa facultade do sentimento, e a cuarta coa facultade da vida. A quinta está máis preto da oitava que a cuarta, e soa mellor porque o seu excedente está máis preto da unidade.\\
Analogamente, a facultade do sentimento está máis preto do pensamento que a facultade da vida, pois participa nunha certa medida da conciencia. Algunhas cousas teñen ser pero non sentimento; outras teñen sentimento pero non pensamento. Por outra banda, todas as cousas que senten teñen tamén ser, e todas as que teñen pensamento posúen tamén sentimento e ser. Así, na harmonía, onde está presente a cuarta non hai necesariamente unha quinta, nin onde está a quinta, unha oitava; pero unha quinta contén sempre unha cuarta, e unha oitava unha quinta e unha cuarta. A cuestión é que os poderes da vida e o sentimento correspóndense cos intervalos incompletos \emph{emmeleis} e a súa combinación, e o poder do pensamento co completo.
}
\end{multicols}

% Exercicio sobre o texto:
% ------------------------
\begin{ejercicio}[]
Con que concepción ou teoría das vistas no punto \ref{o-pensamento-musical--na-antiguxfcidade-cluxe1sica} da páxina \pageref{o-pensamento-musical--na-antiguxfcidade-cluxe1sica} relacionas último relato de Claudio Ptolomeo? \ldots
%\par
 \vspace*{0.50cm} % espazo vertical
\end{ejercicio}
%

