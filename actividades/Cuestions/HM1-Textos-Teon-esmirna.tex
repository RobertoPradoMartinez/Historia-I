% COMENTARIOS DE TEXTOS: TEÓN DE ESMIRNA
% Para incluír en exercicios de clase
% Desenvolve o tema do Ethos

% --------------------------------------------
% Texto Teón - Expositio rerum mathematicarum
% --------------------------------------------

\paragraph{\texorpdfstring{Teón de Esmirna, \emph{Expositio rerum
mathematicarum} (s. II
dC)}{Teón de Esmirna, Expositio rerum mathematicarum (s. II dC)}}\label{teuxf3n-de-esmirna--expositio-rerum-mathematicarum--s.-ii-dc}

\begin{quote}
Teón de Esmirna é tamén filósofo neopitagórico e neoplatónico. Neste
fragmento expón con claridade a teoría da \emph{harmonía das esferas}.
\end{quote}


\begin{multicols}{3}
\setlength{\columnseprule}{1pt}
{\small

\noindent
Velaquí a opinión dalgúns pitagóricos relativa á posición e a orde das
esferas ou círculos en que se moven os planetas. O círculo da Luna está
máis próximo á Terra, o de Hermes é o segundo por encima, logo vén o de
Venus, o do Sol é o cuarto, veñen a continuación os de Marte e Júpiter,
e o de Saturno é o último e o máis próximo ao das estrelas afastadas.
Eles afirman, en efecto, que a órbita do Sol ocupa o lugar intermedio
entre os planetas por tratarse do corazón do universo e o máis apto para
dirixir. {[}\ldots{}{]}

Segundo a doutrina de Pitágoras, ao estar, en efecto, o mundo
armónicamente ordenado, os corpos celestes, que están distantes uns
doutros segundo as proporcións dos sons consoantes, producen, polo
movemento e a velocidade das súas revolucións, os sons harmónicos
correspondentes.
}
\end{multicols}


% Exercicio sobre o texto:
% ------------------------
\begin{ejercicio}[]
Que teorías ou teorías segundo o punto \ref{o-pensamento-musical--na-antiguxfcidade-cluxe1sica} da páxina \pageref{o-pensamento-musical--na-antiguxfcidade-cluxe1sica}  identificas neste fragmento de texto? \\
Xustifica a túa resposta \ldots
%\par
 \vspace*{1.50cm} % espazo vertical
\end{ejercicio}
%
