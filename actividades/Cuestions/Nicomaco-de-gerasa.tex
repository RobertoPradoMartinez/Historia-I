% Textos da Antigüidade clásica
% -----------------------------
%
\colorbox{gray}{GeoGebra
Nicómaco de Gerasa, Enchiridion Harmonices (ss. I-II dC)
Nicómaco foi un filósofo da escola neopitagórica de fins do século I e comezos do II DC. Neste texto relátanos unha das tradicións que se conservaron sobre Pitágoras ao longo dos séculos, e que explica o descubrimento da base matemática das consonancias musicais.
}
Un día [Pitágoras] saíu a pasear, perdido nas súas reflexións e nos pensamentos que os seus esquemas lle suxeriron, preguntándose se podería inventar unha axuda para o oído, segura e libre de erro, como a que posúen os sentidos da vista e o tacto, un no compás, a regra, ou mesmo, podemos dicir, a dioptra; o outro nas escalas ou a invención das medidas. Sucedeu que por unha coincidencia providencial pasou xunto ao taller dun ferreiro, e ouviu alí con bastante claridade como os martelos de ferro golpeaban o yunque e emitían confusamente intervalos que, coa excepción dun, eran consonancias perfectas. Recoñeceu entre aqueles sons as consonancias de diapason (oitava), diapente (quinta) e diatessaron (cuarta). En canto ao intervalo entre a cuarta e a quinta, observou que era en si mesmo disonante, pero polo demais complementario da maior destas dúas consonancias. Entusiasmado, entrou no taller coma se un deus estivese a axudalo nos seus plans, e despois de varios experimentos descubriu que era a diferenza de pesos a que provocaba as diferenzas de altura, e non o esforzo dos ferreiros, nin a forma dos martelos, nin o movemento do ferro traballado. Co maior coidado, determinou os pesos dos martelos e a súa forza impulsora, que atopou perfectamente idéntica; logo volveu á súa casa.

Fixou un só cravo no ángulo formado por dúas paredes, para evitar incluso aquí a máis lixeira diferenza, e por temor a que varios cravos, ao ter cada un a súa propia substancia, puidesen invalidar o experimento. Deste cravo colgou catro cordas idénticas en substancia, número de fíos, espesor e torsión, e suspendeu do extremo máis baixo de cada unha delas un peso. Fixo, ademais, que a lonxitude das cordas fose exactamente a mesma, e logo, pulsándoas xuntas dúas a dúas, escoitou as consonancias arriba mencionadas que variaban con cada par de cordas. A corda estirada polo peso maior, comparada coa que soportaba o máis pequeno, daba lugar ao intervalo dunha oitava. Agora ben, a primeira representaba 12 unidades do peso dado, e a última 6. Demostrou deste xeito que a oitava está nun cociente dobre, como os pesos mesmos fixérono sospeitar. A corda maior, comparada coa máis pequena, que representaba 8 unidades, facía soar a quinta, e probou que estaban nun cociente de sesquitercia, ao ser esa o cociente dos pesos. Logo comparouna coa seguinte, con respecto ao peso que soportaba. A máis grande das outras dúas cordas, de 9 unidades, facía soar a cuarta; así estableceu que estaba na proporción sesquitercia inversa, e que esta mesma corda estaba no cociente de sesquiáltera coa máis pequena, pois 9 a 6 é a mesma cociente, así como a segunda corda máis pequena con 8 unidades está nun cociente de sesquitercia coa de 6 unidades, e nun cociente de sesquiáltera coa de 12 unidades.

Por conseguinte, confirmouse que o intervalo entre a quinta e a cuarta —a cantidade pola que a quinta excede á cuarta— está no cociente de sesquioctava, 9:8. A oitava era o sistema formado pola unión dunha e outra, a saber, a quinta e a cuarta situadas unha á beira doutra. Así, a proporción dobre componse da sesquiáltera e a sesquitercia, 12:8:6; ou, á inversa, pola unión da cuarta e a quinta, de maneira que a oitava está composta da sesquitercia e a sesquiáltera nesta orde, 12:9:6.

Aristoxeno de Tarento, Elementos de harmonía (s. IV aC)
Aristoxeno é o máis orixinal dos teóricos musicais gregos. Foi discípulo de Aristóteles, e nos seus escritos sobre música rexeita as teorías pitagóricas, baseadas no número, para propor un estudo dos sons musicais baseado na percepción. Aínda que a liña maioritaria do pensamento musical grego é a pitagórica, Aristoxeno tivo tamén numerosos seguidores, chegando a súa influencia ata finais da época helenística, no tratado Sobre a música de Arístides Quintiliano (s. II DC).

Falemos agora da harmonía e dos seus partes.

Hai que sinalar, de maneira xeral, que toda teoría que se refira a un canto calquera debe explicar como a voz, pola tensión e a distensión, forma naturalmente os intervalos, pois pretendemos que a voz se mova cun movemento natural e non forme un intervalo por azar. Para isto tentaremos basear as nosas demostracións na experiencia; non faremos nisto como os nosos predecesores.

Uns razoan de maneira moi estraña: rexeitan o xuízo do oído, cuxa exactitude non admiten; buscan razóns puramente abstractas. Ao seu entender, hai certas proporcións numéricas, certas leis de velocidades relativas de vibración das que dependen o agudo e o grave; e, partindo de aí, fan os razoamentos máis extraordinarios e máis afastados dos datos da experiencia.

Outros dan as súas opinións como oráculos, sen razoamento nin demostración: nin sequera saben enunciar convenientemente os propios feitos naturais.

Pola nosa banda, trataremos de recoller todos os feitos que son evidentes para quen coñecen a música, para despois demostrar as consecuencias que resultan destes feitos fundamentais.

Claudio Ptolomeo, Harmónicos (s. II dC)
Aínda que é máis coñecido como astrónomo, Ptolomeo foi tamén matemático, geógrafo e filósofo. En Harmónicos resume as teorías musicais de toda a antigüidade.

Neste texto reúnense as dúas liñas fundamentais do pensamento musical grego: a pitagórica, baseada na especulación matemática, e a aristoxénica, baseada na percepción.

Creo que demostrei suficientemente que os intervalos harmónicos ata o emmeleis están definidos intrinsecamente por certas proporcións fundamentais, e respondín tamén á pregunta de que proporción corresponde a cada un deles. Quen se interesou profundamente pola causa perceptiva dos nosos cálculos, así como pola súa investigación práctica —é dicir, polos métodos que examinei para usar o monocordio— non pode dubidar xa de que en todas as afinacións a corroboración do oído é boa. A consecuencia natural é que calquera que practique estes cálculos, se conserva algunha sensibilidade para a beleza, debe asombrarse ante o poder e a beleza que habita nas harmonías; con todo, isto coincide tamén completamente cos cálculos do intelecto, e coa maior precisión descobre e produce as afinacións no uso práctico. Tamén será presa, por dicilo así, dun sacro anhelo de comprender e entender as verdadeiras relacións desta facultade con outros fenómenos do noso mundo. Por conseguinte, tentaremos tratar esta última parte da nosa tarefa científica da maneira máis ampla posible, para dar expresión ao carácter sublime desta marabillosa facultade.

Teón de Esmirna, Expositio rerum mathematicarum (s. II dC)
Teón de Esmirna é tamén filósofo neopitagórico e neoplatónico. Neste fragmento expón con claridade a teoría da harmonía das esferas.

Velaquí a opinión dalgúns pitagóricos relativa á posición e a orde das esferas ou círculos en que se moven os planetas. O círculo da Luna está máis próximo á Terra, o de Hermes é o segundo por encima, logo vén o de Venus, o do Sol é o cuarto, veñen a continuación os de Marte e Júpiter, e o de Saturno é o último e o máis próximo ao das estrelas afastadas. Eles afirman, en efecto, que a órbita do Sol ocupa o lugar intermedio entre os planetas por tratarse do corazón do universo e o máis apto para dirixir. […]

Segundo a doutrina de Pitágoras, ao estar, en efecto, o mundo armónicamente ordenado, os corpos celestes, que están distantes uns doutros segundo as proporcións dos sons consoantes, producen, polo movemento e a velocidade das súas revolucións, os sons harmónicos correspondentes.

Censorino, Sobre o día do nacemento (s. III dC)
Censorino foi un grammaticus (mestre) romano. A súa obra De die natale foi escrita como agasallo de aniversarios para o seu patrón. O interese deste texto, pertencente á esta obra, é que nos presenta o punto de vista do non especialista: o autor non é músico, nin matemático, nin filósofo; representa así o pensamento xeral da poboación de cultura media de finais da época imperial romana.

En apoio disto temos a afirmación de Pitágoras de que o mundo enteiro está feito segundo un plan musical e que o sete astros errantes entre o ceo e a Terra, que afectan o nacemento dos mortais, móvense rítmicamente e en posicións que corresponden a intervalos musicais, e emiten varios sons consoantes coa súa altitude que dan lugar conxuntamente á melodía máis exquisita. Pero esta é inaudible para nós debido á grandiosidade do son, que nosos limitados oídos son incapaces de aprehender.

[…] Pitágoras cría que a distancia da Terra á Lúa era duns 126.000 estadios, e que isto era o intervalo dun ton. Entón, desde a Lúa ao planeta Mercurio […] hai a metade desa distancia, ou un semitono. De Mercurio a […] Venus hai aproximadamente o mesmo, é dicir, outro semitono; por tanto, o Sol está ao triplo de distancia, un total de ton e medio. A estrela do Sol está así a unha distancia de tres tons e medio da Terra, formando unha quinta, e a dous tons e medio da Luna, formando unha cuarta.

Desde o Sol ao planeta Marte […] o intervalo é o mesmo que da Terra á Luna, é dicir, un ton; de Marte ao planeta Júpiter […] hai a metade diso, un semitono. De Júpiter ao planeta Saturno […] a distancia é outro semitono e de alí ao ceo máis alto onde están os signos do Zodíaco, de nuevo un semitono. Así pois, do ceo máis alto ao Sol o intervalo é dunha cuarta (dous tons e medio), e do punto máis alto da Terra ao mesmo ceo é de seis tons, que forman a consonancia dunha oitava. […] Todo este universo é unha harmonía. Esta é a razón de que Dorilao escribise que o mundo é o instrumento de Deus.

Platón, A república (ss. V-IV aC)
Platón é un dos principais pensadores da antigüidade clásica; o seu pensamento mantívose durante séculos ata o final do helenismo, e de aí pasou ao pensamento cristián; mantense en parte durante a Idade Media e rexorde no Renacemento. Neste texto, Sócrates (que expresa as ideas do autor) conversa co músico Glaucón sobre os diversos estilos musicais da súa época. É unha exposición clara da teoría do ethos.

Entón Glaucón botouse a rir e dixo:

—Pola miña banda, Sócrates, temo que non vou acharme incluído nese mundo de que falas; pois polo momento non estou en condicións de conxecturar que é o que imos dicir, aínda que o sospeito.

—De todos os xeitos —contestei—, supoño que isto primeiro si estarás en condicións de afirmalo: que a melodía se compón de tres elementos, que son letra, harmonía e ritmo.

—Si —dixo—. Iso polo menos seino.

—Agora ben, teño entendido que as palabras da letra en nada difiren das non acompañadas con música canto á necesidade de que unhas e outras se ateñan á mesma maneira e normas establecidas hai pouco.

—É verdade —dixo.

—Polo que toca á harmonía e ritmo, han de acomodarse á letra.

—Como non?

—Agora ben, dixemos que nas nosas palabras non necesitabamos para nada de trenos e queixumes.

—Non, efectivamente.

—Cales son, pois, as harmonías lastimeras? Dimas ti, que es músico.

—Lídaa mixta —enumerou—, lídaa tensa e outras semellantes.

—Teremos, por tanto, que suprimilas, non? —dixen—. Porque non son aptas nin aínda para mulleres de mediana condición, canto menos para homes.

—Exacto.

—Tampouco hai nada menos apropiado para os gardiáns que a embriaguez, molicie e preguiza.

—Como vai habelo?

—Pois ben, cales das harmonías son peiraos e convivales?

—Hai variedades da jonia e lida —dixo— que adoitan ser cualificadas de laxas.

—E serviríasche algunha vez destas harmonías, querido, ante un público de guerreiros?

—De ningún xeito —negou—. Pero paréceme que omites a doria e a frigia.

—É que eu non entendo de harmonías —dixen—; mais permite aquela que sexa capaz de imitar debidamente a voz e acentos dun heroe que, en acción de guerra ou outra esforzada empresa, sofre un revés ou unha ferida ou a morte ou outro infortunio semellante e, con todo, aínda en tales circunstancias deféndese firme e valientemente contra a súa mala fortuna. E outra que imite a alguén que, nunha acción pacífica e non forzada, senón espontánea, tenta convencer a outro de algo ou lle suplica, con preces se é un deus ou con advertencias ou amoestacións se se trata dun home; ou ao contrario, que atende aos rogos, leccións ou reconvenciones doutro e, logrando, como consecuencia diso, o que apetecía, non se envanece, así a todo, observa en todo momento sensatez e moderación e móstrase satisfeito coa súa sorte. Estas dúas harmonías, violenta e pacífica, que mellor poden imitar as voces de xentes desdichadas ou felices, prudentes ou valerosas, son as que debes deixar.

—Pois ben —dixo—, as harmonías que desexas conservar non son outras que as que eu citaba agora mesmo.

—Entón —seguín—, a execución das nosas melodías e cantos non precisará de moitas cordas nin do panarmónico.

—Non creo —dixo.

—Non teremos, pois, que manter construtores de triángulos, péctides e demais instrumentos policordes e poliarmónicos.

—Parece que non.

—E que? Admitirás na cidade aos flauteros e flautistas? Non é a frauta o instrumento que máis sones distintos ofrece, ata o punto de que os mesmos instrumentos panarmónicos son imitación súa?

—En efecto, o é —dixo.

—Non quédanche, pois —dixen—, máis que a lira e cítara como instrumentos útiles na cidade; no campo, os pastores poden empregar unha especie de frauta de pan.

Papiro de Hibeh (s. IV aC)
Trátase dun papiro anónimo atopado na localidade de Hibeh (O Hiba, en Exipto) e datado no século IV AC, é dicir, en época clásica, contemporáneo posiblemente de Platón. Aparentemente é un fragmento dun discurso dun músico ante un auditorio de músicos, en que se expón unha dura crítica aos teóricos musicais con especial referencia á teoría do ethos.

Marabilleime a miúdo, oh cidadáns!, de que vós non vos deades conta do modo equivocado en que algúns consideran as artes que vós mesmos practicades. Estes, definíndose estudiosos das harmoniai, examinan e xulgan os cantos, confrontando uns con outros, e algúns sen razón critícanos; outros, tamén sen razón, encómianos. Estes din que non llos debe xulgar pola súa habilidade no tocar ou no cantar —para as execucións musicais eles admiten renderse ante os outros e reivindican, en cambio, como a súa exclusiva ocupación a especulación teórica sobre a música—, mentres que, segundo parece, afánanse moito nestas actividades nas cales se din inferiores aos outros e toman a treo aquelas en as que pretenden ser particularmente versados. Din que algunhas melodías fan aos oíntes temperantes; outras os volven juiciosos; outras, xustos, valerosos ou viles, e non saben que nin o genos cromático podería facer tornar viles nin o enarmónico valerosos a quen os empregan na súa música. Quen non sabe que os etolios, os dólopes e os que se reúnen (na Anfizionia pilaico-délfica) xunto ás Termópilas e que nos seus cantos usan o genos diatónico son máis valerosos que os actores tráxicos habituados a cantar no genos enarmónico? Non é certo que o genos cromático envileza e que o enarmónico infunda valor. A súa é de xeito sinxelo impudicia; dedican gran parte do seu tempo á música, pero tocan peor que os citaristas, cantan peor que os cantantes, expresan os seus xuízos moito peor que calquera xuíz e, en suma, todo o que fan fano moito peor que os outros, tamén no tocante á denominada ciencia harmónica, á cal, con todo, afirman dedicar toda a súa atención: cando ouven a música non logran dicir unha soa palabra, déixanse embargar pola emoción e marcan co pé o ritmo segundo os sons do instrumento que acompañan co canto. E non senten ridículos cando din que algunhas melodías teñen algo do loureiro, outras da hiedra…

Claudio Ptolomeo, Harmónicos (s. II dC)
Neste segundo texto, Ptolomeo presenta unha visión singular da teoría do ethos: os intervalos musicais corresponden ás facultades da alma.

As facultades orixinais da alma son tres: a facultade do pensamento, a facultade do sentimento e a facultade da vida. Os intervalos orixinais idénticos e consoantes son tamén tres: a identidade da oitava e as consonancias da quinta e a cuarta. Podemos, por tanto, comparar a oitava coa facultade do pensamento —pois en ambos prevalece a simplicidade, a igualdade e a equivalencia—, a quinta coa facultade do sentimento, e a cuarta coa facultade da vida. A quinta está máis preto da oitava que a cuarta, e soa mellor porque o seu excedente está máis preto da unidade. Analogamente, a facultade do sentimento está máis preto do pensamento que a facultade da vida, pois participa nunha certa medida da conciencia. Algunhas cousas teñen ser pero non sentimento; outras teñen sentimento pero non pensamento. Por outra banda, todas as cousas que senten teñen tamén ser, e todas as que teñen pensamento posúen tamén sentimento e ser. Así, na harmonía, onde está presente a cuarta non hai necesariamente unha quinta, nin onde está a quinta, unha oitava; pero unha quinta contén sempre unha cuarta, e unha oitava unha quinta e unha cuarta. A cuestión é que os poderes da vida e o sentimento correspóndense cos intervalos incompletos emmeleis e a súa combinación, e o poder do pensamento co completo.
