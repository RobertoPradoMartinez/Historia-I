% -------------------------------
% BANCO DE PREGUNTAS DE HISTORIA
% -------------------------------
% Tema 0.- INTRODUCIÓN: COMO ESCOITAR A MÚSICA
%
% --------------------------------
% EXERCICIO 01: Concepto de estilo
% --------------------------------
\newproblem{T1ES-01}{
Cando falamos de «estilo musical», estamos a referirnos a ...
    \begin{enumerate}[a)]
    \item \label{T1ES-01:sol}
    Un conxunto de trazos musicais comúns a un conxunto de obras que definen unha tendencia musical, asociada habitualmente a un período ou xénero. %(x)
    \item
    Un conxunto de trazos musicais que varían segundo o uso que se fai da música, o seu carácter, o contexto onde se orixina, etc. 
    \item
    Un conxunto de trazos musicais que determinan a función que ten a música na sociedade, segundo o contexto no que se orixina.
    \item
    Un conxunto de trazos musicais comúns a un conxunto de obras segundo os medios empregados para interpretar a música. 
    \end{enumerate}
    }
%
% Solución:
% ---------
    {% solución:
    (\ref{T1ES-01:sol}) {\color{orange}{\hrulefill}}
%    comentario:
    \\ \small{% Texto do comentario:
Un estilo é por definición, un conxunto de trazos musicais comúns a  un conxunto de obras e que definen unha tendencia musical. Podemos falar, por exemplo, de un estilo renacentista ou un estilo romántico; ou ben un estilo operístico ou relixioso. O habitual, é unir ambolos dous conceptos: así falaremos, por exemplo, do estilo sinfónico romántico ou do estilo da ópera barroca.
{\color{orange}{\hrulefill}}
    }
    }
% ---
%
% ----------------------------------------
% EXERCICIO 02: Concepto de xénero musical
% ----------------------------------------
\newproblem{T1ES-02}{
Indica cales dos seguintes aspectos son factores característicos que definen un «xénero musical». 
    \begin{enumerate}[a)]
    \item 
    O uso que se fai da música, o contexto, o seu carácter, etc. 
    \item
    A función social da música, etc.
    \item \label{T1ES-02:sol}
    Os indicados nas dúas opcións anteriores. %(x)
    \item
    Os indicados na primeira opción.
    \end{enumerate}
    }
%
% Solución:
% ---------
    {% solución:
    (\ref{T1ES-02:sol}) {\color{orange}{\hrulefill}}
%    comentario:
    \\ \small{% Texto do comentario:
    Un xénero musical, ven determinado entre outros factores polo uso que se fai da música, o contexto, o seu carácter, a función, etc. Debemos lembrar que en certas ocasións, fai referencia a formas musicais que alcanzaron unha gran importancia e desenvolvemento en determinadas épocas, como é o caso da ópera ou a sinfonía. 
    {\color{orange}{\hrulefill}}
    }
    }
% ---
%
% --------------------------------
% EXERCICIO 03: Concepto de timbre
% --------------------------------
\newproblem{T1ES-03}{
Se nunha audición, analizamos os instrumentos que escoitamos, en que aspecto estamos a fixar a nosa atención?
    \begin{enumerate}[a)]
    \item 
    Textura
    \item \label{T1ES-03:sol}
    Timbre %(x)
    \item
    Forma
    \item
    Ritmo
    \end{enumerate}
    }
%
% Solución:
% ---------
    {% solución:
    (\ref{T1ES-03:sol}) {\color{orange}{\hrulefill}}
%    comentario:
    \\ \small{% Texto do comentario:
    Cando analizamos os instrumentos que escoitamos nunha composición estamos prestando atención ao timbre, factor este que nos permite identificar tanto a familia á que pertence como o propio instrumento.
    {\color{orange}{\hrulefill}}
    }
    }
% ---
%
% ---------------------------------
% EXERCICIO 04: Concepto de textura
% ---------------------------------
\newproblem{T1ES-04}{
Cando ao realizar a escoita e comentario dunha audición tomamos nota de como se ensamblan as diferentes voces musicais (monodia, polifonía sinxela con bordón, melodía con acompañamento, contrapunto, etc.) estamos a analizar ...
    \begin{enumerate}[a)]
    \item 
    Timbres
    \item
    Formas
    \item
    Ritmo
    \item \label{T1ES-04:sol}
    Ningunha das anteriores %(x)
    \end{enumerate}
    }
%
% Solución:
% ---------
    {% solución:
    (\ref{T1ES-04:sol}) {\color{orange}{\hrulefill}}
%    comentario:
    \\ \small{% Texto do comentario:
    En este caso, analizamos a textura da obra.
{\color{orange}{\hrulefill}}
    }
    }
% ---
%
% ---------------------------------
% EXERCICIO 05: Concepto de textura
% ---------------------------------
\newproblem{T1ES-05}{
Cando ao realizar a escoita e comentario dunha audición tomamos nota de como se ensamblan as diferentes voces musicais (monodia, polifonía sinxela con bordón, melodía con acompañamento, contrapunto, etc.) estamos a analizar ...
    \begin{enumerate}[a)]
    \item 
    A forma da obra
    \item \label{T1ES-05:sol}
    A textura da obra %(x)
    \item
    Os aspectos rítmicos da obra
    \item
    A melodía da obra
    \end{enumerate}
    }
%
% Solución:
% ---------
    {% solución:
    (\ref{T1ES-05:sol}) {\color{orange}{\hrulefill}}
%    comentario:
    \\ \small{% Texto do comentario:
    Se tomamos nota de como se ensamblan as diferentes voces musicais, diremos que estamos analizando a textura da obra (monodia, polifonía, etc.)
    {\color{orange}{\hrulefill}}
    }
    }
% ---
%
