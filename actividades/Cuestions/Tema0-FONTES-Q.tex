% -------------------------------
% BANCO DE PREGUNTAS DE HISTORIA
% -------------------------------
% Tema 0.- INTRODUCIÓN: FONTES DE INFORMACIÓN
%
% ---------------------------------------------
% EXERCICIO 01: Definición fonte de información
% ---------------------------------------------
\newproblem{T1FON-01}{
    Podemos afirmar que as fontes de información son o conxunto de obxectos, documentos, testemuños, (...) que ofrecen información relevante e significativa sobre feitos ocorridos no pasado, e permiten igualmente coñecer a música das culturas antigas. Cales das seguintes consideras, son fontes de información iconográficas?
    \begin{enumerate}[a)]
    \item 
    Pintura, escultura e restos de intrumentos
    \item \label{T1FON-01:sol}
    Pintura, escultura e outras obras das artes visuais %(x)
    \item
    Pintura, escultura, tratados e métodos
    \item
    Ningunha das anteriores
    \end{enumerate}
    }
%
% Solución:
% ---------
    {% solución:
    (\ref{T1FON-01:sol}) {\color{orange}{\hrulefill}}
%    comentario:
    \\ \small{% Texto do comentario:
    Pintura, escultura e outras obras das artes visuais, proporcionan información sobre instrumentos musicais, contextos e prácticas de interpretación, danzas, etc.
    {\color{orange}{\hrulefill}}
    }
    }
% ---
%
% ---------------------------------------------
% EXERCICIO 02: Tipos de fontes de información
% ---------------------------------------------
\newproblem{T1FON-02}{
    Aínda que non é correcto supoñer que en condicións de vida iguais desenvólvense culturas musicais iguais, ás veces o coñecemento das músicas tradicionais actuais pode proporcionar detalles sobre técnicas de interpretación de instrumentos antigos ou sobre movementos de danza. Cal é a rama da Musicoloxía, que a partir do estudo da música de pobos e tribus non occidentais e de músicas de tradición oral actuais, axuda á comprensión da actividade musical antiga?
    \begin{enumerate}[a)]
    \item 
    Arqueoloxía
    \item \label{T1FON-02:sol}
    Etnomusicoloxía %(x)
    \item
    Iconografía
    \item
    Paleo-organoloxía
    \end{enumerate}
    }
%
% Solución:
% ---------
    {% solución:
    (\ref{T1FON-02:sol}) {\color{orange}{\hrulefill}}
%    comentario:
    \\ \small{% Texto do comentario:
    A Etnomusicoloxía é a rama da Musicoloxía que se ocupa do estudo da música de tradición oral da actualidade así como da música de tribus e pobos non occidentais, que poden achegar coñecemento sobre a música, contextos e detalles sobre técnicas de interpretación de instrumentos antigos.
    {\color{orange}{\hrulefill}}
    }
    }
% ---
%
