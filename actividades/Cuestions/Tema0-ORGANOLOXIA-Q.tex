% -------------------------------
% -------------------------------
% BANCO DE PREGUNTAS DE HISTORIA
% -------------------------------
% Tema 0.- INTRODUCIÓN: ORGANOLOXÍA
%
% ---------------------------------------
% EXERCICIO 01.- Definición de organoloxía
% ---------------------------------------
\newproblem{T0OR-01}{
    A rama da Musicoloxía encargada do estudo, investigación e clasificación dos instrumentos musicais coñécese co nome de
    \begin{enumerate}[a)]
    \item \label{T0OR-01:sol}
    Organoloxía %(x)
    \item
    Arqueoloxía
    \item
    Etnomusicoloxía
    \item
    Iconografía
    \end{enumerate}
    }
%
% Solución:
% ---------
    {% solución:
    (\ref{T0OR-01:sol}) {\color{orange}{\hrulefill}}
%    comentario:
    \\ \small{% Texto do comentario:
    A Organoloxía é a rama da Musicoloxía encargada do estudo, investigación e clasificación dos instrumentos musicais.
    {\color{orange}{\hrulefill}}
    }
    }
% ---
%
% ----------------------------------------
% EXERCICIO 02.- Definición de organoloxía
% ----------------------------------------
\newproblem{T0OR-02}{
    A Organoloxía é  
    \begin{enumerate}[a)]
    \item 
    A rama da Historia da Música que clasifica os instrumentos musicais
    \item
    A rama da Etnomusicoloxía que investiga sobre os intrumentos musicais prehistóricos
    \item
    A rama da Musicoloxía que investiga a iconografía prehistórica
    \item \label{T0OR-02:sol}
    A rama da Musicoloxía, que investiga, estuda e clasifca os instrumentos musicais %(x)
    \end{enumerate}
    }
%
% Solución:
% ---------
    {% solución:
    (\ref{T0OR-02:sol}) {\color{orange}{\hrulefill}}
%    comentario:
    \\ \small{% Texto do comentario:
    A Organoloxía é a rama da Musicoloxía encargada do estudo, investigación e clasificación dos instrumentos musicais.
    {\color{orange}{\hrulefill}}
    }
    }
% ---
%
% ------------------------------------------
% EXERCICIO 03.- organoloxía - clasificación
% ------------------------------------------
\newproblem{T0OR-03}{
    A base para a investigación organolóxica dos instrumentos musicais, parte da clasifcación Hornbostel-Sachs, que se basea na forma en que os instrumentos emiten as vibracións que producen o son. Segundo esto, o sistema de clasificación moderno de instrumentos musicais ten en conta o material co que se constrúen os instrumentos e tamén a forma en que producen o son, e polo tanto podemos clasificar os instrumentos musicais en ...
    \begin{enumerate}[a)]
    \item 
    Corda, vento, percusión e outros
    \item
    Cordófonos, idiófonos e membranófonos
    \item \label{T0OR-03:sol}
    Idiófonos, membranófonos, cordófonos, aerófonos, e electrófonos %(x)
    \item
    Cordófonos, aerófonos, electrófonos e membranófonos
    \end{enumerate}
    }
%
% Solución:
% ---------
    {% solución:
    (\ref{T0OR-03:sol}) {\color{orange}{\hrulefill}}
%    comentario:
    \\ \small{% Texto do comentario:
    O sistema de clasificación moderno de instrumentos musicais, ten en conta o material co que se constrúen os instrumentos e tamén a forma en que producen o son, e polo tanto podemos clasificar os instrumentos musicais en idiófonos, membranófonos, cordófonos, aerófonos, e electrófonos
    {\color{orange}{\hrulefill}}
    }
    }
% ---
%
% ------------------------------------------
% EXERCICIO 04.- Organoloxía - clasificación
% ------------------------------------------
\newproblem{T0OR-04}{
    Cal é o corpo vibrante dos idiófonos?
    \begin{enumerate}[a)]
    \item \label{T0OR-04:sol}
    O propio instrumento %(x)
    \item
    Un parche ou membrana tensada
    \item
    Unha columna de aire
    \item
    A fluctuación da corrente eléctrica
    \end{enumerate}
    }
%
% Solución:
% ---------
    {% solución:
    (\ref{T0OR-04:sol}) {\color{orange}{\hrulefill}}
%    comentario:
    \\ \small{% Texto do comentario:
    O corpo vibrante dos Idiófonos é o propio instrumento.
    {\color{orange}{\hrulefill}}
    }
    }
% ---
%
% ------------------------------------------
% EXERCICIO 05.- Organoloxía - clasificación
% ------------------------------------------
\newproblem{T0OR-05}{
    Cal é o corpo vibrante e como se clasifican segundo se toquen os membranófonos?
    \begin{enumerate}[a)]
    \item 
    O propio instrumento tensado; clasifícanse en directamente percutidos
    \item
    Un parche ou membrana; clasifícanse en indirectamente percutidos
    \item \label{T0OR-05:sol}
    Un parche ou membrana tenso; clasifícanse en directamente ou indirectamente percutidos %(x)
    \item
    O propio instrumento; clasifícanse en directa ou indirectamente percutidos
    \end{enumerate}
    }
%
% Solución:
% ---------
    {% solución:
    (\ref{T0OR-05:sol}) {\color{orange}{\hrulefill}}
%    comentario:
    \\ \small{% Texto do comentario:
    Os membranófonos producen o son pola vibración de un parche ou membrana tensado e a súa clasificación varía segundo se trate de percusión directa ou indirecta.
    {\color{orange}{\hrulefill}}
    }
    }
% ---
%
% ------------------------------------------
% EXERCICIO 06.- Organoloxía - clasificación
% ------------------------------------------
\newproblem{T0OR-06}{
    Cal é o corpo vibrante e como se clasifican segundo se toquen os cordófonos?
    \begin{enumerate}[a)]
    \item
    O corpo vibrante é o propio instrumento, que ao vibrar produce son. Segundo como se toquen, clasifícanse en percutidos, punteados, refregados, (...)
    \item
    O corpo vibrante é o propio instrumento, que ao vibrar produce son. Segundo como se toquen, clasifícanse en directamente percutidos, punteados, refregados, (...)
    \item \label{T0OR-06:sol}
    O corpo vibrante é unha corda tensada, que pode ser de tripa animal, fibras vexetais, fibras sintéticas, filamentos metálicos, etc. Segundo como se toquen, clasifícanse en percutidos, punteados, refregados, (...) %(x)
    \item
    O corpo vibrante é unha corda tensada, que por fluctuacións de corrente ao vibrar produce son. Segundo como se toquen clasifícanse en punteados, refregados, (...)
    \end{enumerate}
    }
%
% Solución:
% ---------
    {% solución:
    (\ref{T0OR-06:sol}) {\color{orange}{\hrulefill}}
%    comentario:
    \\ \small{% Texto do comentario:
    Segundo como se toquen, clasifícanse en cordófonos percutidos, punteados, refregados, (...). O corpo vibrante é unha corda tensada que vibra producindo o un determinado son segundo o material de que esté feita. 
    {\color{orange}{\hrulefill}}
    }
    }
% ---
%
% ------------------------------------------
% EXERCICIO 07.- Organoloxía - clasificación
% ------------------------------------------
\newproblem{T0OR-07}{
Cal é o corpo vibrante, e como se clasifican segundo se produce o son os aerófonos?
    \begin{enumerate}[a)]
    \item
    Unha columna de aire que vibra so por interación humana
    \item
    Unha columna de aire que vibra so por interación mecánica
    \item \label{T0OR-07:sol}
    Unha columna de aire que vibra ben por interación humana ou mecánica %(x)
    \item
    Ningunha é correcta
    \end{enumerate}
    }
%
% Solución:
% ---------
    {% solución:
    (\ref{T0OR-07:sol}) {\color{orange}{\hrulefill}}
%    comentario:
    \\ \small{% Texto do comentario:
    Os aerófonos clasifícanse segundo o son se produza por interación humana ou mecánica; como exemplos claros atopamos o órgano de tubos para o segundo caso e a frauta para o primeiro.
    {\color{orange}{\hrulefill}}
    }
    }
% ---
%
% ---------------------------------------
% EXERCICIO 08.- Organoloxía - definición
% ---------------------------------------
\newproblem{T0OR-08}{
O sistema moderno de clasifcación de instrumentos musicais, ten en conta o material co que se constrúen os instrumentos e tamén a forma en que producen o son. Clasifca os instrumentos en cinco grupos principais divididos á vez en subgrupos. A ciencia encargada do seu estudo, investigación e clasifcación é unha rama da musicoloxía: a etnomusicoloxía.
    \begin{enumerate}[a)]
    \item
    Verdadeiro
    \item \label{T0OR-08:sol}
    Falso %(x)    
    \end{enumerate}
    }
%
% Solución:
% ---------
    {% solución:
    (\ref{T0OR-08:sol}) {\color{orange}{\hrulefill}}
%    comentario:
    \\ \small{% Texto do comentario:
    A Organoloxía é a rama da Musicoloxía encargada do estudo, investigación e clasificación dos instrumentos musicais.
    {\color{orange}{\hrulefill}}
    }
    }
% ---
%
% -----------------------------------------------------
% EXERCICIO 09.- Organoloxía - Sistema de clasificación
% -----------------------------------------------------
\newproblem{T0OR-09}{
    No sistema moderno de clasifcación de instrumentos musicais, o principio de división seguido é primeiramente o modo de producir o son segundo o corpo que causa as vibracións, e en segundo lugar o sistema de execución e construcción.
    \begin{enumerate}[a)]
    \item \label{T0OR-09:sol}
    Verdadeiro %(x)
    \item
    Falso
    \end{enumerate}
    }
%
% Solución:
% ---------
    {% solución:
    (\ref{T0OR-09:sol}) {\color{orange}{\hrulefill}}
%    comentario:
    \\ \small{% Texto do comentario:
    A clasificación moderna ten en conta o corpo vibrante e por conseguinte o modo de producir o son; en segundo lugar considera a forma en que se toquen seguido das propias características constructivas do instrumento.
    {\color{orange}{\hrulefill}}
    }
    }
% ---
%
% ---------------------------------------
% EXERCICIO 10.- Organoloxía - Patrimonio
% ---------------------------------------
\newproblem{T0OR-10}{
    A difcultade de coñecer realmente a organoloxía empregada na prehistoria é debida á utilización de materiais perecedoiros na construcción dos instrumentos musicais, o que causou que se fosen degradando co paso do tempo. Malia todo, grazas aos restos conservados podemos saber que se empregaban instrumentos como as pedras de entrechoque, tambores feitos de troncos de árbores, trompas feitas de cornos de animáis, etc. Todos eles ...
    \begin{enumerate}[a)]
    \item 
    forman parte do patrimonio musical primixenio e constitúen unha fonte de información iconográfica valiosísima
    \item \label{T0OR-10:sol}
    forman parte do patrimonio musical primixenio pois son fontes de información arqueolóxicas %(x)
    \item
    non forman parte do patrimonio musical primixenio pois non se conservan, malia seren fontes de información organolóxicas
    \item
    forman parte do patrimonio musical primixenio malia non seren fontes de información organolóxicas
    \end{enumerate}
    }
%
% Solución:
% ---------
    {% solución:
    (\ref{T0OR-10:sol}) {\color{orange}{\hrulefill}}
%    comentario:
    \\ \small{% Texto do comentario:
    Grazas ás achegas que nos proporcionan os restos de instrumentos conservados, atopados en xacementos arqueolóxicos, podemos saber que instrumentos se empregaban na prehistoria e as características organolóxicas dos mesmos, a pesares da súa degradación co paso do tempo.
    {\color{orange}{\hrulefill}}
    }
    }
% ---
%
%
% --------------------------------
% EXERCICIO 
% --------------------------------
\newproblem{T0OR-0}{

    \begin{enumerate}[a)]
    \item \label{T0OR-0:sol}
    
    \item
    
    \item
    
    \item
    
    \end{enumerate}
    }
%
% Solución:
% ---------
    {% solución:
    (\ref{T0OR-0:sol}) {\color{orange}{\hrulefill}}
%    comentario:
    \\ \small{% Texto do comentario:
    Aquí o texto do comentario ...
    {\color{orange}{\hrulefill}}
    }
    }
% ---
%
