% 
% EXERCICIOS DO TEMA 0.- INTRODUCCIÓN - 
% Conceptos sobre a organoloxía
%
% Repaso de conceptos de organoloxía


%\begin{multicols}{2}

% EXERCICIO: ORGANOLOXÍA - CONCEPTOS
% ----------------------------------
\begin{ejercicio}[Clasificación dos instrumentos musicais]
 \begin{enumerate}[1)]
  \item 
    A rama da Musicoloxía encargada do estudo, investigación e clasificación dos instrumentos musicais coñécese co nome de \dotfill
  \item
    Tomando como base o sistema Hornbostel-Sachs, os instrumentos musicais clasifícanse segundo o corpo vibrante, a forma de producir o son e segundo se toquen, en: (completa)

% ---- Táboa de clasificación organolóxica ----    
    \par
%    \vspace{0.20cm}
    \begin{center}
   \begin{tabular}{llclcc}
   {\scriptsize{\textsf{CLASIFICACIÓN}}} & & {\scriptsize{\textsf{CORPO VIBRANTE}}} & & {\scriptsize{\textsf{FORMA DE PRODUCIR O SON}}} &  {\scriptsize{\textsf{INSTRUMENTO}}} \\
   \hline
   & & & & & \\
   \dotfill & & \small{\texttt{madeira / metal}} & & \small{\texttt{percusión directa}} & \dotfill \\
   & & & & &\\
   \dotfill & & \small{\texttt{columna de aire}} & & \dotfill & \dotfill \\   
   & & & & &\\
   \dotfill & & \small{\texttt{corda tensada}} & & \small{\texttt{percusión indirecta}} & \dotfill \\
   & & & & &\\
   \dotfill & & \small{\texttt{parche / mebrana}} & & \small{\texttt{percusión directa}} & \dotfill \\   
   & & & & &\\
   \hline
    \end{tabular}
    \end{center}
% ---- fin da táboa ----
    \begin{center}
%    \texttt{idiófonos, aerófonos, cordófonos, membranófonos, electrófonos} \par
    {\small\texttt{castañuelas, órgano, piano, pandeiro, órgano eléctrico}}     
    \end{center}    
  \end{enumerate}
\end{ejercicio}



% Repaso de conceptos de xeneros e formas da música
%\begin{multicols}{2}

% EXERCICIO: RAMAS MUSICOLOXÍA
% ----------------------------

\begin{ejercicio}[Patrimonio primixenio da música]
 \begin{enumerate}[1)]
  \item 
  Que é o patrimonio musical primixenio? 
  Indica algún exemplo. \par
  \vspace*{3.50cm}
 
 \item
 Le a seguinte a afirmación con atención.
 \begin{quote}
 \small{
 Do estudo etnolóxico comparado de tribos da actualidade, deducimos que as primeiras manifestacións de música, nas súas orixes, é probable que empregasen algún tipo de polifonía simple con bordón, pero seguramente, empregasen escalas de dous a sete sons, con melodías curtas e sen complexidades; máis ben sinxelas, empregando intervalos básicos de cuartas, quintas e oitavas.
 }
 \end{quote}
Que rama da Musicoloxía, se ocupa de levar a cabo esas deducións?
\begin{enumerate}[a)]
 \item 
 Arqueoloxía musical
 \item
 Iconografía
 \item
 Etnomusicoloxía % correcta
 \item
 Paleo-organoloxía
\end{enumerate}

 \end{enumerate}
\end{ejercicio}
%
