% -------------------------------
% BANCO DE PREGUNTAS DE HISTORIA
% -------------------------------
% Tema 0.- INTRODUCIÓN: PERIODIZACIÓN
%
% --------------------------------------
% EXERCICIO 01: Periodización - Concepto
% --------------------------------------
\newproblem{T1PE-01}{
    A que nos referimos cando falamos de periodización en Historia da Música?
    \begin{enumerate}[a)]
    \item 
    Á división en etapas —períodos— da historia, que ten a súa orixe nos humanistas da Idade Media, época na que tamén aparecen as primeiras historias da arte.
    \item
    Á división en etapas —períodos— da historia, que ten a súa orixe coa ilustración (século XVIII), época na que tamén aparece o capitalismo.
    \item \label{T1PE-01:sol}
    Á división en etapas —períodos— da historia, que ten a súa orixe nos humanistas do Renacemento, época na que tamén aparecen as primeiras historias da arte. %(x)
    \item
    Á división en etapas —períodos— da historia, que ten a súa orixe coa ilustración no Renacemento, época na que tamén aparecen as primeiras historias da arte.
    \end{enumerate}
    }
%
% Solución:
% ---------
    {% solución:
    (\ref{T1PE-01:sol}) {\color{orange}{\hrulefill}}
%    comentario:
    \\ \small{% Texto do comentario:
    O concepto periodización nas ciencias sociais, fai referencia a dividir a historia --ou outro eido do coñecemento-- (ciencia, literatura, arte, etc.) en distintos períodos que posúan unha serie de trazos comúns entre si, o suficientemente importantes como para facelos cualitativamente distintos a outros períodos. Ten a súa orixe nos humanistas do Renacemento época na que tamén aparecen as primeiras historias da arte. A historia da música é moi posterior: os primeiros libros apareceron cara finais do século XVIII, e o desenvolvemento da «historiografía musical» é principalmente do século XIX. Esta aparición tardía, fai que a Historia da música adopte habitualmente as periodizacións principalmente da Historia da arte.
    {\color{orange}{\hrulefill}}
    }
    }
% ---
%
% -------------------------------------------
% EXERCICIO 02: Periodización -  Idade Antiga
% -------------------------------------------
\newproblem{T1PE-02}{
    A historia divídese comunmente en catro períodos (idades) que debemos coñecer. Por que se caracteriza a Idade Antiga? 
    \begin{enumerate}[a)]
    \item 
    Entre outros feitos, polas teorías gregas sobre a música, que infuíron de forma importante na música europea anterior ao século V.
    \item
    Entre outros feitos, comeza coa disgregación dos territorios occidentais do Imperio Romano e, malia que hai moitos datos sobre a música europea occidental neste período, o máis relevante, será a aparición da notación musical no século X e sobre todo a notación sobre liñas paralelas a partir do século XI.
    \item
    Entre outras características, comeza coa disgregación dos territorios occidentais do Imperio Romano e entre outros feitos de relevancia, sinalaremos as teorías gregas sobre a música.
    \item \label{T1PE-02:sol}
    Entre outros feitos, polas teorías gregas sobre a música, que infuíron de forma importante na música europea medieval, renacentista e barroca. %(x)
    \end{enumerate}
    }
%
% Solución:
% ---------
    {% solución:
    (\ref{T1PE-02:sol}) {\color{orange}{\hrulefill}}
%    comentario:
    \\ \small{% Texto do comentario:
    A maior parte das culturas antigas non escribiron a súa música e as escasas notacións musicais que existen son practicamente descoñecidas, coa excepción importante da Grecia helenística (a partir do século IV a.c). Para a historia da música occidental, o máis interesante desta época son as teorías gregas sobre a música, que infuíron de forma importante na música europea medieval, renacentista e barroca.
    {\color{orange}{\hrulefill}}
    }
    }
% ---
%
% -----------------------------------------
% EXERCICIO 03: Periodización - Idade Media
% -----------------------------------------
\newproblem{T1PE-03}{
    A historia divídese comunmente en catro períodos (idades) que debemos coñecer. De eles, a Idade Media ...
    \begin{enumerate}[a)]
    \item \label{T1PE-03:sol}
    comeza coa disgregación dos territorios occidentais do Imperio Romano e, malia que hai moitos datos sobre a música europea occidental neste período, o máis relevante, será a aparición da notación musical no século X e sobre todo a notación sobre liñas paralelas a partir do século XI. %(x)
    \item
    comeza coa disgregación dos territorios occidentais do Imperio Romano e entre outros feitos de relevancia, sinalaremos as teorías gregas sobre a música.
    \item
    comeza coincidindo coa disgregación dos territorios occidentais do Imperio Romano (s. VI)
    \item
    caracterízase entre outros feitos, polas teorías gregas sobre a música, que infuíron de forma importante na música europea anterior ao século V.
    \end{enumerate}
    }
%
% Solución:
% ---------
    {% solución:
    (\ref{T1PE-03:sol}) {\color{orange}{\hrulefill}}
%    comentario:
    \\ \small{% Texto do comentario:
    As comunicacións entre os territorios occidentais do Imperio Romano redúcense, a inestabilidade é grande e os desenvolvementos culturais teñen que partir ás veces de cero; comeza así unha progresiva disgregación e decadencia que marcará os inicios da Idade Media, onde entre outras moitas características, destacaremos no eido musical, a aparición da notación musical (século X) así como a notación sobre liñas paralelas (século XI). 
    {\color{orange}{\hrulefill}}
    }
    }
% ---
%
% -------------------------------------------
% EXERCICIO 04: Periodización - Idade Moderna
% -------------------------------------------
\newproblem{T1PE-04}{
    Indica cales de entre as seguintes afirmacións, son características que definen a Idade Moderna.
    \begin{enumerate}[a)]
    \item 
    Desaparición do Imperio Bizantino (continuador do romano), chegada de Cristóbal Colón a América e o movemento cultural do Humanismo.
    \item
    Abrangue aproximadamente o período comprendido entre os séculos XVI e XVIII e será cando se estableza a nomenclatura «antiga», «media» e «moderna» para designar estas tres etapas históricas
    \item \label{T1PE-04:sol}
    As dúas respostas anteriores son correctas %(x)
    \item
    Ningunha das respostas anteriores é correcta
    \end{enumerate}
    }
%
% Solución:
% ---------
    {% solución:
    (\ref{T1PE-04:sol}) {\color{orange}{\hrulefill}}
%    comentario:
    \\ \small{% Texto do comentario:
    Entre outras características que definen a Idade Moderna (séculos XVI ao XVIII) musicalmente debemos lembrar que será aquí cando se poñen as bases do sistema tonal.  Culturalmente, a fin da Idade Media está condicionada polo movemento coñecido como Humanismo que predominará no Renacemento. A nomenclatura «antiga», «media» e «moderna» para designar as tres etapas históricas atribúese a estes humanistas, que estrañaban o esplendor cultural dos antigos gregos e romanos e desprezaban a época intermedia que os separaba. Será cando desaparece o Imperio Bizantino e Colón descubre América. 
    {\color{orange}{\hrulefill}}
    }
    }
% ---
%
% -------------------------------------------
% EXERCICIO 05: Periodización - Idade Moderna
% -------------------------------------------
\newproblem{T1PE-05}{
    A historia divídese comunmente en catro períodos (idades) que debemos coñecer. Cales das seguintes caracteríticas definen a Idade Moderna?   
    \begin{enumerate}[a)]
    \item 
    Musicalmente caracterízase polo desenvolvemento do sistema tonal, que terá o seu máximo esplendor no Barroco tardío e no Clasicismo.
    \item
    Abrangue aproximadamente o período comprendido entre os séculos XVI e XVIII e será cando se estableza a nomenclatura «antiga», «media» e «moderna» para designar estas tres etapas históricas.
    \item \label{T1PE-05:sol}
    As dúas respostas anteriores son correctas %(x)
    \item
    Ningunha das anteriores é correcta
    \end{enumerate}
    }
%
% Solución:
% ---------
    {% solución:
    (\ref{T1PE-05:sol}) {\color{orange}{\hrulefill}}
%    comentario:
    \\ \small{% Texto do comentario:
    Entre outras características que definen a Idade Moderna (séculos XVI ao XVIII) musicalmente debemos lembrar que será aquí cando se crean as bases do sistema tonal. Culturalmente, a fin da Idade Media está condicionada polo movemento humanista, que predomina no Renacemento. A nomenclatura «antiga», «media» e «moderna» para designar as tres etapas históricas atribúese a estes humanistas, que estrañaban o esplendor cultural dos antigos gregos e romanos, e desprezaban a época intermedia que os separaba. Será cando desaparece o Imperio Bizantino e Colón chega ás Américas.
    {\color{orange}{\hrulefill}}
    }
    }
% ---
%
% -------------------------------------------------
% EXERCICIO 06: Periodización - Idade Contemporánea
% -------------------------------------------------
\newproblem{T1PE-06}{
    En que idade, das consideradas na periodización da Historia, podemos afirmar que os músicos comezan a ser artistas independentes?
    \begin{enumerate}[a)]
    \item 
    Idade Media
    \item
    Idade Moderna
    \item \label{T1ES-06:sol}
    Idade Contemporánea %(x)
    \item
    En ningunha das anteriores
    \end{enumerate}
    }
%
% Solución:
% ---------
    {% solución:
    (\ref{T1ES-06:sol}) {\color{orange}{\hrulefill}}
%    comentario:
    \\ \small{% Texto do comentario:
    Na Idade Contemporánea, os músicos comezan a desenvolverse como artistas independientes que non dependen exclusivamente da aristocracia, debido entre outras á aparición do concerto público, á impresión editorial, e outras moitas.
    {\color{orange}{\hrulefill}}
    }
    }
% ---
%
% -----------------------------------------
% EXERCICIO 07: Periodización - Renacemento
% -----------------------------------------
\newproblem{T1PE-07}{
    A música comprendida entre os séculos XV e XVI, corresponde co período ...
    \begin{enumerate}[a)]
    \item \label{T1ES-07:sol}
    Renacemento %(x)
    \item
    Barroco
    \item
    Clasicismo
    \item
    Ningún dos anteriores
    \end{enumerate}
    }
%
% Solución:
% ---------
    {% solución:
    (\ref{T1ES-07:sol}) {\color{orange}{\hrulefill}}
%    comentario:
    \\ \small{% Texto do comentario:
    Segundo a periodización que temos en conta na Historia da Música, o período comprendido entre o 1420 e inicios do século XVII (1580 apróx.) corresponde ao Renacemento.
    {\color{orange}{\hrulefill}}
    }
    }
% ---
%
% ------------------------------------------
% EXERCICIO 08: Periodización - Romanticismo
% ------------------------------------------
\newproblem{T1PE-08}{
    A Historia da Música, considera que o período coñecido como Romanticismo abrangue 
    \begin{enumerate}[a)]
    \item 
    O século XVIII, con varias etapas diferenciadas.
    \item
    Os séculos XVI e XVII, sen diferenciar etapas.
    \item
    O século XVI, con varias etapas diferenciadas.
    \item \label{T1ES-08:sol}
    O século XIX, con varias etapas diferenciadas. %(x)
    \end{enumerate}
    }
%
% Solución:
% ---------
    {% solución:
    (\ref{T1ES-08:sol}) {\color{orange}{\hrulefill}}
%    comentario:
    \\ \small{% Texto do comentario:
    Este período da Historia da Música occidental, abrangue case a totalidade do século XIX (ano 1820) e a primeira década do século XX. Entre algúns dos artistas destacados da época, atopamos a Chopin, Liszt, Schubert, Roberto Schumann, Clara Wiek, Fanny e Felix Mendelssohn, Berlioz, Wagner, Paganini, Verdi, etc.
    {\color{orange}{\hrulefill}}
    }
    }
% ---
%
% --------------------------------------------------
% EXERCICIO 09: Periodización - Música contemporánea
% --------------------------------------------------
\newproblem{T1PE-09}{
    Cal é o término que de forma xeral, empregamos para referirnos á música comprendida entre o ano 1890 e os nosos días?
    \begin{enumerate}[a)]
    \item \label{T1ES-09:sol}
    Música contemporánea %(x)
    \item
    Música clásica
    \item
    Música romántica
    \item
    Música barroca
    \end{enumerate}
    }
%
% Solución:
% ---------
    {% solución:
    (\ref{T1ES-09:sol}) {\color{orange}{\hrulefill}}
%    comentario:
    \\ \small{% Texto do comentario:
    A grandes rasgos, sen ter en conta as diferentes etapas e a súa diferenciación estilística, a música comprendida entre o século XX e XXI coñécese como música contemporánea. Ben é certo, que dentro deste período atoparemos diferentes correntes estilísticas na música europea occidental.
    {\color{orange}{\hrulefill}}
    }
    }
% ---
%
% ----------------------------------------------
% EXERCICIO 10: Periodización - Románico, Gótico
% ----------------------------------------------
\newproblem{T1PE-10}{
    Indica a opción correcta segundo a seguinte afirmación: "A arte Románica (séculos X-XII) e Gótica (séculos XII-XV), corresponden á Idade Media; o Neoclasicismo (século XVIII ata inicios do XIX) ..."
    \begin{enumerate}[a)]
    \item 
    Na música corresponde co Barroco
    \item \label{T1ES-10:sol}
    Na música denomínase simplemente Clasicismo %(x)
    \item
    Na música corresponde co Renacemento
    \item
    Ningunha das anteriores é correcta
    \end{enumerate}
    }
%
% Solución:
% ---------
    {% solución:
    (\ref{T1ES-10:sol}) {\color{orange}{\hrulefill}}
%    comentario:
    \\ \small{% Texto do comentario:
    A Historia da Música, adoita términos que coinciden en moitas ocasións cos empregados na Historia da Arte. O «Neoclasicismo» ou «estilo neoclásico» foi un movemento cultural, artístico e literario orixinado en Italia que se estende por Europa, desde mediados do século XVIII, como reacción ao estilo barroco e durou até ás primeiras décadas do século XIX, en que foi substituído polo Romanticismo.
    {\color{orange}{\hrulefill}}
    }
    }
% ---
%
