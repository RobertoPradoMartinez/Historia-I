% -------------------------------
% BANCO DE PREGUNTAS DE HISTORIA
% -------------------------------
% Tema 0.- INTRODUCIÓN: PERSPECTIVAS SOBRE A MÚSICA
%
% -----------------------------------------
% EXERCICIO 01: Teorías orixe - Logoxénicas
% -----------------------------------------
\newproblem{T0PERS-01}{
    Adóitase situar o comezo da historia no momento da aparición da escritura, hai uns seis mil anos aproximadamente. Desde a súa orixe, a música é un fenómeno moi complexo que está formado por diferentes parámetros e manifestacións que xorden na historia da humanidade de xeito progresivo. Dos resultados obtidos da análise dos datos das diversas fontes de información, baralláronse varias teorías sobre a orixe da música. Cal de estas teorías considera que a música xorde como un elemento de comunicación?
    \begin{enumerate}[a)]
    \item 
    As máxico-relixiosas
    \item \label{T0PERS-01:sol}
    As logoxénicas %(x)
    \item
    As quinéticas
    \item
    Ningunha das anteriores
    \end{enumerate}
    }
%
% Solución:
% ---------
    {% solución:
    (\ref{T0PERS-01:sol}) {\color{orange}{\hrulefill}}
%    comentario:
    \\ \small{% Texto do comentario:
As teorías logoxénicas, defenden que a música é un elemento e comunicación asociado á linguaxe. As quinéticas, defenden a idea de que o corpo é en si mesmo un instrumento musical, pois dispón de cordas vocais e instrumentos de percusión corporal; as máxico-relixiosas atribúen unha función máxica á música. Por outra banda, certos autores consideraban que os sons poden empregarse para localizarse na escuridade ou comunicarse a distancia como se fai hoxe en lugares de difícil acceso a causa da orografía do terreo; estas son as teorías fisiolóxico-comunicativas.
{\color{orange}{\hrulefill}}
    }
    }
% ---
%
% -----------------------------------------------
% EXERCICIO 02: Teorías sobre música - Quinéticas
% -----------------------------------------------
\newproblem{T0PERS-02}{
    Que teoría ou teorías defenden a idea de que o corpo é en si mesmo un instrumento musical, dado que dispón de cordas vocais e instrumentos de percusión corporal?
    \begin{enumerate}[a)]
    \item 
    As fisiolóxico-comunicativas
    \item
    As máxico-relixiosas
    \item
    As logoxénicas
    \item \label{T0PERS-02:sol}
    As quinéticas %(x)
    \end{enumerate}
    }
%
% Solución:
% ---------
    {% solución:
    (\ref{T0PERS-01:sol}) {\color{orange}{\hrulefill}}
%    comentario:
    \\ \small{% Texto do comentario:
    As quinéticas, defenden a idea de que o corpo é en si mesmo un instrumento musical, pois dispón de cordas vocais e instrumentos de percusión corporal, a diferenza das logoxénicas que consideran a música un elemento de comunicación asociado á linguaxe, ou as máxico-relixiosas que atribúen unha función máxica á música. Por outra banda, certos autores consideraban que os sons poden empregarse para localizarse na escuridade ou comunicarse a distancia como se fai hoxe en lugares de difícil acceso a causa da orografía do terreo; estas son as teorías fisiolóxico-comunicativas.
    {\color{orange}{\hrulefill}}
    }
    }
% ---
%
% --------------------------------------------------------------
% EXERCICIO 03: Teorías sobre música - Fisiolóxico-comunicativas
% --------------------------------------------------------------
\newproblem{T0PERS-03}{
    As teorías fisiolóxico-comunicativas, defenden que ...
    \begin{enumerate}[a)]
    \item \label{T0PERS-03:sol}
    Os sons poden empregarse para localizarse na escuriade e comunicarse na distancia. %(x)
    \item
    A música é un elemento de comunicación asociado á linguaxe.
    \item
    A música ten a súa orixe na coordinación para o traballo colectivo. 
    \item
    O corpo é en si mesmo un instrumento musical
    \end{enumerate}
    }
%
% Solución:
% ---------
    {% solución:
    (\ref{T0PERS-03:sol}) {\color{orange}{\hrulefill}}
%    comentario:
    \\ \small{% Texto do comentario:
    As quinéticas, defenden a idea de que o corpo é en si mesmo un instrumento musical, pois dispón de cordas vocais e instrumentos de percusión corporal, a diferenza das logoxénicas que consideran a música un elemento de comunicación asociado á linguaxe, ou as máxico-relixiosas que atribúen unha función máxica á música. Por outra banda, certos autores consideraban que os sons poden empregarse para localizarse na escuridade ou comunicarse a distancia como se fai hoxe en lugares de difícil acceso a causa da orografía do terreo; estas son as teorías fisiolóxico-comunicativas.
    {\color{orange}{\hrulefill}}
    }
    }
% ---
%
% -----------------------------------------
% EXERCICIO 04: Perspectivas sobre a música
% -----------------------------------------
\newproblem{T0PERS-04}{
    Entre as moitas reflexións sobre a orixe da música, destacamos a seguinte: "A música é, ao mesmo tempo, unha arte e unha ciencia. Como arte, non é senón a manifestación do belo por medio dos sons; pero esta manifestación descansa nunha ciencia exacta, formada polo conxunto de leis que rexen a produción dos sons, ao mesmo tempo que as súas relacións de altura e duración." A quen atribúes este pensamento sobre a música?
    \begin{enumerate}[a)]
    \item \label{T0PERS-04:sol}
    Riemann (século XIX) %(x)
    \item
    Platón (séculos V-VI a.C)
    \item
    Wagner (século XIX)
    \item
    Santo Tomás (século XIII)
    \end{enumerate}
    }
%
% Solución:
% ---------
    {% solución:
    (\ref{T0PERS-04:sol}) {\color{orange}{\hrulefill}}
%    comentario:
    \\ \small{% Texto do comentario:
    Entre as refexións sobre a música que teremos en conta para comprender o seu signifcado, atopamos esta de Hugo Riemann no século XIX, que considera a música como ciencia e arte ao mesmo tempo.
    {\color{orange}{\hrulefill}}
    }
    }
% ---
%