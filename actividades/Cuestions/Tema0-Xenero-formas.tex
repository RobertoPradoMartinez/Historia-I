% 
% EXERCICIOS DO TEMA 0.- INTRODUCCIÓN - 
% Conceptos sobre a forma, xéneros musicais e estilos
%
% Repaso de conceptos de xeneros e formas da música

\begin{multicols}{2}

% EXERCICIO: FORMAS E ESTILOS
% ---------------------------
\begin{ejercicio}[Formas e estilos]
 \begin{enumerate}[1)]
  \item 
  Se nunha audición, analizamos os instrumentos que escoitamos nunha composición musical, en que aspecto estamos a fixar a nosa atención?

  \begin{enumerate}[a)]
   \item 
   Na textura
   \item % solución correcta
   No timbre
   \item
   Na forma
   \item
   No ritmo
  
  \end{enumerate}
  
  \item
   Cando escoitamos unha audición e tratamos de identificar a estrutura que ten, partes ou movementos, estamos a analizar a súa \ldots
  
  \begin{enumerate}[a)]
   \item 
   Textura
   \item 
   Timbre
   \item % solución correcta
   Forma
   \item
   Ritmo
  \end{enumerate}
 
 \end{enumerate}

\end{ejercicio}



% Repaso de conceptos de xeneros e formas da música
%\begin{multicols}{2}

% EXERCICIO: FONTES DE INFORMACIÓN
% --------------------------------

\begin{ejercicio}[Fontes de información]
 \begin{enumerate}[1)]
  \item 
  Indica cales son as principais fontes de información que consideramos no estudo da Historia da Música.

  \begin{enumerate}[1.]
   \item \dotfill 
   \item \dotfill 
   \item \dotfill 
   \item \dotfill 
  \end{enumerate}
 
 \item
 As pinturas, esculturas e outras obras de arte son consideradas fontes de información \par
 \dotfill
 \par
 Que é para ti unha fonte de información histórica?
 \par \vspace*{2.3cm}
 \end{enumerate}
\end{ejercicio}
%
\end{multicols}

% EXERCICIO: Liña temporal da historia
% ------------------------------------

\begin{ejercicio}[Liña temporal da historia]
 \begin{enumerate}[1)]
  \item 
A división en etapas ---períodos--- da historia, ten a súa orixe nos humanistas europeos do Renacemento. Indica a periodización:

\begin{enumerate}[a)]
 \item
 Idade Antiga: \dotfill ao  \dotfill \hspace{0.5cm}
 \item
 Idade Media:  \dotfill ao  \dotfill  \hspace{0.5cm}
 \item
 Idade Moderna:  \dotfill ao  \dotfill  \hspace{0.5cm}
 \item
 Idade Contemporánea:  \dotfill ao  \dotfill  \hspace{0.5cm} 
\end{enumerate}
\item
Completa a seguinte periodización:
\par
\begin{center}
\begin{tabular}{lcl}
%\hline
Período ou etapa &  & Cronoloxía \\
\hline
Románico & $\Rightarrow$ &  \\
       & $\Rightarrow$ & XII- XV \\
Renacemento & $\Rightarrow$ & \\
& $\Rightarrow$ & XVII- XVIII \\
Neoclasicismo & $\Rightarrow$ & \dotfill e comezo \dotfill \\
 & $\Rightarrow$ & final do XVIII e parte do XIX \\
Positivismo e Realismo & $\Rightarrow$ &  \\
\hline
\end{tabular}
\end{center}
\end{enumerate}

\end{ejercicio}
%

