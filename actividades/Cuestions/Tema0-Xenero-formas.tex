% 
% EXERCICIOS DO TEMA 0.- INTRODUCCIÓN - 
% Conceptos sobre a forma, xéneros musicais e estilos
%
% Repaso de conceptos de xeneros e formas da música
\begin{multicols}{2}

\begin{ejercicio}[Formas e estilos]
 \begin{enumerate}[1)]
  \item 
  Se nunha audición, analizamos os instrumentos que escoitamos nunha composición musical, en que aspecto estamos a fixar a nosa atención?

  \begin{enumerate}[a)]
   \item 
   Na textura
   \item % solución correcta
   No timbre
   \item
   Na forma
   \item
   No ritmo
  
  \end{enumerate}
  
  \item
  Se nos fixamos na estrutura dunha peza, é dicir, as súas diferentes partes, en que parámetro musical me estou centrando?
  
  \begin{enumerate}[a)]
   \item 
   Na textura
   \item 
   No timbre
   \item % solución correcta
   Na forma
   \item
   No ritmo
  \end{enumerate}
 
 \end{enumerate}

\end{ejercicio}

\end{multicols}

% Repaso de conceptos de xeneros e formas da música
%\begin{multicols}{2}

\begin{ejercicio}[Fontes para o estudo da música]
 \begin{enumerate}[1)]
  \item 
  Indica cales son as principais fontes de información que consideraremos para o estudo da Historia da Música.
  \begin{multicols}{2}
  \begin{enumerate}[1.]
   \item \dotfill
   \item \dotfill
   \item \dotfill
   \item \dotfill
  
  \end{enumerate}
  \end{multicols}
  \item
  Se nos fixamos na estrutura dunha peza, é dicir, as súas diferentes partes, en que parámetro musical me estou centrando?
  
  \begin{enumerate}[a)]
   \item 
   Na textura
   \item 
   No timbre
   \item % solución correcta
   Na forma
   \item
   No ritmo
  \end{enumerate}
 
 \end{enumerate}

\end{ejercicio}
