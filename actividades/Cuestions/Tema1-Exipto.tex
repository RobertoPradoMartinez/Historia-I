% -------------------------------
% BANCO DE PREGUNTAS DE HISTORIA
% -------------------------------
% Tema 1.- MÚSICA EN EXIPTO
%
% EXERCICIO 01:
%
\newproblem{T1EX-01}{
No antigo Exipto, a voz era moi importante nas cerimonias relixiosas e empregábase nos ritos para comunicarse co "máis alá". Que instrumento, do que se empregaba para acompañar a voz, era considerado o máis "apreciado" polos exipcios segundo os teus apuntamentos?
  \begin{enumerate}[a)]
    \item Aulos
    \item Lira
    \item Arpa 
    \item Cítara
    \end{enumerate}
    }
    {c)}
% comentario da resposta:
%    \\ \small{Indica o comentario}
%
% EXERCICIO 02: 
%
\newproblem{T1EX-02}{
O Conservatorio de Viveiro, organizará unha exposición de instrumentos antigos en breve. O centro precisa da colaboración do alumnado de Historia da música de primeiro curso, para catalogar unha arpa procedente dunha cámara funeraria, da que sabemos que ten abundantes ornamentacións pictóricas; foi conservada como parte dunha ofrenda fúnebre, e sabemos que procede do val do Nilo.\\
Dados os datos anteriores, de que civilización proceden?
  \begin{enumerate}[a)]
    \item Grega
    \item Mesopotámica 
    \item Exipcia
    \item Azteca
    \end{enumerate}
    }
    {c)}
% comentario da resposta:
%    \\ \small{Indica o comentario}
%
%
