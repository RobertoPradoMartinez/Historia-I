% -------------------------------
% BANCO DE PREGUNTAS DE HISTORIA
% -------------------------------
% Tema 1.- MÚSICA EN GRECIA:
%
% EXERCICIO 01: ORIXE DIVINO DA MÚSICA
%
\newproblem{T1GR-01}{
Tendo en conta os teus apuntamentos sobre a música en Grecia, como consideras a seguinte afirmación?
    \begin{quote}
        Segundo a mitoloxía grega, a música, tiña orixe divina con Apolo e Orfeo como os seus inventores, [...] posuía poderes máxicos.
    \end{quote}
  \begin{enumerate}[a)]
    \item Falsa
    \item Verdadeira
    \end{enumerate}
    }
    {b)}
% comentario da resposta:
%    \\ \small{Indica o comentario}
%
% EXERCICIO 02: CARACTERÍSTICAS MÚSICA GREGA
%
\newproblem{T1GR-02}{
Na música das antigas civilizacións grega e romana, están as bases cultura europea posterior. Da música dos gregos, destacamos tres aspectos fundamentais, cales?
  \begin{enumerate}[a)]
    \item Era monódica, improvisada e asociada coa poesía e a danza
    \item Era polifónica e estaba asociada coa poesía
    \item Era monódica, non improvisada e asociada só coa danza
    \item Era monódica, non improvisada e asociada á danza
    \end{enumerate}
    }
    {a)}
% comentario da resposta:
%    \\ \small{Indica o comentario}
%
% EXERCICIO 03: FUNDAMENTO NUMÉRICO DA MÚSICA
%
\newproblem{T1GR-03}{
Quen demostrará as súas teorías sobre o fundamento numérico da música alá polo século VI a.C, a pesares de non deixar nada escrito?
  \begin{enumerate}[a)]
    \item Aristóteles
    \item Damón 
    \item Pitágoras de Samos
    \item Platón
    \end{enumerate}
    }
    {c)}
% comentario da resposta:
%    \\ \small{Indica o comentario}
%
% EXERCICIO 04: OS PITAGÓRICOS
%
\newproblem{T1GR-04}{
Que escola ou pensadores, defendían que
    \begin{quote}
        [...] o universo enteiro ten unha estrutura matemática, \textit{todo se reduce a números}.
    \end{quote}
\begin{enumerate}[a)]
    \item Os Pitagóricos, discípulos de Pitágoras de Samos
    \item Os Presocráticos, seguidores de Sócrates
    \item Os Boecianos, discípulos de Severino Boecio
    \item non se sabe \ldots
    \end{enumerate}
    }
    {a)}
% comentario da resposta:
%    \\ \small{Indica o comentario}
%
% EXERCICIO 05: SISTEMA MUSICAL MODAL GREGO
%
\newproblem{T1GR-05}{
Das diferentes fontes conservadas, sabemos que o sistema musical grego é \ldots
  \begin{enumerate}[a)]
    \item modal, baseado no uso de tetracordos descendentes
    \item modal, baseado no uso de tetracordos ascendentes
    \item tonal, baseado no uso de tetracordos descendentes
    \item atonal, baseado nu uso de tetracodos descendentes
    \end{enumerate}
    }
    {a)}
% comentario da resposta:
%    \\ \small{Indica o comentario}
%
% EXERCICIO 06: UNIDADE MODAL E TETRACORDO
%
\newproblem{T1GR-06}{
Que modo dos que coñeces do sistema musical grego, non emprega dous tetracordos iguais?
  \begin{enumerate}[a)]
    \item Modo dórico, correspondente á oitava Mi-Mi
    \item Modo frixio, correspondente á oitava Re-Re
    \item Modo lidio, correspondente á oitava Dó-Dó
    \item Modo mixolidio, correspondente á oitava Si-Si
    \end{enumerate}
    }
    {d)}
% comentario da resposta:
%    \\ \small{Indica o comentario}
%
% EXERCICIO 07: UNIDADE MODAL E TETRACORDO
%
\newproblem{T1GR-07}{
Que modo dos que coñeces do sistema musical grego, emprega os dous tetracordos iguais?
  \begin{enumerate}[a)]
    \item Modo dórico, correspondente á oitava Re-Re
    \item Modo frixio, correspondente á oitava Si-Si
    \item Modo lidio, correspondente á oitava Dó-Dó
    \item Modo mixolidio, correspondente á oitava Si-Si
    \end{enumerate}
    }
    {c)}
% comentario da resposta:
%    \\ \small{Indica o comentario}
%
%
% EXERCICIO 08: SISTEMA MUSICAL GREGO
%
\newproblem{T1GR-08}{
Como se coñece o sistema musical grego? Indica o nome correcto.
  \begin{enumerate}[a)]
    \item Sistema monódico teleion
    \item Sistema enharmónico teleion
    \item Sistema diatónico teleion
    \item Sistema teleion
    \end{enumerate}
    }
    {c)}
% comentario da resposta:
%    \\ \small{Indica o comentario}
%
%
% EXERCICIO 09: TEORÍA MUSICAL GERGA
%
\newproblem{T1GR-09}{
Podemos considerar que a música grega era:
  \begin{enumerate}[a)]
    \item Principalmente tonal, con notación numérica de ritmo marcado e baseada nos modos
    \item Principalmente modal, con notación alfabética de ritmo libre e baseada nos modos
    \item Principalmente tonal, con notación alfabética de ritmo marcado e baseada nos modos
    \item Principalmente modal, con notación numérica de ritmo libre e baseada nos modos
    \end{enumerate}
    }
    {b)}
% comentario da resposta:
%    \\ \small{Indica o comentario}
%
