% -------------------------------
% BANCO DE PREGUNTAS DE HISTORIA
% -------------------------------
% Tema 1.- ORGANOLOXÍA:
%
% EXERCICIO 01: INSTRUMENTOS ANTIGOS
%
\newproblem{T1OR-01}{
Cal se considera o instrumento nacional sumerio?
  \begin{enumerate}[a)]
    \item A lira
    \item A arpa
    \item O Laúde
    \item O Aulos
    \end{enumerate}
    }
    {a)}
%
% comentario da resposta:
%   \\ \small{Indica o comentario}
%
% EXERCICIO 02: INSTRUMENTOS MESOPOTAMIA
%
\newproblem{T1OR-02}{
Que instrumentos aerófonos, de entre os seguintes gardan relación coa música en Mesopotamia?
  \begin{enumerate}[a)]
    \item Frautas
    \item Chirimías dobles
    \item Trompetas rectas
    \item todas
    \end{enumerate}
    }
    {d)}
%
% comentario da resposta:
%   \\ \small{Indica o comentario}
%
%
% EXERCICIO 03: TESTEMUÑOS DE MÚSICA PRIMIXENIA
%
\newproblem{T1OR-03}{
Podemos considerar que os restos de instrumentos da época <<paleolítica>>, os rexistros escritos so século II a.C. e os escritos sobre a música na antigüidade, como testemuños máis primitivos da música?
  \begin{enumerate}[a)]
    \item Non, dado que os restos conservados están en moi mal estado 
    \item Só os restos de instrumentos, non se conservan fontes literarias
    \item Si, porque son fontes de información histórica
    \item Non, ningunha resposta é correcta
    \end{enumerate}
    }
    {c)}
% comentario da resposta:
%    \\ \small{Indica o comentario}
%
% EXERCICIO 04: PATRIMONIO PRIMIXENIO
%
\newproblem{T1OR-04}{
De entre os indicados nos apuntes, cales dos seguintes instrumentos, consideras que forman parte do patrimonio primixenio da Historia da Música?
  \begin{enumerate}[a)]
    \item Tambores, Cornos, \textit{Sonajas}, Frautas de cana, (...)
    \item Arpas, Raspas, \textit{Sonajas}, Aulos (...)
    \item Arcos, Raspas, \textit{Sonajas}, \textit{Krotalos}, (...)
    \item Arpas, Arcos, \textit{Sonajas}, Aulos, \textit{Krotalos} (...)
    \end{enumerate}
    }
    {a)}
% comentario da resposta:
%    \\ \small{Indica o comentario}
%
% EXERCICIO 05: FRAUTAS DE FALANXE
%
\newproblem{T1OR-05}{
Os arqueólogos do Museo de arqueoloxía de Lugo, atoparon recentemente no sótano do Conservatorio Xoan Montes unha especie de frauta de falanxe, ao parecer de procedencia animal, que produce un són que asemella un instrumento de alerta mais que de música en si mesmo. Establecendo similitudes e partindo dos teus coñecementos sobre a Historia da música, a que <<época>> ou <<era>> poden pertencer?
  \par
  \begin{enumerate}[a)]
    \item Era paleolítica
    \item Era neolítica
    \item Idade media
    \item Época antiga
    \end{enumerate}
    }
    {a)}
%
% comentario da resposta:
%   \\ \small{Indica o comentario}
%
% EXERCICIO 06: TAMBORES DE MAN
%
\newproblem{T1OR-06}{
A que época ou era poden pertencer os primeiros tambores de man feitos de arcilla ou barro?
  \par
  \begin{enumerate}[a)]
    \item Era paleolítica
    \item Idade de bronce
    \item Era neolítica
    \item Idade antiga
    \end{enumerate}
    }
    {c)}
%
% comentario da resposta:
%   \\ \small{Indica o comentario}
%
% EXERCICIO 07: TAMBORES DE MAN
%
\newproblem{T1OR-07}{
Os primeiros tambores de man, estaban feitos de arcilla. Eran percutidos inicialmente coas propias mans, polo que considerando que o corpo vibrante é o propio instrumento, podemos consideralos segundo a clasificación moderna como:
  \par
  \begin{enumerate}[a)]
    \item Membranófonos directamente percutidos
    \item Membranófonos indirectamente percutidos
    \item Idiófonos directamente percutidos
    \item Idiófonos indirectamente percutidos
    \end{enumerate}
    }
    {c)}
%
% comentario da resposta:
%   \\ \small{Indica o comentario}
%
% EXERCICIO 08: MESOPOTAMIA FONTES DE INFORMACIÓN
%
\newproblem{T1OR-08}{
Cales das seguintes consideras que son fontes de información da civilización e cultura mesopotámica?
  \par
  \begin{enumerate}[a)]
    \item Documentos literarios, restos de instrumentos
    \item Relieves de pedra, sellos cilíndricos
    \item Documentos literarios e relieves de pedra
    \item Tódalas anteriores
    \end{enumerate}
    }
    {d)}
%
% comentario da resposta:
%   \\ \small{Indica o comentario}
%
% EXERCICIO 09: ORGNOLOXÍA EXIPTO
%
\newproblem{T1OR-09}{
As fontes de información arqueolóxias da cultura exipcia, aportan numerosa información sobre a música que se realizaba e se practicaba. Historiadores e musicólogos, chegaron á conclusión de que nos grandes templos, celebrábanse cerimonias relixiosas con himnos e cánticos ás divinidades. Cal era considerado <<[...] o máis poderoso instrumento para chegar ata as forzas do mundo invisible[...]>> e con cal dos indicados, se acompañaba?
  \par
  \begin{enumerate}[a)]
    \item A frauta, acompañada coa voz
    \item A voz, acompañada coa frauta
    \item A voz, acompañada coa arpa
    \item A arpa, acompañada coa voz
    \end{enumerate}
    }
    {c)}
%
% comentario da resposta:
%   \\ \small{Indica o comentario}
%
% EXERCICIO 10: CAZA ORIXE A INSTRUMENTOS
%
\newproblem{T1OR-10}{
A caza, segundo se indica nos teus apuntamentos de clase, puido dar orixe a:
  \par
  \begin{enumerate}[a)]
    \item Instrumentos de corda
    \item Instrumentos de corda e vento
    \item Instrumentos de vento
    \item Música vocal
    \end{enumerate}
    }
    {b)}
% comentario da resposta:
%   \\ \small{Indica o comentario}
%
% EXERCICIO 11: ORGANOLOXÍA LITUUS 
%
\newproblem{T1OR-11}{
No Museo de arqueoloxía de Lugo, visitando unha das galerías, atopamos un instrumento que chama a nosa atención. Observamos a ficha do instrumento:
\begin{quote}
Aerófono de procedencia etrusca, de corpo recto e campá curva cara arriba, moi empregado xunto co \textit{corno}, \textit{bucina} e o \textit{syrinx} no ámbito militar [...]
\end{quote}
Dados os datos anteriores, sabemos que se trata do \textit{Lituus} (Lituo). Tendo en conta a súa procedencia, a que cultura das seguintes pertence?
  \begin{enumerate}[a)]
    \item Grega
    \item Celta
    \item Romana 
    \item Persa
    \end{enumerate}
    }
    {c)}
% comentario da resposta:
%    \\ \small{Indica o comentario}
%
