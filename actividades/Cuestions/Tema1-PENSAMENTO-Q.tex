% -------------------------------
% BANCO DE PREGUNTAS DE HISTORIA
% -------------------------------
% Importante:
% - Non modificar o nome dos problemas
% - Se hai modificacións da erros á hora de montar o exercicio
%
% Tema 1.- MÚSICA EN GRECIA: PENSAMENTO MUSICAL
% Exercicios de resposta única
%
% EXERCICIO 01: HARMONÍA DAS ESFERAS
%
\newproblem{T1PENS-01}{
Os gregos, sabían que o son procede do movemento. Que teoría das que coñeces, defende que 
    \begin{quote}
        [...] o movemento de cada esfera debía producir un son distinto; segundo as teorías pitagoricas, as distancias entre elas coincidían coas proporcións simples da música, o conxunto dos sons das oito esferas configuraba unha \textit{harmonía} [...]
    \end{quote}
  \begin{enumerate}[a)]
    \item a teoría do \textit{ethos}
    \item a teoría da \textit{harmonía das esferas}
    \item a \textit{concepción matemática da música}
    \item a teoría matemática do \textit{ethos}
    \end{enumerate}
    }
    {b)}
% comentario da resposta:
%    \\ \small{Indica o comentario}
%
% EXERCICIO 02: TEORÍA DO ETHOS
%
\newproblem{T1PENS-02}{
Le con atención o seguinte texto:
    \begin{quote}
       Igual que o \textit{cosmos}, o ser humano está composto de proporcións matemáticas, que han regular a relación entre o corpo e alma. A música, pode reflexar a estrutura psíquica dun ser humano e así relacionarse cos diferentes estados de ánimo.
    \end{quote}
A que teoría das que coñeces, fai referencia o texto anterior?
  \begin{enumerate}[a)]
    \item a teoría do \textit{ethos}
    \item a teoría da \textit{harmonía das esferas}
    \item a \textit{concepción matemática da música}
    \item a teoría matemática do \textit{ethos}
    \end{enumerate}
    }
    {a)}
% comentario da resposta:
%   \\ \small{Indica o comentario}
%
% EXERCICIO 02: HARMONÍA DAS ESFERAS
%
\newproblem{T1PENS-03}{
Que teoría das que coñeces, defende que 
    \begin{quote}
        [...] o movemento de cada unha debía producir un son distinto; segundo as teorías pitagoricas, as distancias entre elas coincidían coas proporcións simples da música, o conxunto dos sons das oito esferas configuraba unha \textit{harmonía} [...]
    \end{quote}
  \begin{enumerate}[a)]
    \item a teoría do \textit{ethos}
    \item a \textit{harmonía das esferas}
    \item a \textit{concepción matemática da música}
    \item a teoría matemática do \textit{ethos}
    \end{enumerate}
    }
    {b)}
% comentario da resposta:
%    \\ \small{Indica o comentario}
%
% comentario da resposta:
%   \\ \small{Indica o comentario}
%
% EXERCICIO 04:
%
\newproblem{T1PENS-04}{
A que civilización, cultura e período atribúes a teoría anterior?
  \par
  \begin{enumerate}[a)]
    \item Grecorromana, Idade antiga
    \item Romana, Idade antiga
    \item Grega, Idade antiga
    \item Grecorromana, Idade media
    \end{enumerate}
    }
    {c)}
%
% comentario da resposta:
%   \\ \small{Indica o comentario}
