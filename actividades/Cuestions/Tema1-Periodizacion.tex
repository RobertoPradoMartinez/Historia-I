% ------------------------------
% BANCO DE PREGUNTAS DE HISTORIA
% ------------------------------
% Tema 1.- PERIODIZACIÓN:
%
% EXERCICIO 01: IDADE ANTIGA
%
\newproblem{T1PE-01}{
Adoitamos situar o comezo da <<Historia>> no momento en que aparece a escritura, hai uns 6000 anos aproximadamente. O período anterior, coñécese co nome de <<Prehistoria>>.\\
Segundo os teus apuntamentos, a liña temporal da <<Idade Antiga>> chega ata o:
  \begin{enumerate}[a)]
    \item século VI a.C
    \item século V d.C
    \item século VII a.C
    \item século VII d.C
    \end{enumerate}
    }
    {b)
    % comentario da resposta:
%    \\ \small{Indica o comentario}
    }
%
% EXERCICIO 02: IDADE MODERNA
%
\newproblem{T1PE-02}{
Tendo en conta os apuntes de clase, entre que séculos situarías a <<Idade Moderna>>?
  \begin{enumerate}[a)]
    \item XVI - XVIII
    \item VI - XV
    \item XIX - XXI
    \item XVI - XIX
    \end{enumerate}
    }
    {a)}
    % comentario da resposta:
%    \\ \small{Indica o comentario}    
%
% EXERCICIO 03: MÚSICA MEDIEVAL
%
\newproblem{T1PE-03}{
Cando en Historia da música, falamos de <<música medieval>> referímonos:
  \begin{enumerate}[a)]
    \item á música occidental dos séculos XV e XVI
    \item á música oriental dos séculos V e XV
    \item á música oriental dos séculos VI e XVI
    \item á música occidental dos séculos V e XV
    \end{enumerate}
    }
    {b)}
    % comentario da resposta:
%    \\ \small{Indica o comentario}
%
% EXERCICIO 04: ROMÁNICO E GÓTICO
%
\newproblem{T1PE-04}{
Románico (s.X-XII) e Gótico (XII-XV) son etapas culturais que situarías na Idade:
  \begin{enumerate}[a)]
    \item Moderna
    \item Contemporánea
    \item Antiga
    \item Media
    \end{enumerate}
    }
    {d)}
    % comentario da resposta:
%    \\ \small{Indica o comentario}
%
% EXERCICIO 05: PREHISTORIA
%
\newproblem{T1PE-05}{
Musicólogos e historiadores, afirman que o comezo da <<Historia>> corresponde coa aparición da escritura. O período anterior, coñécese co nome de \ldots
  \begin{enumerate}[a)]
    \item Prehistoria
    \item Idade Media
    \item Idade Antiga
    \item Idade Moderna
    \end{enumerate}
    }
    {a)
    % comentario da resposta:
%    \\ \small{Indica o comentario}
    }
%
%
% EXERCICIO 06: CAMBIAR! REPETIDA!!
%
\newproblem{T1PE-06}{
Tendo en conta os apuntes de clase, entre que séculos situarías a <<Idade Moderna>>?
  \begin{enumerate}[a)]
    \item XVI - XVIII
    \item VI - XV
    \item XIX - XXI
    \item XVI - XIX
    \end{enumerate}
    }
    {a)
    % comentario da resposta:
%    \\ \small{Indica o comentario}    
    }
%
% EXERCICIO 07: 
%
\newproblem{T1PE-07}{
Tendo en conta os apuntes de clase, entre que séculos situarías a <<Idade Moderna>>?
  \begin{enumerate}[a)]
    \item XVI - XVIII
    \item VI - XV
    \item XIX - XXI
    \item XVI - XIX
    \end{enumerate}
    }
    {a)
    % comentario da resposta:
%    \\ \small{Indica o comentario}    
    }
%
% EXERCICIO 08: 
%
\newproblem{T1PE-08}{
Tendo en conta os apuntes de clase, entre que séculos situarías a <<Idade Moderna>>?
  \begin{enumerate}[a)]
    \item XVI - XVIII
    \item VI - XV
    \item XIX - XXI
    \item XVI - XIX
    \end{enumerate}
    }
    {a)
    % comentario da resposta:
%    \\ \small{Indica o comentario}    
    }
%
% EXERCICIO 09: 
%
\newproblem{T1PE-09}{
Tendo en conta os apuntes de clase, entre que séculos situarías a <<Idade Moderna>>?
  \begin{enumerate}[a)]
    \item XVI - XVIII
    \item VI - XV
    \item XIX - XXI
    \item XVI - XIX
    \end{enumerate}
    }
    {a)
    % comentario da resposta:
%    \\ \small{Indica o comentario}    
    }
%
% EXERCICIO 10: 
%
\newproblem{T1PE-10}{
Tendo en conta os apuntes de clase, entre que séculos situarías a <<Idade Moderna>>?
  \begin{enumerate}[a)]
    \item XVI - XVIII
    \item VI - XV
    \item XIX - XXI
    \item XVI - XIX
    \end{enumerate}
    }
    {a)
    % comentario da resposta:
%    \\ \small{Indica o comentario}    
    }
%
%
% EXERCICIO 11: PERIODIZACIÓN DE GRECIA
%
\newproblem{T1PE-11}{
A notación musical, nace á vez que a escritura. Moitas culturas non escribiron a súa música e as escasas notacións que chegaron ata os nosos días son practicamene descoñecidas coa excepción da cultura grega. Dentro dos diferentes períodos históricos, (épocas, idades, ...) en cal dos seguintes sitúas a cultura grega?
  \begin{enumerate}[a)]
    \item Prehistoria
    \item Idade antiga
    \item Idade da memoria
    \item Idade media
    \end{enumerate}
    }
    {b)
    % comentario da resposta:
%    \\ \small{Indica o comentario}    
    }
%
%
% EXERCICIO 12: DECADENCIA DE ROMA
%
\newproblem{T1PE-12}{
<<Coa decadencia do Imperio Romano, as comunicacións entre os territorios redúcense e a inestabilidade faise cada vez máis presente; a pesares de todo aparece a notación musical e posteriormente a notación sobre liñas paralelas [...]>>\\
Estamos a falar da:
  \begin{enumerate}[a)]
    \item Idade antiga
    \item Idade media
    \item Idade contemporánea
    \item Idade moderna
    \end{enumerate}
    }
    {b)
    % comentario da resposta:
%    \\ \small{Indica o comentario}    
    }
%