% -------------------------------
% BANCO DE PREGUNTAS DE HISTORIA
% -------------------------------
% Tema 1.- MÚSICA EN ROMA:
%
% EXERCICIO 1: ROMA INFLUENCIAS
%
\newproblem{T1RO-01}{
Da cultura romana descoñecemos se efectuaron grandes contribucións significativas á teoría ou práctica da música, pero si sabemos que a súa cultura tivo influencias etruscas.
  \par
  \begin{enumerate}[a)]
    \item Falso, só tiveron influencias gregas
    \item Verdadeiro, e tamén influencia da cultura grega
    \item Falso, os romanos non tiveron influencia etrusca
    \item Verdadeiro, os romanos só tiveron influencia etrusca
    \end{enumerate}
    }
    {b)
    % comentario da resposta:
%    \\ \small{Indica o comentario}    
    }
%
% EXERCICIO 2: 
%
\newproblem{T1RO-02}{
Modelo para exames...
  \par
  \begin{enumerate}[a)]
    \item Opcion 1
    \item Opción 2
    \item Opción 3
    \item Opción 4
    \end{enumerate}
    }
    {a)
    % comentario da resposta:
%    \\ \small{Indica o comentario}    
    }
%
% EXERCICIO 3:
%
\newproblem{T1RO-03}{
Modelo para exames...
  \par
  \begin{enumerate}[a)]
    \item Opcion 1
    \item Opción 2
    \item Opción 3
    \item Opción 4
    \end{enumerate}
    }
    {a)
    % comentario da resposta:
%    \\ \small{Indica o comentario}    
    }
%
% EXERCICIO 4:
%
\newproblem{T1RO-04}{
Modelo para exames...
  \par
  \begin{enumerate}[a)]
    \item Opcion 1
    \item Opción 2
    \item Opción 3
    \item Opción 4
    \end{enumerate}
    }
    {a)
    % comentario da resposta:
%    \\ \small{Indica o comentario}    
    }
%
% EXERCICIO 05: 
%
\newproblem{T1RO-05}{
Tendo en conta os apuntes de clase, entre que séculos situarías a <<Idade Moderna>>?
  \begin{enumerate}[a)]
    \item XVI - XVIII
    \item VI - XV
    \item XIX - XXI
    \item XVI - XIX
    \end{enumerate}
    }
    {a)
    % comentario da resposta:
%    \\ \small{Indica o comentario}    
    }
%
% EXERCICIO 06: 
%
\newproblem{T1RO-06}{
Tendo en conta os apuntes de clase, entre que séculos situarías a <<Idade Moderna>>?
  \begin{enumerate}[a)]
    \item XVI - XVIII
    \item VI - XV
    \item XIX - XXI
    \item XVI - XIX
    \end{enumerate}
    }
    {a)
    % comentario da resposta:
%    \\ \small{Indica o comentario}    
    }
%
% EXERCICIO 07: 
%
\newproblem{T1RO-07}{
Tendo en conta os apuntes de clase, entre que séculos situarías a <<Idade Moderna>>?
  \begin{enumerate}[a)]
    \item XVI - XVIII
    \item VI - XV
    \item XIX - XXI
    \item XVI - XIX
    \end{enumerate}
    }
    {a)
    % comentario da resposta:
%    \\ \small{Indica o comentario}    
    }
%
% EXERCICIO 08: 
%
\newproblem{T1RO-08}{
Tendo en conta os apuntes de clase, entre que séculos situarías a <<Idade Moderna>>?
  \begin{enumerate}[a)]
    \item XVI - XVIII
    \item VI - XV
    \item XIX - XXI
    \item XVI - XIX
    \end{enumerate}
    }
    {a)
    % comentario da resposta:
%    \\ \small{Indica o comentario}    
    }
%
% EXERCICIO 09: 
%
\newproblem{T1RO-09}{
Tendo en conta os apuntes de clase, entre que séculos situarías a <<Idade Moderna>>?
  \begin{enumerate}[a)]
    \item XVI - XVIII
    \item VI - XV
    \item XIX - XXI
    \item XVI - XIX
    \end{enumerate}
    }
    {a)
    % comentario da resposta:
%    \\ \small{Indica o comentario}    
    }
%
% EXERCICIO 10: 
%
\newproblem{T1RO-10}{
Tendo en conta os apuntes de clase, entre que séculos situarías a <<Idade Moderna>>?
  \begin{enumerate}[a)]
    \item XVI - XVIII
    \item VI - XV
    \item XIX - XXI
    \item XVI - XIX
    \end{enumerate}
    }
    {a)
    % comentario da resposta:
%    \\ \small{Indica o comentario}    
    }
%