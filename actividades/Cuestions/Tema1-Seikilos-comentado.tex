% ---------------------------------
% Exemplo de comentario de audición
% ---------------------------------
% A modo orientativo, ofrécese o comantario de audición do Epitafio de Seikilos
%

%\begin{ejercicio}[Audición comentada \textit{Epitafio Seikilos}]
\subsection*{Comentario de audición}\label{seikilos-comentado}
Le con atención o seguinte comentario de audición sobre \textit{Epitafio de Seikilos}, fixándote nos diferentes aspectos que se comentan.
%
   \begin{enumerate}[1.-]

       \item \textbf{Contextualización}
       \par
O Epitafio de Seikilos é un fragmento de inscrición epigráfica grega atopado nunha columna de mármore posta sobre a tumba que fixera construír Seikilos para a súa esposa Euterpe, preto de Trales (en Asia Menor).
\par
Conservada actualmente no Museo Nacional de Dinamarca, foi descuberta en 1883 por Sir W. M. Ramsay en Turquía e conservada no museo da antiga cidade de Esmirna. Durante o holocausto de Asia Menor (1919-1922), no que a cidade de Esmirna foi destruida, non se tivo coñecemento do seu paradeiro; posteriormente foi recuperada con sinais de desgaste e deterioro na súa base e coa última liña do texto borrada.
\par
Este manuscrito constitúe un exemplo de forma de composición musical grega, co engadido de ser a melodía escrita máis antiga que se coñece. A inscrición, contén un texto en grego sobre o que se desenvolve a melodía

   \item \textbf{Timbre} \par
A canción é melancólica, clasificada como \textit{skolion} ou "canción para beber".

    \item \textbf{Textura} \par
    \begin{itemize}
        \item 
        A composición está construída e organizada segundo principios modais.
        \item 
        Está en modo *Mixolidio actual e desenvólvese nun ámbito de oitava xusta. 
        \item
        Aparecen todos os sons da escala La4 + Mi5, con *Fa e *Do sostidos.
        \item
        O son que aparece con máis frecuencia é La4 (oito aparicións), seguido de Mi5 (seis aparicións). 
        \item
        O La4 é o son máis grave, que cerra a composición. 
        \item 
        O ámbito estreito, a escaseza de saltos e a (presumible) utilización dun instrumento para dobrar a liña vocal fan que interpretación a da melodía non revista complexidade técnica algunha.
    \end{itemize}

    \item \textbf{Melodía} \par
        \begin{itemize}
            \item 
            É de ámbito estreito: (a distancia entre a nota máis grave e a nota máis aguda é dunha oitava), que discorre sobre todo por graos conxuntos (intervalos de segunda e terceira).
            \item 
            Entre os saltos melódicos, só pode destacarse o de quinta ascendente co que se inicia a composición. 
            \item 
            A melodía está dividida en catro fases, exactamente iguais en duración (12 tempos cada unha). 
            \item 
            Todas as frases, excepto a última, terminan cun son prolongado, e todas as frases, excepto a primeira, poden considerarse cerradas.
        \end{itemize}
        
    \item \textbf{Ritmo} \par
    \begin{itemize}
        \item 
        Descoñécese a velocidade (tempo) da canción, xa que non está explicada na notación.
        \item 
        O tempo básico, ou unidade de duración (chronos protos), é a duración breve,transcrita en notación ortocrónica como corchea.
        \item
        Cada frase musical, e cada verso do poema, están constituídos por 12 *chronos protos. 
        \item 
        As tres últimas frases teñen unha construción rítmica moi semellante.
        \item
        Os sons da peza poden ter tres duracións: \begin{enumerate}
            \item 
            a trancrita como corchea (chronos protos),
            \item a transcrita como negra (diseme ou dúas chronos protos) e
            \item a negra con puntillo (triseme chronos protos).
            \end{enumerate}
    \end{itemize}

    \item \textbf{Forma} \par
        \begin{itemize}
            \item 
            Estamos ante unha composición que comeza con unha breve introdución de percusión, seguida da melodía principal introducida pola corda, que posteriormente realizará a voz. 
            Trátase dunha forma de reducidas dimensións con intervención da percusión, corda e voz; neste caso é unha forma menor e de ritmo libre seguindo as características da música da Grecia Clásica.
            \par
            
        \end{itemize}

    \item \textbf{Relación música texto}
        \begin{itemize}
            \item 
            A inscrición cantada, segundo a pronuncia do grego \textit{koiné}.
            \item 
            Como é característico da música da Grecia Antiga, melodía e texto forman un todo unificado.
            \item 
            Cada frase musical coincide con cada un dos catro versos que constitúen a composición literaria. 
            \item 
            A relación entre o texto e a música é de estilo silábica (sílaba por nota) con pequenos adornos.
        \end{itemize}     
        
   \end{enumerate}
%\end{ejercicio}
\newpage
