%
% AUDICIÓNS DO PRIMEIRO TRIMESTRE:
% ================================
% Creamos as fichas de audición
% Cargamos as audicións no exame
% Utiliza:
%\useproblem{Audicion-N} % (N = no.audición)
%\begin{multicols}{2}
%%%%%%%%%%%%%%%%%%%%%%%%
%%%%% AUDICIÓN 1 %%%%%%%
%%%%%%%%%%%%%%%%%%%%%%%%

\begin{defproblem}{Audicion-01}

\begin{ejercicio}[]
	\begin{enumerate}[1.-]
   \begin{multicols}{2}
% Pregunta:
% 
%			\vspace*{0.1cm}
		\item
			AUTOR: \dotfill 
		\item 
		    OBRA: 
		    \begin{enumerate}[a)]
		    \item
			\textbf{Título}: \dotfill
		    \item
			\textbf{Timbre}: \par
				\small{Segundo as características da obra e audición, diferenciamos: (indica a correcta)}
				\begin{enumerate}
				 \item
				 \small{Coro de voces masculinas}
				 \item
				 \small{Coro de voces masculinas \emph{a capella}}
				 \item
				 \small{Coro de voces femininas}
				 \item
				 \small{Coro de voces mixto \emph{a capella}}
				\end{enumerate}

			\normalsize
%			\vspace*{0.2cm}	
		    \item 
		    \textbf{Textura}: \\
		    \small{Indica con un $x$ a textura que corresponda.} \par
			\begin{tabular}{lllll} 
			    \ldots & \small{melódica horizontal, monódica} & & \\
			    \ldots & \small{melódica horizontal, polifónica} & & \\
			    \ldots & \small{melodica vertical, homofónica} & & \\
			    \ldots & \small{non melódica} & & \\
			\end{tabular}
            \normalsize
            \item
			\textbf{Forma}: \\
				\small{Indica con un $x$ as que correspondan:} \par
			\begin{tabular}{lllll} 
			     \small{Maior} & \ldots \\ 
			     \small{Menor} & \ldots \\ 
			     \small{Vocal} & \ldots \\ 
			     \small{Instrumental} & \ldots \\ 
			     \small{Mixta} & \ldots \\ 
			     \small{Estruturada (forma fixa)} & \ldots \\ 
			     \small{Libre (sen estrutura coñecida)} & \ldots \\ 
			\end{tabular}
            \normalsize
%			\vspace*{0.2cm}			
		    \item
			\textbf{Estilo}: \\
				\small{Indica con un $x$ o estilo de canto e interpretación da obra:} \par
			\begin{tabular}{llllll}
			     \ldots & \small{Silábico} & & & \ldots & Directo  \\
			     \ldots & \small{Neumático} & & & \ldots & Antifonal   \\
			     \ldots & \small{Melismático} & & & \ldots & Responsorial  \\
			\end{tabular}
            \normalsize
		    \item
			\textbf{Clasificación}: \\
				\small{
				Segundo o ámbito e estilo da obra, estamos ante:
				%\ldots
				}
				\begin{enumerate}
				 \item
				 \small{Un \emph{introito} do <<propio>> da misa}
				 \item
				 \small{Un \emph{gradual} do <<propio>> da misa}
				 \item
				 \small{Unha \emph{sequentia}}
				 \item
				 \small{Un drama litúrxico}
				\end{enumerate}
		    \end{enumerate}
%			
       \end{multicols}
\par \vspace{0.3cm}
	\end{enumerate}
\end{ejercicio}

% Solucións da Audición 1:
% ------------------------
\begin{onlysolution}
    \begin{solution}
Autor: o que sexa\\
Obra: a que sexa\\
    \end{solution}
\end{onlysolution}

\end{defproblem}
% 
% Fin da ficha de audición
\begin{defproblem}{Audicion-01}
\begin{ejercicio}[]
	\begin{enumerate}[1.-]
        \vspace*{0.3cm}
		\item
			Autor: \dotfill
			\vspace*{0.3cm}
		\item
			Obra:
			\begin{enumerate}[a)]
			    \item Título: \dotfill \vspace*{0.3cm}
			    \item Período: \dotfill \vspace*{0.3cm}
			    \item Forma: \dotfill \vspace*{0.3cm}
			    \item Timbre: \dotfill \vspace*{0.3cm} 		
			    \item Textura: \dotfill \vspace*{0.3cm}
			    \item Estilo: \dotfill \vspace*{0.3cm}
			    \item Xénero: \dotfill \vspace*{0.3cm}
			\end{enumerate}
		\item 
		    Resume as principais características que definen a obra:
			\vspace*{8.0cm}			
	\end{enumerate}
\end{ejercicio}

\begin{ejercicio}[]
	\begin{enumerate}[1.-]
        \vspace*{0.3cm}
		\item
			Autor: \dotfill
			\vspace*{0.3cm}
		\item
			Obra:
			\begin{enumerate}[a)]
			    \item Título: \dotfill \vspace*{0.3cm}
			    \item Período: \dotfill \vspace*{0.3cm}
			    \item Forma: \dotfill \vspace*{0.3cm}
			    \item Timbre: \dotfill \vspace*{0.3cm} 		
			    \item Textura: \dotfill \vspace*{0.3cm}
			    \item Estilo: \dotfill \vspace*{0.3cm}
			    \item Xénero: \dotfill \vspace*{0.3cm}
			\end{enumerate}
		\item 
		    Resume as principais características que definen a obra:
			\vspace*{8.0cm}			
	\end{enumerate}
\end{ejercicio}
\end{defproblem}
