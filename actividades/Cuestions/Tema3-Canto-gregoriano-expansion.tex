% -------------------------------
% BANCO DE PREGUNTAS DE HISTORIA
% -------------------------------
%
% Tema 3.- CANTO GREGORIANO
%
% CUESTIÓN: Canto gregoriano: características do canto
% ----------------------------------------------------
%
%CUESTIÓN: Canto gregoriano: expansión - tropos
% ---------------------------------------------
\newproblem{T3GREGO1-01}{
Tropo, designa actualmente un conxunto de técnicas de ampliación do repertorio de canto gregoriano. Cando falamos da técnica de tropar por adición de música, non nos estamos a referir a  ...
    \begin{enumerate}[a)]
    \item
    engadir melismas a algunha das últimas ou primeiras sílabas dun canto 
    \item \label{sol:T3GREGO1-01}
    engadir texto a unha pasaxe dun canto melismático e pasalo a silábico
    \item
    engadir unha pasaxe de música intercalado entre os versos do canto 
    \item
    engadir unha pasaxe música ao principio dos versos do canto 
    \end{enumerate}
}
   {\ref{sol:T3GREGO1-01}}
% comentario da resposta:
%    \\ \small{Indica o comentario}
%

%Cuestión: Canto gregoriano: expansión - tropos
% ---------------------------------------------
\newproblem{T3GREGO1-02}{
Tropo, designa actualmente un conxunto de técnicas de ampliación do repertorio de canto gregoriano. Cando falamos da técnica de tropar engadindo texto e música, estamos a referirnos a \ldots

    \begin{enumerate}[a)]
    \item engadir melismas a algunha das últimas ou primeiras sílabas dun canto 
    \item engadir texto a unha pasaxe dun canto melismático e pasalo a silábico
    \item engadir unha pasaxe de texto e música intercalado entre os versos do canto 
    \item engadir unha pasaxe de texto e música ao principio dos versos do canto 
    \end{enumerate}
}
   {c)}
% comentario da resposta:
%    \\ \small{Indica o comentario}
%

%Cuestión: Canto gregoriano: expansión - secuencias
% ------------------------------------------------
%\newproblem{T3GREGO1-03}{

%As secuencias forman parte das técnicas e formas que aparecen como evolución do canto chá. Podemos afirmar que son ...

%    \begin{enumerate}[a)]
%    \item Cantos independientes de nova composición en estilo silábico e repetición pareada
%    \item Cantos independientes de nova composición en estilo neumático e repetición pareada
%    \item Cantos independientes de nova composición en estilo melismático e repetición pareada
%    \item Cantos independientes de nova composición en estilo silábico sen repetición
%    \end{enumerate}
%}
%   {a)}
% comentario da resposta:
%    \\ \small{Indica o comentario}
%

%Cuestión: Canto gregoriano: expansión - secuencias
% ------------------------------------------------
\newproblem{T3GREGO1-03}{
A obra \textit{Dies Irae}, axústase a unha das formas paralitúrxicas que aparecen como evolución do canto chá. A cal?

    \begin{enumerate}[a)]
    \item 
    Tropos
    \item \label{sol:T3GREGO1-03}
    Secuencias
    \item
    Drama litúrxico
    \item
    Himnos e cánticos
    \end{enumerate}
}
   {\ref{sol:T3GREGO1-03}}
% comentario da resposta:
%    \\ \small{Indica o comentario}
%

% Cuestión: Canto gregoriano: expansión - secuencias
% ------------------------------------------------
%\newproblem{T3GREGO1-04}{

%O XIX Concilio ecuménico da Igrexa católica celebrado en Trento (Italia) coñecido como Concilio de Trento, celebrado en períodos descontinuos entre 1545 e 1563, prohibe varias obras que se baseaban en certas formas de expansión do canto litúrxico, indultando a coñecida misa de defuntos Dies Irae.  A que técnica de expansión do repertorio de este canto, se axusta a mencionada obra?

%    \begin{enumerate}[a)]
%    \itemTropo
%    \item Secuencia
%    \item Drama litúrxico
%    \item Introito
%    \end{enumerate}
%}
%   {solución?}
% comentario da resposta:
%    \\ \small{Indica o comentario}
%

% Cuestión: Canto gregoriano: expansión - secuencias
% ------------------------------------------------
\newproblem{T3GREGO1-04}{
O XIX Concilio ecuménico da Igrexa católica celebrado en Trento (Italia) coñecido como Concilio de Trento, celebrado en períodos descontinuos entre 1545 e 1563, prohibe varias obras que se baseaban en certas formas de expansión do canto litúrxico, indultando a coñecida misa de defuntos Dies Irae.  A que técnica de expansión do repertorio de este canto, se axusta a mencionada obra?

    \begin{enumerate}[a)]
    \item 
    Tropo
    \item\label{sol:T3GREGO1-04}
    Secuencia
    \item
    Drama litúrxico
    \item
    Introito
    \end{enumerate}
}
   {\ref{sol:T3GREGO1-04}}
% comentario da resposta:
%    \\ \small{Indica o comentario}
%