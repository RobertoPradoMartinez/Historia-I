% -------------------------------
% BANCO DE PREGUNTAS DE HISTORIA
% -------------------------------
%
% Tema 3.- CANTO GREGORIANO
% 
% CUESTIÓN: Canto gregoriano: características do canto
% ----------------------------------------------------

\newproblem{T3GREGO-01}{

    Sinala cales das seguintes, son características do canto coñecido como <<gregoriano>>:

    \begin{enumerate}[a)]

        \item trátase dun canto monódico non mensural
        \item trátase dun canto monódico mensural
        \item trátase dun canto polifónico non mensural
        \item trátase dun canto polifónico mensural
        
    \end{enumerate}
}
{a)}
% comentario da resposta:
%    \\ \small{Indica o comentario}
%
% 
% CUESTIÓN: Canto gregoriano: NON características do canto
% --------------------------------------------------------

\newproblem{T3GREGO-02}{

    Sinala cales das seguintes, non son características do canto coñecido como <<gregoriano>>:

    \begin{enumerate}[a)]

        \item trátase dun canto monódico non mensural
        \item trátase dun canto monódico a capella
        \item trátase dun canto monódico en latín
        \item trátase dun canto monódico a capella mensural
        
    \end{enumerate}
}
{d)}
% comentario da resposta:
%    \\ \small{Indica o comentario}
%
% CUESTIÓN: Canto gregoriano: estilos de canto
% --------------------------------------------

\newproblem{T3GREGO-03}{

    Segundo a relación entre o texto e música, distinguimos no canto <<gregoriano>> tres estilos:

    \begin{enumerate}[a)]
        
        \item Directo, antifonal, responsorial
        \item Indirecto, antifonal, responsorial
        \item Silábico, neumático, melismático
        \item Silábico, neumático, non melismático
        
    \end{enumerate}

}
{c)}
% comentario da resposta:
%    \\ \small{Indica o comentario}
%
% CUESTIÓN: Canto gregoriano: estilos de interpretación
% -----------------------------------------------------

\newproblem{T3GREGO-04}{

    Segundo o estilo de interpretación, distinguimos no canto <<gregoriano>> tres estilos :

    \begin{enumerate}[a)]

        \item Directo, antifonal, responsorial
        \item Indirecto, antifonal, responsorial
        \item Silábico, neumático, melismático
        \item Silábico, neumático, non melismático
        
    \end{enumerate}
}
{a)}
% comentario da resposta:
%    \\ \small{Indica o comentario}
%
%CUESTIÓN: Canto gregoriano: estilos de interpretación
% ----------------------------------------------------

\newproblem{T3GREGO-05}{

    Na abadía benedictina de Santo Domingo de Silos (Burgos) dous coros están a realizar un ensaio do introito  <<Puer Natus est nobis>>. Observamos no ensaio, que cantan alternando os dous coros. Segundo a interpretación, podemos afirmar que:

    \begin{enumerate}[a)]

        \item Cantan en estilo directo
        \item Cantan en estilo indirecto
        \item Cantan en estilo responsorial
        \item Cantan en estilo antifonal 
        
    \end{enumerate}
}
{d)}
% comentario da resposta:
%    \\ \small{Indica o comentario}
%
% Cuestión: Canto gregoriano: estilos de canto
% --------------------------------------------

\newproblem{T3GREGO-06}{

    Como sabes, o repertorio gregoriano, mestura varios estilos de canto. Á hora de determinar o estilo, teremos en conta que un deles vai predominar sobre os outros e, polo tanto, caracteriza ese canto.

    Analizada unha folla dun códex do repertorio gregoriano, atopamos un Introito que presenta un texto moi ornamentado, onde prácticamente cada sílaba ten melismas (moi extensos). Observamos ademáis, que esta ornamentación predomina na meirande parte da partitura. Poderiamos concluír que se trata, tendo en conta estas características, dun estilo de canto:

    \begin{enumerate}[a)]

        \item Antifonal
        \item Neumático
        \item Melismático
        \item Responsorial
        
    \end{enumerate}
}
{c)}
% comentario da resposta:
%    \\ \small{Indica o comentario}
%
%Cuestión: Canto gregoriano: repertorio do canto
% ----------------------------------------------
\newproblem{T3GREGO-07}{

    O repertorio gregoriano está formado fundamentalmente por cantos que se interpretaban nas dúas grandes cerimonias litúrxicas. Os cantos do propio ...

    \begin{enumerate}[a)]

        \item eran interpretados pola Scholae, forman parte dos cantos da misa
        \item eran interpretados pola Scholae, forman parte dos cantos do oficio
        \item eran interpretados polos fieles, forman parte dos cantos da misa
        \item eran interpretados pola fieles, forman parte dos cantos do oficio
        
    \end{enumerate}
}
{a)}
% comentario da resposta:
%    \\ \small{Indica o comentario}
%
% Cuestión: Canto gregoriano: cantos do ordinario
% -----------------------------------------------

\newproblem{T3GREGO-08}{

    Kyrie, Gloria, (...) son cantos moi antigos do repertorio gregoriano, que reciben o nome das palabras coas que comezan os textos do canto.
    Indica a resposta correcta:

    \begin{enumerate}[a)]

        \item Forman parte do repertorio dos cantos do ordinario da misa
        \item Forman parte do repertorio dos cantos do propio da misa
        \item Forman parte do repertorio dos cantos do ordinario dos oficios
        \item Forman parte do repertorio dos cantos do propio dos oficios
    
    \end{enumerate}
}
{a)}
% comentario da resposta:
%    \\ \small{Indica o comentario}
%
% Cuestión: Canto gregoriano: cantos do oficio (antífonas)
% --------------------------------------------------------
\newproblem{T3GREGO-09}{

    Que nome reciben os cantos de ámbito reducido, estilo silábico, cantados como introdución e conclusión de salmos e cánticos que forman parte dos cantos do oficio do repertorio gregoriano?

    \begin{enumerate}[a)]
        
        \item Agnus Dei
        \item Antífonas
        \item Responsorios
        \item Himnos
        
    \end{enumerate}

}
{b)}
% comentario da resposta:
%    \\ \small{Indica o comentario}

% Cuestión: Canto gregoriano: autoría dos cantos
% ----------------------------------------------

\newproblem{T3GREGO-10}{

    A meirande parte de melodías de canto chá podemos afirmar que son ...

    Indica a correcta.

    \begin{enumerate}[a)]

        \item Monódicas, non mensurais, sen autor coñecido escritas en cinco liñas
        \item Monódicas, ritmo libre, sen autor coñecido baseadas no sistema modal
        \item Monódicas, mensurais, sen autor coñecido baseadas no sistema modal
        \item Monódicas, non mensurais, de autor coñecido baseadas no sistema modal
        
    \end{enumerate}
}
{b)}
% comentario da resposta:
%    \\ \small{Indica o comentario}
% ------
%
% --------------- entregados en exercicios ----------------
%
\newproblem{T3GREGO-11}{

    Sinala cales das seguintes, son características do canto coñecido como <<gregoriano>>:

    \begin{enumerate}[a)]

        \item trátase dun canto monódico mensural
        \item trátase dun canto polifónico non mensural
        \item trátase dun canto homofónico mensural
        \item ningunha das anteriores
        
    \end{enumerate}
}
{d)}
% comentario da resposta:
%    \\ \small{Indica o comentario}
%
% Cuestión: Canto gregoriano: cantos do oficio (antífonas)
% --------------------------------------------------------
\newproblem{T3GREGO-12}{

    Antínfonas, salmos e responsorios son cantos\ldots

    \begin{enumerate}[a)]
        
        \item Vocais relixiosos do oficio
        \item Vocais relixiosos da misa
        \item Vocais relixiosos do propio
        \item Vocais relixiosos do ordinario
        
    \end{enumerate}

}
{b)}
% comentario da resposta:
%    \\ \small{Indica o comentario}
