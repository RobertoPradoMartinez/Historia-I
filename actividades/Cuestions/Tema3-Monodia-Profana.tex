% -------------------------------
% BANCO DE PREGUNTAS DE HISTORIA
% -------------------------------
%
% Tema 3.- MONODIA PROFANA - MÚSICA TROBADORESCA
%
% CUESTIÓN: FORMA MUSICAL TROVADORESCA: Canción estrófica
% -------------------------------------------------------
\newproblem{T3TROBA-01}{
Que forma musical medieval profana das que coñeces, consiste en dividir o texto en varias estrofas, seguindo o mesmo esquema métrico, coa mesma melodía e estrutura AAB, coñecida como <<forma bar>>?

    \begin{enumerate}[a)]
      \item 
      Tropo
      \item 
      Secuencia
      \item 
      Canción estrófica
      \item 
      Drama litúrxico
    \end{enumerate}
}
   {c)}
% comentario da resposta:
%    \\ \small{Indica o comentario}
%
%
% CUESTIÓN: FORMA MUSICAL TROBADORESCA: Canción estrófica
% -------------------------------------------------------
\newproblem{T3TROBA-02}{
As principais características que definen o estilo das cancións trobadorescas son: 

    \begin{enumerate}[a)]
        \item 
        melodías baseadas no sistema tonal, estilo silábico, sen acompañamento instrumental
        \item 
        melodías baseadas no sistema tonal, estilo ornamentado, con acompañamento instrumental
        \item 
        melodías baseadas no sistema modal, estilo ornamentado, sen acompañamento instrumental
        \item 
        melodías baseadas no sistema modal, estilo silábico e con acompañamento instrumental
    \end{enumerate}
}
  {d)}
% comentario da resposta:
%    \\ \small{Indica o comentario}
%
% CUESTIÓN: FORMA MUSICAL TROBADORESCA: Canción estrófica
% -------------------------------------------------------
\newproblem{T3TROBA-03}{
Destro dos diferentes xéneros de canción trobadoresca, podemos afirmar que …
    \begin{enumerate}[a)]
        \item 
        O sirventés trata temas sociais, satíricos ou morais
        \item 
        O sirventés trata temas dramáticos, máis que líricos
        \item 
        O planh trata temas satíricos ou morais  
        \item 
        A pastorela, trata temas políticos, satíricos ou morais
    \end{enumerate}
}
  {a)}
% comentario da resposta:
%    \\ \small{Indica o comentario}
%
% CUESTIÓN: FORMA MUSICAL TROBADORESCA: Chanson de toile
% -------------------------------------------------------
\newproblem{T3TROBA-04}{
Sinala a opción que consideras correcta, sobre a seguinte afirmación: 
    \begin{quote}
    Lai e Chanson de toile son dous xéneros de canción profana medieval, que \ldots  
    \end{quote}
    \begin{enumerate}[a)]
        \item 
        Pertencen ao xénero da canción de danza dos trobeiros langue d'oil
        \item 
        Pertencen ao xénero da canción narrativa dos trobeiros en langue d'oil 
        \item 
        Pertencen ao xénero de canción de danza dos trobadores langue d'oc
        \item 
        Pertencen ao xénero de canción narrativa dos trobadores langue d'oc
    \end{enumerate}
}
  {b)}
% comentario da resposta:
%    \\ \small{Indica o comentario}
%
% CUESTIÓN: MÚSICA TROBADORESCA: Orixe
% -------------------------------------------------------
\newproblem{T3TROBA-05}{
Podemos afirmar que cara finais do século XI sucede algo que ten o seu centro cultural en Saint Martial de Limoges \ldots A que nos estamos a referir?

    \begin{enumerate}[a)]
        \item Inicio da música profana e movemento trobeiro
        \item Fin da monodia relixiosa e do movemento trobadoresco
        \item Inicio da música profana e movemento trobadoresco        
        \item Fin da monodia relixiosa e do movemento trobeiro 
    \end{enumerate}
}
  {c)}
% comentario da resposta:
%    \\ \small{Indica o comentario}
%
