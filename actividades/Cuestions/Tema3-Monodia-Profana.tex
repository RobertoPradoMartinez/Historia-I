Tema 3.- MONODIA PROFANA - MÚSICA TROBADORESCA

CUESTIÓN: FORMA MUSICAL TROVADORESCA: Canción estrófica

\newproblem{T3NOTA-01}{
A que forma musical medieval profana das que coñeces, consiste en dividir o texto en varias estrofas seguindo o mesmo esquema métrico, coa mesma melodía en estrutura AAB, coñecida como «forma bar»

Canción estrófica (x)

Tropo

Secuencia

Drama litúrxico

}
   {a)}
% comentario da resposta:
%    \\ \small{Indica o comentario}
%




CUESTIÓN: FORMA MUSICAL TROBADORESCA: Canción estrófica

As principais características que definen o estilo das cancións trobadorescas son: 

melodías baseadas no sistema tonal, estilo silábico, sen acompañamento instrumental

melodías baseadas no sistema tonal, estilo ornamentado, con acompañamento instrumental

melodías baseadas no sistema modal, estilo ornamentado, sen acompañamento instrumental

melodías baseadas no sistema modal, estilo silábico e con acompañamento instrumental (x)



CUESTIÓN: FORMA MUSICAL TROBADORESCA: Canción estrófica

Destro dos diferentes xéneros de canción trobadoresca, podemos afirmar que …

O sirventés trata temas sociais, satíricos ou morais (x)

O sirventés trata temas dramáticos, máis que líricos

O planh trata temas satíricos ou morais  

A pastorela, trata temas políticos, satíricos ou morais



CUESTIÓN: FORMA MUSICAL TROBADORESCA: Chanson de toile

Sinala a opción que consideras correcta, sobre a seguinte afirmación: 

Lai e Chanson de toile son dous xéneros de canción profana medieval, que …

Pertencen ao xénero da canción narrativa dos trobeiros en langue d'oil

Pertencen ao xénero da canción de danza dos trobeiros langue d'oil

Pertencen ao xénero de canción de danza dos trobadores langue d'oc

Pertencen ao xénero de canción narrativa dos trobadores langue d'oc



CUESTIÓN: MÚSICA TROBADORESCA: Orixe

Podemos afirmar que cara finais do século XI sucede algo que ten o seu centro cultural en Saint Martial de Limoges… A que nos estamos a referir?

Inicio da música profana e movemento trobadoresco

Inicio da música profana e movemento trobeiro

Fin da monodia relixiosa e do movemento trobadoresco

Fin da monodia relixiosa e do movemento trobeiro 

