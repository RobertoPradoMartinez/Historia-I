% --------------------------------
% BANCO DE PREGUNTAS DE HISTORIA I
% --------------------------------
%
% Tema 3.- MONODIA PROFANA MEDIEVAL
%
% CUESTIÓN: Notación medieval - neumas
% ------------------------------------
\newproblem{T3NOTA-01}{
Alá polo século XI, Guido d’Arezzo crea varias técnicas que facilitaban a lectura a primeira vista e polo tanto a aprendizaxe dos cantos, entre elas \ldots
    \begin{enumerate}[a)]
    \item 
    Situar os neumas nunha pauta de catro liñas paralelas a distancia dunha terceira
    \item 
    Situar os neumas nunha pauta de catro liñas paralelas a distancia dunha segunda
    \item 
    Situar os neumas nunha pauta de cinco liñas paralelas a distancia dunha segunda
    \item 
    Situar os neumas nunha pauta de cinco liñas paralelas a distancia dunha terceira
    \end{enumerate}
}
   {a)}
% comentario da resposta:
%    \\ \small{Indica o comentario}
%
% CUESTIÓN: Notación medieval - claves
% ------------------------------------
\newproblem{T3NOTA-02}{
O sistema de notación guidoniano, considera o uso de letras clave. Cales son e onde se sitúan na pauta de liñas paralelas?
    \begin{enumerate}[a)]
    \item Clave de Fa e Dó as dúas na terceira liña, á dereita da pauta
    \item Clave de Fa e Dó as dúas na segunda liña, á esquerda da pauta
    \item Clave de Fa e Dó en calqueira liña, á esquerda da pauta
    \item Clave de Fa e Dó en calqueira liña, á dereita da pauta
    \end{enumerate}
}
   {c)}
% comentario da resposta:
%    \\ \small{Indica o comentario}
%
% CUESTIÓN: Notación medieval - neumas
% ------------------------------------
\newproblem{T3NOTA-03}{
Que son os <<neumas>>
    \begin{enumerate}[a)]
    \item Notacións modernas que se escribían sobre as liñas do propio texto que se debía cantar, debuxaban o perfil melódico do canto pero non con precisión a melodía.
    \item Notacións máis antigas que se escribían sobre as liñas do propio texto que se debía cantar, debuxaban o perfil melódico do canto e con precisión a melodía.
    \item Notacións máis antigas que se escribían sobre as liñas do propio texto que se debía cantar, debuxaban o perfil melódico do canto pero non con precisión a melodía.
    \item Ningunha das respostas é correcta.
    \end{enumerate}
}
   {c)}
% comentario da resposta:
%    \\ \small{Indica o comentario}
%
% CUESTIÓN: Notación medieval - custus
% ------------------------------------
\newproblem{T3NOTA-04}{
Que nome recibe un pequeno signo situado á dereita de cada pauta que indica a primeira nota da seguinte pauta e facilita a entoación correcta do intervalo, no sistema recopilado por Guido de Arezzo?
    \begin{enumerate}[a)]
    \item \emph{custus}
    \item clave
    \item neuma
    \item hexacordo
    \end{enumerate}
}
   {a)}
% comentario da resposta:
%    \\ \small{Indica o comentario}
%
%
% CUESTIÓN: Notación medieval - neumas
% ------------------------------------
\newproblem{T3NOTA-05}{
Nos modos medievais as catro especies de quinta, son a base dos catro modos básicos. Segundo se combine cada especie, obtemos o que se coñece como modo. Nos modos auténticos:
    \begin{enumerate}[a)]
    \item A especie de cuarta vai antes de de quinta
    \item A especie de quinta vai antes da de terceira
    \item A especie de quinta vai antes da de cuarta
    \item A especie de terceira vai antes da de quinta
    \end{enumerate}
}
   {c}
% comentario da resposta:
%    \\ \small{Indica o comentario}
%
% ---------------------------------
% TODO:ADAPTAR PARA CREAR SOLUCIÓNS
% Solución:
% ---------
%    {% solución:
%    (\ref{T1PE-04:sol}) {\color{orange}{\hrulefill}}
%    comentario:
%    \\ \small{% Texto do comentario:
%    SOLUCIÓN
%    {\color{orange}{\hrulefill}}
%    }
%    }
% ---
