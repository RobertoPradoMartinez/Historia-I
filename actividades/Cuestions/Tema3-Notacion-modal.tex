% --------------------------------
% BANCO DE PREGUNTAS DE HISTORIA I
% --------------------------------
%
% Tema 3.- MONODIA PROFANA MEDIEVAL
%
% CUESTIÓN: Notación medieval - neumas
% ------------------------------------
\newproblem{T3NOTA-01}{
As notacións máis antigas utilizaban uns signos chamados neumas, que se escribían sobre as liñas do propio texto que se debía cantar. Alá polo século XI, Guido d’Arezzo crea varias técnicas que facilitaban a lectura a primeira vista e polo tanto a aprendizaxe dos cantos, entre elas \ldots
    \begin{enumerate}[a)]
    \item 
    Situar os neumas nunha pauta de catro liñas paralelas a distancia dunha terceira
    \item 
    Situar os neumas nunha pauta de catro liñas paralelas a distancia dunha segunda
    \item 
    Situar os neumas nunha pauta de cinco liñas paralelas a distancia dunha segunda
    \item 
    Situar os neumas nunha pauta de cinco liñas paralelas a distancia dunha terceira
    \end{enumerate}
}
   {a)}
% comentario da resposta:
%    \\ \small{Indica o comentario}
%
% CUESTIÓN: Notación medieval - claves
% ------------------------------------
\newproblem{T3NOTA-02}{
O sistema de notación guidoniano, considera o uso de letras clave. Cales son e onde se sitúan na pauta de liñas paralelas?
    \begin{enumerate}[a)]
    \item Clave de Fa e Dó as dúas na terceira liña, á dereita da pauta
    \item Clave de Fa e Dó as dúas na segunda liña, á esquerda da pauta
    \item Clave de Fa e Dó en calqueira liña, á esquerda da pauta
    \item Clave de Fa e Dó en calqueira liña, á dereita da pauta
    \end{enumerate}
}
   {c)}
% comentario da resposta:
%    \\ \small{Indica o comentario}
%
