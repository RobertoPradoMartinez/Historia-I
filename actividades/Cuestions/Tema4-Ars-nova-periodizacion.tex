% -------------------------------
% BANCO DE PREGUNTAS DE HISTORIA
% -------------------------------
% TODO: Redactar solucións!!
% Tema 4.- ARS NOVA
% 
% CUESTIÓN: Ars nova - Periodización da Ars Nova
% ----------------------------------------------
%
\newproblem{T4ARS-NOVA-01}{
Entre o 1320 e o 1380, prodúcese un momento revolucionario na música que se carcteriza por un ambiente de ruptura co pasado; este período, que precede e marca os inicios do <<Renacemento>> denomínase comunmente como \ldots 
    \begin{enumerate}[a)]
    \item 
    \emph{Ars antiqva} debido ao tratado de Philippe de Vitry (c.a 1322)
    \item
    \emph{Ars nova} debido ao tratado de Guillaume de Machaut (c.a 1322)
    \item \label{T4ARS-NOVA-01:sol}
    \emph{Ars nova} debido ao tratado de Philippe de Vitry (c.a 1322)
    \item
    \emph{Ars antiqva} debido ao tratado de Guillaume de Machaut (c.a 1322)
    \end{enumerate}
}
    { % Solución:
    (\ref{T4ARS-NOVA-01:sol}) {\color{orange}{\hrulefill}} \\
    \small{ % Comentario:
    O período que antecede ao <<Renacemento>> --coñecido como \emph{Ars nova}-- exténdese dende o 1320 ata o 1380, e recibe o nome do tratado publicado por Vitry (c.a 1322). 
    {\color{orange}{\hrulefill}}
    }
    }
% 
% CUESTIÓN: ARS NOVA - AUTORES E MÚSICOS DESTACADOS
% -------------------------------------------------
%
\newproblem{T4ARS-NOVA-02}{
Vitry e Machaut (en Francia) xunto con Landini, Ciconia e Bolonia (en Italia) son considerados como os principais representantes da:
    \begin{enumerate}[a)]
    \item 
    \emph{Ars antiqva} (1320 - 1380) que se desenvolve en Italia
    \item 
    \emph{Ars nova} (1320 - 1380) que se inicia en Italia e se desenvolve en Francia
    \item 
    \emph{Ars antiqva} (1320 - 1380) que se inicia en Francia
    \item \label{T4ARS-NOVA-02:sol}
    \emph{Ars nova} (1320 - 1380) que se inicia en Francia e se desenvolve en Italia
    \end{enumerate}
}
    {% Solución:
    (\ref{T4ARS-NOVA-02:sol}) {\color{orange}{\hrulefill}} \\
    \small{% Comentario:
    A nova arte que se orixina en París, entre o 1320 e 1380 terá en Francia como principiais representantes a Vitry e Machaut, se ben o desenvolvemento italiano da \emph{ars nova}, terá como representantes a músicos da época como Landini, Ciconia e Bolonia, entre outros. 
    {\color{orange}{\hrulefill}}
    }
    }
%
% 
% CUESTIÓN: ARS NOVA - CITA DE PERIODIZACIÓN
% ------------------------------------------
%
\newproblem{T4ARS-NOVA-03}{
Nos apuntamentos de Historia, podemos atopar a seguinte afiración:
    \begin{quote}
     En esta época predominaba a música profana e esto revela a debilidade interna da Igrexa no século XIV
    \end{quote}
A que época se refire?
    \begin{enumerate}[a)]
    \item \label{T4ARS-NOVA-03:sol}
    \emph{Ars nova}
    \item 
    \emph{Ars antiqva}
    \item
    Renacemento
    \item 
    Barroco
    \end{enumerate}
}
    {% Solución:
    (\ref{T4ARS-NOVA-03:sol}) {\color{orange}{\hrulefill}} \\
    \small{% Comentario:
    Comentario sobre a resposta:
    {\color{orange}{\hrulefill}}
    }
    }
% 
% CUESTIÓN: ARS NOVA - XÉNEROS PERIODIZACIÓN
% ------------------------------------------
%
\newproblem{T4ARS-NOVA-04}{
O motete (do francés \emph{motet}) será considerado como o xénero principal da época, en especial o \emph{motete doble} francés de temática amorosa, política ou social, entre outras. 
A que época nos estamos a referir?
    \begin{enumerate}[a)]
    \item 
    \emph{Ars antiqva}
    \item
    Renacemento
    \item \label{T4ARS-NOVA-04:sol}
    \emph{Ars nova}
    \item 
    Barroco
    \end{enumerate}
}
    {% Solución:
    (\ref{T4ARS-NOVA-04:sol}) {\color{orange}{\hrulefill}} \\
    \small{% Comentario:
    Comentario sobre a resposta:
    {\color{orange}{\hrulefill}}
    }
    }
% 
