\begin{defproblem}{Audicion-01}
%\begin{multicols}{2}
\begin{ejercicio}[]
	\begin{enumerate}[1.-]
        \vspace*{0.3cm}
		\item
			Autor: \dotfill
			\vspace*{0.3cm}
		\item
			Obra:
			\begin{enumerate}[a)]
                \item Título: \dotfill
			    \begin{multicols}{2}
%			    \item Período: 
%			    \begin{itemize}
%			     \item
%			     (\hspace{0.25cm}) século XV 
%			    \item 
%			     (\hspace{0.25cm}) século XIV
%			     \item
%			     (\hspace{0.25cm}) século XII
%			     \item
%			     (\hspace{0.25cm}) século XVI			     
%			    \end{itemize}

			    \item Forma: 
%			    \dotfill 
			    \vspace*{0.3cm}
			     \begin{itemize}
			     \item
			     (\hspace{0.25cm}) Virelai
			    \item 
			     (\hspace{0.25cm}) Estampida
			     \item
			     (\hspace{0.25cm}) Motete isorrítmico
			     \item
			     (\hspace{0.25cm}) Frottola
			     \item
			     (\hspace{0.25cm}) Misa (Kyrie)
			     \item
			     (\hspace{0.25cm}) Villancico			     
			     \end{itemize}
			    \item Timbre: 
%			    \dotfill 
			    \vspace*{0.3cm} 		
			     \begin{itemize}
			     \item
			     (\hspace{0.25cm}) Voz, corda, vento e percusión
                 \item 
			     (\hspace{0.25cm}) Voz, corda e percusión
			     \item
			     (\hspace{0.25cm}) Voz, corda e vento
                 \item
			     (\hspace{0.25cm}) Voz e corda                
			     \item
			     (\hspace{0.25cm}) Voces mixtas
			     \item
			     (\hspace{0.25cm}) Voces
			     \end{itemize}			    
			    \item Estilo:
%			    \dotfill 
               \vspace*{0.3cm}
                \begin{itemize}
                \item
			     (\hspace{0.25cm}) Contrapuntístico \\ (polifonía medieval)
                 \item 
			     (\hspace{0.25cm}) Trovadoresco \\ (monodia profana medieval)
			     \item
			     (\hspace{0.25cm}) Polifonía relixiosa renacentista
			     \item
			     (\hspace{0.25cm}) Polifonía profana renacentista
			     \end{itemize}                
			    \item Xénero:
%			    \dotfill 
			    \vspace*{0.3cm}
                \begin{itemize}
                \item
			     (\hspace{0.25cm}) Relixioso (música sacra)
                 \item 
			     (\hspace{0.25cm}) Música profana 
			     \item
			     (\hspace{0.25cm}) Música vocal relixiosa
			     \item
			     (\hspace{0.25cm}) Canción profana (popular)
			     \end{itemize}
			    \item Textura:
%			    \dotfill 
			    \vspace*{0.3cm}
                \begin{itemize}
                \item
			     (\hspace{0.25cm}) Melódica horizontal - polifónica
                 \item 
			     (\hspace{0.25cm}) Melódica horizontal - monódica
			     \item
			     (\hspace{0.25cm}) Melódica vertical - homofónica
			     \item
			     (\hspace{0.25cm}) Non melódica
			     \end{itemize}			     
			    \end{multicols}
			\end{enumerate}
%		\item 
%		    Resume as principais características que definen a obra:
			\vspace*{2.0cm}			
	\end{enumerate}
\end{ejercicio}

\begin{ejercicio}[]
	\begin{enumerate}[1.-]
        \vspace*{0.3cm}
		\item
			Autor: \dotfill
			\vspace*{0.3cm}
		\item
			Obra:
			\begin{enumerate}[a)]
                \item Título: \dotfill
			    \begin{multicols}{2}
%			    \item Período: 
%			    \begin{itemize}
%			     \item
%			     (\hspace{0.25cm}) século XV 
%			    \item 
%			     (\hspace{0.25cm}) século XIV
%			     \item
%			     (\hspace{0.25cm}) século XII
%			     \item
%			     (\hspace{0.25cm}) século XVI			     
%			    \end{itemize}

			    \item Forma: 
%			    \dotfill 
			    \vspace*{0.3cm}
			     \begin{itemize}
			     \item
			     (\hspace{0.25cm}) Virelai
			    \item 
			     (\hspace{0.25cm}) Estampida
			     \item
			     (\hspace{0.25cm}) Motete isorrítmico
			     \item
			     (\hspace{0.25cm}) Frottola
			     \item
			     (\hspace{0.25cm}) Misa (Kyrie)
			     \item
			     (\hspace{0.25cm}) Villancico			     
			     \end{itemize}
			    \item Timbre: 
%			    \dotfill 
			    \vspace*{0.3cm} 		
			     \begin{itemize}
			     \item
			     (\hspace{0.25cm}) Voz, corda, vento e percusión
                 \item 
			     (\hspace{0.25cm}) Voz, corda e percusión
			     \item
			     (\hspace{0.25cm}) Voz, corda e vento
                 \item
			     (\hspace{0.25cm}) Voz e corda                
			     \item
			     (\hspace{0.25cm}) Voces mixtas
			     \item
			     (\hspace{0.25cm}) Voces
			     \end{itemize}			    
			    \item Estilo:
%			    \dotfill 
               \vspace*{0.3cm}
                \begin{itemize}
                \item
			     (\hspace{0.25cm}) Contrapuntístico \\ (polifonía medieval)
                 \item 
			     (\hspace{0.25cm}) Trovadoresco \\ (monodia profana medieval)
			     \item
			     (\hspace{0.25cm}) Polifonía relixiosa renacentista
			     \item
			     (\hspace{0.25cm}) Polifonía profana renacentista
			     \end{itemize}                
			    \item Xénero:
%			    \dotfill 
			    \vspace*{0.3cm}
                \begin{itemize}
                \item
			     (\hspace{0.25cm}) Relixioso (música sacra)
                 \item 
			     (\hspace{0.25cm}) Música profana 
			     \item
			     (\hspace{0.25cm}) Música vocal relixiosa
			     \item
			     (\hspace{0.25cm}) Canción profana (popular)
			     \end{itemize}
			    \item Textura:
%			    \dotfill 
			    \vspace*{0.3cm}
                \begin{itemize}
                \item
			     (\hspace{0.25cm}) Melódica horizontal - polifónica
                 \item 
			     (\hspace{0.25cm}) Melódica horizontal - monódica
			     \item
			     (\hspace{0.25cm}) Melódica vertical - homofónica
			     \item
			     (\hspace{0.25cm}) Non melódica
			     \end{itemize}			     
			    \end{multicols}
			\end{enumerate}
%		\item 
%		    Resume as principais características que definen a obra:
			\vspace*{2.0cm}			
	\end{enumerate}
\end{ejercicio}

%\end{multicols}
\end{defproblem}
