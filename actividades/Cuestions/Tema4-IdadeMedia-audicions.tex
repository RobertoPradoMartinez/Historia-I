% -------------------------------
% BANCO DE PREGUNTAS DE HISTORIA
% -------------------------------
%
% Tema 4.- IDADE MEDIA
% 
% CUESTIÓN: IDADE-MEDIA - AUDICIÓNS
% -------------------------------
%
\newproblem{T4IM-AU-01}{
Que famosa obra de música profana e estilo trovadoresco para voz, corda e percusión, de entre as que figuran nas fichas de audición do trimestre, obedece á forma \emph{estampida}?
    \begin{enumerate}[a)]
    \item 
    \emph{Stella Splendens} de Josquin des Prez (1450/55 - 1521)
    \item \label{T4IM-AU-01}
    \emph{Kalenda Maia} de Rimbaut de Vaqueiras (1165 - 1207)
    \item
    \emph{Mas vale trocar} Rimbaut de Vaqueiras (1165 - 1207)
    \item  
    \emph{O vos omnes} de Tomás Luis de Victoria (1548 - 1611)
    \end{enumerate}
}
{\ref{T4IM-AU-01}}
% comentario da resposta:
%    \\ \small{Indica o comentario}
%
%
% CUESTIÓN: IDADE-MEDIA - AUDICIÓNS
% -------------------------------
%
\newproblem{T4IM-AU-02}{
A que forma musical do século XIV, reponde a obra de música sacra en estilo contrapuntísitico para voz, vento, corda e percusión \emph{Stella Splendens}? 
    \begin{enumerate}[a)]
    \item 
    Estampida
    \item \label{T4IM-AU-02}
    Virelai
    \item 
    Villancico (panxoliña)
    \item 
    Frottola
    \end{enumerate}
}
{\ref{T4IM-AU-02}}
% comentario da resposta:
%    \\ \small{Indica o comentario}
%
%
