% -------------------------------
% BANCO DE PREGUNTAS DE HISTORIA
% -------------------------------
%
% Tema 4.- MÚSICA NO RENACEMENTO
% ------------------------------
% Comentario.- 
%
% EXERCICIO.- 
% -------------------------------------
%\newproblem{T4RENA-01}{
%
Adoita situarse a orixe do Renacemento como fenómeno cultural e artístico, nas cidades italianas do norte e en Roma, xunto coas de Flandes e Países Baixos e, por tanto, relacionado con áreas de forte desenvolvemento urbano e comercial. Desde estes focos iniciais, o Renacemento estenderíase a toda Europa paulatinamente. A cronoloxía clásica distingue entre \href{http://es.wikipedia.org/wiki/Quattrocento}{\emph{quattrocento}} e \href{http://es.wikipedia.org/wiki/Cinquecento}{\emph{cinquecento}.}
%
\par
\vspace*{0.25cm}
%
\begin{ejercicio}[Periodización - Renacemento]
Lé con atención o seguinte texto e responde a cuestión.
%\begin{multicols}{2}
    \begin{quote}
    O concepto de \href{http://es.wikipedia.org/wiki/Renacimiento}{Renacemento} aparece para a historiografía da arte e da cultura en época tan distante como o século XIX. Este termo refírese á recuperación da cultura da \href{http://es.wikipedia.org/wiki/Antig\%C3\%BCedad_cl\%C3\%A1sica}{Antigüidade} \href{http://es.wikipedia.org/wiki/Antig\%C3\%BCedad_cl\%C3\%A1sica}{clásica} tras o longo período, supostamente escuro, que suporía a \href{http://es.wikipedia.org/wiki/Edad_Media}{Idade Media.} [\ldots] \\
    Efectivamente, os séculos XV e XVI trouxeron grandes cambios, así como a formación dalgúns dos principios estruturais que estiveron operativos na cultura europea até as revolucións burguesas dos séculos XVIII e XIX. De feito, basta lembrar que a historiografía fai comezar no século (\ldots) unha nova \href{http://es.wikipedia.org/wiki/Edad_Moderna}{Idade Moderna.}
\end{quote}
\begin{flushright}
 J.Jurado: \textit{Apuntamentos para a historia da música}
\end{flushright}
%
Tendo en conta a periodización que coñeces, a que século se está a referir o autor da cita? \ldots
\par
\vspace*{0.15cm}
    \begin{tabular}{c c c c}
        a) XIV & b) XV & c) XIII & d) XVI  \\
    \end{tabular}

%\end{multicols}
\end{ejercicio}
%}
% {a)}
% comentario da resposta:
%    \\ \small{Indica o comentario}
%

