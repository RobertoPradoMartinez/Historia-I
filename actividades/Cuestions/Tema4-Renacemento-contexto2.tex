% -----------------------------------------
% CUESTIÓN: TRAZOS DA CULTURA RENACENTISTA 
% -----------------------------------------
% Comentario.- 
%
%
% EXERCICIO.- 
% -------------------------------------
%\newproblem{T4RENA-02}{
%
Facendo unha síntese, podemos enunciar os trazos xerais da cultura do
Renacemento: 
\begin{multicols}{2}
Crecemento económico e demográfico: o Renacemento ve nacer os principios da economía capitalista (bancos, letras, crédito...). Paralelamente, obsérvase un crecemento das cidades e das clases que lle son propias: a burguesía. O propio concepto moderno de Estado alcanza a súa formulación nas obras de \href{http://es.wikipedia.org/wiki/Maquiavelo}{Maquiavelo.} A nivel social, a expansión das clases burguesas favoreceu a demanda dunha arte laica, en detrimento do protagonismo da arte relixioso que dominara o período anterior.
\par
Desenvolvemento científico e tecnolóxico, ilustrado pola revolución \href{http://es.wikipedia.org/wiki/Cop\%C3\%A9rnico}{copernicana} e a creatividade de personaxes como \href{http://es.wikipedia.org/wiki/Leonardo_da_Vinci}{Leonardo da Vinci.} sentan as bases do \href{http://es.wikipedia.org/wiki/M\%C3\%A9todo_cient\%C3\%ADfico}{método científico} e experimental que se desenvolve con forza e que abre a porta á ciencia moderna e á sociedade tecnolóxica.
\par
A nivel cultural pode falarse dun xiro fundamental coa invención da \href{http://es.wikipedia.org/wiki/Imprenta}{imprenta} a mediados do XV e as posibilidades de expansión de ideas que iso supón. O vigor das universidades e a circulación de información favoreceron a expansión do \href{http://es.wikipedia.org/wiki/Humanismo}{Humanismo.} O Humanismo comprende toda unha antropoloxía dentro da cal a persoa pasa a ocupar un lugar central como punto desde o que se observa e valora a realidade. O pensamento individual asume a responsabilidade de elaborar unha interpretación correcta do mundo mediante un medio crítico baseado na experimentación. O punto de vista crítico do Humanismo respecto diso dunha interpretación dogmática do mundo é, en boa parte, o desencadenamento da crise relixiosa e dos movementos protestantes do XVI.
\par
A creación artística será un dos aspectos máis rechamantes do período. En primeiro lugar polo novo concepto de arte: o artista xa non é un artesán ao servizo da inspiración divina, senón un creador que aspira ao status de home de ciencia. Poucos períodos da Historia de Occidente coñeceron un ritmo de produción de obras de arte tan intenso en calidade e cantidade como este, porque posta ao servizo da exaltación do poder persoal (príncipes, papas...) a creación artística convértese nun elemento de prestixio privilexiado, facendo do mecenado unha institución obrigada para calquera poderoso.\\
\par
\vspace*{0.25cm}
(J.Jurado \textit{Apuntamentos para a Historia da Música}. Ed. Dos Acordes. Vigo 2017.)
\\
%\hrulefill
%
\end{multicols}

\begin{ejercicio}[Trazos da cultura do Renacemento]
Realiza un resumo das principais características que consideras definen a cultura renacentista, tendo en conta o texto do Profesor Jurado. 
\par
\vspace*{9.5cm}
\end{ejercicio}
%}
% {a)}
% comentario da resposta:
%    \\ \small{Indica o comentario}
%
