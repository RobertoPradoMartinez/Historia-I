\subsection*{A escola franco-flamenca}

\begin{multicols}{2}

A guerra dos Cen Anos conduciu a unha diminución da importancia musical de Francia, que fai desviar a hexemonía musical desta nación e de Italia cara a Inglaterra, Borgoña e, sobre todo, Flandes. Estas cortes serán escenario de festas relixiosas e profanas nas que a música ocupa un
posto relevante. Nelas fúndanse capelas musicais principescas a imitación da papal.
\par
Na evolución da música do continente atopamos dúas tendencias: unha, central, herdeira do \emph{Ars Nova} e outra, periférica, de influencia inglesa. Foi esencial neste sentido a achega de Dunstable, que estaba ao servizo do duque de Bedford e, por tanto, pertencía ás tropas invasoras do continente, e que entrou en contacto con músicos franco-flamencos e borgoñones.\\
A súa música caracterízase por:
%
\begin{itemize}
\item
  \textbf{Ritmos regulares} e \textbf{melodías sinxelas}
\item
  \textbf{Motetes a tres voces} (algún a catro); aínda emprega a isorritmia nalgún deles
\item
  \textbf{Cancións a tres voces} con influencias tanto italianas como da \emph{Chanson}.
\end{itemize}
\end{multicols}
%
% CUESTIÓN:
% ---------
\begin{ejercicio}[Compositores franco-flamencos ou borgoñones]
Dentro da \textbf{escola franco-flamenca}, adóitanse distinguir varias xeracións de compositores. Completa os seguintes apartados:
\begin{multicols}{2}

\begin{itemize}
    \item A \textbf{primeira xeración}, ten como principais representantes a \dotfill \\
    Principais características: 
    \par
    \vspace*{2.5cm}
    \item  A \textbf{segunda xeración} está representada por \dotfill \\
    Principais características: 
    \par
    \vspace*{2.5cm}
    \item A \textbf{terceira xeración} está representada sobre todo por \dotfill \\ 
    Principais características:
    \par
    \vspace*{2.5cm}
    \item  Na \textbf{cuarta xeración}, os compositores máis destacados son \dotfill \\
    Principais características:
    \par
    \vspace*{2.5cm}
    \item  A \textbf{quinta xeración}, ten como principal representante a \dotfill \\
    Principais características:
    \par
    \vspace*{2cm}    
\end{itemize}
\end{multicols}
\end{ejercicio}