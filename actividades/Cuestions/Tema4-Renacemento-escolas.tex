% A ESCOLA FRANCO-FLAMENCA
% ========================
\subsection*{A escola franco-flamenca}

\begin{multicols}{2}

A guerra dos Cen Anos conduciu a unha diminución da importancia musical de Francia, que fai desviar a hexemonía musical desta nación e de Italia cara a Inglaterra, Borgoña e, sobre todo, Flandes. Estas cortes serán escenario de festas relixiosas e profanas nas que a música ocupa un
posto relevante. Nelas fúndanse capelas musicais principescas a imitación da papal.
\par
Na evolución da música do continente atopamos dúas tendencias: unha, central, herdeira do \emph{Ars Nova} e outra, periférica, de influencia inglesa. Foi esencial neste sentido a achega de Dunstable, que estaba ao servizo do duque de Bedford e, por tanto, pertencía ás tropas invasoras do continente, e que entrou en contacto con músicos franco-flamencos e borgoñones.\\
A súa música caracterízase por:
%
\begin{itemize}
\item
  \textbf{Ritmos regulares} e \textbf{melodías sinxelas}
\item
  \textbf{Motetes a tres voces} (algún a catro); aínda emprega a isorritmia nalgún deles
\item
  \textbf{Cancións a tres voces} con influencias tanto italianas como da \emph{Chanson}.
\end{itemize}
\end{multicols}
%
% CUESTIÓN: Compositores franco-flamencos
% ---------------------------------------
\begin{ejercicio}[Compositores franco-flamencos ou borgoñones]
Dentro da \textbf{escola franco-flamenca}, adóitanse distinguir varias xeracións de compositores. Completa os seguintes apartados:
\begin{multicols}{2}

\begin{itemize}
    \item A \textbf{primeira xeración}, ten como principais representantes a \dotfill \\
    Principais características: 
    \par
    \vspace*{2.5cm}
    \item  A \textbf{segunda xeración} está representada por \dotfill \\
    Principais características: 
    \par
    \vspace*{2.5cm}
    \item A \textbf{terceira xeración} está representada sobre todo por \dotfill \\ 
    Principais características:
    \par
    \vspace*{2.5cm}
    \item  Na \textbf{cuarta xeración}, os compositores máis destacados son \dotfill \\
    Principais características:
    \par
    \vspace*{2.5cm}
    \item  A \textbf{quinta xeración}, ten como principal representante a \dotfill \\
    Principais características:
    \par
    \vspace*{2cm}    
\end{itemize}
\end{multicols}
\end{ejercicio}

% CUESTIÓN: Resumo renacemento en Italia
% ---------------------------------------
\begin{ejercicio}[Renacemento en Italia]
Resume as principais características que consideras son representativas do <<Renacemento italiano>>.\\
Indica as principais formas de música profana que coñeces.
\par
\vspace{9cm}
\end{ejercicio}

% CUESTIÓN: resumo reacemento en España
% -------------------------------------
\begin{ejercicio}[Renacemento en España]
Resume as principais características do <<Renacemento español>>.\\
Indica as principais formas e músicos representativos desta época.
\par
\vspace{9cm}
\end{ejercicio}

% CUESTIÓN: renacemento en Francia
% --------------------------------
\begin{ejercicio}[Renacemento en Francia]
Resume as principais características que consideras son representativas do <<Renacemento francés>>.\\
Indica as principais formas de música profana que coñeces.
\par
\vspace{5cm}
\end{ejercicio}

% CUESTIÓN: renacemento en Inglaterra
% -----------------------------------
\begin{ejercicio}[Renacemento en Inglaterra]
Resume as principais características do <<Renacemento inglés>>.\\
Indica as principais formas e músicos representativos desta época.
\par
\vspace{5cm}
\end{ejercicio}

% CUESTIÓN: renacemento en Alemaña
% --------------------------------
\begin{ejercicio}[Renacemento en Alemaña]
Resume as principais características do <<Renacemento alemán>>.\\
Indica as principais formas e músicos representativos desta época.
\par
\vspace{5cm}
\end{ejercicio}

% A ESCOLA ALEMÁ
% ==============

\subsection*{A música en Alemaña \emph{vs} a Escola Romana}

A música na alemaña da época renacentista, ven determinada por un feito dunha grande repercusión dentro do seo da igrexa cristiá; a fixación das \href{http://es.wikipedia.org/wiki/Las_95_tesis}{95 teses de Wittemberg} de Lutero, publicadas un 31 de outubro de 1517. 

\begin{figure}[h] %this figure will be at the right
    \caption{Esto é unha imaxe de exemplo en Latex}
    \centering
    \includegraphics[width=0.25\textwidth]{..}
\end{figure}


\begin{multicols}{2}
    
Con este feito, iníciase o proceso que habería de desembocar no cisma protestante. Será a partir do 1521, cando o \emph{luteranismo} se difunde por toda Europa central. Dentro do eido musical, a repercusión deste feito está determinada pola importancia que terá a \emph{coral} dentro da música sacra no seo da Igrexa.

A coral é en orixe un canto relixioso monódico, en alemán, con ritmo sinxelo, melodía de ámbito curto, por graos conxuntos e estilo silábico. A súa sinxeleza, e o feito de que moitas destas melodías fosen populares, garantiron unha ampla difusión deste estilo vocal. A execución da coral ía desde os máis sinxelos cánticos monódicos (cantados ao unísono) até as grandes presentacións de \emph{corais} harmonizadas, onde á melodía cantada pola comunidade, súmase agora o acompañamento de órgano --ou instrumentos-- e dun coro especializado.

Ao longo da evolución da \emph{coral}, distinguimos:

    \begin{itemize}
        \item 
        \textbf{Coral de tenor} (como C.F.), propio do século XVI, tanto polifónico como homofónico. De estilo sinxelo, é cadrado, sobre baixo harmónico e con cadencias claras.
        \item
        \textbf{Coral de soprano} (como C.F., é dicir, paráfrasis), a catro partes con acompañamento de órgano; aparece a finais do XVI.
        \item
        \textbf{Motete-coral}, da mesma época, tomando a coral como C.F. en estilo motete (contrapunto imitativo).
    \end{itemize}
\end{multicols}