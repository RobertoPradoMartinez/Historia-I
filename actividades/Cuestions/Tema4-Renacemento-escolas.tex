% A ESCOLA FRANCO-FLAMENCA
% ========================
\subsection*{A escola franco-flamenca}

\begin{multicols}{2}

A guerra dos Cen Anos conduciu a unha diminución da importancia musical de Francia, que fai desviar a hexemonía musical desta nación e de Italia cara a Inglaterra, Borgoña e, sobre todo, Flandes. Estas cortes serán escenario de festas relixiosas e profanas nas que a música ocupa un
posto relevante. Nelas fúndanse capelas musicais principescas a imitación da papal.
\par
Na evolución da música do continente atopamos dúas tendencias: unha, central, herdeira do \emph{Ars Nova} e outra, periférica, de influencia inglesa. Foi esencial neste sentido a achega de Dunstable, que estaba ao servizo do duque de Bedford e, por tanto, pertencía ás tropas invasoras do continente, e que entrou en contacto con músicos franco-flamencos e borgoñones.\\
A súa música caracterízase por:
%
\begin{itemize}
\item
  \textbf{Ritmos regulares} e \textbf{melodías sinxelas}
\item
  \textbf{Motetes a tres voces} (algún a catro); aínda emprega a isorritmia nalgún deles
\item
  \textbf{Cancións a tres voces} con influencias tanto italianas como da \emph{Chanson}.
\end{itemize}
\end{multicols}
%
% CUESTIÓN: Compositores franco-flamencos
% ---------------------------------------
\begin{ejercicio}[Compositores franco-flamencos ou borgoñones]
Dentro da \textbf{escola franco-flamenca}, adóitanse distinguir varias xeracións de compositores. Completa os seguintes apartados:
\begin{multicols}{2}

\begin{itemize}
    \item A \textbf{primeira xeración}, ten como principais representantes a \dotfill \\
    Principais características: 
    \par
    \vspace*{2.5cm}
    \item  A \textbf{segunda xeración} está representada por \dotfill \\
    Principais características: 
    \par
    \vspace*{2.5cm}
    \item A \textbf{terceira xeración} está representada sobre todo por \dotfill \\ 
    Principais características:
    \par
    \vspace*{2.5cm}
    \item  Na \textbf{cuarta xeración}, os compositores máis destacados son \dotfill \\
    Principais características:
    \par
    \vspace*{2.5cm}
    \item  A \textbf{quinta xeración}, ten como principal representante a \dotfill \\
    Principais características:
    \par
    \vspace*{2cm}    
\end{itemize}
\end{multicols}
\end{ejercicio}

% CUESTIÓN: Resumo renacemento en Italia
% ---------------------------------------
\begin{ejercicio}[Renacemento en Italia]
Resume as principais características que consideras son representativas do <<Renacemento italiano>>.\\
Indica as principais formas de música profana que coñeces.
\par
\vspace{9cm}
\end{ejercicio}

% CUESTIÓN: resumo reacemento en España
% -------------------------------------
\begin{ejercicio}[Renacemento en España]
Resume as principais características do <<Renacemento español>>.\\
Indica as principais formas e músicos representativos desta época.
\par
\vspace{9cm}
\end{ejercicio}

% CUESTIÓN: renacemento en Francia
% --------------------------------
\begin{ejercicio}[Renacemento en Francia]
Resume as principais características que consideras son representativas do <<Renacemento francés>>.\\
Indica as principais formas de música profana que coñeces.
\par
\vspace{5cm}
\end{ejercicio}

% CUESTIÓN: renacemento en Inglaterra
% -----------------------------------
\begin{ejercicio}[Renacemento en Inglaterra]
Resume as principais características do <<Renacemento inglés>>.\\
Indica as principais formas e músicos representativos desta época.
\par
\vspace{5cm}
\end{ejercicio}

% CUESTIÓN: renacemento en Alemaña
% --------------------------------
\begin{ejercicio}[Renacemento en Alemaña]
Resume as principais características do <<Renacemento alemán>>.\\
Indica as principais formas e músicos representativos desta época.
\par
\vspace{5cm}
\end{ejercicio}

% A ESCOLA ALEMÁ
% ==============

\subsection*{Lutero e a reforma}

A música na alemaña da época renacentista, ven determinada por un feito dunha grande repercusión dentro do seo da igrexa cristiá; a fixación das \href{http://es.wikipedia.org/wiki/Las_95_tesis}{95 teses de Wittemberg} de \textbf{Martiño Lutero}, publicadas un 31 de outubro de 1517. 
Con este feito, iníciase o proceso que habería de desembocar no \textbf{cisma protestante}. Será a partir do 1521, cando o \emph{luteranismo} se difunde por toda Europa central. Dentro do eido musical, a repercusión deste feito está determinada pola importancia que terá a \emph{coral} dentro da música sacra no seo da Igrexa.

\subsubsection*{A \emph{coral} alemá}

\begin{multicols}{2}

A coral é en orixe un canto relixioso monódico, en alemán, con ritmo sinxelo, melodía de ámbito curto, por graos conxuntos e estilo silábico. A súa sinxeleza, e o feito de que moitas destas melodías fosen populares, garantiron unha ampla difusión deste estilo vocal. A execución da coral ía desde os máis sinxelos cánticos monódicos (cantados ao unísono) até as grandes presentacións de \emph{corais} harmonizadas, onde á melodía cantada pola comunidade, súmase agora o acompañamento de órgano --ou instrumentos-- e dun coro especializado.
Ao longo da evolución da \emph{coral}, distinguimos:

    \begin{itemize}
        \item 
        \textbf{Coral de tenor} (como C.F.), propio do século XVI, tanto polifónico como homofónico. De estilo sinxelo, é cadrado, sobre baixo harmónico e con cadencias claras.
        \item
        \textbf{Coral de soprano} (como C.F., é dicir, paráfrasis), a catro partes con acompañamento de órgano; aparece a finais do XVI.
        \item
        \textbf{Motete-coral}, da mesma época, tomando a coral como C.F. en estilo motete (contrapunto imitativo).
    \end{itemize}
\end{multicols}

\subsection*{A Contrarreforma}

A resposta da igrexa católica ás reformas protestantes foi a convocatoria do \href{http://es.wikipedia.org/wiki/Concilio_de_Trento}{Concilio de Trento}, que se celebrou na cidade do mesmo nome entre os anos 1545-1563. Será nos dous últimos anos da súa celebración, cando se abordan as cuestións musicais e se debate sobre a música que se empregaría na liturxia.

\subsubsection*{A Escola Romana}

\begin{multicols}{2}

As sesións do Concilio de Trento puxeron de manifesto a necesidade dunha reforma en profundidade da práctica musical no seo da liturxia. Os procedementos das misas de parodia, os excesos dos instrumentos, o artificio das voces (\ldots) chegaban a facer incomprensible o texto sacro, polo que conduciron a un afastamento perigoso para o dogma e o adoutrinamento de fieis que se procuraba na liturxia. Estas orientacións de estilo deron lugar ao que se coñece como Escola romana.

Por \textbf{Escola Romana} enténdese ao grupo de compositores que actuaron durante o século XVI na Capela Papal en Roma, onde o máximo representante é \textbf{G.P. da Palestrina}. A súa música caracterízase por: unha melodía sinxela e \emph{cantabile}, ritmo fluído e tranquilo, texturas imitativas pero non excesivamente densas, harmonías triádicas, textos comprensibles con preferencia ao emprego de tenores gregorianos como C.F. (aínda que emprega tamén paráfrasis, parodia e composición libre). O seu estilo é sobrio austero e equilibrado. A súa música foi considerada a expresión máis perfecta do estilo eclesiástico. A súa arte resume todo o século que lle precede, englobando todas as técnicas da composición polifónica e os seus xéneros.

A obra de Palestrina é case integramente relixiosa, aínda que tamén se  atopan \emph{madrigales} sacros e profanos. Foi admirado en toda Europa, aínda que a súa calidade non reside tanto na novidade dos seus métodos como na intelixencia coa que os utilizou.

\subsubsection*{A Escola Veneciana}

Si a Escola Romana significou a perfección total, dentro dunha evolución, máis que innovación, a Escola Veneciana presenta características singulares que a diferencian claramente do resto das escolas polifónicas.

Aparece por primeira vez música para dúas ou máis coros de voces \href{http://es.wikipedia.org/wiki/Estilo_policoral_veneciano}{\emph{coro spezzato}},  creando con iso uns efectos antifonales de coros que dialogan. O escenario no que se desenvolvía a música desta escola, era a catedral de estilo bizantino de San Marcos, cuberta de cúpulas e de magníficos mosaicos, que proporcionaba un marco incomparable para a experimentación da \textbf{técnica policoral}; a estes coros engadíase en moitas ocasións instrumentos para logar maior forza. O efecto policoral, tal e como se practicou en Venecia en tempos dos \href{http://open.spotify.com/track/6r9zLucRqaSSyV4XS4t7HL}{Gabrieli}, está na base da técnica do \emph{concertato}, que se consolida como o ideal estético do Barroco a partir dos inicios do século XVII. O aumento no número de voces e o uso de dous coros (técnica do \emph{doble coro}) proporcionoulle gran brillo e esplendor.

Todo isto é a representación do ambiente festivo-relixioso, cívico e social que se desenvolve ao redor da Basílica de San Marcos e a nobreza veneciana. O promotor da nova escola veneciana é o flamenco \href{http://es.wikipedia.org/wiki/Adrian_Willaert}{Willaert}, mestre de capela na citada basílica de San Marcos. A maiores do novo estilo do dobre coro, os seus progresos foron considerables no campo da música instrumental. Outros autores, á parte dos Gabrieli, foron \href{http://es.wikipedia.org/wiki/Gioseffo_Zarlino}{Zarlino} e Cipriano de Rore. 

\end{multicols}

\vspace{0.25cm}

% CUESTIÓN: Coral alemá
% ---------------------
\begin{ejercicio}[A música da Reforma - a \emph{coral}]
    \begin{multicols}{2}
     \begin{enumerate}
        \item 
        Que supón a <<reforma luterana>> no contexto musical alemán?.
        \item 
        Que entendes por \emph{coral}? \ldots 
        \item 
        Que tipos de \emph{coral} coñeces? \ldots Indica en que consisten.
     \end{enumerate}
    \end{multicols}
     \hrulefill
\par
\vspace{7cm}
\end{ejercicio}
