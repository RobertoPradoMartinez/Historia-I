%
% CUESTIÓNS SOBRE AUDICIÓNS PARA INCLUÍR EN PRESENTACIÓN DE PROBAS
% ================================================================
% Relación de textos sobre audicións comentadas a incluír na proba B
% 
% Para incluír na folla de exercicios:
% \useproblem{Cuestion-N} % (N = referencia)
%
% TODO: PENDIENTE DE REDACTAR INTROS DAS AUDICIÓNS
%
% -------------------------------------------------------------------------------
% Cuestión-AU501:  CONTEXTO NUPER ROSARUM FLORES - GUILLAUME DUFAY
% Mov.:
% -------------------------------------------------------------------------------
\begin{defproblem}{Texto-AU501}
\begin{quotation}
\small{
O autor da obra, está considerado a figura central da Escola borgoñona; é o
máis famoso e influente compositor da escena musical europea de mediados do
século.

}
%\vspace*{0.5cm}
\end{quotation}

% TODO:
% --------
% Solución:
% --------
\begin{onlysolution}
    \begin{solution}
Autor: o que sexa\\
Obra: a que sexa\\
    \end{solution}
\end{onlysolution}

\end{defproblem}
%
%
%
%-------------------------------------------------------------------------------
% Cuestión-AU502:  CONTEXTO NUPER ROSARUM FLORES - GUILLAUME DUFAY
% Mov.:
%
%-------------------------------------------------------------------------------
\begin{defproblem}{Texto-AU502}
\begin{quotation}
\small{
Estamos ante unha obra que pertende a unha forma de grande éxito na súa época.
Esta música, e en concreto esta pequena (\ldots) satírica, foi obxecto de
moitas interpretacións xeralmente referidas á proverbial tacañería do mecenas
Galeazzo Sforza á hora de pagarlles o soldo aos seus músicos.
%\footnote{(Wikipedia, 2023).}
}
%\vspace*{0.5cm}
\end{quotation}

% TODO:
% --------
% Solución:
% --------
\begin{onlysolution}
    \begin{solution}
Autor: o que sexa\\
Obra: a que sexa\\
    \end{solution}
\end{onlysolution}

\end{defproblem}
%
%
