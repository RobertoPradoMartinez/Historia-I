%
% AUDICIÓNS DO TEMA 5
% ====================
% Creamos as fichas de audición
% Cargamos as audicións no exame
% Utiliza:
%\useproblem{Audicion-N} % (N = no.audición)
%
% ===================================================
% AUDICIÓN:
% ===================================================
% Ficha de audición con texto ao inicio:
% --------------------------------------
% Dependento da Cuestión asignamos un autor ou outro
%
% CUESTIÓNS SOBRE AUDICIÓNS PARA INCLUÍR EN PRESENTACIÓN DE PROBAS
% ================================================================
% Relación de textos sobre audicións comentadas a incluír na proba B
% 
% Para incluír na folla de exercicios:
% \useproblem{Cuestion-N} % (N = referencia)
%
% TODO: PENDIENTE DE REDACTAR INTROS DAS AUDICIÓNS
%
% -------------------------------------------------------------------------------
% Cuestión-AU501:  CONTEXTO NUPER ROSARUM FLORES - GUILLAUME DUFAY
% Mov.:
% -------------------------------------------------------------------------------
\begin{defproblem}{Texto-AU501}
\begin{quotation}
\small{
O autor da obra, está considerado a figura central da Escola borgoñona; é o
máis famoso e influente compositor da escena musical europea de mediados do
século.

}
%\vspace*{0.5cm}
\end{quotation}

% TODO:
% --------
% Solución:
% --------
\begin{onlysolution}
    \begin{solution}
Autor: o que sexa\\
Obra: a que sexa\\
    \end{solution}
\end{onlysolution}

\end{defproblem}
%
%
%
%-------------------------------------------------------------------------------
% Cuestión-AU502:  CONTEXTO NUPER ROSARUM FLORES - GUILLAUME DUFAY
% Mov.:
%
%-------------------------------------------------------------------------------
\begin{defproblem}{Texto-AU502}
\begin{quotation}
\small{
Estamos ante unha obra que pertende a unha forma de grande éxito na súa época.
Esta música, e en concreto esta pequena (\ldots) satírica, foi obxecto de
moitas interpretacións xeralmente referidas á proverbial tacañería do mecenas
Galeazzo Sforza á hora de pagarlles o soldo aos seus músicos.
%\footnote{(Wikipedia, 2023).}
}
%\vspace*{0.5cm}
\end{quotation}

% TODO:
% --------
% Solución:
% --------
\begin{onlysolution}
    \begin{solution}
Autor: o que sexa\\
Obra: a que sexa\\
    \end{solution}
\end{onlysolution}

\end{defproblem}
%
%


\begin{defproblem}{Audicion-501}

\begin{ejercicio}[]
%
% Texto de presentación antes de ficha:
% Aquí seleccionamos a audición que queremos que entre:
% ---
    \useproblem{Texto-AU501}
% ---
%
\begin{enumerate}[1.-]
	\begin{multicols}{2}
% Pregunta:
% 
%		\vspace*{0.1cm}
		\item
			AUTOR: \dotfill 
		\item 
		    OBRA: 
		    \begin{enumerate}[a)]
		    \item
			\textbf{Título}: \dotfill
		    \item
			\textbf{Timbre}: \par
				\small{Segundo as características da obra e audición, diferenciamos: (indica a correcta)}
			\begin{itemize}
				 \item
				 \small{Voz, corda, vento e percusión}
				 \item
				 \small{Voz, corda e percusión}
				 \item
				 \small{Voz, corda e vento}
				 \item
				 \small{Voz e corda}
				 \item
				 \small{Voces}
			\end{itemize}

			\normalsize
%			\vspace*{0.2cm}	
		    \item 
		    \textbf{Textura}: \\
		    \small{Sinala a textura da obra segundo corresponda.} \par
			\begin{itemize}
                \item
			     Melódica horizontal - polifónica
                 \item
			     Melódica horizontal - monódica
			     \item
			     Melódica vertical - homofónica
			     \item
				Non melódica
			\end{itemize}
            \normalsize
            \item
			\textbf{Forma}: \\
				\small{Indica a forma á que se axusta a obra:} \par
			\begin{itemize}
			     \item
			     Virelai
			    \item
			     Estampida
			     \item
			     Motete isorrítmico
			     \item
			     Frottola
			     \item
			     Madrigal
			     \item
			     Misa: \emph{Kyrie}
			     \item
			     Villancico pastoril
			\end{itemize}
            \normalsize
%			\vspace*{0.2cm}			
		    \item
			\textbf{Estilo}: \\
				\small{Indica o estilo da obra:} \par
			\begin{itemize}
                \item
			     Contrapuntístico: polifonía medieval
                 \item
			     Trobadoresco: monodia medieval
			     \item
			     Polifonía relixiosa renacentista
			     \item
			     Polifonía profana renacentista
			\end{itemize}
            \normalsize
		    \item
			\textbf{Xénero}: \\
				\small{
				Estamos ante:
				%\ldots
				}
			\begin{itemize}
                \item
			     Música relixiosa (sacra)
                 \item
			     Música vocal relixiosa (música sacra)
			     \item
			     Música profana
			     \item
			     Canción profana (popular)
			\end{itemize}
		    \end{enumerate}
    \end{multicols}
%\par
%\vspace{0.3cm}
    \item
    COMENTARIO:\\
    {\small Redacta coas túas palabras os datos da ficha de audición a modo de presentación da obra.}
    \par
    \vspace{2.80cm}
	\end{enumerate}
\end{ejercicio}

% Solucións da Audición 1:
% ------------------------
\begin{onlysolution}
    \begin{solution}
Autor: o que sexa\\
Obra: a que sexa\\
    \end{solution}
\end{onlysolution}

\end{defproblem}
% 
% Fin da ficha de audición
%
%
% ===================================================
% AUDICIÓN:
% ===================================================
% Ficha de audición con texto ao inicio:
% --------------------------------------
% Dependento da Cuestión asignamos un autor ou outro
%
% CUESTIÓNS SOBRE AUDICIÓNS PARA INCLUÍR EN PRESENTACIÓN DE PROBAS
% ================================================================
% Relación de textos sobre audicións comentadas a incluír na proba B
% 
% Para incluír na folla de exercicios:
% \useproblem{Cuestion-N} % (N = referencia)
%
% TODO: PENDIENTE DE REDACTAR INTROS DAS AUDICIÓNS
%
% -------------------------------------------------------------------------------
% Cuestión-AU501:  CONTEXTO NUPER ROSARUM FLORES - GUILLAUME DUFAY
% Mov.:
% -------------------------------------------------------------------------------
\begin{defproblem}{Texto-AU501}
\begin{quotation}
\small{
O autor da obra, está considerado a figura central da Escola borgoñona; é o
máis famoso e influente compositor da escena musical europea de mediados do
século.

}
%\vspace*{0.5cm}
\end{quotation}

% TODO:
% --------
% Solución:
% --------
\begin{onlysolution}
    \begin{solution}
Autor: o que sexa\\
Obra: a que sexa\\
    \end{solution}
\end{onlysolution}

\end{defproblem}
%
%
%
%-------------------------------------------------------------------------------
% Cuestión-AU502:  CONTEXTO NUPER ROSARUM FLORES - GUILLAUME DUFAY
% Mov.:
%
%-------------------------------------------------------------------------------
\begin{defproblem}{Texto-AU502}
\begin{quotation}
\small{
Estamos ante unha obra que pertende a unha forma de grande éxito na súa época.
Esta música, e en concreto esta pequena (\ldots) satírica, foi obxecto de
moitas interpretacións xeralmente referidas á proverbial tacañería do mecenas
Galeazzo Sforza á hora de pagarlles o soldo aos seus músicos.
%\footnote{(Wikipedia, 2023).}
}
%\vspace*{0.5cm}
\end{quotation}

% TODO:
% --------
% Solución:
% --------
\begin{onlysolution}
    \begin{solution}
Autor: o que sexa\\
Obra: a que sexa\\
    \end{solution}
\end{onlysolution}

\end{defproblem}
%
%


\begin{defproblem}{Audicion-502}

\begin{ejercicio}[]
%
% Texto de presentación antes de ficha:
% Aquí seleccionamos a audición que queremos que entre:
% ---
   \useproblem{Texto-AU502}
% ---
%
\begin{enumerate}[1.-]
	\begin{multicols}{2}
% Pregunta:
%
%		\vspace*{0.1cm}
		\item
			AUTOR: \dotfill
		\item
		    OBRA:
		    \begin{enumerate}[a)]
		    \item
			\textbf{Título}: \dotfill
		    \item
			\textbf{Timbre}: \par
				\small{Segundo as características da obra e audición, diferenciamos: (indica a correcta)}
			\begin{itemize}
				 \item
				 \small{Voz, corda, vento e percusión}
				 \item
				 \small{Voz, corda e percusión}
				 \item
				 \small{Voz, corda e vento}
				 \item
				 \small{Voz e corda}
				 \item
				 \small{Voces}
			\end{itemize}

			\normalsize
%			\vspace*{0.2cm}
		    \item
		    \textbf{Textura}: \\
		    \small{Sinala a textura da obra segundo corresponda.} \par
			\begin{itemize}
                \item
			     Melódica horizontal - polifónica
                 \item
			     Melódica horizontal - monódica
			     \item
			     Melódica vertical - homofónica
			     \item
				Non melódica
			\end{itemize}
            \normalsize
            \item
			\textbf{Forma}: \\
				\small{Indica a forma á que se axusta a obra:} \par
			\begin{itemize}
			     \item
			     Virelai
			    \item
			     Estampida
			     \item
			     Motete isorrítmico
			     \item
			     Frottola
			     \item
			     Madrigal
			     \item
			     Misa: \emph{Kyrie}
			     \item
			     Villancico pastoril
			\end{itemize}
            \normalsize
%			\vspace*{0.2cm}
		    \item
			\textbf{Estilo}: \\
				\small{Indica o estilo da obra:} \par
			\begin{itemize}
                \item
			     Contrapuntístico: polifonía medieval
                 \item
			     Trobadoresco: monodia medieval
			     \item
			     Polifonía relixiosa renacentista
			     \item
			     Polifonía profana renacentista
			\end{itemize}
            \normalsize
		    \item
			\textbf{Xénero}: \\
				\small{
				Estamos ante:
				%\ldots
				}
			\begin{itemize}
                \item
			     Música relixiosa (sacra)
                 \item
			     Música vocal relixiosa (música sacra)
			     \item
			     Música profana
			     \item
			     Canción profana (popular)
			\end{itemize}
		    \end{enumerate}
    \end{multicols}
%\par
%\vspace{0.3cm}
    \item
    COMENTARIO:\\
    {\small Redacta coas túas palabras os datos da ficha de audición a modo de presentación da obra.}
    \par
    \vspace*{5.0cm}
	\end{enumerate}
\end{ejercicio}

% Solucións da Audición 1:
% ------------------------
\begin{onlysolution}
    \begin{solution}
Autor: o que sexa\\
Obra: a que sexa\\
    \end{solution}
\end{onlysolution}

\end{defproblem}
%
% Fin da ficha de audición
