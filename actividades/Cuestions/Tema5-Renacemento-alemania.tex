% -------------------------------
% BANCO DE PREGUNTAS DE HISTORIA
% -------------------------------
%
% Tema 5.- RENACEMENTO
% 
% CUESTIÓN: RENACEMENTO - ALEMAÑA
% -------------------------------
%
\newproblem{T5RENA-DE-01}{
Podemos afirmar que no século XVI, o xénero da canción en Alemaña ven determinado \ldots
    \begin{enumerate}[a)]
    \item 
    Pola canción polifónica de c.f. de tipo tradicional con variantes
    \item 
    Canción eclesiástica protestante influenciada por Martiño Lutero
    \item
    Pola canción monódica de c.f. de tipo tradicional con variantes
    \item \label{T5RENA-DE-01:sol} 
    Tanto por a) como por b)
    \end{enumerate}
}
    { % Solución:
    (\ref{T5RENA-DE-01:sol}) {\color{orange}{\hrulefill}} \\
    \small{ % Comentario:
    Comentario solución: 
    {\color{orange}{\hrulefill}}
    }
    }
%
%
% CUESTIÓN: RENACEMENTO - ALEMAÑA
% -------------------------------
%
\newproblem{T5RENA-DE-02}{
Podemos afirmar que a \emph{coral} alemá é, na súa orixe \ldots
    \begin{enumerate}[a)]
    \item 
    un canto profano, monódico, de ritmo sinxelo en estilo neumático, cantado en alemán
    \item 
    un canto sacro, polifónico, de ritmo sinxelo en estilo silábico, cantado en alemán
    \item \label{T5RENA-DE-02:sol}
    un canto sacro, monódico, de ritmo sinxelo en estilo silábico, cantado en alemán
    \item 
    un canto sacro, monódico, de ritmo sinxelo en estilo neumático, cantado en alemán
    \end{enumerate}
}
    { % Solución:
    (\ref{T5RENA-DE-02:sol}) {\color{orange}{\hrulefill}} \\
    \small{ % Comentario:
    Comentario solución: 
    {\color{orange}{\hrulefill}}
    }
    }
%
