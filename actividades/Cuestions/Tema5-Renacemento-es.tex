% -------------------------------
% BANCO DE PREGUNTAS DE HISTORIA
% -------------------------------
%
% Tema 5.- RENACEMENTO
% 
% CUESTIÓN: RENACEMENTO - ESPAÑA
% ------------------------------
%
\newproblem{T5RENA-ES-01}{
A música profana española, exprésase a través de tres formas principais:
    \begin{enumerate}[a)]
    \item 
    Romance, frottola e villancico
    \item \label{T5RENA-ES-01:sol}
    Romance, villancico e ensalada    
    \item
    Villancico, frottola e ensalada
    \item 
    Villancico, villanella e ensalada
    \end{enumerate}
}
    { % Solución:
    (\ref{T5RENA-ES-01:sol}) {\color{orange}{\hrulefill}} \\
    \small{ % Comentario:
    Comentario solución: 
    {\color{orange}{\hrulefill}}
    }
    }
%
% CUESTIÓN: RENACEMENTO - ESPAÑA
% ------------------------------
%
\newproblem{T5RENA-ES-02}{
O villancico (ou panxoliña) é considerada \ldots
    \begin{enumerate}[a)]
    \item 
    Unha forma de música sacra de orixe e temática popular
    \item \label{T5RENA-ES-02:sol}
    Unha forma de música profana de orixe e temática popular
    \item 
    Unha forma de música profana de orixe sacra e temática popular
    \item 
    Unha forma de música sacra de orixe popular e temática sacra
    \end{enumerate}
}
    { % Solución:
    (\ref{T5RENA-ES-02:sol}) {\color{orange}{\hrulefill}} \\
    \small{ % Comentario:
    Comentario solución: 
    {\color{orange}{\hrulefill}}
    }
    }
%
% CUESTIÓN: RENACEMENTO - ESPAÑA
% ------------------------------
%
\newproblem{T5RENA-ES-03}{
Cal das seguintes obras  que consta de tres partes (\emph{estribillo} - copla - \emph{estribillo}), se axusta a unha forma de música profana de temática popular do renacemento español?
    \begin{enumerate}[a)]
    \item 
    \emph{Stella Splendens}
    \item 
    \emph{Kalenda Maia}
    \item \label{T5RENA-ES-03:sol}
    \emph{Mas vale trocar}
    \item 
    \emph{El Grillo}
    \end{enumerate}
}
    { % Solución:
    (\ref{T5RENA-ES-03:sol}) {\color{orange}{\hrulefill}} \\
    \small{ % Comentario:
    Comentario solución: 
    {\color{orange}{\hrulefill}}
    }
    }
%
