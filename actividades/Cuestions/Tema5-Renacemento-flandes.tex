% -------------------------------
% BANCO DE PREGUNTAS DE HISTORIA
% -------------------------------
%
% Tema 5.- RENACEMENTO
% 
% CUESTIÓN: RENACEMENTO - FLANDES
% ---------------------------------------------
%
\newproblem{T5RENA-FLANDES-01}{
Clemens von Papa e Adrian Willaert son dous grandes músicos representativos da \ldots
    \begin{enumerate}[a)]
    \item 
    escola franco-flamenca
    \item \label{T5RENA-FLANDES-01:sol}
    escola veneciana
    \item 
    escola española
    \item
    escola romana
    \end{enumerate}
}
    { % Solución:
    (\ref{T5RENA-FLANDES-01:sol}) {\color{orange}{\hrulefill}} \\
    \small{ % Comentario:
    Comentario solución: 
    {\color{orange}{\hrulefill}}
    }
    }
%

% CUESTIÓN: RENACEMENTO - FLANDES
% ---------------------------------------------
%
\newproblem{T5RENA-FLANDES-02}{
O motete isorrítmico \emph{Nuper rosarum flores} foi creado para a consagración da catedral de florencia cando Filippo Brunelleschi rematou a súa impresionante cúpula. A que escola pertence o seu autor?
    \begin{enumerate}[a)]
    \item 
    Escola veneciana
    \item 
    Escola romana
    \item \label{T5RENA-FLANDES-02:sol}
    Escola franco-flamenca
    \item
    Escola alemana
    \end{enumerate}
}
    { % Solución:
    (\ref{T5RENA-FLANDES-02:sol}) {\color{orange}{\hrulefill}} \\
    \small{ % Comentario:
    Comentario solución: 
    {\color{orange}{\hrulefill}}
    }
    }
%
% CUESTIÓN: RENACEMENTO - FLANDES
% ---------------------------------------------
%
\newproblem{T5RENA-FLANDES-03}{
Cal dos seguintes músicos e compositores flamenco --mestre de capela, na basílica de San Marcos-- é considerado o promotor da nova <<escola veneciana>>?
    \begin{enumerate}[a)]
    \item 
    Clemens von Papa 
    \item 
    Guilles Binchois
    \item \label{T5RENA-FLANDES-03:sol}
    Adrian Willaert
    \item
    Giovanni Pierluigi da Palestrina
    \end{enumerate}
}
    { % Solución:
    (\ref{T5RENA-FLANDES-03:sol}) {\color{orange}{\hrulefill}} \\
    \small{ % Comentario:
    Comentario solución: 
    {\color{orange}{\hrulefill}}
    }
    }
%
