% -------------------------------
% BANCO DE PREGUNTAS DE HISTORIA
% -------------------------------
%
% Tema 5.- RENACEMENTO
% 
% CUESTIÓN: RENACEMENTO - FRANCIA
% -------------------------------
%
\newproblem{T5RENA-FRA-01}{
Que xénero, considerado o principal entre compositores franco-flamencos a comezos do XVI en Italia (e considerado o principal en Francia), incorpora influencias do madrigal con expresivas interpretacións de textos e cromatismos, no curso do século XVI?  
    \begin{enumerate}[a)]
    \item \label{T5RENA-FRA-01:sol}
    A \emph{Chansón} francesa
    \item 
    O \emph{romance} español
    \item
    A \emph{villanella} italiana
    \item  
    A \emph{coral} alemana
    \end{enumerate}
}
    { % Solución:
    (\ref{T5RENA-FRA-01:sol}) {\color{orange}{\hrulefill}} \\
    \small{ % Comentario:
    Comentario solución: 
    {\color{orange}{\hrulefill}}
    }
    }
%

% CUESTIÓN: RENACEMENTO - FRANCIA
% -------------------------------
%
\newproblem{T5RENA-FRA-02}{
Cal dos seguintes compositores do século XVI, consideras forma parte da do renacemento francés?
    \begin{enumerate}[a)]
    \item \label{T5RENA-FRA-02:sol}
    Jannequin
    \item 
    Fletxa
    \item 
    Busnois
    \item 
    Binchois
    \end{enumerate}
}
    { % Solución:
    (\ref{T5RENA-FRA-02:sol}) {\color{orange}{\hrulefill}} \\
    \small{ % Comentario:
    Comentario solución: 
    {\color{orange}{\hrulefill}}
    }
    }
%
%
