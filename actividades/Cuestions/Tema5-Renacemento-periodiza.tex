% -------------------------------
% BANCO DE PREGUNTAS DE HISTORIA
% -------------------------------
%
% Tema 5.- RENACEMENTO
% 
% CUESTIÓN: RENACEMENTO - PERIODIZACIÓN
% -------------------------------------
%
\newproblem{T5RENA-PER-01}{
Consultada a Wikipedia, atopamos a seguinte definción:
\begin{quote}
 La música del Renacimiento o música renacentista es la música escrita durante los siglos [\ldots]. Las características estilísticas de la música renacentista son su textura polifónica, que sigue las leyes del contrapunto, y está regida por el sistema modal heredado del canto gregoriano.
\end{quote}
\begin{flushleft}
 \emph{Música del Renacimiento} - Wikipedia (Maio, 2022)
\end{flushleft}
A qué séculos fai referencia?
    \begin{enumerate}[a)]
    \item 
    XII e XIII
    \item
    XIII e XIV
    \item \label{T5RENA-PER-01:sol}
    XV e XVI
    \item
    XVI e XVII
    \end{enumerate}
}
    { % Solución:
    (\ref{T5RENA-PER-01:sol}) {\color{orange}{\hrulefill}} \\
    \small{ % Comentario:
    Comentario solución: 
    {\color{orange}{\hrulefill}}
    }
    }
%
% CUESTIÓN: RENACEMENTO - PERIODIZACIÓN
% -------------------------------------
%
\newproblem{T5RENA-PER-02}{
Unha das principais características deste período será o desprazamento do centro de creación; inicialmente en Francia desprazarase cara Italia, que asumirá a dirección --pasando polo ámbito franco-flamenco-- debido ao constante intercambio cultural e musical dos compositores da época, motivado polas frecuente viaxes a Italia.\\
A que período nos estamos a referir?
    \begin{enumerate}[a)]
    \item \label{T5RENA-PER-02:sol}
    Renacemento
    \item 
    \emph{Ars antiqva}
    \item
    \emph{Ars nova}
    \item
    Barroco
    \end{enumerate}
}
    { % Solución:
    (\ref{T5RENA-PER-02:sol}) {\color{orange}{\hrulefill}} \\
    \small{ % Comentario:
    Comentario solución: 
    {\color{orange}{\hrulefill}}
    }
    }
%
% CUESTIÓN: RENACEMENTO - PERIODIZACIÓN
% -------------------------------------
%
\newproblem{T5RENA-PER-03}{
Que famoso artista polifacético, coñecido sobre todo por ser o primeiro historiador da arte moderna é considerado como o pai do término <<Renacemento>> (c.a 1550), empregado habitualmente para referirse á arte da época en Italia?
    \begin{enumerate}[a)]
    \item 
    Johannes Gutemberg (c.a 1400 - 1468)
    \item \label{T5RENA-PER-03:sol}
    Giorgio Vasari (1511 - 1574)
    \item
    Vincenzo Galilei (1520 - 1591) 
    \item
    Giovanni P. da Palestrina (1525 - 1594)
    \end{enumerate}
}
    { % Solución:
    (\ref{T5RENA-PER-03:sol}) {\color{orange}{\hrulefill}} \\
    \small{ % Comentario:
    Comentario solución: 
    {\color{orange}{\hrulefill}}
    }
    }
%
