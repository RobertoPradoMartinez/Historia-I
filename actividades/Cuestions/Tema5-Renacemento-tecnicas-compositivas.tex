% -------------------------------
% BANCO DE PREGUNTAS DE HISTORIA
% -------------------------------
%
% Tema 5.- RENACEMENTO
% 
% CUESTIÓN: RENACEMENTO - TÉCNICAS COMPOSITIVAS
% ---------------------------------------------
%
\newproblem{T5RENA-TECNICAS-01}{
Dentro das principais técnicas de composición da música renacentista, atopamos a técnica da variación, que consiste en \ldots
    \begin{enumerate}[a)]
    \item 
    Expoñer unha breve melodía nunha voz que repetirá inmediatamente outra voz, ben ao unísono ou ben a distancia de cuarta ou quinta
    \item \label{T5RENA-TEC-01:sol}
    Expoñer un tema e repetilo, variando a melodía, a rítmica ou harmonía, etc. 
    \item 
    Expoñer unha breve melodía nunha voz que variará inmediatamente outra voz, a distancia de cuarta ou quinta
    \item
    Expoñer unha melodía (c.f.) nunha voz, mentres as outras tecen o entramado polifónico ao seu redor
    \end{enumerate}
}
    { % Solución:
    (\ref{T5RENA-TEC-01:sol}) {\color{orange}{\hrulefill}} \\
    \small{ % Comentario:
    Comentario solución: 
    {\color{orange}{\hrulefill}}
    }
    }
%
%
% CUESTIÓN: RENACEMENTO - TÉCNICAS COMPOSITIVAS
% ---------------------------------------------
%
\newproblem{T5RENA-TECNICAS-02}{
En que consiste a técnica de composición por \emph{cantus firmus} (c.f.)?
    \begin{enumerate}[a)]
    \item 
    En expoñer nunha voz unha breve melodía que repite outra voz, inmediatamente despois, normalmente a distancia de cuarta ou quinta
    \item 
    Expoñer unha breve melodía nunha voz que variará inmediatamente outra voz, a distancia de cuarta ou quinta
    \item \label{T5RENA-TEC-02:sol}
    Expoñer unha melodía (c.f.) nunha voz, mentres as outras tecen ao seu redor o entramado polifónico
    \item
    Expoñer unha melodía (c.f.) nunha voz, mentres as outras a repiten ao unísono
    \end{enumerate}
}
    { % Solución:
    (\ref{T5RENA-TEC-02:sol}) {\color{orange}{\hrulefill}} \\
    \small{ % Comentario:
    Comentario solución: 
    {\color{orange}{\hrulefill}}
    }
    }
%
% CUESTIÓN: RENACEMENTO - TÉCNICA VARIACIÓN
% -----------------------------------------
%
\newproblem{T5RENA-TECNICAS-03}{
Dentro das técnicas compositivas da música renacentista, sinala de entre as seguintes, a consiste en expoñer un tema e repetilo, ben sexa variando a melodía, a rítmica ou a harmonía:
    \begin{enumerate}[a)]
    \item 
    Técnica do \emph{cantus firmus}
    \item \label{T5RENA-TEC-03:sol}
    Técnica da variación
    \item 
    Técnica do contrapunto imitativo
    \item
    Técnica da homofonía
    \end{enumerate}
}
    { % Solución:
    (\ref{T5RENA-TEC-03:sol}) {\color{orange}{\hrulefill}} \\
    \small{ % Comentario:
    Comentario solución: 
    {\color{orange}{\hrulefill}}
    }
    }
%
% CUESTIÓN: RENACEMENTO - TÉCNICA C.F.
% ------------------------------------
%
\newproblem{T5RENA-TECNICAS-04}{
Sobre as técnicas compositivas que coñeces da música do renacemento, indica aquela que consiste na exposición dunha breve melodía nunha voz --inicialmente de canto chá-- sobre a que as demáis voces, irán creando e desenvolvendo un entramado polifónico:
    \begin{enumerate}[a)]
    \item \label{T5RENA-TEC-04:sol}
    Técnica do \emph{cantus firmus}
    \item 
    Técnica da variación
    \item 
    Técnica do contrapunto imitativo
    \item
    Técnica da homofonía
    \end{enumerate}
}
    { % Solución:
    (\ref{T5RENA-TEC-04:sol}) {\color{orange}{\hrulefill}} \\
    \small{ % Comentario:
    Comentario solución: 
    {\color{orange}{\hrulefill}}
    }
    }
%
