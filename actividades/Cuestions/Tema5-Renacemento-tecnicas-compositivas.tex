% -------------------------------
% BANCO DE PREGUNTAS DE HISTORIA
% -------------------------------
%
% Tema 5.- RENACEMENTO
% 
% CUESTIÓN: RENACEMENTO - TÉCNICAS COMPOSITIVAS
% ---------------------------------------------
%
\newproblem{T5RENA-TECNICAS-01}{
Dentro das principais técnicas de composición da música renacentista, atopamos a técnica da variación, que consiste en \ldots
    \begin{enumerate}[a)]
    \item 
    Expoñer unha breve melodía nunha voz que repetirá inmediatamente outra voz, ben ao unísono ou ben a distancia de cuarta ou quinta
    \item \label{T5RENA-TEC-01}
    Expoñer un tema e repetilo, variando a melodía, a rítmica ou harmonía, etc. 
    \item 
    Expoñer unha breve melodía nunha voz que variará inmediatamente outra voz, a distancia de cuarta ou quinta
    \item
    Expoñer unha melodía (c.f.) nunha voz, mentres as outras tecen o entramado polifónico ao seu redor
    \end{enumerate}
}
{\ref{T5RENA-TEC-01}}
% comentario da resposta:
%    \\ \small{Indica o comentario}
%
%
% CUESTIÓN: RENACEMENTO - TÉCNICAS COMPOSITIVAS
% ---------------------------------------------
%
\newproblem{T5RENA-TECNICAS-02}{
En que consiste a técnica de composición por \emph{cantus firmus} (c.f.)?
    \begin{enumerate}[a)]
    \item 
    En expoñer nunha voz unha breve melodía que repite outra voz, inmediatamente despois, normalmente a distancia de cuarta ou quinta
    \item 
    Expoñer unha breve melodía nunha voz que variará inmediatamente outra voz, a distancia de cuarta ou quinta
    \item \label{T5RENA-TEC-02}
    Expoñer unha melodía (c.f.) nunha voz, mentres as outras tecen ao seu redor o entramado polifónico
    \item
    Expoñer unha melodía (c.f.) nunha voz, mentres as outras a repiten ao unísono
    \end{enumerate}
}
{\ref{T5RENA-TEC-02}}
% comentario da resposta:
%    \\ \small{Indica o comentario}
%
%
