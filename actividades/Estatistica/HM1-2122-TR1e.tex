%
%%%%%%%%%%%%%%%%%%%%%%%%%%%%%%%%%%%%%%%%%%%%%%%%%%
% PLANTILLA ESTATÍSTICAS DE HISTORIA DA MÚSICA I
% A finalidade é un modelo para realizar estatísticas
%%%%%%%%%%%%%%%%%%%%%%%%%%%%%%%%%%%%%%%%%%%%%%%%%%
%
% Clase de documento:
\documentclass[letterpaper,12pt,notitlepage,spanish]{article}
%
% Arquivo de configuración (non modificar):
% -----------------------------------------
\input{../Modelos/include/config-HM1estatistica.tex}
% -----------------------------------------
\usepackage{graphicx}
\begin{document}
%
% DATOS FOLLA ESTATÍSTICA:
% ------------------------
%
% TÍTULO OU ENCABEZAMENTO DA FOLLA
%
\begin{center}
%\large{
%Cuestionario estatístico \\ avaliación inicial e coñecementos previos
%} \\
\vspace*{0.5cm}
%
% NÚMERO DE FOLLA:
%
%\normalsize % Número de hoja:
%(Hoja no. 1)
%\\
%
\begin{flushleft}
\begin{tabular}{| l | l | c | l | l | l |}
\hline
Nome e Apelidos: & \small{(non necesario indicar este dato)} & Especialidade: & \hrulefill & Curso: & \\ \hline 
\small{(non indicar este dato)}: & \\ \hline
Especialidade: &  \\\hline
Titor/a &  \\ \hline
\end{tabular}
\end{flushleft}
%
%

\vspace{1.10cm}
	\begin{flushleft}
	Nome e Apelidos: \hrulefill \\ 
	Curso: \hrulefill\\
	\vspace*{0.50cm}
		\begin{center}
		\small{Instruccións para o cuestionario}\\		
		\end{center}
%	\hrulefill \\
	\vspace*{0.25cm}
%
\small{ % INSTRUCCIONES:
%\texttt{ %
Este cuestionario ten por finalidade realizar unha avaliación incial de coñecementos previos sobre a materia e outros aspectos relevantes, meramente estatísticos. \\
Lé con atención e completa o cuestionario seguinte. \\
%}
} % fin instrucciones.
%
	\vspace*{0.25cm}		
 	\end{flushleft}
\end{center}
%
% ESPACIO PARA REDACTAR LOS EJERCICIOS:
% -------------------------------------  
%
% MODELO PARA LOS EJERCICIOS.- DESCRIPTOR
% \begin{ejercicio}[Título del ejercicio]
%		\begin{enumerate}[1.-]
%
% Pregunta:
%
%			\item 
%
% Pregunta:
%
%			\item
%
%		\end{enumerate}
% \end{ejercicio}
%

% EJERICIO.- ORIGEN DE LA MÚSICA
%
\begin{ejercicio}[El Origen de la música]	
	\begin{enumerate}[1.-]
%
% Pregunta:
%
		\item  
		Los comienzos de la música se desconocen. Según la mitología se cree que la música es de origen:\par
	\begin{tabular}{l l l l l l}
	a) animal &  &  &   & c) desconocido &  \\
	b) bárbaro &  &  &  & d) divino &  \\
	\end{tabular} 
%
% Pregunta:
%
	\item 	Indica si son verdaderas o falsas las siguientes afirmaciones: \par		
		\begin{enumerate}[a)]
			\item La idea occidental de la música se remonta a la antigüedad griega \ldots
			\item La literatura, la pintura y la arqueología, son fuentes de información histórica \ldots 
			\item Los restos arqueológicos no se consideran fuentes de información en música \ldots
			\item Los hallazgos de instrumentos son considerados como "patrimonio primigenio" \ldots
		\end{enumerate} 
	\end{enumerate}
\end{ejercicio}
%
% EJERCICIO.- ORGANOLOGÍA
%
\begin{ejercicio}[Organología - Instrumentos antiguos]
	\begin{enumerate}[1.-]
%
% Pregunta:
%
		\item 
		En función de como producen el sonido, los instrumentos se clasifican en familias. Sitúa los siguientes instrumentos en su correspondiente familia: 
	\textit{sonajas, trompas, flautas, arcos, tambores}
%
		\begin{enumerate}[a)]
		\item cuerda: \ldots
		\item viento: \ldots
		\item percusión: \ldots
		\end{enumerate}
%		\vspace{3cm}		
%
% Pregunta:
%
		\item Completa las afirmaciones, empleando las opciones que se dan a continuación: \\
		 \textit{troncos de árboles huecos, flautas de falanges, arcilla, trompetas y sonajas}
		\begin{enumerate}[a)]
		\item Los primeros tambores prehistóricos estaban hechos de \ldots
		\item Los instrumentos musicales más antiguos del paleolítico son \ldots
		\item Dentro de los instrumentos conservados de la edad de bronce, encontramos \ldots 
		\item Los primeros tambores de la era neolítica estaban hechos de \ldots
		\end{enumerate}
%
	\end{enumerate}
\end{ejercicio}
%
% EJERCICIO.-  FUENTES DE INFORMACIÓN 
% 
\begin{ejercicio}[La música en Mesopotamia]
	\begin{enumerate}[1.-]
%
%Pregunta:
		\item Cita tres instrumentos de viento de la civilización mesopotámica y describe sus principales características:
		\begin{enumerate}[a)]
		\item \ldots
		\vspace{1.5cm}
		\item \ldots
		\vspace{1.5cm}
		\item \ldots
		\vspace{1.5cm}
		\end{enumerate}
%
% Pregunta:
		\item Cita tres instrumentos de cuerda de Mesopotamia y describe sus principales características:
		\begin{enumerate}[a)]
		\item \ldots
		\vspace{1.5cm}
		\item \ldots
		\vspace{1.5cm}
		\item \ldots
		\vspace{1.5cm}
		\end{enumerate} 
	\end{enumerate}
\end{ejercicio}
%
% EJERCICIO.-  ANTIGUO EGIPTO
%
\begin{ejercicio}[La música en el Antiguo Egipto]
	\begin{enumerate}[1.-]
%
% Pregunta:
%
		\item En el antiguo Egipto, la voz era muy importante en las ceremonias religiosas y se empleaba en los ritos para comunicarse con el "más allá". ¿Qué instrumento, de los que se empleaban para acompañar a la voz, era considerado el más apreciado por los egipcios?\par		
		\begin{tabular}{c c c c c c c}
		a) el aulos & & 
		b) la lira & & 
		c) el arpa & & 
		d) la cítara \\
		\end{tabular}
%
% Pregunta: 
%
		\item ¿Cuáles son las principales fuentes de información que conservamos del antiguo Egipto?
		\vspace{2.25cm}
	\end{enumerate}
\end{ejercicio}
%
% EJERCICIO.- MÚSICA EN LA ANTIGUA GRECIA
%
\begin{ejercicio}[La música en la Antigua Grecia]
	\begin{enumerate}[1.-]
%
% Pregunta:
%
		\item 
		La civilización griega antigua, creía que los primeros músicos eran: \par
		\begin{tabular}{c c c c}
		a) los persas & b) los egipcios & c) sus dioses y semidioses & d) los incas \\
		\end{tabular}
%
%Pregunta:
%
		\item Indica si son verdaderas o falsas las siguientes afirmaciones:
		\begin{enumerate}[a)]
		\item La música en la Grecia antigua era principalmente monofónica
		\item La música en la Grecia antigua era principalmente polifónica
		\item En Grecia se empleó el contrapunto a partir del siglo I d.C.
		\item La música en Grecia era tanto monofónica como polifónica
		\end{enumerate}
	\end{enumerate}
\end{ejercicio}
%
% EJERCICIO.- TEORÍA MUSICAL GRIEGA
%
 \begin{ejercicio}[Teoría musical griega]
		\begin{enumerate}[1.-]
%
% Pregunta:
%
			\item La teoría musical griega se centra en el estudio del ritmo y de la melodía; el objeto principal de estudio es el intervalo. Pitágoras establece 3 intervalos principales, que correspondían a 3 proporciones matemáticas simples. Completa: \par
		\begin{center}
			\begin{tabular}{c c c c c}
			\vspace{0.25cm}
			\textbf{Intervalo}  &  &  \textbf{Proporción}  &  &  \textbf{Nombre griego}  \\ 
			\vspace{0.25cm}
			$ 8^a $  &  &  $ 2:1 $  &  &  \dotfill  \\ 
			\vspace{0.25cm}
			\dotfill &  &  $ 3:2 $  &  &  \dotfill \\ 
			\vspace{0.25cm}
			$ 4^a $  &  &  $ 4:2 $  &  & \textit{diatessaron}     \\
			%\vspace{0.25cm}
			%\hline 
			\end{tabular}
		\end{center}
%
% Pregunta:
%
			\item  Indica si son verdaderas [$V$] o falsas [$F$] las afirmaciones siguientes.
		\begin{enumerate}[a)]
		\item	La unidad modal básica del sistema musical griego es el \textbf{tetracordo}, que consiste en un conjunto de $4$ notas que abarcan una $4^a$J asc. (ascendente) \ldots
		\item	La unidad modal básica del sistema musical griego es el \textbf{tetracordo}, que consiste en un conjunto de $4$ notas que abarcan una $4^a$J desc. (descendente) \ldots
		\item	La combinación de $2$ tetracordios forma una escala \ldots
		\item	Los géneros principales del tetracordo son $3$: \textit{diatónico}, \textit{cromático} y \textit{enarmónico}  \ldots
		\end{enumerate}
%
		\end{enumerate}
 \end{ejercicio}
%
% EJERCICIO.- PENSAMIENTO MUSICAL GRIEGO
%
 \begin{ejercicio}[Pensamiento musical griego]
		\begin{enumerate}[1.-]
%
% Pregunta:
%
		\item Indica el nombre de las dos teorías sobre la música de la Grecia antigua que influyeron en el pensamiento musical occidental.
		\vspace*{1.10cm}
		\item Indica qué teoría, desarrollada por \textbf{Damón} en el (siglo V a.C) y defendida por \textbf{Platón}, afirma que "\textit{la música puede influir y modificar el comportamiento y la personalidad de los seres humanos}" \ldots
		\vspace*{0.50cm}
%
% Pregunta:
%%
		\end{enumerate}
 \end{ejercicio}
%

%
% EJERCICIO.- ORGANOLOGÍA
%
\begin{ejercicio}[
Organología - Instrumentos musicales en Grecia
]
	\begin{enumerate}[1.-]
%
% Pregunta:
%
	\item ¿Cuáles eran los instrumentos musicales más empleados en la antigua Grecia? \par
%
		\begin{tabular}{c c c c c c c}
		a) la lira y arpa & & 
		%\textbf{b) lira, cítara y aulos} & & 
		b) lira, cítara y aulos & & 
		c) la chirimía y la lira & & 
		d) la cítara y el arpa \\
		\end{tabular}
%
% Pregunta:
%
	\item Cuando hablamos del \textit{aulos} nos estamos refiriendo a:
		\begin{enumerate}[a)]
			\item el teatro donde se representa la música griega
			\item una flauta con sonido similar al de un oboe
			\item una flauta doble con sonido similar al de un oboe
			\item un instrumento similar al \textit{hydraulis}
		\end{enumerate}
	\end{enumerate}
\end{ejercicio}
%
% EJERCICIO.- LA MÚSICA EN ROMA 
%
\begin{ejercicio}[La música en Roma]
	\begin{enumerate}[1.-]
%
% Pregunta: 
%
	\item Indica si son verdaderas o falsas las siguientes afirmaciones.
		\begin{enumerate}[a)]
			\item La música romana era original y distinta de la música griega \ldots
			\item La música romana se inspiró en la música egipcia \ldots
			\item La música en Roma se inspiró en la cultura y música griega \ldots
			\item La música de los romanos influyó de forma importante en la música griega \ldots
		\end{enumerate}
%
% Pregunta: 
%
	\item Los romanos, desarrollaron instrumentos musicales de viento metal, principalmente para: \par
%
		\begin{tabular}{c c c c c c c}
		a) ceremonias religiosas & & 
		b) cacerías & & 
		c) el ejército & & 
		d) los faraones egipcios \\
		\end{tabular}
%
	\end{enumerate}
\end{ejercicio}
%
% EJERCICIO.- IGLESIA CRISTIANA PRIMITIVA
%
\begin{ejercicio}[La música en la Iglesia cristiana primitiva]
	\begin{enumerate}[1.-]
%
% Pregunta:
%
		\item Entre las fuentes de música eclesiástica cristiana primitiva encontramos:
%
			\begin{enumerate}[a)]
				\item la música del lejano oriente
				\item la música del templo judío
				\item la música de las mezquitas árabes
				\item la música prehistórica
			\end{enumerate}
%
%
% Pregunta:
%
	\item En el siglo IV a.C. se produce un cambio muy importante con el \textbf{Edicto de Milán}, que supuso para los cristianos el libre ejercicio de su religión. ¿Cuál era la función de los instrumentos musicales en la Iglesia cristiana primitiva? 
%
			\begin{enumerate}[a)]	
				\item acompañaban el canto cristiano
				\item acompañaban el canto de la liturgia de la palabra
				\item eran vinculados al culto pagano y prohibidos en la iglesia
				\item acompañaban el canto de salmos e himnos
			\end{enumerate}
%	
	\end{enumerate}
\end{ejercicio}
%
%
\end{document}
%Fin de Hoja de ejercicios
%