% --------------------------
% MODELO CUESTIONARIOS EXAME
% --------------------------
% ACTUALIZADO: 17 de Marzo 2023
%
% TODO: É preciso ver máis sobre o paquete 'PROBSOLN'
%
% Axustes para cuestionarios aleatorios:
% ======================================
% Guión para non repetir cuestións de exercicios, etc.
%
% Indicamos ano académico:
% -----------------------
\SetStartYear{2018}
\PSNrandseed{\GetStartYear} %
%
% Exclude problems that have been used this year or the previous
% two academic years. (Creates a file called 'excluded.tex' to
% store labels of used problems. Also creates a file called
% \jobname.prb that stores labels of problems used in this
% document, so they don't get excluded on subsequent runs.)
%\ExcludePreviousFile[1]{excluded}
%
 % To clear the used problems file (\jobname.prb) of the labels
 % generated in the previous run, comment the above and uncomment
 % below:
%
 % TODO: Revisar os arquivos excluídos
\ClearUsedFile{excluded}
%
%
% CUESTIÓNS:
% ----------
% No. total de preguntas = 20
% TODO: REVISAR AS SOLUCIÓNS DO CUESTIONARIO
%
% Carga aletoria
% --------------
\loadrandomproblems[Tema3-PERIODIZACION]{4}{
../../Cuestions/Tema3-Periodizacion-Q.tex}
%
\loadrandomproblems[Tema3-CANTO-CHA-EXPANSION]{3}{
../../Cuestions/Tema3-Canto-gregoriano-expansion-Q.tex}
%
\loadrandomproblems[Tema3-NOTACION-MODAL]{5}{
../../Cuestions/Tema3-Notacion-modal-Q.tex}
%
\loadrandomproblems[Tema3-CANTO-CHA]{8}{
../../Cuestions/Tema3-Canto-gregoriano-Q.tex}
%
%\loadrandomproblems[Tema0-PERSPECTIVAS]{2}{
%../../Cuestions/Tema0-PERSPECTIVAS-Q.tex}
%
%\loadrandomproblems[Tema1-EXIPTO]{1}{
%../../Cuestions/Tema1-EXIPTO-Q.tex}
%
%\loadrandomproblems[Tema1-GRECIA]{1}{
%../../Cuestions/Tema1-GRECIA-Q.tex}
%
%\loadrandomproblems[Tema1-ORGANOLOXIA]{2}{
%../../Cuestions/Tema1-ORGANOLOXIA-Q.tex}
%
%\loadrandomproblems[Tema1-PENSAMENTO]{1}{
%../../Cuestions/Tema1-PENSAMENTO-Q.tex}
%
%\loadrandomproblems[Tema1-PERIODIZACION]{1}{
%../../Cuestions/Tema1-PERIODIZACION-Q.tex }
%
%\loadrandomproblems[Tema1-ROMA]{1}{
%../../Cuestions/Tema1-ROMA-Q.tex}
%
% Carga todas
% -----------
%\loadallproblems
%\loadallproblems[Tema1-PENS]{}
%\loadallproblems[Tema1-GR]{}
%\loadallproblems[Tema1-OR]{}
%\loadallproblems[Tema1-EX]{}
%\loadallproblems[Tema1-RO]{}
%
% Carga selectiva
% ---------------
%\loadselectedproblems[Tema1-EX]{T1EX-01,T1EX-02}{../../Cuestions/Tema1-Exipto.tex}
%\loadselectedproblems[Tema1-GR]{T1GR-05,T1GR-07,T1GR-08,T1GR-02}{
%../../Cuestions/Tema1-Grecia.tex}
%\loadselectedproblems[Tema1-PE]{T1PE-05,T1PE-11,T1PE-12}{
%../../Cuestions/Tema1-Periodizacion.tex}
%\loadselectedproblems[Tema1-PENS]{T1PENS-03,T1PENS-04}{
%../../Cuestions/Tema1-Pensamento.tex}
%\loadselectedproblems[Tema1-RO]{T1RO-01}{../../Cuestions/Tema1-Roma.tex}
%
%
% Carga exluínte
% --------------
%\loadexceptproblems[Tema1-CO]{T1CO-01,T1CO-02,T1CO-03,T1CO-04}{../../Cuestions/Tema1-Compositores.tex}
%
%
%
% -------------------------------
% Estrutura e numeración da proba
% -------------------------------
% Organiza aquí todas as cuestións
%
\begin{multicols}{2}
	\begin{enumerate}[1.-]
{\small % Reducido texto para 4 páxinas (impresión a3)
%%
\foreachproblem[Tema3-PERIODIZACION]{
\item\label{prob:\thisproblemlabel}
\thisproblem}
%%
\foreachproblem[Tema3-NOTACION-MODAL]{
\item\label{prob:\thisproblemlabel}
\thisproblem}
%%
\foreachproblem[Tema3-CANTO-CHA-EXPANSION]{
\item\label{prob:\thisproblemlabel}
\thisproblem}
%%
\foreachproblem[Tema3-CANTO-CHA]{
\item\label{prob:\thisproblemlabel}
\thisproblem} 
%%
%\foreachproblem[Tema0-PERSPECTIVAS]{
%\item\label{prob:\thisproblemlabel}
%\thisproblem}
%%
%\foreachproblem[Tema1-EXIPTO]{
%\item\label{prob:\thisproblemlabel}
%\thisproblem}
%%
%\foreachproblem[Tema1-GRECIA]{
%\item\label{prob:\thisproblemlabel}
%\thisproblem}
%%
%\foreachproblem[Tema1-ORGANOLOXIA]{
%\item\label{prob:\thisproblemlabel}
%\thisproblem}
%
%\foreachproblem[Tema1-PENSAMENTO]{
%\item\label{prob:\thisproblemlabel}
%\thisproblem}
%
%\foreachproblem[Tema1-PERIODIZACION]{
%\item\label{prob:\thisproblemlabel}
%\thisproblem}
%
%\foreachproblem[Tema1-ROMA]{
%\item\label{prob:\thisproblemlabel}
%\thisproblem}
%
}
    \end{enumerate}
\end{multicols}
%
