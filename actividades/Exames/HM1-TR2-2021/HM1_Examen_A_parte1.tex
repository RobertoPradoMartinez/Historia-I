%
%%%%%%%%%%%%%%%%%%%%%%%%%%%%%%%%%%%%%%%%%%%%%%%%%%
% PLANTILLA EXÁMENES DE HISTORIA DE LA MÚSICA I
% Este es un modelo para redactar exámenes
% Está basada en la plantilla para realizar ejercicios
%
% Pasos para cubrir la plantilla:
% 1) Realizar una copia de este modelo
% 2) Renombrar el archivo
% 3) Incluye las cuestiones que consideres
%%%%%%%%%%%%%%%%%%%%%%%%%%%%%%%%%%%%%%%%%%%%%%%%%%
%
% Esta plantilla es para crear exámenes de esta materia
% Los exámenes constan de dos partes: A(teórica) y B(práctica)
%
% Descomentar según se necesite utilizar un modelo u otro
%
% Clase de documento:
\documentclass[letterpaper,12pt,notitlepage,spanish]{article}
%
% CONFIGURACIÓN:
% Ruta de figuras y demás:
% configurar cuando se redacte el documento final
% indicar ruta donde se encuentre lo que se quiera adjuntar.
%
%\graphicspath{ {/home/user/images/} }
%
% Archivo externo de configuración
% --------------------------------
%%%%%%%%%%%%%%%%%%%%%%%%%%%%%%%%%%%%%%%%%%%
%% ---------- MODELO EJERCICIOS ---------- 
%% MATERIA: HISTORIA
%% CURSO: 
%% AÑO ACADÉMICO: 
%% CENTRO: 
%%%%%%%%%%%%%%%%%%%%%%%%%%%%%%%%%%%%%%%%%%%
%% 
%% MODELO PARA REDACTAR EJERCICIOS
%% ===============================
%% 
%% Clase de documento
%% ------------------
%\documentclass[letterpaper,12pt,notitlepage,spanish]{article}
%\documentclass[12pt,a4paper,notitlepage]{article}
%
% Márgenes de documento
% ---------------------
\usepackage[left=2.0cm, right=2.0cm, lines=45, top=2.5cm, bottom=2.0cm]{geometry}
%
% Paquetes necesarios
% -------------------
\usepackage[utf8]{inputenc} % acentos en ES
\usepackage[spanish,activeacute, es-tabla]{babel}
\usepackage{enumerate} % entornos de listas
\usepackage{multicol}  % varias columnas texto
\usepackage{fancyhdr}  % encabezado personalizado
\usepackage{fancybox}  % entornos con cajas
\usepackage{pdfpages}  % páginas pdf
\usepackage{lastpage}  % numeración páxinas
\usepackage{amssymb}   % completar líneas
\usepackage{xcolor}  % color en exames
%
\usepackage{environ} 
\usepackage{probsoln} % paquete para soluciones
%\showanswers % para mostrar soluciones
% --->
% Recuadros y figuras
% -------------------
% Código de "OndaHostil"
\usepackage{tcolorbox}        %recuadros colores
\tcbuselibrary{listingsutf8}  % estilos recuadro
% Definir cuadro de ancho del texto
\newtcolorbox{cadro1}[1]{colback=red!5!white,colframe=red!75!black,fonttitle=\bfseries,title=#1}
% OPCIONES
% colback: color de fondo
% colframe: color de borde
% fonttitle: estilo de título
% title: título de la cuadro o referencia a argumento

% Cuadro estrecho
\newtcbox{cadro2}[1]{colback=grey!5!white,colframe=grey!75!white,fonttitle=\bfseries,title=#1}

% Cadro en gris
\newtcolorbox{datos-alumnado}[1]{colback=gray!10!white,colframe=gray!50!black,fonttitle=\bfseries, title=NOME E APELIDOS/CALIFICACIÓN:
\vspace*{0.10cm}}

% Cadro en azul para instruccións:
\definecolor{lila1}{RGB}{193,124,250} % lila
\definecolor{cornflowerblue}{RGB}{100,149,237} % azul suave
%
\newtcolorbox{instruccions}[1]{colback=gray!2!white,colframe=gray!50!cornflowerblue,fonttitle=\bfseries, title=\centering{INSTRUCCIÓNS PARA REALIZAR A PROBA}
\vspace*{0.10cm}}
% Cuadro numerado para ejemplos
\newtcolorbox[auto counter,number within=section]{example}[2][]
{colback=green!5!white,colframe=green!75!black,fonttitle=\bfseries, title=Exemplo~\thetcbcounter: #2,#1}

\usepackage[colorlinks, linkcolor=black]{hyperref}
%
% SOLUCIÓNS AOS EXAMES:
% ---------------------
%% Esto es lo importante. Ponemos la solución al margen.
%\NewEnviron{solutionnew}{%
%%  \leavevmode\marginpar{\raggedright\footnotesize \textbf{Solución:}\\ \BODY}
%%  \textbf{Solución:}\\ \BODY} % sol. con salto de liña
%  \small{Solución:} \BODY} % sol. na mesma liña
%  {}
%\renewenvironment{solution}{\solutionnew}{\endsolutionnew}
% ---
% Solución resaltada en cor lima:
% -------------------------------
%Esto es lo importante. Ponemos la solución al margen.
\NewEnviron{solutionnew}{%
%  \leavevmode\marginpar{\raggedright\footnotesize \textbf{Solución:}\\ \BODY}
%  \textbf{Solución:}\\ \BODY} % sol. con salto de liña
  \small{\colorbox{lime}{Solución:} \BODY}} % sol. na mesma liña
  {}
\renewenvironment{solution}{\solutionnew}{\endsolutionnew}
% ---
% --------
% Lineas de encabezado y pié
% --------------------------
\renewcommand{\headrulewidth}{0.5pt}
%\renewcommand{\headrulewidth}{1.0pt}
\renewcommand{\footrulewidth}{0.5pt}
%\renewcommand{\footrulewidth}{1.0pt}
\pagestyle{fancy} % estilo de página
%
% Recuadros y figuras
% -------------------
\newcommand\Loadedframemethod{TikZ}
\usepackage[framemethod=\Loadedframemethod]{mdframed}
\usepackage{tikz}
\usetikzlibrary{calc,through,backgrounds}
\usetikzlibrary{matrix,positioning}
%Desssins geometriques
\usetikzlibrary{arrows}
\usetikzlibrary{shapes.geometric}
\usetikzlibrary{datavisualization}
\usetikzlibrary{automata} % LATEX and plain TEX
\usetikzlibrary{shapes.multipart}
\usetikzlibrary{decorations.pathmorphing} 
\usepackage{pgfplots}
\usepackage{physics}
\usepackage{titletoc}
\usepackage{mathpazo} 
\usepackage{algpseudocode}
\usepackage{algorithmicx} 
\usepackage{bohr} 
\usepackage{xlop} 
\usepackage{bbding} 
%\usepackage{minibox} 
% Texto árabe
\usepackage{mathdesign}
\usepackage{bbding} 
% --
% Tipograía:
% ----------
% Fuente HEURÍSTICA (cómoda de leer)
%\usepackage{heuristica}
% Fuente LIBERTINE (cómoda para apuntes)
%\usepackage{libertineRoman}
\usepackage[proportional]{libertine}
% Fuente ROMANDE (estilo antiguo pero no muy cómoda)
%\usepackage{romande} %

%=====================Algo setup
\algblock{If}{EndIf}
\algcblock[If]{If}{ElsIf}{EndIf}
\algcblock{If}{Else}{EndIf}
\algrenewtext{If}{\textbf{si}}
\algrenewtext{Else}{\textbf{sinon}}
\algrenewtext{EndIf}{\textbf{finsi}}
\algrenewtext{Then}{\textbf{alors}}
\algrenewtext{While}{\textbf{tant que}}
\algrenewtext{EndWhile}{\textbf{fin tant que}}
\algrenewtext{Repeat}{\textbf{r\'ep\'eter}}
\algrenewtext{Until}{\textbf{jusqu'\`a}}
\algcblockdefx[Strange]{If}{Eeee}{Oooo}
[1]{\textbf{Eeee} "#1"}
{\textbf{Wuuuups\dots}}

\algrenewcommand\algorithmicwhile{\textbf{TantQue}}
\algrenewcommand\algorithmicdo{\textbf{Faire}}
\algrenewcommand\algorithmicend{\textbf{Fin}}
\algrenewcommand\algorithmicrequire{\textbf{Variables}}
\algrenewcommand\algorithmicensure{\textbf{Constante}}% replace ensure by constante
\algblock[block]{Begin}{End}
\newcommand\algo[1]{\textbf{algorithme} #1;}
\newcommand\vars{\textbf{variables } }
\newcommand\consts{\textbf{constantes}}
\algrenewtext{Begin}{\textbf{debut}}
\algrenewtext{End}{\textbf{fin}}
%================================
%================================

\setlength{\parskip}{1.25cm}
\setlength{\parindent}{1.25cm}
\tikzstyle{titregris} =
[draw=gray,fill=gray, shading = exersicetitle, %
text=gray, rectangle, rounded corners, right,minimum height=.3cm]
\pgfdeclarehorizontalshading{exersicebackground}{100bp}
{color(0bp)=(green!40); color(100bp)=(black!5)}
\pgfdeclarehorizontalshading{exersicetitle}{100bp}
{color(0bp)=(red!40);color(100bp)=(black!5)}
\newcounter{exercise}
\renewcommand*\theexercise{exercice \textbf{Ejercicio}~n\arabic{exercise}}
\makeatletter
\def\mdf@@exercisepoints{}%new mdframed key:
\define@key{mdf}{exercisepoints}{%
\def\mdf@@exercisepoints{#1}
}
\mdfdefinestyle{exercisestyle}{%
outerlinewidth=1em,outerlinecolor=white,%
leftmargin=-1em,rightmargin=-1em,%
middlelinewidth=0.5pt,roundcorner=3pt,linecolor=black,
apptotikzsetting={\tikzset{mdfbackground/.append style ={%
shading = exersicebackground}}},
innertopmargin=0.1\baselineskip,
skipabove={\dimexpr0.1\baselineskip+0\topskip\relax},
skipbelow={-0.1em},
needspace=0.5\baselineskip,
frametitlefont=\sffamily\bfseries,
settings={\global\stepcounter{exercise}},
singleextra={%
\node[titregris,xshift=0.5cm] at (P-|O) %
{~\mdf@frametitlefont{\theexercise}~};
\ifdefempty{\mdf@@exercisepoints}%
{}%
{\node[titregris,left,xshift=-1cm] at (P)%
{~\mdf@frametitlefont{\mdf@@exercisepoints points}~};}%
},
firstextra={%
\node[titregris,xshift=1cm] at (P-|O) %
{~\mdf@frametitlefont{\theexercise}~};
\ifdefempty{\mdf@@exercisepoints}%
{}%
{\node[titregris,left,xshift=-1cm] at (P)%
{~\mdf@frametitlefont{\mdf@@exercisepoints points}~};}%
},
}
\makeatother


%%%%%%%%%

%%%%%%%%%%%%%%%
\mdfdefinestyle{theoremstyle}{%
outerlinewidth=0.01em,linecolor=black,middlelinewidth=0.5pt,%
frametitlerule=true,roundcorner=2pt,%
apptotikzsetting={\tikzset{mfframetitlebackground/.append style={%
shade,left color=white, right color=blue!20}}},
frametitlerulecolor=black,innertopmargin=1\baselineskip,%green!60,
innerbottommargin=0.5\baselineskip,
frametitlerulewidth=0.1pt,
innertopmargin=0.7\topskip,skipabove={\dimexpr0.2\baselineskip+0.1\topskip\relax},
frametitleaboveskip=1pt,
frametitlebelowskip=1pt
}
\setlength{\parskip}{0mm}
\setlength{\parindent}{10mm}
\mdtheorem[style=theoremstyle]{ejercicio}{\textbf{Ejercicio}}
%================Liste definition--numList-and alphList=============
\newcounter{alphListCounter}
\newenvironment
{alphList}
{\begin{list}
{\alph{alphListCounter})}
{\usecounter{alphListCounter}
\setlength{\rightmargin}{0cm}
\setlength{\leftmargin}{0.5cm}
\setlength{\itemsep}{0.2cm}
\setlength{\partopsep}{0cm}
\setlength{\parsep}{0cm}}
}
{\end{list}}
\newcounter{numListCounter}
\newenvironment
{numList}
{\begin{list}
{\arabic{numListCounter})}
{\usecounter{numListCounter}
\setlength{\rightmargin}{0cm}
\setlength{\leftmargin}{0.5cm}
\setlength{\itemsep}{0cm}
\setlength{\partopsep}{0cm}
\setlength{\parsep}{0cm}}
}
{\end{list}}
%
%% -- Fin del archivo de configuración  --
%%

% --------------------------------
\usepackage{graphicx}
%
\begin{document}
%
% DATOS DE HOJA DE EXAMEN
% ---------------------------
%
\begin{center}
%
\Large{ % TÍTULO HOJA 
PRUEBA TEÓRICA - 2ª EVALUACIÓN \\ (Parte A)} \\ % 
\vspace*{0.5cm}
\normalsize % NÚMERO DE HOJA: (no se aplica aquí)
%(Hoja no. 1)
%\\
\vspace{1.10cm}
	\begin{flushleft}
	Nombre y Apellidos: \hrulefill\\
	\vspace*{0.15cm}
		\begin{center}
		\small{\texttt{INSTRUCCIONES}
		}\\		
		\end{center}
%	\hrulefill \\
	\vspace*{0.15cm}
%
\small{ % Instrucciones:
	\texttt{
La prueba teórica consta de 10 ejercicios tipo test de una sola respuesta verdadera.\\
Cada ejercicio consta de dos apartados. Cada apartado puntúa de la siguiente forma:
\begin{enumerate}[1)]
	\item  \textbf{0.5} puntos, si la respuesta es correcta
	\item  \textbf{-0.25} puntos si la respuesta no es correcta.
\end{enumerate}
Tiempo para realizar la prueba = \textbf{30 minutos}.
}} % fin instrucciones.
%
	\vspace*{0.15cm}
		\begin{center}
		\small{\texttt{EJERCICIOS}
		}\\		
		\end{center}		
 	\end{flushleft}
\end{center}
%
% --------------------------------
% ENUNCIADOS DE LA PRUEBA TEÓRICA:
% --------------------------------
%
% -----------
% EJERCICIO 1 
% -----------
\begin{ejercicio}[]
	\begin{enumerate}[1.-]
%
% Pregunta:
% 
		\item
		Pregunta número 1:\par
%
%\vspace*{0.15cm} % Salto de 0.15 cm para separar
%
	\begin{tabular}{llll}
		a) Antigüedad griega y romana & & c) Mesopotamia & \\
		b) Mesopotamia y Egipto & & d) antigüedad griega, Asia Menor y lejano Oriente
	\end{tabular}
%	
% Solución: d) antigüedad griega, Asia Menor y lejano Oriente
%
% Pregunta:
%
\item Teniendo en cuenta algunas fuentes de información, como la arqueología, parece que algunos instrumentos han existido desde tiempos prehistóricos. ¿A qué nos estamos refiriendo cuando hablamos de \textit{patrimonio primigenio}? \par
%
%\vspace*{0.15cm} % Salto de 0.15 cm para separar
%
	\begin{tabular}{llll}
		a) registros escritos del 3er milenio a.C. & & & c) escritos literarios sobre música\\
		b) representaciones artísticas de instrumentos & & & d) hallazgos de instrumentos antiguos\\
	\end{tabular}
%
% Solución: hallazgos de instrumentos muy antiguos
%
	\end{enumerate}
\end{ejercicio}
%
% -----------
% EJERCICIO 2
% -----------
% EJERCICIO.- 
\begin{ejercicio}[]
	\begin{enumerate}[1.-]
%
% Pregunta:
%
		\item La caza pudo dar origen a: \par
%
%\vspace*{0.15cm} % Salto de 0.15 cm para separar
%		
	\begin{tabular}{llll}
		a) únicamente instrumentos de cuerda & & & c) instrumentos de percusión\\
		b) instrumentos de viento y cuerda & & & d) música vocal\\
	\end{tabular}
%
% Solución: instrumentos de viento y cuerda
%
% Pregunta:
%
		\item Los primeros instrumentos hallados en excavaciones arqueológicas, estaban hechos de:\par
%
%\vspace*{0.15cm} % Salto de 0.15 cm para separar
%		
	\begin{tabular}{llll}
		a) huesos de reno & b) huesos de caballo & c) huesos de aves  & d) espinas de peces\\
%
% Solución: huesos de reno
%

	\end{tabular}
%
	\end{enumerate}
\end{ejercicio}
%
% EJERCICIO 3
%
\begin{ejercicio}[] % inicio ejercicio
	\begin{enumerate}[1.-]
%
% Pregunta:
%
		\item Señala la opción que consideres correcta:
%
		\begin{enumerate}[a)]
		\item Los primeros hallazgos de instrumentos se remontan a la era paleolítica
		 % verdadero
		\item Los primeros hallazgos de instrumentos se remontan a la era neolítica
		 % falso
		\item Los primeros hallazgos de instrumentos se remontan a la edad de bronce
		 % falso (no narra una historia)
		\item Los primeros hallazgos de instrumentos se remontan a Mesopotamia
		 % verdadero
		\end{enumerate}
%		\vspace{0.5cm}
%
% Solución: Los primeros hallazgos de instrumentos se remontan a la era paleolítica
%
% Pregunta:
%
		\item ¿Cuál es el origen de la música? Señala la correcta:
%		
	\begin{enumerate}[a)]
		\item La \textit{Armonía de las esferas}, afirma que es de origen divino
		 % Falso
		\item El origen de la música surge de la imitación de la naturaleza
		 % Falso
		\item El origen de la música surge de la expresión de emociones humanas
		 % Falso
		\item El origen de la música se desconoce, hay varias teorías al respecto 
		 % Verdadero
%
% Solución: El origen de la música se desconoce, hay varias teorías al respecto
%		
	\end{enumerate}
\end{enumerate}
%%
\end{ejercicio} % fin ejercicio
%
%
% EJERCICIO 4
%
\begin{ejercicio}
	\begin{enumerate}
%
% Pregunta:
%
		\item La antigua civilización mesopotámica estaba muy avanzada hace ya seis mil años. A lo largo del IV milenio a.C. la liturgia estaba constituida por \textit{himnos} y posteriormente se añadieron \textit{salmos} y \textit{lamentaciones} llegando más tarde a formar una liturgia completa. Se sabe que influye en el desarrollo de la música de forma muy importante. Para celebrar dicha liturgia se incorporan en el tempo:
		\par
%		
%\vspace*{0.15cm} % Salto de 0.15 cm para separar
%		
	\begin{tabular}{llll}
		a) cantores masculinos &  &  & c) cantores pre-adolescentes\\
		c) cantores femeninos & & & d) arpistas \\
%
% Solución: cantores femeninos
%
	\end{tabular}

%
% Pregunta:
%
	\item Indica la opción correcta:
%
		\begin{enumerate}[a)]
			\item La música mesopotámica influye en el desarrollo de la música a partir del s. IV a.C.
			 % Verdadero
			\item La música mesopotámica influye en el desarrollo de la música a partir del s. VI a.C. 
			 % Falso
			\item La música mesopotámica influye en el desarrollo de la música a partir del s. II a.C. 
			 % Falso
			\item La música mesopotámica influye en el desarrollo de la música a partir del s. I d.C.
			 % Falso   
		\end{enumerate}
%
% Solución: La música mesopotámica influye en el desarrollo de la música a partir del s. IV a.C.
%
	\end{enumerate}
\end{ejercicio}
%
%
% EJERCICIO 5
%
\begin{ejercicio}[] % inicio ejercicio
%
	\begin{enumerate}[1.-]
%
% Pregunta:
%
		\item El instrumento más representativo de Mesopotamia era: \par
%		
%		\vspace*{0.15cm}
	\begin{tabular}{llll}
		a) la lira & b) el arpa & c) la flauta & d) el laúd\\
	\end{tabular}
%
% Solución: la lira
%
% Pregunta:
%
		\item El arpa que conocemos de Mesopotamia tenía: \par
%		
%		\vspace*{0.15cm}
	\begin{tabular}{llll}
		a) cuatro cuerdas & &  & c) entre cuatro y siete cuerdas\\
		c) menos de cuatro cuerdas & &  & d) entre seis y doce cuerdas\\
	\end{tabular}
%
% Solución: entre 4 -7 cuerdas
%
	\end{enumerate}
%
\end{ejercicio} % fin ejercicio
%
%
% EJERCICIO 6
%
\begin{ejercicio}
	\begin{enumerate}
%
% Pregunta:
%
%		
		\item El instrumento más representativo del Antiguo Egipto fue: \par
%		
%		\vspace*{0.15cm}
	\begin{tabular}{llll}
		a) la lira & b) el arpa & c) el aulos & d) el tambor\\
	\end{tabular}
%
% Solución: el arpa
%
% Pregunta:
%
		\item ¿Dónde se han encontrado fuentes de información arqueológica del Antiguo Egipto? \par
%		
%		\vspace*{0.15cm}
	\begin{tabular}{llll}
		a) en los templos sumerios &  & c) en el fondo del Nilo & \\
		b) cámaras funerarias egipcias & & d) en templos de Irán
	\end{tabular}
%
% Solución: cámaras funerarias
%
	\end{enumerate}
\end{ejercicio}
%
%
% EJERCICIO 7
%
\begin{ejercicio}
	\begin{enumerate}
%
% Pregunta:
%
		\item Los egipcios creían que el instrumento más poderoso para las ceremonias religiosas era: \par
%		
%		\vspace*{0.15cm}
	\begin{tabular}{llll}
		a) La voz & b) el aulos & c) el arpa & d) el tambor\\
	\end{tabular}
%
% Solución: la voz
%
%
% Pregunta:
%
	\item La música en el Antiguo Egipto se practicaba en: \par
%		
%		\vspace*{0.15cm}
	\begin{tabular}{llll}
		a) El templo & b) la corte & c) el pueblo & d) todas son correctas\\
	\end{tabular}
%
% Solución: La música en el Antiguo Egipto se practicaba en el templo, la corte y el pueblo
%
\end{enumerate}
\end{ejercicio}
%

%
% EJERCICIO 8
%
\begin{ejercicio}
	\begin{enumerate}
%
%
	\item La música griega se asociaba con: \par
%		
%		\vspace*{0.15cm}
	\begin{tabular}{llll}
		a) las emociones & b) la danza & c) la expresión oral y la danza & d) expresión oral\\
	\end{tabular}
%
% Solución: Expresión oral y danza
%
%
% Pregunta:
%
		\item El primer griego considerado como el alma de las teorías sobre la relación entre la música, el cosmos, el orden del universo y el alma del ser humano: \par
%		
%		\vspace*{0.20cm}
	\begin{tabular}{llll}
		a) Aristóteles & b) Platón & c) Pitágoras  & d) Sócrates\\
	\end{tabular}

%
% Solución: Pitágoras
%
	\end{enumerate}
\end{ejercicio}
%
% EJERCICIO 9
%
\begin{ejercicio}
	\begin{enumerate}
%
% Pregunta:
		\item Una fuente importante de música cristiana primitiva era: \par
%		
%		\vspace*{0.20cm}
	\begin{tabular}{llll}
		a) la música etrusca &  &  & c) la música romana \\
		b) los salmos &  &  & d) la música de la antigüedad tardía
	\end{tabular}

%
% Solución: La música de la antigüedad tardía
%
%
% Pregunta:
%
	\item Entre las fuentes de música eclesiástica cristiana primitiva encontramos: \par
%
%		
%		\vspace*{0.20cm}
	\begin{tabular}{llll}
		a) música del lejano oriente &  & c) música del templo judío &  \\
		b) música de las mezquitas & & d) música prehistórica \\
	\end{tabular}

%
% Solución: Música del templo judío
%
	\end{enumerate}
\end{ejercicio}
%
% EJERCICIO 10
%
\begin{ejercicio}
	\begin{enumerate}
%
% Pregunta:
%
	\item La música romana se inspiró en: \par
%
%		\vspace*{0.20cm}
	\begin{tabular}{llll}
		a) la egipcia & b) la griega  & c) la mesopotámica & d) la persa \\
	\end{tabular}

%
% Solución: la griega
%
% Pregunta:
%
	\item Los romanos desarrollaron instrumentos de viento metal principalmente para: \par
%
%		
%		\vspace*{0.20cm}
	\begin{tabular}{llll}
		a) cacerías & b) el teatro & c) ceremonias religiosas & d) el ejército  \\
	\end{tabular}

%
% Solución: El ejército
%
	\end{enumerate}
\end{ejercicio}
%
%
\end{document}
% Fin de Hoja de ejercicios