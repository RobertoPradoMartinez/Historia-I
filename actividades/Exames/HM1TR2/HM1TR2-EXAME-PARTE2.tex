%
%%%%%%%%%%%%%%%%%%%%%%%%%%%%%%%%%%%%%%%%%%%%%%%%%%
% PLANTILLA EJERCICIOS DE HISTORIA DE LA MÚSICA I
% Este es un modelo para redactar los ejercicios
% 
% Pasos para cubrir la plantilla:
% 1) Realizar una copia de este modelo
% 2) Renombrar el archivo:
%		"HM1_Hoja(número).tex"
% 3) El número de Hoja debe ser correlativo
% 
%%%%%%%%%%%%%%%%%%%%%%%%%%%%%%%%%%%%%%%%%%%%%%%%%%
%
% Esta plantilla es para crear ejercicios de esta materia
% Se recomienda crear un archivo por cada tema
% Descomentar según se necesite utilizar un modelo de ejercicio u otro
% Clase de documento:
\documentclass[letterpaper,12pt,notitlepage,spanish]{article}
%
% CONFIGURACIÓN:
% Ruta de figuras y demás:
% configurar cuando se redacte el documento final
% indicar ruta donde se encuentre lo que se quiera adjuntar.
%
%\graphicspath{ {/home/user/images/} }
%
% Archivo externo de configuración
% --------------------------------
%%%%%%%%%%%%%%%%%%%%%%%%%%%%%%%%%%%%%%%%%%%
%% ---------- MODELO EJERCICIOS ---------- 
%% MATERIA: HISTORIA
%% CURSO: 
%% AÑO ACADÉMICO: 
%% CENTRO: 
%%%%%%%%%%%%%%%%%%%%%%%%%%%%%%%%%%%%%%%%%%%
%% 
%% MODELO PARA REDACTAR EJERCICIOS
%% ===============================
%% 
%% Clase de documento
%% ------------------
%\documentclass[letterpaper,12pt,notitlepage,spanish]{article}
%\documentclass[12pt,a4paper,notitlepage]{article}
%
% Márgenes de documento
% ---------------------
\usepackage[left=2.0cm, right=2.0cm, lines=45, top=2.5cm, bottom=2.0cm]{geometry}
%
% Paquetes necesarios
% -------------------
\usepackage[utf8]{inputenc} % acentos en ES
\usepackage[spanish,activeacute, es-tabla]{babel}
\usepackage{enumerate} % entornos de listas
\usepackage{multicol}  % varias columnas texto
\usepackage{fancyhdr}  % encabezado personalizado
\usepackage{fancybox}  % entornos con cajas
\usepackage{pdfpages}  % páginas pdf
\usepackage{amssymb}   % completar líneas
%
\usepackage{environ} 
%\usepackage[noanswers]{probsoln} % sin soluciones
\usepackage[answers]{probsoln} % con soluciones
%\showanswers % para mostrar soluciones
%
%Esto es lo importante. Ponemos la solución al margen.
\NewEnviron{solutionnew}{%
%  \leavevmode\marginpar{\raggedright\footnotesize \textbf{Solución:}\\ \BODY}
%  \textbf{Solución:}\\ \BODY} % sol. con salto de liña
  \small{Solución:} \BODY} % sol. na mesma liña
  {}
\renewenvironment{solution}{\solutionnew}{\endsolutionnew}
% ---

%
% Lineas de encabezado y pié
% --------------------------
\renewcommand{\headrulewidth}{0.5pt}
%\renewcommand{\headrulewidth}{1.0pt}
\renewcommand{\footrulewidth}{0.5pt}
%\renewcommand{\footrulewidth}{1.0pt}
\pagestyle{fancy} % estilo de página
%
% Recuadros y figuras
% -------------------
% Código de "OndaHostil"
\usepackage{tcolorbox}        %recuadros colores
\tcbuselibrary{listingsutf8}  % estilos recuadro
% Definir cuadro de ancho del texto
\newtcolorbox{cadro1}[1]{colback=red!5!white,colframe=red!75!black,fonttitle=\bfseries,title=#1}
% OPCIONES
% colback: color de fondo
% colframe: color de borde
% fonttitle: estilo de título
% title: título de la cuadro o referencia a argumento

% Cuadro estrecho
\newtcbox{cadro2}[1]{colback=grey!5!white,colframe=grey!75!white,fonttitle=\bfseries,title=#1}

% Cadro en gris
\newtcolorbox{datos-alumnado}[1]{colback=gray!10!white,colframe=gray!50!black,fonttitle=\bfseries, title=NOME E APELIDOS/CALIFICACIÓN:
\vspace*{0.10cm}}

% Cadro en gris para instruccións:
\newtcolorbox{instruccions}[1]{colback=gray!2!white,colframe=gray!50!black,fonttitle=\bfseries, title=\centering{INSTRUCCIÓNS PARA REALIZAR A PROBA}
\vspace*{0.10cm}}
% Cuadro numerado para ejemplos
\newtcolorbox[auto counter,number within=section]{example}[2][]
{colback=green!5!white,colframe=green!75!black,fonttitle=\bfseries, title=Exemplo~\thetcbcounter: #2,#1}

\usepackage[colorlinks, linkcolor=black]{hyperref}
%
% --------
\newcommand\Loadedframemethod{TikZ}
\usepackage[framemethod=\Loadedframemethod]{mdframed}
\usepackage{tikz}
\usetikzlibrary{calc,through,backgrounds}
\usetikzlibrary{matrix,positioning}
%Desssins geometriques
\usetikzlibrary{arrows}
\usetikzlibrary{shapes.geometric}
\usetikzlibrary{datavisualization}
\usetikzlibrary{automata} % LATEX and plain TEX
\usetikzlibrary{shapes.multipart}
\usetikzlibrary{decorations.pathmorphing} 
\usepackage{pgfplots}
\usepackage{physics}
\usepackage{titletoc}
\usepackage{mathpazo} 
\usepackage{algpseudocode}
\usepackage{algorithmicx} 
\usepackage{bohr} 
\usepackage{xlop} 
\usepackage{bbding} 
%\usepackage{minibox} 
% Texto árabe
\usepackage{mathdesign}
\usepackage{bbding} 
% --
% Tipograía:
% ----------
% Fuente HEURÍSTICA (cómoda de leer)
%\usepackage{heuristica}
% Fuente LIBERTINE (cómoda para apuntes)
\usepackage{libertineRoman}
%\usepackage[proportional]{libertine}
% Fuente ROMANDE (estilo antiguo pero no muy cómoda)
%\usepackage{romande} %
% 
% Encabezado y pié de página (textos)
% -----------------------------------
% Modelo 1:
% ---------
% texto de encabezado izquierda:
%\lhead{\normalfont{Historia de la Música II}}
% texto encabezado centro:
%\chead{\textbf{Ejercicios}}
% texto de encabezado derecha:
%\rhead{\normalfont{curso: 2020/2021}}
% texto pié izquierdo:
%\lfoot{\small{\textit{}}}
% texto pié centrado:
%\cfoot{\textsc{Pág. \thepage }}
% texto pié derecho
%\rfoot{\textit{Pr. $\mathcal{A}$.Kaal}}
% ----------
% Modelo 2:
% ---------
% Encabezado y pié de página (textos)
% -----------------------------------
% texto de encabezado izquierda:
%
\lhead{
	\hrule
	\vspace*{0.20cm}
	\normalfont{Historia da Música I}
	\vspace*{0.10cm}
	%\hrule
}
% texto encabezado centro:
\chead{
	\textbf{Cuestionario de Exame}
	\vspace*{0.08cm}}
% texto de encabezado derecha:
\rhead{
	\normalfont{curso: 2021/2022}
	\vspace*{0.08cm}}
%
% texto pié izquierdo:
%\lfoot{
	%\begin{center}
		%\vspace*{0.20cm}
		%\hrule
		%\small{
		%Conservatorio Profesional de Música de Viveiro - Avda. da mariña s/n - (27850) Viveiro - Lugo
		%	}
	%\end{center}
%}
% texto pié centrado:
\cfoot{
	%\vspace*{0.30cm}
	%\hrule
	%\vspace*{0.90cm}
	\small{- Páx. \thepage -  }\\
	%\small{Conservatorio Profesional de Música de Viveiro}\\
	%\small{avda. da Mariña s/n}
}

% --------
%=====================Algo setup
\algblock{If}{EndIf}
\algcblock[If]{If}{ElsIf}{EndIf}
\algcblock{If}{Else}{EndIf}
\algrenewtext{If}{\textbf{si}}
\algrenewtext{Else}{\textbf{sinon}}
\algrenewtext{EndIf}{\textbf{finsi}}
\algrenewtext{Then}{\textbf{alors}}
\algrenewtext{While}{\textbf{tant que}}
\algrenewtext{EndWhile}{\textbf{fin tant que}}
\algrenewtext{Repeat}{\textbf{r\'ep\'eter}}
\algrenewtext{Until}{\textbf{jusqu'\`a}}
\algcblockdefx[Strange]{If}{Eeee}{Oooo}
[1]{\textbf{Eeee} "#1"}
{\textbf{Wuuuups\dots}}

\algrenewcommand\algorithmicwhile{\textbf{TantQue}}
\algrenewcommand\algorithmicdo{\textbf{Faire}}
\algrenewcommand\algorithmicend{\textbf{Fin}}
\algrenewcommand\algorithmicrequire{\textbf{Variables}}
\algrenewcommand\algorithmicensure{\textbf{Constante}}% replace ensure by constante
\algblock[block]{Begin}{End}
\newcommand\algo[1]{\textbf{algorithme} #1;}
\newcommand\vars{\textbf{variables } }
\newcommand\consts{\textbf{constantes}}
\algrenewtext{Begin}{\textbf{debut}}
\algrenewtext{End}{\textbf{fin}}
%================================
%================================

\setlength{\parskip}{1.25cm}
\setlength{\parindent}{1.25cm}
\tikzstyle{titregris} =
[draw=gray,fill=gray, shading = exersicetitle, %
text=gray, rectangle, rounded corners, right,minimum height=.3cm]
\pgfdeclarehorizontalshading{exersicebackground}{100bp}
{color(0bp)=(green!40); color(100bp)=(black!5)}
\pgfdeclarehorizontalshading{exersicetitle}{100bp}
{color(0bp)=(red!40);color(100bp)=(black!5)}
\newcounter{exercise}
\renewcommand*\theexercise{exercice \textbf{Audición}~n\arabic{exercise}}
\makeatletter
\def\mdf@@exercisepoints{}%new mdframed key:
\define@key{mdf}{exercisepoints}{%
\def\mdf@@exercisepoints{#1}
}
\mdfdefinestyle{exercisestyle}{%
outerlinewidth=1em,outerlinecolor=white,%
leftmargin=-1em,rightmargin=-1em,%
middlelinewidth=0.5pt,roundcorner=3pt,linecolor=black,
apptotikzsetting={\tikzset{mdfbackground/.append style ={%
shading = exersicebackground}}},
innertopmargin=0.1\baselineskip,
skipabove={\dimexpr0.1\baselineskip+0\topskip\relax},
skipbelow={-0.1em},
needspace=0.5\baselineskip,
frametitlefont=\sffamily\bfseries,
settings={\global\stepcounter{exercise}},
singleextra={%
\node[titregris,xshift=0.5cm] at (P-|O) %
{~\mdf@frametitlefont{\theexercise}~};
\ifdefempty{\mdf@@exercisepoints}%
{}%
{\node[titregris,left,xshift=-1cm] at (P)%
{~\mdf@frametitlefont{\mdf@@exercisepoints points}~};}%
},
firstextra={%
\node[titregris,xshift=1cm] at (P-|O) %
{~\mdf@frametitlefont{\theexercise}~};
\ifdefempty{\mdf@@exercisepoints}%
{}%
{\node[titregris,left,xshift=-1cm] at (P)%
{~\mdf@frametitlefont{\mdf@@exercisepoints points}~};}%
},
}
\makeatother


%%%%%%%%%

%%%%%%%%%%%%%%%
\mdfdefinestyle{theoremstyle}{%
outerlinewidth=0.01em,linecolor=black,middlelinewidth=0.5pt,%
frametitlerule=true,roundcorner=2pt,%
apptotikzsetting={\tikzset{mfframetitlebackground/.append style={%
shade,left color=white, right color=blue!20}}},
frametitlerulecolor=black,innertopmargin=1\baselineskip,%green!60,
innerbottommargin=0.5\baselineskip,
frametitlerulewidth=0.1pt,
innertopmargin=0.7\topskip,skipabove={\dimexpr0.2\baselineskip+0.1\topskip\relax},
frametitleaboveskip=1pt,
frametitlebelowskip=1pt
}
\setlength{\parskip}{0mm}
\setlength{\parindent}{10mm}
\mdtheorem[style=theoremstyle]{ejercicio}{\textbf{Audición}}
%================Liste definition--numList-and alphList=============
\newcounter{alphListCounter}
\newenvironment
{alphList}
{\begin{list}
{\alph{alphListCounter})}
{\usecounter{alphListCounter}
\setlength{\rightmargin}{0cm}
\setlength{\leftmargin}{0.5cm}
\setlength{\itemsep}{0.2cm}
\setlength{\partopsep}{0cm}
\setlength{\parsep}{0cm}}
}
{\end{list}}
\newcounter{numListCounter}
\newenvironment
{numList}
{\begin{list}
{\arabic{numListCounter})}
{\usecounter{numListCounter}
\setlength{\rightmargin}{0cm}
\setlength{\leftmargin}{0.5cm}
\setlength{\itemsep}{0cm}
\setlength{\partopsep}{0cm}
\setlength{\parsep}{0cm}}
}
{\end{list}}
%
%% -- Fin del archivo de configuración  --
%%
% --------------------------------
\usepackage{graphicx}
\setlength{\columnsep}{1.25cm} % separamos
%
\begin{document}
\hideanswers % ocultamos solucións cuestionario

% Cabeceira do exame:
% -------------------
%%%%%%%%%%%%%%%%%%%%%%%%%%%%%%%%%%%%%%%%%%%%%%
%% CABECEIRA PARA EXAME DE 5º CURSO DE HISTORIA %%
%%%%%%%%%%%%%%%%%%%%%%%%%%%%%%%%%%%%%%%%%%%%%%
%
\thispagestyle{empty}
\begin{center}
    \Large{ % TÍTULO FOLLA EXERCICIOS
    Conservatorio Profesional de Música de Viveiro\\
    \vspace*{0.30cm}
    \large{
    Historia da Música 1º - PROBA ORDINARIA - 2a 
    Avaliación}\\
}
    \vspace*{0.50cm}
\end{center}
\normalsize
% Espazo datos alumnado:
% ----------------------
%
    \begin{tabular}{l l l}
    Nome e Apelidos: ............................................................................................... & Curso: .......................... \\
    \end{tabular}
\par
\vspace*{0.50cm}
%
% Fin da cabeceira do exame % 2a Avaliación

% Cabeceira Parte B:
% ------------------
%%%%%%%%%%%%%%%%%%%%%%%%%%%%%%%%%%%%%%%%%%%%%%%%%%
%% CABECEIRA PARA EXAME DE 4º CURSO DE HISTORIA %%
%% PARTE B                                      %%
%%%%%%%%%%%%%%%%%%%%%%%%%%%%%%%%%%%%%%%%%%%%%%%%%%
%
\begin{center}
    \large (Parte B)
\end{center}
\vspace*{0.15cm}

% Instruccións para audicións:
% ----------------------------
%%%%%%%%%%%%%%%%%%%%%%%%%%%%%%%%%%%%%%%%%%%%%%%%%%
%% CABECEIRA PARA EXAME DE 4º CURSO DE HISTORIA %%
%% Audicións Parte B                            %%
%%%%%%%%%%%%%%%%%%%%%%%%%%%%%%%%%%%%%%%%%%%%%%%%%%
%
\begin{instruccions}

\begin{center}
\texttt{Procedemento:}    
\end{center}
\begin{enumerate}
    \item
    AUDICIÓN 1: \textbf{1 minuto}. 
    REFLEXIÓN: \textbf{1 minuto} completar ficha 1.
    \item 
    AUDICIÓN 2: \textbf{1 minuto}. 
    REFLEXIÓN: \textbf{1 minuto} completar ficha 2.
\end{enumerate}
\begin{center}
    Repetirase o precedemento outra vez.
\end{center} 
\end{instruccions}
\vspace*{0.15cm}
% Fin da cabeceira parte B


% ESPACIO PARA AS AUDICIÓNS:
% --------------------------
%
%
% AUDICIÓNS DO PRIMEIRO TRIMESTRE:
% ================================
% Creamos as fichas de audición
% Cargamos as audicións no exame
% Utiliza:
%\useproblem{Audicion-N} % (N = no.audición)
%\begin{multicols}{2}
%%%%%%%%%%%%%%%%%%%%%%%%
%%%%% AUDICIÓN 1 %%%%%%%
%%%%%%%%%%%%%%%%%%%%%%%%

\begin{defproblem}{Audicion-01}

\begin{ejercicio}[]
	\begin{enumerate}[1.-]
   \begin{multicols}{2}
% Pregunta:
% 
%			\vspace*{0.1cm}
		\item
			AUTOR: \dotfill 
		\item 
		    OBRA: 
		    \begin{enumerate}[a)]
		    \item
			\textbf{Título}: \dotfill
		    \item
			\textbf{Timbre}: \par
				\small{Segundo as características da obra e audición, diferenciamos: (indica a correcta)}
				\begin{enumerate}
				 \item
				 \small{Coro de voces masculinas}
				 \item
				 \small{Coro de voces masculinas \emph{a capella}}
				 \item
				 \small{Coro de voces femininas}
				 \item
				 \small{Coro de voces mixto \emph{a capella}}
				\end{enumerate}

			\normalsize
%			\vspace*{0.2cm}	
		    \item 
		    \textbf{Textura}: \\
		    \small{Indica con un $x$ a textura que corresponda.} \par
			\begin{tabular}{lllll} 
			    \ldots & \small{melódica horizontal, monódica} & & \\
			    \ldots & \small{melódica horizontal, polifónica} & & \\
			    \ldots & \small{melodica vertical, homofónica} & & \\
			    \ldots & \small{non melódica} & & \\
			\end{tabular}
            \normalsize
            \item
			\textbf{Forma}: \\
				\small{Indica con un $x$ as que correspondan:} \par
			\begin{tabular}{lllll} 
			     \small{Maior} & \ldots \\ 
			     \small{Menor} & \ldots \\ 
			     \small{Vocal} & \ldots \\ 
			     \small{Instrumental} & \ldots \\ 
			     \small{Mixta} & \ldots \\ 
			     \small{Estruturada (forma fixa)} & \ldots \\ 
			     \small{Libre (sen estrutura coñecida)} & \ldots \\ 
			\end{tabular}
            \normalsize
%			\vspace*{0.2cm}			
		    \item
			\textbf{Estilo}: \\
				\small{Indica con un $x$ o estilo de canto e interpretación da obra:} \par
			\begin{tabular}{llllll}
			     \ldots & \small{Silábico} & & & \ldots & Directo  \\
			     \ldots & \small{Neumático} & & & \ldots & Antifonal   \\
			     \ldots & \small{Melismático} & & & \ldots & Responsorial  \\
			\end{tabular}
            \normalsize
		    \item
			\textbf{Clasificación}: \\
				\small{
				Segundo o ámbito e estilo da obra, estamos ante:
				%\ldots
				}
				\begin{enumerate}
				 \item
				 \small{Un \emph{introito} do <<propio>> da misa}
				 \item
				 \small{Un \emph{gradual} do <<propio>> da misa}
				 \item
				 \small{Unha \emph{sequentia}}
				 \item
				 \small{Un drama litúrxico}
				\end{enumerate}
		    \end{enumerate}
%			
       \end{multicols}
\par \vspace{0.3cm}
	\end{enumerate}
\end{ejercicio}

% Solucións da Audición 1:
% ------------------------
\begin{onlysolution}
    \begin{solution}
Autor: o que sexa\\
Obra: a que sexa\\
    \end{solution}
\end{onlysolution}

\end{defproblem}
% 
% Fin da ficha de audición
\begin{defproblem}{Audicion-01}
\begin{ejercicio}[]
	\begin{enumerate}[1.-]
        \vspace*{0.3cm}
		\item
			Autor: \dotfill
			\vspace*{0.3cm}
		\item
			Obra:
			\begin{enumerate}[a)]
			    \item Título: \dotfill \vspace*{0.3cm}
			    \item Período: \dotfill \vspace*{0.3cm}
			    \item Forma: \dotfill \vspace*{0.3cm}
			    \item Timbre: \dotfill \vspace*{0.3cm} 		
			    \item Textura: \dotfill \vspace*{0.3cm}
			    \item Estilo: \dotfill \vspace*{0.3cm}
			    \item Xénero: \dotfill \vspace*{0.3cm}
			\end{enumerate}
		\item 
		    Resume as principais características que definen a obra:
			\vspace*{8.0cm}			
	\end{enumerate}
\end{ejercicio}

\begin{ejercicio}[]
	\begin{enumerate}[1.-]
        \vspace*{0.3cm}
		\item
			Autor: \dotfill
			\vspace*{0.3cm}
		\item
			Obra:
			\begin{enumerate}[a)]
			    \item Título: \dotfill \vspace*{0.3cm}
			    \item Período: \dotfill \vspace*{0.3cm}
			    \item Forma: \dotfill \vspace*{0.3cm}
			    \item Timbre: \dotfill \vspace*{0.3cm} 		
			    \item Textura: \dotfill \vspace*{0.3cm}
			    \item Estilo: \dotfill \vspace*{0.3cm}
			    \item Xénero: \dotfill \vspace*{0.3cm}
			\end{enumerate}
		\item 
		    Resume as principais características que definen a obra:
			\vspace*{8.0cm}			
	\end{enumerate}
\end{ejercicio}
\end{defproblem}

%
% Ficha audición 1
% ----------------
\useproblem{Audicion-01}
\newpage
%
% Ficha audición 2
% ----------------
\vspace*{1.10cm}
\useproblem{Audicion-01}
\vspace*{3.30cm}
%
% Instruccións fin de proba:
% --------------------------
%%%%%%%%%%%%%%%%%%%%%%%%%%%%%%%%%%%%%%%%%%
%%   INSTRUCCIÓNS FINAIS DA PARTE B     %%
%%               PARTE A                %%
%%%%%%%%%%%%%%%%%%%%%%%%%%%%%%%%%%%%%%%%%%
%
\begin{center}
  \large{FIN DA PROBA DE AUDICIÓN}
  \par
  \vspace{1.0cm}
  Entrega este cuestionario \\
  indicando \textbf{nome} e \textbf{apelidos} \\
  dentro do cuestionario da Parte A \\
  \par
  \vspace{0.75cm}
  \textbf{Os cuestionarios sen nome e apelidos considéranse non realizados}
\end{center}
%\newpage
%

%\newpage

%\end{multicols}
\end{document}
% Fin de Hoja de ejercicios
