%
%%%%%%%%%%%%%%%%%%%%%%%%%%%%%%%%%%%%%%%%%%%%%%%%%%
% PLANTILLA EJERCICIOS DE HISTORIA DE LA MÚSICA I
% Este es un modelo para redactar los ejercicios
% 
% Pasos para cubrir la plantilla:
% 1) Realizar una copia de este modelo
% 2) Renombrar el archivo:
%		"HM1_Hoja(número).tex"
% 3) El número de Hoja debe ser correlativo
% 
%%%%%%%%%%%%%%%%%%%%%%%%%%%%%%%%%%%%%%%%%%%%%%%%%%
%
% Esta plantilla es para crear ejercicios de esta materia
% Se recomienda crear un archivo por cada tema
% Descomentar según se necesite utilizar un modelo de ejercicio u otro
% Clase de documento:
\documentclass[letterpaper,12pt,notitlepage,spanish]{article}
%
% CONFIGURACIÓN:
% Ruta de figuras y demás:
% configurar cuando se redacte el documento final
% indicar ruta donde se encuentre lo que se quiera adjuntar.
%
%\graphicspath{ {/home/user/images/} }
%
% Archivo externo de configuración
% --------------------------------
\input{../../Modelos/include/config-HM1examen_A_Parte2.tex}
% --------------------------------
\usepackage{graphicx}
\setlength{\columnsep}{1.25cm} % separamos
%
\begin{document}
\hideanswers % ocultamos solucións cuestionario

% Cabeceira do exame:
% -------------------
%%%%%%%%%%%%%%%%%%%%%%%%%%%%%%%%%%%%%%%%%%%%%%
%% CABECEIRA PARA EXAME DE 5º CURSO DE HISTORIA %%
%%%%%%%%%%%%%%%%%%%%%%%%%%%%%%%%%%%%%%%%%%%%%%
%
\thispagestyle{empty}
\begin{center}
    \Large{ % TÍTULO FOLLA EXERCICIOS
    Conservatorio Profesional de Música de Viveiro\\
    \vspace*{0.30cm}
    \large{
    Historia da Música 1º - PROBA ORDINARIA - 2a 
    Avaliación}\\
}
    \vspace*{0.50cm}
\end{center}
\normalsize
% Espazo datos alumnado:
% ----------------------
%
    \begin{tabular}{l l l}
    Nome e Apelidos: ............................................................................................... & Curso: .......................... \\
    \end{tabular}
\par
\vspace*{0.50cm}
%
% Fin da cabeceira do exame

% Cabeceira Parte B:
% ------------------
%%%%%%%%%%%%%%%%%%%%%%%%%%%%%%%%%%%%%%%%%%%%%%%%%%
%% CABECEIRA PARA EXAME DE 4º CURSO DE HISTORIA %%
%% PARTE B                                      %%
%%%%%%%%%%%%%%%%%%%%%%%%%%%%%%%%%%%%%%%%%%%%%%%%%%
%
\begin{center}
    \large (Parte B)
\end{center}
\vspace*{0.15cm}

% Instruccións para audicións:
% ----------------------------
%%%%%%%%%%%%%%%%%%%%%%%%%%%%%%%%%%%%%%%%%%%%%%%%%%
%% CABECEIRA PARA EXAME DE 4º CURSO DE HISTORIA %%
%% Audicións Parte B                            %%
%%%%%%%%%%%%%%%%%%%%%%%%%%%%%%%%%%%%%%%%%%%%%%%%%%
%
\begin{instruccions}

\begin{center}
\texttt{Procedemento:}    
\end{center}
\begin{enumerate}
    \item
    AUDICIÓN 1: \textbf{1 minuto}. 
    REFLEXIÓN: \textbf{1 minuto} completar ficha 1.
    \item 
    AUDICIÓN 2: \textbf{1 minuto}. 
    REFLEXIÓN: \textbf{1 minuto} completar ficha 2.
\end{enumerate}
\begin{center}
    Repetirase o precedemento outra vez.
\end{center} 
\end{instruccions}
\vspace*{0.15cm}
% Fin da cabeceira parte B


% ESPACIO PARA AS AUDICIÓNS:
% --------------------------
% AUDICIÓN 1
\begin{defproblem}{Audicion-01}
\begin{multicols}{2}
\begin{ejercicio}[]
	\begin{enumerate}[1.-]
        \vspace*{0.3cm}
		\item
			Autor: \dotfill
			\vspace*{0.3cm}
		\item
			Obra:
			\begin{enumerate}[a)]
			    \item Título: \dotfill \vspace*{0.3cm}
			    \item Período: \dotfill \vspace*{0.3cm}
			    \item Forma: \dotfill \vspace*{0.3cm}
			    \item Timbre: \dotfill \vspace*{0.3cm} 		
			    \item Textura: \dotfill \vspace*{0.3cm}
			    \item Estilo: \dotfill \vspace*{0.3cm}
			    \item Xénero: \dotfill \vspace*{0.3cm}
			\end{enumerate}
%		\item 
%		    Resume as principais características que definen a obra:
			\vspace*{2.0cm}			
	\end{enumerate}
\end{ejercicio}

\begin{ejercicio}[]
	\begin{enumerate}[1.-]
        \vspace*{0.3cm}
		\item
			Autor: \dotfill
			\vspace*{0.3cm}
		\item
			Obra:
			\begin{enumerate}[a)]
			    \item Título: \dotfill \vspace*{0.3cm}
			    \item Período: \dotfill \vspace*{0.3cm}
			    \item Forma: \dotfill \vspace*{0.3cm}
			    \item Timbre: \dotfill \vspace*{0.3cm} 		
			    \item Textura: \dotfill \vspace*{0.3cm}
			    \item Estilo: \dotfill \vspace*{0.3cm}
			    \item Xénero: \dotfill \vspace*{0.3cm}
			\end{enumerate}
%		\item 
%		    Resume as principais características que definen a obra:
			\vspace*{2.0cm}			
	\end{enumerate}
\end{ejercicio}
\end{multicols}
\end{defproblem}
\useproblem{Audicion-01} % cargamos audición
\newpage
%
% AUDICIÓN 2
%
\useproblem{Audicion-01} % cargamos ficha 2
\vspace*{3.30cm}
\begin{center}
    \large{FIN DA PROBA DE AUDICIÓN}
    \par
    \vspace{1.0cm}
    Entrega este cuestionario \\
    indicando \textbf{nome} e \textbf{apelidos} \\
    dentro do cuestionario da Parte A \\
    \par
    \vspace{0.75cm}
    \textbf{Os cuestionarios sen nome e apelidos considéranse non realizados}
\end{center}
\newpage

%\end{multicols}
\end{document}
% Fin de Hoja de ejercicios