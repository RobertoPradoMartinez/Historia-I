%%%%%%%%%%%%%%%%%%%%%%%%%%%%%%%%%%%%%%%%%%%%%%%%%%%%%
%% DOCUMENTO PRINCIPAL PARA CREAR OS EXAMES DE HM1 %%
%%%%%%%%%%%%%%%%%%%%%%%%%%%%%%%%%%%%%%%%%%%%%%%%%%%%%
%%              Versión 2021/2022                  %%
%%                                                 %%
%% Notas sobre a versión: Ver README.md            %%
%%%%%%%%%%%%%%%%%%%%%%%%%%%%%%%%%%%%%%%%%%%%%%%%%%%%%
%
% ---------------
% PREÁMBULO LATEX 
% ---------------
%
% Tipo de documento e características:
% ------------------------------------
\documentclass[letterpaper,12pt,notitlepage,spanish]{article}
%
% Dependencias:
% -------------
%\graphicspath{ {/home/user/images/} }        % carpeta de figuras
\usepackage{graphicx}                         % paquete para gráficos
\setlength{\columnsep}{0.75cm}                % separación de columnas
%
% Configuración personalizada para exame:
% ---------------------------------------
%%%%%%%%%%%%%%%%%%%%%%%%%%%%%%%%%%%%%%%%%%%
%% ---------- MODELO EJERCICIOS ---------- 
%% MATERIA: HISTORIA
%% CURSO: 
%% AÑO ACADÉMICO: 
%% CENTRO: 
%%%%%%%%%%%%%%%%%%%%%%%%%%%%%%%%%%%%%%%%%%%
%% 
%% MODELO PARA REDACTAR EJERCICIOS
%% ===============================
%% 
%% Clase de documento
%% ------------------
%\documentclass[letterpaper,12pt,notitlepage,spanish]{article}
%\documentclass[12pt,a4paper,notitlepage]{article}
%
% Márgenes de documento
% ---------------------
\usepackage[left=2.0cm, right=2.0cm, lines=45, top=2.5cm, bottom=2.0cm]{geometry}
%
% Paquetes necesarios
% -------------------
\usepackage[utf8]{inputenc} % acentos en ES
\usepackage[spanish,activeacute, es-tabla]{babel}
\usepackage{enumerate} % entornos de listas
\usepackage{multicol}  % varias columnas texto
\usepackage{fancyhdr}  % encabezado personalizado
\usepackage{fancybox}  % entornos con cajas
\usepackage{pdfpages}  % páginas pdf
\usepackage{lastpage}  % numeración páxinas
\usepackage{amssymb}   % completar líneas
\usepackage{xcolor}  % color en exames
%
\usepackage{environ} 
\usepackage{probsoln} % paquete para soluciones
%\showanswers % para mostrar soluciones
% --->
% Recuadros y figuras
% -------------------
% Código de "OndaHostil"
\usepackage{tcolorbox}        %recuadros colores
\tcbuselibrary{listingsutf8}  % estilos recuadro
% Definir cuadro de ancho del texto
\newtcolorbox{cadro1}[1]{colback=red!5!white,colframe=red!75!black,fonttitle=\bfseries,title=#1}
% OPCIONES
% colback: color de fondo
% colframe: color de borde
% fonttitle: estilo de título
% title: título de la cuadro o referencia a argumento

% Cuadro estrecho
\newtcbox{cadro2}[1]{colback=grey!5!white,colframe=grey!75!white,fonttitle=\bfseries,title=#1}

% Cadro en gris
\newtcolorbox{datos-alumnado}[1]{colback=gray!10!white,colframe=gray!50!black,fonttitle=\bfseries, title=NOME E APELIDOS/CALIFICACIÓN:
\vspace*{0.10cm}}

% Cadro en azul para instruccións:
\definecolor{lila1}{RGB}{193,124,250} % lila
\definecolor{cornflowerblue}{RGB}{100,149,237} % azul suave
%
\newtcolorbox{instruccions}[1]{colback=gray!2!white,colframe=gray!50!cornflowerblue,fonttitle=\bfseries, title=\centering{INSTRUCCIÓNS PARA REALIZAR A PROBA}
\vspace*{0.10cm}}
% Cuadro numerado para ejemplos
\newtcolorbox[auto counter,number within=section]{example}[2][]
{colback=green!5!white,colframe=green!75!black,fonttitle=\bfseries, title=Exemplo~\thetcbcounter: #2,#1}

\usepackage[colorlinks, linkcolor=black]{hyperref}
%
% SOLUCIÓNS AOS EXAMES:
% ---------------------
%% Esto es lo importante. Ponemos la solución al margen.
%\NewEnviron{solutionnew}{%
%%  \leavevmode\marginpar{\raggedright\footnotesize \textbf{Solución:}\\ \BODY}
%%  \textbf{Solución:}\\ \BODY} % sol. con salto de liña
%  \small{Solución:} \BODY} % sol. na mesma liña
%  {}
%\renewenvironment{solution}{\solutionnew}{\endsolutionnew}
% ---
% Solución resaltada en cor lima:
% -------------------------------
%Esto es lo importante. Ponemos la solución al margen.
\NewEnviron{solutionnew}{%
%  \leavevmode\marginpar{\raggedright\footnotesize \textbf{Solución:}\\ \BODY}
%  \textbf{Solución:}\\ \BODY} % sol. con salto de liña
  \small{\colorbox{lime}{Solución:} \BODY}} % sol. na mesma liña
  {}
\renewenvironment{solution}{\solutionnew}{\endsolutionnew}
% ---
% --------
% Lineas de encabezado y pié
% --------------------------
\renewcommand{\headrulewidth}{0.5pt}
%\renewcommand{\headrulewidth}{1.0pt}
\renewcommand{\footrulewidth}{0.5pt}
%\renewcommand{\footrulewidth}{1.0pt}
\pagestyle{fancy} % estilo de página
%
% Recuadros y figuras
% -------------------
\newcommand\Loadedframemethod{TikZ}
\usepackage[framemethod=\Loadedframemethod]{mdframed}
\usepackage{tikz}
\usetikzlibrary{calc,through,backgrounds}
\usetikzlibrary{matrix,positioning}
%Desssins geometriques
\usetikzlibrary{arrows}
\usetikzlibrary{shapes.geometric}
\usetikzlibrary{datavisualization}
\usetikzlibrary{automata} % LATEX and plain TEX
\usetikzlibrary{shapes.multipart}
\usetikzlibrary{decorations.pathmorphing} 
\usepackage{pgfplots}
\usepackage{physics}
\usepackage{titletoc}
\usepackage{mathpazo} 
\usepackage{algpseudocode}
\usepackage{algorithmicx} 
\usepackage{bohr} 
\usepackage{xlop} 
\usepackage{bbding} 
%\usepackage{minibox} 
% Texto árabe
\usepackage{mathdesign}
\usepackage{bbding} 
% --
% Tipograía:
% ----------
% Fuente HEURÍSTICA (cómoda de leer)
%\usepackage{heuristica}
% Fuente LIBERTINE (cómoda para apuntes)
%\usepackage{libertineRoman}
\usepackage[proportional]{libertine}
% Fuente ROMANDE (estilo antiguo pero no muy cómoda)
%\usepackage{romande} %

%=====================Algo setup
\algblock{If}{EndIf}
\algcblock[If]{If}{ElsIf}{EndIf}
\algcblock{If}{Else}{EndIf}
\algrenewtext{If}{\textbf{si}}
\algrenewtext{Else}{\textbf{sinon}}
\algrenewtext{EndIf}{\textbf{finsi}}
\algrenewtext{Then}{\textbf{alors}}
\algrenewtext{While}{\textbf{tant que}}
\algrenewtext{EndWhile}{\textbf{fin tant que}}
\algrenewtext{Repeat}{\textbf{r\'ep\'eter}}
\algrenewtext{Until}{\textbf{jusqu'\`a}}
\algcblockdefx[Strange]{If}{Eeee}{Oooo}
[1]{\textbf{Eeee} "#1"}
{\textbf{Wuuuups\dots}}

\algrenewcommand\algorithmicwhile{\textbf{TantQue}}
\algrenewcommand\algorithmicdo{\textbf{Faire}}
\algrenewcommand\algorithmicend{\textbf{Fin}}
\algrenewcommand\algorithmicrequire{\textbf{Variables}}
\algrenewcommand\algorithmicensure{\textbf{Constante}}% replace ensure by constante
\algblock[block]{Begin}{End}
\newcommand\algo[1]{\textbf{algorithme} #1;}
\newcommand\vars{\textbf{variables } }
\newcommand\consts{\textbf{constantes}}
\algrenewtext{Begin}{\textbf{debut}}
\algrenewtext{End}{\textbf{fin}}
%================================
%================================

\setlength{\parskip}{1.25cm}
\setlength{\parindent}{1.25cm}
\tikzstyle{titregris} =
[draw=gray,fill=gray, shading = exersicetitle, %
text=gray, rectangle, rounded corners, right,minimum height=.3cm]
\pgfdeclarehorizontalshading{exersicebackground}{100bp}
{color(0bp)=(green!40); color(100bp)=(black!5)}
\pgfdeclarehorizontalshading{exersicetitle}{100bp}
{color(0bp)=(red!40);color(100bp)=(black!5)}
\newcounter{exercise}
\renewcommand*\theexercise{exercice \textbf{Ejercicio}~n\arabic{exercise}}
\makeatletter
\def\mdf@@exercisepoints{}%new mdframed key:
\define@key{mdf}{exercisepoints}{%
\def\mdf@@exercisepoints{#1}
}
\mdfdefinestyle{exercisestyle}{%
outerlinewidth=1em,outerlinecolor=white,%
leftmargin=-1em,rightmargin=-1em,%
middlelinewidth=0.5pt,roundcorner=3pt,linecolor=black,
apptotikzsetting={\tikzset{mdfbackground/.append style ={%
shading = exersicebackground}}},
innertopmargin=0.1\baselineskip,
skipabove={\dimexpr0.1\baselineskip+0\topskip\relax},
skipbelow={-0.1em},
needspace=0.5\baselineskip,
frametitlefont=\sffamily\bfseries,
settings={\global\stepcounter{exercise}},
singleextra={%
\node[titregris,xshift=0.5cm] at (P-|O) %
{~\mdf@frametitlefont{\theexercise}~};
\ifdefempty{\mdf@@exercisepoints}%
{}%
{\node[titregris,left,xshift=-1cm] at (P)%
{~\mdf@frametitlefont{\mdf@@exercisepoints points}~};}%
},
firstextra={%
\node[titregris,xshift=1cm] at (P-|O) %
{~\mdf@frametitlefont{\theexercise}~};
\ifdefempty{\mdf@@exercisepoints}%
{}%
{\node[titregris,left,xshift=-1cm] at (P)%
{~\mdf@frametitlefont{\mdf@@exercisepoints points}~};}%
},
}
\makeatother


%%%%%%%%%

%%%%%%%%%%%%%%%
\mdfdefinestyle{theoremstyle}{%
outerlinewidth=0.01em,linecolor=black,middlelinewidth=0.5pt,%
frametitlerule=true,roundcorner=2pt,%
apptotikzsetting={\tikzset{mfframetitlebackground/.append style={%
shade,left color=white, right color=blue!20}}},
frametitlerulecolor=black,innertopmargin=1\baselineskip,%green!60,
innerbottommargin=0.5\baselineskip,
frametitlerulewidth=0.1pt,
innertopmargin=0.7\topskip,skipabove={\dimexpr0.2\baselineskip+0.1\topskip\relax},
frametitleaboveskip=1pt,
frametitlebelowskip=1pt
}
\setlength{\parskip}{0mm}
\setlength{\parindent}{10mm}
\mdtheorem[style=theoremstyle]{ejercicio}{\textbf{Ejercicio}}
%================Liste definition--numList-and alphList=============
\newcounter{alphListCounter}
\newenvironment
{alphList}
{\begin{list}
{\alph{alphListCounter})}
{\usecounter{alphListCounter}
\setlength{\rightmargin}{0cm}
\setlength{\leftmargin}{0.5cm}
\setlength{\itemsep}{0.2cm}
\setlength{\partopsep}{0cm}
\setlength{\parsep}{0cm}}
}
{\end{list}}
\newcounter{numListCounter}
\newenvironment
{numList}
{\begin{list}
{\arabic{numListCounter})}
{\usecounter{numListCounter}
\setlength{\rightmargin}{0cm}
\setlength{\leftmargin}{0.5cm}
\setlength{\itemsep}{0cm}
\setlength{\partopsep}{0cm}
\setlength{\parsep}{0cm}}
}
{\end{list}}
%
%% -- Fin del archivo de configuración  --
%%

%
% Fin do preámbulo de LaTeX
% -------------------------
%
%
% MAQUETACIÓN DA PROBA DE TEORÍA PARTE A:
% ---------------------------------------
%
\begin{document}
%
% -----------------------
% 1.- Cabeceira do exame:
% -----------------------
%
% Cabeceira:
% ----------
%
% 1a Avaliación:
% --------------
%%%%%%%%%%%%%%%%%%%%%%%%%%%%%%%%%%%%%%%%%%%%%%%
%% CABECEIRA PARA EXAME DE 5º CURSO DE HISTORIA %%
%%%%%%%%%%%%%%%%%%%%%%%%%%%%%%%%%%%%%%%%%%%%%%
%
\thispagestyle{empty}
\begin{center}
    \Large{ % TÍTULO FOLLA EXERCICIOS
    Conservatorio Profesional de Música de Viveiro\\
    \vspace*{0.30cm}
    \large{
    Historia da Música 1º - PROBA ORDINARIA - 1a 
    Avaliación}\\
}
    \vspace*{0.50cm}
\end{center}
\normalsize
% Espazo datos alumnado:
% ----------------------
%
    \begin{tabular}{l l l}
    Nome e Apelidos: ............................................................................................... & Curso: .......................... \\
    \end{tabular}
\par
\vspace*{0.50cm}
%
% Fin da cabeceira do exame
%
% 2a Avaliación:
% --------------
%%%%%%%%%%%%%%%%%%%%%%%%%%%%%%%%%%%%%%%%%%%%%%
%% CABECEIRA PARA EXAME DE 5º CURSO DE HISTORIA %%
%%%%%%%%%%%%%%%%%%%%%%%%%%%%%%%%%%%%%%%%%%%%%%
%
\thispagestyle{empty}
\begin{center}
    \Large{ % TÍTULO FOLLA EXERCICIOS
    Conservatorio Profesional de Música de Viveiro\\
    \vspace*{0.30cm}
    \large{
    Historia da Música 1º - PROBA ORDINARIA - 2a 
    Avaliación}\\
}
    \vspace*{0.50cm}
\end{center}
\normalsize
% Espazo datos alumnado:
% ----------------------
%
    \begin{tabular}{l l l}
    Nome e Apelidos: ............................................................................................... & Curso: .......................... \\
    \end{tabular}
\par
\vspace*{0.50cm}
%
% Fin da cabeceira do exame
%
% 3a Avaliación:
% --------------
%%%%%%%%%%%%%%%%%%%%%%%%%%%%%%%%%%%%%%%%%%%%%%%
%% CABECEIRA PARA EXAME DE 5º CURSO DE HISTORIA %%
%%%%%%%%%%%%%%%%%%%%%%%%%%%%%%%%%%%%%%%%%%%%%%
%
\thispagestyle{empty}
\begin{center}
    \Large{ % TÍTULO FOLLA EXERCICIOS
    Conservatorio Profesional de Música de Viveiro\\
    \vspace*{0.30cm}
    \large{
    Historia da Música 1º - PROBA ORDINARIA - 3a 
    Avaliación}\\
}
    \vspace*{0.50cm}
\end{center}
\normalsize
% Espazo datos alumnado:
% ----------------------
%
    \begin{tabular}{l l l}
    Nome e Apelidos: ............................................................................................... & Curso: .......................... \\
    \end{tabular}
\par
%\vspace*{0.250cm}
%
% Fin da cabeceira do exame

%
% ---------------------------------------
% 2.- Instruccións para realizar a proba:
% ---------------------------------------
% INSTRUCCIÓNS PARA REALIZAR EXAME
% PARTE A
%
\begin{instruccions}

    \par
    \vspace*{0.15cm}
    \begin{center} % AVISO CENTRAL
    \texttt{
    NON COMECES A PROBA ATA QUE SE INDIQUE.
    }
    \end{center}
    \begin{center} % AVISO MÓBILES
    \textbf{
    Os teléfonos móbiles deben estar silenciados ou apagados.
    }
    \end{center}
    Está \textbf{totalmente prohibido} o uso de calquera dispositivo electrónico, material de apoio non autorizado, falar, ou pedir consello ao resto de compañeiros ou compañeiras no tempo establecido para realizar esta proba.
    \par
    \vspace*{0.2cm}
    O incumprimento do parágrafo anterior suporá automáticamente a \textbf{expulsión da proba} (Partes A e B), dando por non superada a avaliación trimestral da materia. 
    \par
    \vspace*{0.2cm}
    Completa o \textbf{nome e apelidos} en maiúsculas con letra lexible antes de comezar.
    \par
    Toda proba na que non figuren o nome e apelidos non será calificada.
    \par
    \vspace*{0.15cm}
    \textbf{Non se admiten} respostas a lapis nin en bolígrafo que non sexa de cor azul ou negra.\\
    Para responder as cuestións, \textbf{rodea a letra da resposta} que consideres oportuna: se te equivocas ou queres cambiar a túa resposta, \textbf{anula cun X} e rodea a nova resposta.
    \par
    \vspace*{0.3cm}
    \textbf{Se tes dúbidas} sobre algunha cuestión desta proba que se poida aclarar, levanta a man para non desconcentrar ao resto de compañeiros e compañeiras que realizan a proba.
    \par
    \vspace*{0.20cm}
    Se finalizas a proba agarda en silencio ata que o resto de compañeiras e compañeiros rematen. Non se pode abandonar a aula no tempo establecido para realizar ámbalas dúas probas (A e B).
    \par
    \vspace*{0.15cm}
    \begin{center}
    \texttt{SISTEMA DE CALIFICACIÓN.}
    \end{center}
    \par
    \vspace*{0.15cm}
    A \textbf{Parte A} desta proba consta dun total de \textbf{20 preguntas} de resposta única.\\
    A \textbf{Parte B} consta de \textbf{5 preguntas}.
    \par
    \vspace*{0.15cm}
    Tempo total Parte A: 40 minutos.
    \par
    Tempo total Parte B: 10 minutos.
    \par
    \vspace*{0.15cm}
    \textbf{Calificación} da parte A: \\
    0.5 puntos resposta correcta; -0.25 resposta non correcta; 0 puntos, cuestión sen resposta ou resposta nula.\\
%    \vspace*{0.15cm}
     \textbf{Calificación} da parte B:\\
     0.5 puntos resposta correcta; -0.25 resposta non correcta; 0 puntos, cuestión sen resposta ou resposta nula.
   
\end{instruccions}
%
\newpage
%
% Fin das instruccións do exame
%
% ----------------------
% 3.- Cabeceira Parte A:
% ----------------------
%
% 1a Avaliación:
% --------------
% %%%%%%%%%%%%%%%%%%%%%%%%%%%%%%%%%%%%%%%%%%%%%%%%%%
%% CABECEIRA PARA EXAME DE 4º CURSO DE HISTORIA %%
%% PARTE A                                      %%
%%%%%%%%%%%%%%%%%%%%%%%%%%%%%%%%%%%%%%%%%%%%%%%%%%
%
\begin{center}
    \large{
    PROBA ORDINARIA - 1a Avaliación}\\
    \vspace*{0.35cm}
    \large (Parte A)
\end{center}
\vspace*{0.15cm}
%
% 2a Avaliación:
% --------------
%%%%%%%%%%%%%%%%%%%%%%%%%%%%%%%%%%%%%%%%%%%%%%%%%%
%% CABECEIRA PARA EXAME DE 4º CURSO DE HISTORIA %%
%% PARTE A                                      %%
%%%%%%%%%%%%%%%%%%%%%%%%%%%%%%%%%%%%%%%%%%%%%%%%%%
%
\begin{center}
    \large{
    PROBA ORDINARIA - 2a Avaliación}\\
    \vspace*{0.35cm}
    \large (Parte A)
\end{center}
\vspace*{0.15cm}
%
% 3a Avaliación:
% --------------
% \input{../../include/config-HM2cabeceira3-parte-A.tex}
%
%
% -------------------------------------
% 4.- Cuestionario da proba avaliación:
% -------------------------------------
%
% - Amosar respostas:
% -------------------
\hideanswers        % oculta respostas
%\showanswers        % mostra respostas
%
% - Opcións de carga banco cuestións:
% -----------------------------------
%
% A) Cargar todas:
% ----------------
\loadallproblems[Tema3-NOTA]{../../Cuestions/Tema3-Notacion-modal.tex}
\loadallproblems[Tema3-TROBA]{../../Cuestions/Tema3-Monodia-Profana.tex}
\loadallproblems[Tema3-GREGO1]{../../Cuestions/Tema3-Canto-gregoriano-expansion.tex}
%\loadallproblems[Tema1-OR]{../../Cuestions/Tema1-Organoloxia.tex}
%\loadallproblems[Tema1-EX]{../../Cuestions/Tema1-Exipto.tex}
%\loadallproblems[Tema1-RO]{../../Cuestions/Tema1-Roma.tex}
%
% B) Cargar aleatoriamente: 
% -------------------------
% !TODO: REVISAR POR QUE PRODUCE ERRO !! 
%\loadrandomproblems[Tema3-TROBA]{../../Cuestions/Tema3-Monodia-Profana.tex}
%\loadrandomproblems[Tema1-PENS]{../../Cuestions/Tema1-Pensamento.tex}
%\loadrandomproblems[Tema1-GR]{../../Cuestions/Tema1-Grecia.tex}
%\loadrandomproblems[Tema1-OR]{../../Cuestions/Tema1-Organoloxia.tex}
%\loadrandomproblems[Tema1-EX]{../../Cuestions/Tema1-Exipto.tex}
%\loadrandomproblems[Tema1-RO]{../../Cuestions/Tema1-Roma.tex}
%
% C) Cargar manualmente (selectiva):
% ----------------------------------
\loadselectedproblems[Tema3-GREGO]{T3GREGO-11,T3GREGO-12,T3GREGO-05,T3GREGO-08,T3GREGO-10}{../../Cuestions/Tema3-Canto-gregoriano.tex}
%\loadselectedproblems[Tema1-GR]{T1GR-05,T1GR-07,T1GR-08,T1GR-02}{../../Cuestions/Tema1-Grecia.tex}
%\loadselectedproblems[Tema1-PE]{T1PE-05,T1PE-11,T1PE-12}{../../Cuestions/Tema1-Periodizacion.tex}
%\loadselectedproblems[Tema1-PENS]{T1PENS-03,T1PENS-04}{../../Cuestions/Tema1-Pensamento.tex}
%\loadselectedproblems[Tema1-RO]{T1RO-01}{../../Cuestions/Tema1-Roma.tex}
%
% D) Cargar por exclusión:
% ------------------------
%\loadexceptproblems[Tema1-CO]{T1CO-01,T1CO-02,T1CO-03,T1CO-04}{../../Cuestions/Tema1-Compositores.tex}
% 
% NUMERACIÓN PREGUNTAS E ESTRUTURA da PROBA:
% ------------------------------------------
\begin{multicols}{2}
	\begin{enumerate}[1.-]
    \foreachproblem[Tema3-NOTA]{\item\label{prob:\thisproblemlabel}\thisproblem} % Grecia
    \foreachproblem[Tema3-TROBA]{\item\label{prob:\thisproblemlabel}\thisproblem} % Pensamento
    \foreachproblem[Tema3-GREGO]{\item\label{prob:\thisproblemlabel}\thisproblem} % Organoloxía
    \foreachproblem[Tema3-GREGO1]{\item\label{prob:\thisproblemlabel}\thisproblem} % Roma
%    \foreachproblem[Tema1-EX]{\item\label{prob:\thisproblemlabel}\thisproblem}
%    \foreachproblem[Tema1-PE]{\item\label{prob:\thisproblemlabel}\thisproblem} % Periodización
    \end{enumerate}
\end{multicols}
%
%
% --------------------------------------
% 5.- Texto fin de cuestionario Parte A:
% --------------------------------------
% !TODO: PENDIENTE DE FACER !!
% (Incluír solamente se é necesario)
% \input{../../include/config-HM2instruccions-fin.tex}
\end{document}
% Fin de Hoja de ejercicios