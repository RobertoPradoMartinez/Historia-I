

% EXERCICIO PARA AUDICIÓN HIMNO A NÉMESIS
%
\begin{ejercicio}[Audición \textit{Himno a Némesis}]
%
Atendendo aos elementos mínimos que debes ter en conta para elaborar un comentario de audición (indicados no punto 2 do caderno de exercicios), completa a seguinte ficha.
%
    \begin{enumerate}[1.-]
    \item \textbf{Autor.} Indica o autor da obra (se é posible) \dotfill
    \item \textbf{Título da obra.} Indica o título da obra \dotfill
%    \begin{multicols}{2}
        \item 
        \textbf{Timbre.}
%        \begin{enumerate}
%        \item 
        Indica os instrumentos que recoñeces na audición, segundo a clasificación moderna.
        \vspace*{1.0cm}
%            \begin{enumerate}
%            \item Cordófonos \dotfill
%            \item Aerófonos \dotfill
%            \item Idiófonos \dotfill
%            \item Membranófonos\dotfill
%            \item Outros \dotfill
%            \end{enumerate}
%        \end{enumerate}
%    \end{multicols}
%
        \item
        \textbf{Textura.}
        Estamos ante textura de escrita horizontal ou vertical? \dotfill
        \par
        Cal é, das que coñeces? (homofonía, polifonía, monodia, etc.) \dotfill \par
%
        \item
        \textbf{Melodía.}
            Para determinar a textura, fíxate na melodía prestando atención a:
            \begin{enumerate}
            \item Que son é o son que máis se repite? \dotfill
            \item Por que ámbito se move? (2as, 3as, grandes saltos, ...) \dotfill
            \item Que fan os instrumentos con respecto á voz? \dotfill
            \item Cantos instrumentos identificas de cada familia? \dotfill
            \item Algún dos instrumentos leva a voz principal? Se é así, cal(es)?\dotfill
        \end{enumerate}
        \item 
        \textbf{Ritmo.}
        \begin{enumerate}
            \item
            Identifica as figuras e trata de establecer o tempo (pulso, compás) en caso de que non se indique.
            \begin{enumerate}
                \item Que figuras identificas? \dotfill
                \item Identificas un tempo longo, breve, ...? \dotfill
                \item Podemos identificar o compás? 
                \item Se é o caso, cal? \dotfill
            \end{enumerate}
            \item 
            Atendento á rítmica, presta atención aos seguintes aspectos:
            \begin{enumerate}
                \item É ritmo constante ou cambiante? \dotfill
                \item Podemos dicir que é rápido ou lento? \dotfill
                \item Estamos ante un ritmo libre? \dotfill
                \item Quen leva o ritmo? (voz, instrumentos de vento, ...) \dotfill
            \end{enumerate}
        \end{enumerate}
        \item 
        \textbf{Forma.}
        Determina segundo a extensión, instrumentación e estrutura da obra, a forma:
        \begin{enumerate}
            \item Estamos ante unha forma maior ou menor? \dotfill
            \item O timbre fainos pensar que se trata de unha forma? \dotfill
            \item Estamos ante unha forma libre? Por que? \dotfill
        \end{enumerate}
    \end{enumerate}
%
\end{ejercicio}
%
% EXERCICIO PARA AUDICIÓN EPITAFIO SEIKILOS
%
\begin{ejercicio}[Audición \textit{Epitafio de Seikilos}]
%
Atendendo aos elementos mínimos que debes ter en conta para elaborar un comentario de audición (indicados no punto 2 do caderno de exercicios), completa a seguinte ficha.
%
    \begin{enumerate}[1.-]
    \item \textbf{Autor.} Indica o autor da obra (se é posible) \dotfill
    \item \textbf{Título da obra.} Indica o título da obra \dotfill
%    \begin{multicols}{2}
        \item 
        \textbf{Timbre.}
%        \begin{enumerate}
%        \item 
        Indica os instrumentos que recoñeces na audición, segundo a clasificación moderna.
        \vspace*{1.0cm}
%            \begin{enumerate}
%            \item Cordófonos \dotfill
%            \item Aerófonos \dotfill
%            \item Idiófonos \dotfill
%            \item Membranófonos\dotfill
%            \item Outros \dotfill
%            \end{enumerate}
%        \end{enumerate}
%    \end{multicols}
%
        \item
        \textbf{Textura.}
        Estamos ante textura de escrita horizontal ou vertical? \dotfill
        \par
        Cal é, das que coñeces? (homofonía, polifonía, monodia, etc.) \dotfill \par
%
        \item
        \textbf{Melodía.}
            Para determinar a textura, fíxate na melodía prestando atención a:
            \begin{enumerate}
            \item Que son é o son que máis se repite? \dotfill
            \item Por que ámbito se move? (2as, 3as, grandes saltos, ...) \dotfill
            \item Que fan os instrumentos con respecto á voz? \dotfill
            \item Cantos instrumentos identificas de cada familia? \dotfill
            \item Algún dos instrumentos leva a voz principal? Se é así, cal(es)?\dotfill
        \end{enumerate}
        \item 
        \textbf{Ritmo.}
        \begin{enumerate}
            \item
            Identifica as figuras e trata de establecer o tempo (pulso, compás) en caso de que non se indique.
            \begin{enumerate}
                \item Que figuras identificas? \dotfill
                \item Identificas un tempo longo, breve, ...? \dotfill
                \item Podemos identificar o compás? 
                \item Se é o caso, cal? \dotfill
            \end{enumerate}
            \item 
            Atendento á rítmica, presta atención aos seguintes aspectos:
            \begin{enumerate}
                \item É ritmo constante ou cambiante? \dotfill
                \item Podemos dicir que é rápido ou lento? \dotfill
                \item Estamos ante un ritmo libre? \dotfill
                \item Quen leva o ritmo? (voz, instrumentos de vento, ...) \dotfill
            \end{enumerate}
        \end{enumerate}
        \item 
        \textbf{Forma.}
        Determina segundo a extensión, instrumentación e estrutura da obra, a forma:
        \begin{enumerate}
            \item Estamos ante unha forma maior ou menor? \dotfill
            \item O timbre fainos pensar que se trata de unha forma? \dotfill
            \item Estamos ante unha forma libre? Por que? \dotfill
        \end{enumerate}
    \end{enumerate}
%
\end{ejercicio}