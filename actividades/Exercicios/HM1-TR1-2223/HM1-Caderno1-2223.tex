%%%%%%%%%%%%%%%%%%%%%%%%%%%%%%%%%%%%%%%%%%%%%%%%%%%%%%
%% Preámbulo de LaTeX para o caderno de exercicios %%%
%%%%%%%%%%%%%%%%%%%%%%%%%%%%%%%%%%%%%%%%%%%%%%%%%%%%%%
%
% Clase de documento:
\documentclass[a4papper,12pt,notitlepage,spanish]{article}
%
% Cargamos configuración
% ----------------------
% Seleccionar o idioma:
%%%%%%%%%%%%%%%%%%%%%%%%%%%%%%%%%%%%%%%%%%%
%% ---------- MODELO EJERCICIOS ---------- 
%% MATERIA: HISTORIA
%% CURSO: 
%% AÑO ACADÉMICO: 
%% CENTRO: 
%%%%%%%%%%%%%%%%%%%%%%%%%%%%%%%%%%%%%%%%%%%
%% 
%% MODELO PARA REDACTAR EJERCICIOS
%% ===============================
%% 
%% Clase de documento
%% ------------------
%\documentclass[letterpaper,12pt,notitlepage,spanish]{article}
%\documentclass[12pt,a4paper,notitlepage]{article}
%
% Márgenes de documento
% ---------------------
\usepackage[left=2.0cm, right=2.0cm, lines=45, top=2.5cm, bottom=2.0cm]{geometry}
%
% Paquetes necesarios
% -------------------
\usepackage[utf8]{inputenc} % acentos en ES
\usepackage[spanish,activeacute, es-tabla]{babel}
\usepackage{enumerate} % entornos de listas
\usepackage{multicol}  % varias columnas texto
\usepackage{fancyhdr}  % encabezado personalizado
\usepackage{fancybox}  % entornos con cajas
\usepackage{pdfpages}  % páginas pdf
%\usepackage[none]{hyphenat} % evitar separación silábica
%
\usepackage{lipsum} % generar texto aleatorio "loren ipsum"
\usepackage{environ} 
\usepackage{probsoln} % paquete para soluciones
%\showanswers % para mostrar soluciones
%
%Esto es lo importante. Ponemos la solución al margen.
\NewEnviron{solutionnew}{%
%  \leavevmode\marginpar{\raggedright\footnotesize \textbf{Solución:}\\ \BODY}
%  \textbf{Solución:}\\ \BODY} % sol. con salto de liña
  \small{Solución:} \BODY} % sol. na mesma liña
  {}
\renewenvironment{solution}{\solutionnew}{\endsolutionnew}
%
% FIGURAS EN COLUMNAS:
\newenvironment{Figura}
  {\par\medskip\noindent\minipage{\linewidth}}
  {\endminipage\par\medskip}
% ---
%
% Lineas de encabezado y pié
% --------------------------
\renewcommand{\headrulewidth}{0.5pt}
%\renewcommand{\headrulewidth}{1.0pt}
\renewcommand{\footrulewidth}{0.5pt}
%\renewcommand{\footrulewidth}{1.0pt}
\pagestyle{fancy} % estilo de página
%
% Recuadros y figuras
% -------------------
\newcommand\Loadedframemethod{TikZ}
\usepackage[framemethod=\Loadedframemethod]{mdframed}
\usepackage{tikz}
\usetikzlibrary{calc,through,backgrounds}
\usetikzlibrary{matrix,positioning}
%Desssins geometriques
\usetikzlibrary{arrows}
\usetikzlibrary{shapes.geometric}
\usetikzlibrary{datavisualization}
\usetikzlibrary{automata} % LATEX and plain TEX
\usetikzlibrary{shapes.multipart}
\usetikzlibrary{decorations.pathmorphing} 
\usepackage{pgfplots}
\usepackage{physics}
\usepackage{titletoc}
\usepackage{mathpazo} 
\usepackage{algpseudocode}
\usepackage{algorithmicx} 
\usepackage{bohr} 
\usepackage{xlop} 
\usepackage{bbding} 
%\usepackage{minibox} 
% Texto árabe
\usepackage{mathdesign}
\usepackage{bbding} 
% --
% --
%   ---------   Personalización de textos    ---------
%   Opción en Galego.-
\addto\captionsspanish{
\renewcommand{\contentsname}{Índice} 
\renewcommand{\listfigurename}{Índice de ilustracións} 
\renewcommand{\listtablename}{Índice de táboas} 
\renewcommand{\bibname}{Bibliografía} 
\renewcommand{\indexname}{Indice alfabético} 
\renewcommand{\figurename}{Ilustración} 
\renewcommand{\tablename}{Táboa} 
%\renewcommand{\appendixname}{Anexo} 
%\renewcommand{\abstractname}{Resumo}
%\renewcommand{\partname}{BLOQUE} % Cambio Parte BLOQUE
%\renewcommand{\chaptername}{TEMA} % Cambio CAPÍTULO por TEMA en mayúsculas %
}
%   ---------   Fin personalización de textos   ---------

% Tipograía:
% ----------
% Fuente HEURÍSTICA (cómoda de leer)
%\usepackage{heuristica}
% Fuente LIBERTINE (cómoda para apuntes)
\usepackage{libertineRoman}
%\usepackage[proportional]{libertine}
% Fuente ROMANDE (estilo antiguo pero no muy cómoda)
%\usepackage{romande} %
% 
% Encabezado y pié de página (textos)
% -----------------------------------
% Modelo 1:
% ---------
% texto de encabezado izquierda:
%\lhead{\normalfont{Historia de la Música I}}
% texto encabezado centro:
%\chead{\textbf{Ejercicios}}
% texto de encabezado derecha:
%\rhead{\normalfont{curso: 2020/2021}}
% texto pié izquierdo:
%\lfoot{\small{\textit{}}}
% texto pié centrado:
%\cfoot{\textsc{Pág. \thepage }}
% texto pié derecho
%\rfoot{\textit{Pr. $\mathcal{A}$.Kaal}}
% ----------
% Modelo 2:
% ---------
% Encabezado y pié de página (textos)
% -----------------------------------
% texto de encabezado izquierda:
%
%\lhead{
%	\hrule
%	\vspace*{0.20cm}
%	\normalfont{Historia de la Música I}
%	\vspace*{0.10cm}
	%\hrule
%}
% texto encabezado centro:
%\chead{
%	\textbf{Cuestionario de Ejercicios}
%	\vspace*{0.08cm}}
% texto de encabezado derecha:
%\rhead{
%	\normalfont{curso: 2020/2021}
%	\vspace*{0.08cm}}
%
% texto pié izquierdo:
%\lfoot{
	%\begin{center}
		%\vspace*{0.20cm}
		%\hrule
		%\small{
		%Conservatorio Profesional de Música de Viveiro - Avda. da mariña s/n - (27850) Viveiro - Lugo
		%	}
	%\end{center}
%}
% texto pié centrado:
%\cfoot{
	%\vspace*{0.30cm}
	%\hrule
	%\vspace*{0.90cm}
%	\small{- Página \thepage -  }\\
	%\small{Conservatorio Profesional de Música de Viveiro}\\
	%\small{avda. da Mariña s/n}
%}

% ----------
% Modelo 3:
% ---------
% Encabezado y pié de página (textos)
% -----------------------------------
% texto de encabezado izquierda:
%
\lhead{
	\hrule
	\vspace*{0.20cm}
	\normalfont{Historia da Música I}
	\vspace*{0.10cm}
	%\hrule
}
% texto encabezado centro:
\chead{
	\textbf{CADERNO DE EXERCICIOS}
	\vspace*{0.08cm}}
% texto de encabezado derecha:
\rhead{
	\normalfont{curso: 2022/2023}
	\vspace*{0.08cm}}
%
% texto pié izquierdo:
%\lfoot{
%	\begin{center}
%		\vspace*{0.20cm}
%		\hrule
%		\small{
%		Conservatorio Profesional de Música de Viveiro - Avda. da mariña s/n - (27850) Viveiro - Lugo
%			}
%	\end{center}
%}
% texto pié centrado:
\cfoot{
	%\vspace*{0.30cm}
	%\hrule
	%\vspace*{0.90cm}
	\small{- \thepage -  }\\
	%\small{Conservatorio Profesional de Música de Viveiro}\\
	%\small{avda. da Mariña s/n}
}

% --------
%=====================Algo setup
\algblock{If}{EndIf}
\algcblock[If]{If}{ElsIf}{EndIf}
\algcblock{If}{Else}{EndIf}
\algrenewtext{If}{\textbf{si}}
\algrenewtext{Else}{\textbf{sinon}}
\algrenewtext{EndIf}{\textbf{finsi}}
\algrenewtext{Then}{\textbf{alors}}
\algrenewtext{While}{\textbf{tant que}}
\algrenewtext{EndWhile}{\textbf{fin tant que}}
\algrenewtext{Repeat}{\textbf{r\'ep\'eter}}
\algrenewtext{Until}{\textbf{jusqu'\`a}}
\algcblockdefx[Strange]{If}{Eeee}{Oooo}
[1]{\textbf{Eeee} "#1"}
{\textbf{Wuuuups\dots}}

\algrenewcommand\algorithmicwhile{\textbf{TantQue}}
\algrenewcommand\algorithmicdo{\textbf{Faire}}
\algrenewcommand\algorithmicend{\textbf{Fin}}
\algrenewcommand\algorithmicrequire{\textbf{Variables}}
\algrenewcommand\algorithmicensure{\textbf{Constante}}% replace ensure by constante
\algblock[block]{Begin}{End}
\newcommand\algo[1]{\textbf{algorithme} #1;}
\newcommand\vars{\textbf{variables } }
\newcommand\consts{\textbf{constantes}}
\algrenewtext{Begin}{\textbf{debut}}
\algrenewtext{End}{\textbf{fin}}
%================================
%================================

\setlength{\parskip}{1.25cm}
\setlength{\parindent}{1.25cm}
\tikzstyle{titregris} =
[draw=gray,fill=gray, shading = exersicetitle, %
text=gray, rectangle, rounded corners, right,minimum height=.3cm]
\pgfdeclarehorizontalshading{exersicebackground}{100bp}
{color(0bp)=(green!40); color(100bp)=(black!5)}
\pgfdeclarehorizontalshading{exersicetitle}{100bp}
{color(0bp)=(red!40);color(100bp)=(black!5)}
\newcounter{exercise}
%\renewcommand*\theexercise{exercice \textbf{Ejercicio}~n\arabic{exercise}} % CASTELÁN
\renewcommand*\theexercise{exercice \textbf{Exercicio}~n\arabic{exercise}} % GALEGO
\makeatletter
\def\mdf@@exercisepoints{}%new mdframed key:
\define@key{mdf}{exercisepoints}{%
\def\mdf@@exercisepoints{#1}
}
\mdfdefinestyle{exercisestyle}{%
outerlinewidth=1em,outerlinecolor=white,%
leftmargin=-1em,rightmargin=-1em,%
middlelinewidth=0.5pt,roundcorner=3pt,linecolor=black,
apptotikzsetting={\tikzset{mdfbackground/.append style ={%
shading = exersicebackground}}},
innertopmargin=0.1\baselineskip,
skipabove={\dimexpr0.1\baselineskip+0\topskip\relax},
skipbelow={-0.1em},
needspace=0.5\baselineskip,
frametitlefont=\sffamily\bfseries,
settings={\global\stepcounter{exercise}},
singleextra={%
\node[titregris,xshift=0.5cm] at (P-|O) %
{~\mdf@frametitlefont{\theexercise}~};
\ifdefempty{\mdf@@exercisepoints}%
{}%
{\node[titregris,left,xshift=-1cm] at (P)%
{~\mdf@frametitlefont{\mdf@@exercisepoints points}~};}%
},
firstextra={%
\node[titregris,xshift=1cm] at (P-|O) %
{~\mdf@frametitlefont{\theexercise}~};
\ifdefempty{\mdf@@exercisepoints}%
{}%
{\node[titregris,left,xshift=-1cm] at (P)%
{~\mdf@frametitlefont{\mdf@@exercisepoints points}~};}%
},
}
\makeatother

%%%%%%%%%%%%%%%5 Definición modificada %%%%%%%%%%%%%%%%%%%%%%%%%
%
% Modificado para traballar con banco de exercicios
% Elimino as liñas do recadro de exercicios convencional
%
%\mdfdefinestyle{theoremstyle}{%
%outerlinewidth=0.01em,linecolor=white,middlelinewidth=0.5pt,%
%frametitlerule=true,roundcorner=2pt,%
%apptotikzsetting={\tikzset{mfframetitlebackground/.append style={%
%shade,left color=white, right color=blue!20}}},
%frametitlerulecolor=white,innertopmargin=1\baselineskip,%green!60,
%innerbottommargin=0.5\baselineskip,
%frametitlerulewidth=0.1pt,
%innertopmargin=0.7\topskip,skipabove={\dimexpr0.2\baselineskip+0.1\topskip\relax},
%frametitleaboveskip=1pt,
%frametitlebelowskip=1pt
%}
% --------------------------------------

%%%%%%%%%%%%%%% Definición orixinal %%%%%%%%%%%%%%%%%%%%%%%%%
%%       Maquetación con bordes para exercicios            %%
%%
\mdfdefinestyle{theoremstyle}{%
outerlinewidth=0.01em,linecolor=black,middlelinewidth=0.5pt,%
frametitlerule=true,roundcorner=2pt,%
apptotikzsetting={\tikzset{mfframetitlebackground/.append style={%
shade,left color=white, right color=blue!20}}},
frametitlerulecolor=black,innertopmargin=1\baselineskip,%green!60,
innerbottommargin=0.5\baselineskip,
frametitlerulewidth=0.1pt,
innertopmargin=0.7\topskip,skipabove={\dimexpr0.2\baselineskip+0.1\topskip\relax},
frametitleaboveskip=1pt,
frametitlebelowskip=1pt
}
\setlength{\parskip}{0mm}
\setlength{\parindent}{10mm}
%\mdtheorem[style=theoremstyle]{ejercicio}{\textbf{Ejercicio}} % Castelán
\mdtheorem[style=theoremstyle]{ejercicio}{\textbf{Exercicio}} % Galego
%================Liste definition--numList-and alphList=============
\newcounter{alphListCounter}
\newenvironment
{alphList}
{\begin{list}
{\alph{alphListCounter})}
{\usecounter{alphListCounter}
\setlength{\rightmargin}{0cm}
\setlength{\leftmargin}{0.5cm}
\setlength{\itemsep}{0.2cm}
\setlength{\partopsep}{0cm}
\setlength{\parsep}{0cm}}
}
{\end{list}}
\newcounter{numListCounter}
\newenvironment
{numList}
{\begin{list}
{\arabic{numListCounter})}
{\usecounter{numListCounter}
\setlength{\rightmargin}{0cm}
\setlength{\leftmargin}{0.5cm}
\setlength{\itemsep}{0cm}
\setlength{\partopsep}{0cm}
\setlength{\parsep}{0cm}}
}
{\end{list}}
%
%
% --------------- CRONOGRAMAS en LATEX: --------------- 
\usepackage{chronology}

\renewenvironment{chronology}[5][6]{%
    \setcounter{step}{#1}%
    \setcounter{yearstart}{#2}\setcounter{yearstop}{#3}
    \setcounter{deltayears}{\theyearstop-\theyearstart}
    \setlength{\unit}{#4}%
    \setlength{\timelinewidth}{#5}%
    \pgfmathsetcounter{stepstart}%
    {\theyearstart+\thestep-mod(\theyearstart,\thestep)}%
    \pgfmathsetcounter{stepstop}{\theyearstop-mod(\theyearstop,\thestep)}%
    \addtocounter{step}{\thestepstart}%
    \begin{lrbox}{\timelinebox}%
    \begin{tikzpicture}[baseline={(current bounding box.north)}]%
    \draw [|->] (0,0) -- (\thedeltayears*\unit+\unit, 0);%
    \foreach \x in {1,...,\thedeltayears}%
    \draw[xshift=\x*\unit] (0,-.1\unit) -- (0,.1\unit);
   \addtocounter{deltayears}{1}%
    \foreach \x in {\thestepstart,\thestep,...,\thestepstop}{%
        \pgfmathsetlength\xstop{(\x-\theyearstart)*\unit}%
        \draw[xshift=\xstop] (0,-.3\unit) -- (0,.3\unit);%
        \node at (\xstop,0) [below=.2\unit] {\x};}%
    }
{%
\end{tikzpicture}%
\end{lrbox}%
    \raisebox{2ex}{\resizebox{\timelinewidth}{!}{\usebox{\timelinebox}}}}%
%
% Fin do código
%
%% -- Fin del archivo de configuración  --
%%
 % Galego
%\input{../../Modelos/include/config-HM1ejercicios_ES.tex} % Castelán
% --------------------------------
\usepackage{graphicx}
\usepackage{hyperref}
\usepackage{wrapfig}

% Directorio raíz de imaxes por defecto:
% --------------------------------------
% As imaxes son as mesmas que se usan no temario, polo que están nas carpetas de unidades correspondentes:
% P.Ex: /ud-00/figura.png

\graphicspath{ {../../../figures/} } %
%
% INICIO DO DOCUMENTO
% ===================

\begin{document}
% --------------------------------------------------
% Cargamos os exercicios desde o Banco de Exercicios
% -------------------------------------------------- 
% 
\hideanswers % oculta respostas
%
% Carga aquí os exercicios que correspondan a incluir na folla:
% -------------------------------------------------------------
\loadallproblems[Tema1-PE]{../../Cuestions/Tema1-Periodizacion.tex}
\loadallproblems[Tema1-EX]{../../Cuestions/Tema1-Exipto.tex} %
\loadallproblems[Tema1-GR]{../../Cuestions/Tema1-Grecia.tex} %
\loadallproblems[Tema1-PENS]{../../Cuestions/Tema1-Pensamento.tex} %
%\loadallproblems[Tema1-RO]{../../Cuestions/Tema1-Roma.tex} %
%\loadrandomproblems[loops]{2}{loops}% aleatorios
% -------------------------------------------------------------
%
% TÍTULO A FOLLA DE EXERCICIOS:
%
\begin{center}
\Large{
1º Trimestre
} \\
\vspace*{0.5cm}
%
% DATOS DO ALUMNADO:
% -----------------
\vspace{1.10cm}
	\begin{flushleft}
	Nome e Apelidos: \hrulefill\\
	%\vspace*{0.50cm}
%
% INSTRUCCIÓNS:
% ------------
%		\begin{center}
%		\small{Instrucciones para realizar los ejercicios}\\		
%		\end{center}
%	\hrulefill \\
	%\vspace*{0.25cm}
%
%\small{ % INSTRUCCIONES:
%\texttt{Lee con atención y realiza con detenimiento, los siguientes ejercicios teniendo en cuenta lo que se indica en cada uno. \\
%}} % fin instrucciones.
%
	\vspace*{0.25cm}		
 	\end{flushleft}
\end{center}
\vspace*{1.10cm}
%

% ------------------------------------------  
% EXERCICIOS.
% ------------------------------------------
%
\begin{multicols}{2} % a 2 columnas
    \begin{enumerate}
%\useproblem{input}
    \foreachproblem[Tema1-PE]{\item\label{prob:\thisproblemlabel}\thisproblem}
    \foreachproblem[Tema1-EX]{\item\label{prob:\thisproblemlabel}\thisproblem}
    \foreachproblem[Tema1-GR]{\item\label{prob:\thisproblemlabel}\thisproblem}
    \foreachproblem[Tema1-PENS]{\item\label{prob:\thisproblemlabel}\thisproblem}
    \end{enumerate}
\end{multicols}
%
% SOLUCIONS:
% ----------
% Ocultas polo momento. Insertar ao final do caderno
%
%\newpage
%\begin{multicols}{2}
%\showanswers
%\begin{itemize}
%\foreachdataset{\thisdataset}{%
%\foreachproblem[\thisdataset]{\item[\ref{prob:\thisproblemlabel}]\thisproblem}
%}
%\end{itemize}
%\end{multicols}

% 
% EXERCICIOS DO TEMA 0.- INTRODUCCIÓN
%
% Repaso de conceptos de Perspectivas sobre a música

\section{Pensamentos, teorías e perspectivas sobre a música}

\begin{ejercicio}[Perspectivas sobre música- Rene Descartes]

% Ilustramos os exercicios con figuras
% --- Rene Descartes:
\begin{wrapfigure}{l}{0.30\textwidth} 
\begin{center} 
\includegraphics[width=0.20\textwidth]{/ud-00/descartes.png} 
\end{center} 
\caption{\\ \textbf{Rene Descartes} \\ 
Francia, 1596 - Suecia, 1650} 
\label{fig:descartes}
\end{wrapfigure}
% ---

As perspectivas sobre a música varian ao longo das diferentes épocas da historia. 
Algunhas reflexións, convidan a pensar na música como unha arte totalmente ligada a expresión de sentimentos; outras xustifican que, a música hai que considerala unha ciencia.

Desde os inicios da música, os teóricos, filósofos, músicos e grandes ilustrados trataron de comprender o feito musical e establecer relacións que lles axudasen a coñecer mellor o seu significado. É o caso de Pitágoras, Descartes, Kant, Wagner e moitos outros, que coas súas reflexións e perspectivas sobre a música sentan os precedentes do pensamento estétilo e musical, dentro da Historia do pensamento musical. 

Descartes s. {\scriptsize  XVII} define a música do seguinte xeito:

    \begin{quotation}{\small
     \noindent
     A mesma cousa que a uns invita a bailar a outros pode facer chorar. Pois isto non provén senón da asociación de ideas na nosa mente; como aqueles que algunha vez se divertiron bailando con certa peza, tan pronto como a volvan a escoltar volverán ás ganas de bailar; pola contra, se algún só oíu gallardas cando lle aconteceu algo malo, volverá  a entristecerse cando as escoite de novo.}
    \end{quotation}
 
\begin{enumerate}[1)]
 \item \label{perspectiva-descartes}
 Que perspectiva sobre a música adoita Descartes segundo a afirmación anterior?
 \begin{enumerate}[a)]
  \item 
  Música como expresión dos sentimentos
  \item
  Música como arte
  \item \label{sol:1}
  Música como feito musical
  \item
  Ningunha das anteriores
 \end{enumerate}
 \item \label{perpectiva-pitagoras}
 A quen atribúes a seguinte afirmación? \dotfill
     \begin{quote}
    {\small
    Os números son as cousas; agora ben, a música é número. O mundo é música; o cosmos é unha lira sublime de sete cordas.
    }
    \end{quote}
\end{enumerate}

\end{ejercicio}

% --------------------------

\begin{ejercicio}[Perspectivas sobre música - Richard Wagner]

% Ilustramos os exercicios con figuras
% --- Richard Wagner:
\begin{wrapfigure}{r}{0.30\textwidth} 
\begin{center} 
\includegraphics[width=0.20\textwidth]{/ud-00/wagner.png} 
\end{center} 
\caption{\\ \textbf{Richard Wagner} \\ 
Leipzig, 1813 - Venecia, 1883} 
\label{fig:descartes}
\end{wrapfigure}
% ---

Wilhelm Richard Wagner (Leipzig, 22 de maio de 1813 - Venecia, 13 de febreiro de 1883) foi un compositor, director de orquestra, poeta, ensaísta, dramaturgo e teórico musical alemán do Romanticismo.

Unha das súas maiores achegas á música foi o cambio de perspectiva acerca das composicións, que Wagner consideraba como ``obras de arte totais'' nas que sintetizaban todas as grandes artes: visuais, poéticas, escénicas, musicais, \ldots

Foi un dos máximos expoñentes do romanticismo musical alemán, que rompeu cos moldes canónicos do clasicismo. 
No s.{\scriptsize XIX}, Richard Wagner afirmaba sobre a música:

    \begin{quotation}{\small
     \noindent
     O son vén do corazón e a súa linguaxe artística natural é a música. A melodía é a lingua absoluta, a través da que o músico fala a todos os corazóns.}
    \end{quotation}
 
\begin{enumerate}[1)]
 \item \label{perspectiva-descartes}
 Que perspectiva sobre a música adoita Wagner segundo a afirmación anterior?
 \begin{enumerate}[a)]
  \item 
  Música como expresión dos sentimentos
  \item \label{sol:2}
  Música como arte
  \item 
  Música como feito musical
  \item
  Ningunha das anteriores
 \end{enumerate}
 \item \label{perpectiva-pitagoras}
 A quen atribúes a seguinte afirmación sobre a música? \dotfill
     \begin{quote}
    {%\small
    [\ldots] a arte educativa por excelencia que se insire na alma e forma a virtude
    }
    \end{quote}
\end{enumerate}
\end{ejercicio}

% ------

\begin{ejercicio}[Teorías sobre a orixe da música]
Sinala a opción correcta, segundo as afirmacións que se indican nos seguintes puntos.
\begin{enumerate}[1)]
 \item 
 As teorías logoxénicas consideran que a música naceu asociada á linguaxe comunicativa.
  \begin{enumerate}[a)]
   \item \label{sol:3}
   verdadeiro, nace da necesidade de comunicación
   \item 
   falso, tiñan unha función máxico-relixiosa
  \end{enumerate}
  \item
  Que teorías postulan que o corpo humano é un instrumento en si mesmo? 
  \begin{enumerate}[a)]
   \item 
   teorías logoxénicas
   \item 
   teorías máxico-relixiosas
   \item \label{sol:4}
   teorías quinéticas
   \item 
   teorías conspirativas
  \end{enumerate}
 \end{enumerate}

\end{ejercicio}




\newpage
%
% EXERCICIO 2.- Música na antigüidade clásica
%
% 
% EXERCICIOS DO TEMA 0.- INTRODUCCIÓN - 
% Conceptos sobre a forma, xéneros musicais e estilos
%
% Repaso de conceptos de xeneros e formas da música

\begin{multicols}{2}

% EXERCICIO: FORMAS E ESTILOS
% ---------------------------
\begin{ejercicio}[Formas e estilos]
 \begin{enumerate}[1)]
  \item 
  Se nunha audición, analizamos os instrumentos que escoitamos nunha composición musical, en que aspecto estamos a fixar a nosa atención?

  \begin{enumerate}[a)]
   \item 
   Na textura
   \item % solución correcta
   No timbre
   \item
   Na forma
   \item
   No ritmo
  
  \end{enumerate}
  
  \item
   Cando escoitamos unha audición e tratamos de identificar a estrutura que ten, partes ou movementos, estamos a analizar a súa \ldots
  
  \begin{enumerate}[a)]
   \item 
   Textura
   \item 
   Timbre
   \item % solución correcta
   Forma
   \item
   Ritmo
  \end{enumerate}
 
 \end{enumerate}

\end{ejercicio}



% Repaso de conceptos de xeneros e formas da música
%\begin{multicols}{2}

% EXERCICIO: FONTES DE INFORMACIÓN
% --------------------------------

\begin{ejercicio}[Fontes de información]
 \begin{enumerate}[1)]
  \item 
  Indica cales son as principais fontes de información que consideramos no estudo da Historia da Música.

  \begin{enumerate}[1.]
   \item \dotfill 
   \item \dotfill 
   \item \dotfill 
   \item \dotfill 
  \end{enumerate}
 
 \item
 As pinturas, esculturas e outras obras de arte son consideradas fontes de información \par
 \dotfill
 \par
 Que é para ti unha fonte de información histórica?
 \par \vspace*{2.3cm}
 \end{enumerate}
\end{ejercicio}
%
\end{multicols}

% EXERCICIO: Liña temporal da historia
% ------------------------------------

\begin{ejercicio}[Liña temporal da historia]
 \begin{enumerate}[1)]
  \item 
A división en etapas ---períodos--- da historia, ten a súa orixe nos humanistas europeos do Renacemento. Indica a periodización:

\begin{enumerate}[a)]
 \item
 Idade Antiga: \dotfill ao  \dotfill \hspace{0.5cm}
 \item
 Idade Media:  \dotfill ao  \dotfill  \hspace{0.5cm}
 \item
 Idade Moderna:  \dotfill ao  \dotfill  \hspace{0.5cm}
 \item
 Idade Contemporánea:  \dotfill ao  \dotfill  \hspace{0.5cm} 
\end{enumerate}
\item
Completa a seguinte periodización:
\par
\begin{center}
\begin{tabular}{lcl}
%\hline
Período ou etapa &  & Cronoloxía \\
\hline
Románico & $\Rightarrow$ &  \\
       & $\Rightarrow$ & XII- XV \\
Renacemento & $\Rightarrow$ & \\
& $\Rightarrow$ & XVII- XVIII \\
Neoclasicismo & $\Rightarrow$ & \dotfill e comezo \dotfill \\
 & $\Rightarrow$ & final do XVIII e parte do XIX \\
Positivismo e Realismo & $\Rightarrow$ &  \\
\hline
\end{tabular}
\end{center}
\end{enumerate}

\end{ejercicio}
%


\newpage
%
% EXERCICIO 3.- Teoría musical
%
%\input{Exercicio3-Teoria-musical.tex}
%\newpage

\end{document}
%Fin de Hoja de ejercicios
%
