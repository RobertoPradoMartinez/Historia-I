% -----------------------------------
% FICHAS DE AUDICIÓNS PARA COMPLETAR:
% -----------------------------------
\section{Coñece a música do medievo}
%
Escoita con atención as audicións propostas para este trimestre e completa as fichas. Fai un breve resumo das principais características da obra.
Podes atopar máis info na aula virtual de Historia da Música I.
%
\begin{multicols}{2}
%
% AUDICIÓN 1.- PUER NATUS EST NOBIS
% ---------------------------------
% Exemplo de canto chá: Puer natus est nobis - Introito (Modo VII)
%
\begin{ejercicio}[Puer natus est nobis] 
%
Completa a ficha da obra proposta como exercicio de audición.
%
	\begin{enumerate}[1.-]
        \vspace*{0.3cm}
		\item
			Autor: \dotfill
			\vspace*{0.3cm}
		\item
			Obra:
			\begin{enumerate}[a)]
			    \item Título: \dotfill \vspace*{0.3cm}
			    \item Período: \dotfill \vspace*{0.3cm}
			    \item Forma: \dotfill \vspace*{0.3cm}
			    \item Timbre: \dotfill 			\vspace*{0.3cm}
			    \item Textura: \dotfill \vspace*{0.3cm}
			    \item Estilo: \dotfill \vspace*{0.3cm}
			    \item Xénero: \dotfill 
			    \vspace*{0.3cm}
			\end{enumerate}
		\item 
		    Resume as principais características que definen a obra:
			\vspace*{8.0cm}			

	\end{enumerate}
\end{ejercicio}
%
%
\begin{ejercicio}[Dies irae]
%
Completa a ficha da obra proposta como exercicio de audición.
%
	\begin{enumerate}[1.-]
        \vspace*{0.3cm}
		\item
			Autor: \dotfill
			\vspace*{0.3cm}
		\item
			Obra:
			\begin{enumerate}[a)]
			    \item Título: \dotfill \vspace*{0.3cm}
			    \item Período: \dotfill \vspace*{0.3cm}
			    \item Forma: \dotfill \vspace*{0.3cm}
			    \item Timbre: \dotfill \vspace*{0.3cm} 		
			    \item Textura: \dotfill \vspace*{0.3cm}
			    \item Estilo: \dotfill \vspace*{0.3cm}
			    \item Xénero: \dotfill \vspace*{0.3cm}
			\end{enumerate}
		\item 
		    Resume as principais características que definen a obra:
			\vspace*{8.0cm}			

	\end{enumerate}
\end{ejercicio}
%
\end{multicols}
%