%
%%%%%%%%%%%%%%%%%%%%%%%%%%%%%%%%%%%%%%%%%%%%%%%%%%
% PLANTILLA EJERCICIOS DE HISTORIA DE LA MÚSICA I
% Este es un modelo para redactar los ejercicios
% 
% Pasos para cubrir la plantilla:
% 1) Realizar una copia de este modelo
% 2) Renombrar el archivo:
%		"HM1_Hoja(número).tex"
% 3) El número de Hoja debe ser correlativo
% 
%%%%%%%%%%%%%%%%%%%%%%%%%%%%%%%%%%%%%%%%%%%%%%%%%%
%
% Esta plantilla es para crear ejercicios de esta materia
% Se recomienda crear un archivo por cada tema
% Descomentar según se necesite utilizar un modelo de ejercicio u otro
% Clase de documento:
\documentclass[letterpaper,12pt,notitlepage,spanish]{article}
%
% Archivo externo de configuración
% --------------------------------
\input{../../Modelos/include/config-HM1ejercicios.tex}
% --------------------------------
\usepackage{graphicx}
\begin{document}
%
% DATOS DE HOJA DE EJERCICIOS
% ---------------------------
%
% TÍTULO DE LA HOJA DE EJERCICIOS:
%
\begin{center}
\Large{
Ejercicios del tercer trimestre
} \\ 
(siglos XV y XVI) \\
\vspace*{0.5cm}
%
% NÚMERO DE HOJA:
%
\normalsize % Número de hoja:
(Hoja no. 1)

%\vspace{1.10cm}
%	\begin{flushleft}
%	Nombre y Apellidos: \hrulefill\\
%	\vspace*{0.50cm}
%		\begin{center}
%		\small{Instrucciones para realizar los ejercicios}\\		
%		\end{center}
%	\hrulefill \\
%	\vspace*{0.25cm}
%
%\small{ % INSTRUCCIONES:
%\texttt{Lee con atención y realiza con detenimiento, los siguientes ejercicios teniendo en cuenta lo que se indica en cada uno. \\
%}} % fin instrucciones.
%
%	\vspace*{0.25cm}		
% 	\end{flushleft}
\end{center}
%
% ESPACIO PARA REDACTAR LOS EJERCICIOS:
% -------------------------------------  
%
% MODELO PARA LOS EJERCICIOS.- DESCRIPTOR
% \begin{ejercicio}[Título del ejercicio]
%		\begin{enumerate}[1.-]
%
% Pregunta:
%
%			\item 
%
% Pregunta:
%
%			\item
%
%		\end{enumerate}
% \end{ejercicio}
%
% EJERCICIO. 1
\begin{ejercicio}[]	
	\begin{enumerate}[1.-]
%
% Pregunta:
%
    \item 
    Sitúa los períodos musicales siguientes, en sus respectivas épocas:\\
	    \textit{Barroco, Edad Antigua, Renacimiento, Edad Media}.
		\vspace*{0.1cm}
		
			\begin{tabular}{l c r}
				%\hline
			siglos (\small{VI - XV})	& $\ldots$ & \\
				%\hline
			hasta siglo (\small{V})	& $\ldots$ & \\
				%\hline
			siglos (\small{XV y XVI})	& $\ldots$ &  \\
				%\hline
			siglo (\small{XVII})	& $\ldots$ &  \\
				%\hline
			\end{tabular}
	\item
	Sitúa los autores siguientes en su respectiva época:\\
	    Vitry, Machaut, Dufay, Rimbaut de Vaqueiras, Palestrina, Alfonso X.
	    \vspace*{0.1cm}
	    
	    	\begin{tabular}{l c r}
				%\hline
			Edad Media	& $\ldots$ & \\
				%\hline
			Renacimiento	& $\ldots$ & \\
				%\hline
			\end{tabular}
    \end{enumerate}			
\end{ejercicio}

% EJERICIO.- 2
%
\begin{ejercicio}[]	
	\begin{enumerate}[1.-]
%
% Pregunta:
%
		\item  
		A comienzos del siglo XIV aparecen nuevas técnicas musicales en cuanto al ritmo se refiere, que dan lugar a un importante cambio de estilo. Este nuevo estilo aparece reflejado en tratados como \textit{Ars Nova} (Arte nueva) o \textit{Ars Novae Musicae} (Arte de la nueva música), en donde se insiste en el carácter de novedad de la nueva técnica. ¿Quién es el autor del primero?
		\par
	\begin{tabular}{l l l l l l}
	a) Philippe de Vitry  &  &  &   & c) Francisco Guerrero &  \\
	b) Juan de Muris &  &  &  & d) Zarlino &  \\
	\end{tabular} 
%
% Pregunta:
%
	\item 
	\textbf{Philippe de Vitry} (1291 - 1361) y \textbf{Guillaume de Machaut} (1300 - 1377) son dos autores importantes de motetes del:
	\par
	\begin{tabular}{l l l l l l}
	a) Renacimiento &  &  &   & c) \textit{Ars Antiqva} &  \\
	b) \textit{Ars Nova} &  &  &  & d) Renacentismo &  \\
	\end{tabular} 	
	\end{enumerate}
\end{ejercicio}
%
% EJERCICIO.- 3 
%
\begin{ejercicio}[]
	\begin{enumerate}[1.-]
%
% Pregunta:
%
		\item 
		Una de las técnicas innovadoras del \textit{Ars Nova} es la \textbf{isorritmia}, ¿en qué consiste?
          \begin{enumerate}[a)]
	    	\item En la repetición de un patrón melódico llamado(\textit{color}) una y otra vez.
	    	\item En la repetición de un patrón rítmico llamado (\textit{talea}) una y otra vez.
	    	\item En la repetición de un patrón rítmico (\textit{talea}) sobre una melodía gregoriana (\textit{color}).
	    	\item La isorritmia no es una técnica del (\textit{Ars Nova})
	    \end{enumerate}		
%
% Pregunta:
%
		\item 
		Quién es el autor de la \textit{Messe de Notre Dame}? \dotfill
%
	\end{enumerate}
\end{ejercicio}
%
% EJERCICIO.- 4 
% 
\begin{ejercicio}[]
	\begin{enumerate}[1.-]
%
% Pregunta:
		\item 
		La textura, en general, es el modo en que se combinan la melodía, el ritmo y la armonía en una composición o fragmento musical. 
		Indica verdadero o falso según proceda:
		
		\begin{enumerate}[a)]
		\item La textura \textbf{polifónica}, contrapuntística o imitativa está basada  en la idea de la imitación a intervalos armónicos regulares y en la independencia rítmica de las voces. \ldots
		\item Una melodía sin acompañamiento forma una textura monofónica o \textbf{monodia}. \ldots
		\item Si la melodía, como elemento principal de la textura, se presenta con algún tipo de acompañamiento, decimos que se trata de una \textbf{melodía acompañada}. \ldots
		\item En la \textbf{textura homofónica} las voces suenan a la vez con el mismo ritmo, creando un conjunto de acordes (textura acórdica y homorrítmica) es propia de géneros como la *chanson*, la *frottola* o el villancico.\ldots
		\end{enumerate}
%
%Pregunta:
		\item 
		¿Cuáles de las siguientes son características del Renacimiento?
	    	\begin{enumerate}[a)]
	    	\item Música más sencilla rítmicamente, con indicación de compás pero sin líneas divisorias; la isorritmia se emplea únicamente en celebraciones religiosas solemnes (s.XV).
		    \item Melodías fluidas y sencillas de ámbito reducido con motivos recurrentes.
	    	\item Todas ellas 
	    	\item Ninguna de ellas
		    \end{enumerate}
	\end{enumerate}
\end{ejercicio}
%
% EJERCICIO.-  5
%
\begin{ejercicio}[]
	\begin{enumerate}[1.-]
%
% Pregunta:
%
		\item 
		¿Cuáles de los siguientes, crees que son géneros de música profana del Renacimiento?
		\\ Indica la correcta:\par		
		\begin{tabular}{l l l l l l l}
		a) \textit{Chanson}, \textit{Frottola}, Villancico y Motete & & \ldots\\
		b) \textit{Chanson}, \textit{Frottola}, Villancico y Madrigal & & \ldots\\
		c) \textit{Chanson} \textit{Frottola}, Misa y Motete & &  \ldots\\
		d) \textit{Chanson} \textit{Frottola}, Madrigal y Motete & & \ldots\\
		\end{tabular}
%
% Pregunta: 
%
		\item 
		\textit{El Grillo} de Josquin de Prez, es un ejemplo de \par
		\begin{tabular}{l l l l l l l l l l l l}
		a) \textit{Chanson} & & 
		b) \textit{Frottola} & &
		c) Madrigal & &  
		d) Motete & &
		\end{tabular}
	\end{enumerate}
\end{ejercicio}
%
% EJERCICIO.- 6
%
\begin{ejercicio}[]
	\begin{enumerate}[1.-]
%
% Pregunta:
%
		\item 
		Sitúa a cada compositor en su escuela correspondiente:\\
		Byrd, los Gabrieli, Guerrero, Palestrina, 
		\par
		\begin{tabular}{l l l l l l l l l}
		\small{ ROMANA} & & 
		\small{ VENECIANA} & & 
		\small{ ESPAÑOLA} & &  
		\small{ INGLESA} & & \\
		\dotfill & &
		\dotfill & &
		\dotfill & &
		\dotfill & &
		\end{tabular}
	%	\vspace*{0.2cm}
%
%Pregunta:
%
		\item 
		Durante los siglos XV y XVI podemos diferenciar varias generaciones de compositores. \\
		Indica si son verdaderas o falsas las siguientes afirmaciones:
		\begin{enumerate}[a)]
		\item Dufay, Dunstable y Binchois pertenecen a la primera generación (1400-1460). \dotfill
		\item Busnois y Ockeghem pertenecen a la segunda generación (1450-1500). \dotfill
		\item Willaert pertenece a la tercera generación (1490-1520). \dotfill
		\item Josquin des Prez pertenecen a la cuarta generación (1520-1560). \dotfill
		\end{enumerate}
	\end{enumerate}
\end{ejercicio}
%
% EJERCICIO.- 7
%
 \begin{ejercicio}[]
		\begin{enumerate}[1.-]
%
% Pregunta:
%
			\item
			Lutero era partidario de aprovechar la música dentro de la liturgia pero de forma más popular y no como se venía haciendo hasta el momento; propone una misa más popular en lengua alemana, que permitiera la participación activa de la comunidad.\\ Sobre el Coral, podemos afirmar que: (señala la opción correcta) 
		    \begin{enumerate}[a)]
		        \item Se trata de un canto religioso monódico en alemán, con ritmo sencillo y melodías de ámbito reducido que se mueve por grados conjuntos y en estilo silábico.
		        \item Se trata de un canto profano monódico en alemán, con ritmo sencillo y melodías de ámbito reducido que se mueve por grados disjuntos y en estilo silábico.
		        \item Se trata de un canto sacro monódico en alemán, con ritmo sencillo y melodías de ámbito reducido que se mueve por grados disjuntos y en estilo silábico.
		        \item Se trata de un canto religioso polifónico en alemán, con ritmo sencillo y melodías de ámbito reducido que se mueve por grados conjuntos y en estilo silábico.
		    \end{enumerate}
%
% Pregunta:
%
			\item  
			El Concilio de Trento (1545-1563), plantea la necesidad de una reforma en profundidad de la práctica musical ligada a la liturgia para así centrar la atención en lo realmente importante: el texto sagrado. Estas orientaciones que plantea el Concilio, dará lugar a lo que se conoce como Escuela romana. \\
			¿Verdadero o Falso? \ldots
%
		\end{enumerate}
 \end{ejercicio}
%
% EJERCICIO.- 8
%
 \begin{ejercicio}[]
		\begin{enumerate}[1.-]
%
% Pregunta:
%
		\item 
		\textbf{Willaert}, maestro de capilla en la basílica de San Marcos y posteriormente los hermanos \textbf{Gabrieli} y \textbf{Zarlino}, serán los impulsores de la escuela \dotfill
		
		\item 
		El \textit{Service} (servicio) y el \textit{Anthem} (que podríamos traducir cómo Himno), forman parte de los oficios y la liturgia respectivamente. ¿En qué escuela los situarías? \dotfill
%
% Pregunta:
%%
		\end{enumerate}
 \end{ejercicio}
%

%
% EJERCICIO.- 
%
\begin{ejercicio}[]
	\begin{enumerate}[1.-]
%
% Pregunta:
%
	\item 
	A partir de 1420, se hizo habitual un modelo que se puede considerar el primer tipo importante de misa como género musical: \textbf{la misa sobre cantus firmus}. Los cinco cantos —\textit{Kyrie, Gloria, Credo, Sanctus y Agnus Dei}— 
	¿Verdadero o falso? \ldots
%
%
% Pregunta:
%
	\item 
	En los siglos XV y XVI se desarrollan también otros tipos de misa polifónica, entre los que destaca la \textbf{Misa de difuntos}, llamada también de \textit{Requiem}. ¿Verdadero o falso? \ldots
	\end{enumerate}
\end{ejercicio}
%
% EJERCICIO.-  
%
\begin{ejercicio}[]
	\begin{enumerate}[1.-]
%
% Pregunta: 
%
	\item 
	Los instrumentos con capacidades polifónicas —cuerda pulsada y tecla— solían interpretar de forma individal (solista) piezas polifónicas de origen vocal o compuestas específicamente para el instrumento. \\Indica verdadero o falso:
		\begin{enumerate}[a)]
			\item El \textbf{Laúd} es el instrumento de cámara principal en toda Europa, tanto para interpretación solista como para acompañar a la voz. Solía tener seis \textit{órdenes} de cuerdas dobles afinadas por cuartas justas excepto las dos centrales, separadas por una tercera mayor. Tenía forma abombada, mástil corto con trastes móviles y el clavijero casi perpendicular al mástil. \ldots
			\item La \textbf{Vihuela} es el equivalente español del laúd, con mismo número de órdenes y afinación similar, pero con forma parecida a la de la guitarra. \ldots
			\item La \textbf{Guitarra} era mucho más pequeña que la actual con cuatro órdenes, con afinación similar a las cuatro primeras cuerdas de una guitarra actual. \ldots
			\item El \textbf{Clave} es un instrumento de tecla, donde el teclado acciona unos \textit{plectros} que pulsan las cuerdas, por lo que se trata de un instrumento de cuerda pulsada, aunque manejado por teclado. \ldots
		\end{enumerate}
%
% Pregunta: 
%
	\item
	Los conjuntos instrumentales solían utilizar como fuente la música escrita para conjunto vocal, por lo que la notación habitual era la misma que en el canto.
	La costumbre de la época era escribir las voces en cuadernos separados y no en partitura, por lo que para los instrumentos polifónicos era necesario copiar la música a un formato que permitiera ver todas las voces simultáneamente e interpretarlas en un solo instrumento. A esta operación se le denominaba \textbf{intabulación} y las notaciones utilizadas se conocen como \dotfill \par


%
	\end{enumerate}
\end{ejercicio}
%%%	
%
%
\end{document}
%Fin de Hoja de ejercicios
%