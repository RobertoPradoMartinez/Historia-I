% -----------------------------------
% FICHAS DE AUDICIÓNS PARA COMPLETAR:
% -----------------------------------
\subsection*{Exercicio de audición sen partitura}
%
Escoita con atención as audicións propostas para este trimestre, completa as fichas e responde ás cuestións.
%
%
% AUDICIÓN 1.- NUPER ROSARUM FLORES
% ---------------------------------
% 
%
\vspace*{0.25cm}

\begin{ejercicio}[] 
%
Completa a ficha da obra proposta como exercicio de audición.
%
\begin{multicols}{2}
	\begin{enumerate}[1.-]
        \vspace*{0.3cm}
		\item
			Autor: \dotfill
			\vspace*{0.3cm}
		\item
			Obra:
			\begin{enumerate}[a)]
			    \item Título: \dotfill \vspace*{0.3cm}
			    \item Período: \dotfill \vspace*{0.3cm}
			    \item Forma: \dotfill \vspace*{0.3cm}
			    \item Timbre: \dotfill 			\vspace*{0.3cm}
			    \item Textura: \dotfill \vspace*{0.3cm}
			    \item Estilo: \dotfill \vspace*{0.3cm}
			    \item Xénero: \dotfill 
			    \vspace*{0.3cm}
			\end{enumerate}
	\end{enumerate}
\end{multicols}
	\begin{itemize}
%	    \item 
%Identifica a escola e xeración, á que podería pertencer o autor desta obra:
%    \vspace*{0.25cm}
    
%    \begin{tabular}{l l l l}
%        a) Franco-flamenca; $1^a$ xeración \\ 
%        b) Franco-flamenca; $2^a$ xeración \\
%        c) Franco-flamenca; $4^a$ xeración \\
%        d) Franco-flamenca; $3^a$ xeración \\
%    \end{tabular}
    
    \item
    Partindo dos datos dispoñibles na ficha de audición, realiza un breve comentario da obra.
    \vspace*{12cm}
	\end{itemize}

\end{ejercicio}
%
%%
% AUDICIÓN 2.- 
% ----------------------
%
\begin{ejercicio}[]
%
Completa a ficha da obra proposta como exercicio de audición.
%
\begin{multicols}{2}
	\begin{enumerate}[1.-]
        \vspace*{0.3cm}
		\item
			Autor: \dotfill
			\vspace*{0.3cm}
		\item
			Obra:
			\begin{enumerate}[a)]
			    \item Título: \dotfill \vspace*{0.3cm}
			    \item Período: \dotfill \vspace*{0.3cm}
			    \item Forma: \dotfill \vspace*{0.3cm}
			    \item Timbre: \dotfill \vspace*{0.3cm} 		
			    \item Textura: \dotfill \vspace*{0.3cm}
			    \item Estilo: \dotfill \vspace*{0.3cm}
			    \item Xénero: \dotfill \vspace*{0.3cm}
			\end{enumerate}
%		\item 
%		    Resume as principais características que definen a obra:
%			\vspace*{8.0cm}			

	\end{enumerate}
\end{multicols}
	\begin{itemize}
%	    \item 
%Identifica a escola e xeración, á que podería pertencer o autor desta obra:
%    \vspace*{0.25cm}
    
%    \begin{tabular}{l l l l}
%        a) Franco-flamenca; $1^a$ xeración \\ 
%        b) Franco-flamenca; $2^a$ xeración \\
%        c) Franco-flamenca; $4^a$ xeración \\
%        d) Franco-flamenca; $3^a$ xeración \\
%    \end{tabular}
    
    \item
    Partindo dos datos dispoñibles na ficha de audición, realiza un breve comentario da obra.
    \vspace*{14cm}
	\end{itemize}
\end{ejercicio}
%
