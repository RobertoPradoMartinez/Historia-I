%
% ----
\hideanswers % oculta respostas
%
% Cargamos os Exercicios:
% ----------------------

\loadallproblems[Tema4-RENA]{../../Cuestions/Modelo-cuestions.tex}

%\loadallproblems[Tema3-GREGO1]{../../Cuestions/Tema3-Canto-gregoriano-expansion.tex} %
%\loadallproblems[Tema3-NOTA]{../../Cuestions/Tema3-Notacion-modal.tex} %
%\loadallproblems[Tema1-PENS]{../../Cuestions/Tema1-Pensamento.tex} %
%\loadallproblems[Tema1-RO]{../../Cuestions/Tema1-Roma.tex} %
%\loadrandomproblems[loops]{2}{loops}% aleatorios
%
% FOLLA EXERCICIOS: 
% -----------------
% Comentario para realizar o simulacro de exame:
\subsection*{Cuestionario autoavaliación} \label{subsec:autoaval}
Realiza o seguinte exercicio de autoavaliación, prestando atención a aquelas cuestións onde a resposta non sexa clara e ofreza dúbidas. Anota todas aquelas que consideres oportunas para debatilas coas túas compañeiras e compañeiros de clase.
\begin{multicols}{2} % a 2 columnas
    \begin{enumerate}
%\useproblem{input}
    \foreachproblem[Tema4-RENA]{\item\label{prob:\thisproblemlabel}\thisproblem}
%    \foreachproblem[Tema3-GREGO1]{\item\label{prob:\thisproblemlabel}\thisproblem}
 %   \foreachproblem[Tema3-NOTA]{\item\label{prob:\thisproblemlabel}\thisproblem}
%    \foreachproblem[Tema1-PENS]{\item\label{prob:\thisproblemlabel}\thisproblem}
    \end{enumerate}
\end{multicols}
%
% SOLUCIONS:
% ----------
%\newpage
%\begin{multicols}{2}
%\showanswers
%\begin{itemize}
%\foreachdataset{\thisdataset}{%
%\foreachproblem[\thisdataset]{\item[\ref{prob:\thisproblemlabel}]\thisproblem}
%}
%\end{itemize}
%\end{multicols}
