%
%%%%%%%%%%%%%%%%%%%%%%%%%%%%%%%%%%%%%%%%%%%%%%%%%%%%%%%%%%
%   PLANTILLA EXERCICIOS DE HISTORIA DE LA MÚSICA I      %
% Este é un modelo para redactar cadernos de exercicios  %
%%%%%%%%%%%%%%%%%%%%%%%%%%%%%%%%%%%%%%%%%%%%%%%%%%%%%%%%%%
% Pasos para cubrir la plantilla:
% 1) Realizar una copia de este modelo
% 2) Renombrar el archivo:
%		"HM1_Hoja(número).tex"
% 3) El número de Hoja debe ser correlativo
% 
%%%%%%%%%%%%%%%%%%%%%%%%%%%%%%%%%%%%%%%%%%%%%%%%%%%%%%%%%%
%
% Esta plantilla es para crear ejercicios de esta materia
% Se recomienda crear un archivo por cada tema
% Descomentar según se necesite utilizar un modelo de ejercicio u otro
% Clase de documento:
\documentclass[letterpaper,12pt,notitlepage,spanish]{article}
%
% Arquivo externo de configuración
% --------------------------------
% Seleccionar o idioma:
%%%%%%%%%%%%%%%%%%%%%%%%%%%%%%%%%%%%%%%%%%%
%% ---------- MODELO EJERCICIOS ---------- 
%% MATERIA: HISTORIA
%% CURSO: 
%% AÑO ACADÉMICO: 
%% CENTRO: 
%%%%%%%%%%%%%%%%%%%%%%%%%%%%%%%%%%%%%%%%%%%
%% 
%% MODELO PARA REDACTAR EJERCICIOS
%% ===============================
%% 
%% Clase de documento
%% ------------------
%\documentclass[letterpaper,12pt,notitlepage,spanish]{article}
%\documentclass[12pt,a4paper,notitlepage]{article}
%
% Márgenes de documento
% ---------------------
\usepackage[left=2.0cm, right=2.0cm, lines=45, top=2.5cm, bottom=2.0cm]{geometry}
%
% Paquetes necesarios
% -------------------
\usepackage[utf8]{inputenc} % acentos en ES
\usepackage[spanish,activeacute, es-tabla]{babel}
\usepackage{enumerate} % entornos de listas
\usepackage{multicol}  % varias columnas texto
\usepackage{fancyhdr}  % encabezado personalizado
\usepackage{fancybox}  % entornos con cajas
\usepackage{pdfpages}  % páginas pdf
%
\usepackage{lipsum} % generar texto aleatorio "loren ipsum"
\usepackage{environ} 
\usepackage{probsoln} % paquete para soluciones
%\showanswers % para mostrar soluciones
%
%Esto es lo importante. Ponemos la solución al margen.
\NewEnviron{solutionnew}{%
%  \leavevmode\marginpar{\raggedright\footnotesize \textbf{Solución:}\\ \BODY}
%  \textbf{Solución:}\\ \BODY} % sol. con salto de liña
  \small{Solución:} \BODY} % sol. na mesma liña
  {}
\renewenvironment{solution}{\solutionnew}{\endsolutionnew}
%
% FIGURAS EN COLUMNAS:
\newenvironment{Figura}
  {\par\medskip\noindent\minipage{\linewidth}}
  {\endminipage\par\medskip}
% ---
%
% Lineas de encabezado y pié
% --------------------------
\renewcommand{\headrulewidth}{0.5pt}
%\renewcommand{\headrulewidth}{1.0pt}
\renewcommand{\footrulewidth}{0.5pt}
%\renewcommand{\footrulewidth}{1.0pt}
\pagestyle{fancy} % estilo de página
%
% Recuadros y figuras
% -------------------
\newcommand\Loadedframemethod{TikZ}
\usepackage[framemethod=\Loadedframemethod]{mdframed}
\usepackage{tikz}
\usetikzlibrary{calc,through,backgrounds}
\usetikzlibrary{matrix,positioning}
%Desssins geometriques
\usetikzlibrary{arrows}
\usetikzlibrary{shapes.geometric}
\usetikzlibrary{datavisualization}
\usetikzlibrary{automata} % LATEX and plain TEX
\usetikzlibrary{shapes.multipart}
\usetikzlibrary{decorations.pathmorphing} 
\usepackage{pgfplots}
\usepackage{physics}
\usepackage{titletoc}
\usepackage{mathpazo} 
\usepackage{algpseudocode}
\usepackage{algorithmicx} 
\usepackage{bohr} 
\usepackage{xlop} 
\usepackage{bbding} 
%\usepackage{minibox} 
% Texto árabe
\usepackage{mathdesign}
\usepackage{bbding} 
% --
% Tipograía:
% ----------
% Fuente HEURÍSTICA (cómoda de leer)
%\usepackage{heuristica}
% Fuente LIBERTINE (cómoda para apuntes)
\usepackage{libertineRoman}
%\usepackage[proportional]{libertine}
% Fuente ROMANDE (estilo antiguo pero no muy cómoda)
%\usepackage{romande} %
% 
% Encabezado y pié de página (textos)
% -----------------------------------
% Modelo 1:
% ---------
% texto de encabezado izquierda:
%\lhead{\normalfont{Historia de la Música I}}
% texto encabezado centro:
%\chead{\textbf{Ejercicios}}
% texto de encabezado derecha:
%\rhead{\normalfont{curso: 2020/2021}}
% texto pié izquierdo:
%\lfoot{\small{\textit{}}}
% texto pié centrado:
%\cfoot{\textsc{Pág. \thepage }}
% texto pié derecho
%\rfoot{\textit{Pr. $\mathcal{A}$.Kaal}}
% ----------
% Modelo 2:
% ---------
% Encabezado y pié de página (textos)
% -----------------------------------
% texto de encabezado izquierda:
%
%\lhead{
%	\hrule
%	\vspace*{0.20cm}
%	\normalfont{Historia de la Música I}
%	\vspace*{0.10cm}
	%\hrule
%}
% texto encabezado centro:
%\chead{
%	\textbf{Cuestionario de Ejercicios}
%	\vspace*{0.08cm}}
% texto de encabezado derecha:
%\rhead{
%	\normalfont{curso: 2020/2021}
%	\vspace*{0.08cm}}
%
% texto pié izquierdo:
%\lfoot{
	%\begin{center}
		%\vspace*{0.20cm}
		%\hrule
		%\small{
		%Conservatorio Profesional de Música de Viveiro - Avda. da mariña s/n - (27850) Viveiro - Lugo
		%	}
	%\end{center}
%}
% texto pié centrado:
%\cfoot{
	%\vspace*{0.30cm}
	%\hrule
	%\vspace*{0.90cm}
%	\small{- Página \thepage -  }\\
	%\small{Conservatorio Profesional de Música de Viveiro}\\
	%\small{avda. da Mariña s/n}
%}

% ----------
% Modelo 3:
% ---------
% Encabezado y pié de página (textos)
% -----------------------------------
% texto de encabezado izquierda:
%
\lhead{
	\hrule
	\vspace*{0.20cm}
	\normalfont{Historia da Música I}
	\vspace*{0.10cm}
	%\hrule
}
% texto encabezado centro:
\chead{
	\textbf{CADERNO DE EXERCICIOS}
	\vspace*{0.08cm}}
% texto de encabezado derecha:
\rhead{
	\normalfont{curso: 2021/2022}
	\vspace*{0.08cm}}
%
% texto pié izquierdo:
%\lfoot{
	%\begin{center}
		%\vspace*{0.20cm}
		%\hrule
		%\small{
		%Conservatorio Profesional de Música de Viveiro - Avda. da mariña s/n - (27850) Viveiro - Lugo
		%	}
	%\end{center}
%}
% texto pié centrado:
\cfoot{
	%\vspace*{0.30cm}
	%\hrule
	%\vspace*{0.90cm}
	\small{- \thepage -  }\\
	%\small{Conservatorio Profesional de Música de Viveiro}\\
	%\small{avda. da Mariña s/n}
}

% --------
%=====================Algo setup
\algblock{If}{EndIf}
\algcblock[If]{If}{ElsIf}{EndIf}
\algcblock{If}{Else}{EndIf}
\algrenewtext{If}{\textbf{si}}
\algrenewtext{Else}{\textbf{sinon}}
\algrenewtext{EndIf}{\textbf{finsi}}
\algrenewtext{Then}{\textbf{alors}}
\algrenewtext{While}{\textbf{tant que}}
\algrenewtext{EndWhile}{\textbf{fin tant que}}
\algrenewtext{Repeat}{\textbf{r\'ep\'eter}}
\algrenewtext{Until}{\textbf{jusqu'\`a}}
\algcblockdefx[Strange]{If}{Eeee}{Oooo}
[1]{\textbf{Eeee} "#1"}
{\textbf{Wuuuups\dots}}

\algrenewcommand\algorithmicwhile{\textbf{TantQue}}
\algrenewcommand\algorithmicdo{\textbf{Faire}}
\algrenewcommand\algorithmicend{\textbf{Fin}}
\algrenewcommand\algorithmicrequire{\textbf{Variables}}
\algrenewcommand\algorithmicensure{\textbf{Constante}}% replace ensure by constante
\algblock[block]{Begin}{End}
\newcommand\algo[1]{\textbf{algorithme} #1;}
\newcommand\vars{\textbf{variables } }
\newcommand\consts{\textbf{constantes}}
\algrenewtext{Begin}{\textbf{debut}}
\algrenewtext{End}{\textbf{fin}}
%================================
%================================

\setlength{\parskip}{1.25cm}
\setlength{\parindent}{1.25cm}
\tikzstyle{titregris} =
[draw=gray,fill=gray, shading = exersicetitle, %
text=gray, rectangle, rounded corners, right,minimum height=.3cm]
\pgfdeclarehorizontalshading{exersicebackground}{100bp}
{color(0bp)=(green!40); color(100bp)=(black!5)}
\pgfdeclarehorizontalshading{exersicetitle}{100bp}
{color(0bp)=(red!40);color(100bp)=(black!5)}
\newcounter{exercise}
%\renewcommand*\theexercise{exercice \textbf{Ejercicio}~n\arabic{exercise}} % CASTELÁN
\renewcommand*\theexercise{exercice \textbf{Exercicio}~n\arabic{exercise}} % GALEGO
\makeatletter
\def\mdf@@exercisepoints{}%new mdframed key:
\define@key{mdf}{exercisepoints}{%
\def\mdf@@exercisepoints{#1}
}
\mdfdefinestyle{exercisestyle}{%
outerlinewidth=1em,outerlinecolor=white,%
leftmargin=-1em,rightmargin=-1em,%
middlelinewidth=0.5pt,roundcorner=3pt,linecolor=black,
apptotikzsetting={\tikzset{mdfbackground/.append style ={%
shading = exersicebackground}}},
innertopmargin=0.1\baselineskip,
skipabove={\dimexpr0.1\baselineskip+0\topskip\relax},
skipbelow={-0.1em},
needspace=0.5\baselineskip,
frametitlefont=\sffamily\bfseries,
settings={\global\stepcounter{exercise}},
singleextra={%
\node[titregris,xshift=0.5cm] at (P-|O) %
{~\mdf@frametitlefont{\theexercise}~};
\ifdefempty{\mdf@@exercisepoints}%
{}%
{\node[titregris,left,xshift=-1cm] at (P)%
{~\mdf@frametitlefont{\mdf@@exercisepoints points}~};}%
},
firstextra={%
\node[titregris,xshift=1cm] at (P-|O) %
{~\mdf@frametitlefont{\theexercise}~};
\ifdefempty{\mdf@@exercisepoints}%
{}%
{\node[titregris,left,xshift=-1cm] at (P)%
{~\mdf@frametitlefont{\mdf@@exercisepoints points}~};}%
},
}
\makeatother


%%%%%%%%%

%%%%%%%%%%%%%%%
\mdfdefinestyle{theoremstyle}{%
outerlinewidth=0.01em,linecolor=black,middlelinewidth=0.5pt,%
frametitlerule=true,roundcorner=2pt,%
apptotikzsetting={\tikzset{mfframetitlebackground/.append style={%
shade,left color=white, right color=blue!20}}},
frametitlerulecolor=black,innertopmargin=1\baselineskip,%green!60,
innerbottommargin=0.5\baselineskip,
frametitlerulewidth=0.1pt,
innertopmargin=0.7\topskip,skipabove={\dimexpr0.2\baselineskip+0.1\topskip\relax},
frametitleaboveskip=1pt,
frametitlebelowskip=1pt
}
\setlength{\parskip}{0mm}
\setlength{\parindent}{10mm}
%\mdtheorem[style=theoremstyle]{ejercicio}{\textbf{Ejercicio}} % Castelán
\mdtheorem[style=theoremstyle]{ejercicio}{\textbf{Exercicio}} % Galego
%================Liste definition--numList-and alphList=============
\newcounter{alphListCounter}
\newenvironment
{alphList}
{\begin{list}
{\alph{alphListCounter})}
{\usecounter{alphListCounter}
\setlength{\rightmargin}{0cm}
\setlength{\leftmargin}{0.5cm}
\setlength{\itemsep}{0.2cm}
\setlength{\partopsep}{0cm}
\setlength{\parsep}{0cm}}
}
{\end{list}}
\newcounter{numListCounter}
\newenvironment
{numList}
{\begin{list}
{\arabic{numListCounter})}
{\usecounter{numListCounter}
\setlength{\rightmargin}{0cm}
\setlength{\leftmargin}{0.5cm}
\setlength{\itemsep}{0cm}
\setlength{\partopsep}{0cm}
\setlength{\parsep}{0cm}}
}
{\end{list}}
%
%
%% -- Fin del archivo de configuración  --
%% % Galego
%\input{../../Modelos/include/config-HM1ejercicios_ES.tex} % Castelán
% --------------------------------
\usepackage{graphicx}
\usepackage{hyperref}

%
%Ruta absoluta en formato tipo Unix (Linux, OsX)
%\graphicspath{{../../figures/}}
%
\setlength{\columnsep}{1cm} % separación entre columnas
\setlength{\columnseprule}{0.75pt}
\begin{document}
%
% DATOS DE HOJA DE EJERCICIOS
% ---------------------------
%
% TÍTULO DE LA HOJA DE EJERCICIOS:
%
\begin{center}
\Large{
3º Trimestre
} \\
\vspace*{0.5cm}
%
% NÚMERO DE FOLLA:
%
%\normalsize % Número de hoja:
%(XVII - XVIII)
%\\
% Datos do alumnado:
% ------------------
\vspace{1.10cm}
	\begin{flushleft}
	Nome e Apelidos: \hrulefill\\
	%\vspace*{0.50cm}
%		\begin{center}
%		\small{Instrucciones para realizar los ejercicios}\\		
%		\end{center}
%	\hrulefill \\
	%\vspace*{0.25cm}
%
%\small{ % INSTRUCCIONES:
%\texttt{Lee con atención y realiza con detenimiento, los siguientes ejercicios teniendo en cuenta lo que se indica en cada uno. \\
%}} % fin instrucciones.
%
	\vspace*{0.25cm}		
 	\end{flushleft}
\end{center}

% --------------------------------------------
% ESPACIO PARA COMPOÑER A FOLLA DE EXERCICIOS:
% --------------------------------------------
% CABECEIRA EXERCICIOS:
\input{cabeceira-exercicios.txt}
%
% EXERCICIO 1.- Contextualización Renacemento
% -------------------------------------------
%% EXERCICIOS PARA INCLUÍR DENTRO DO CADERNO DE EXERCICIOS %%
%
% EXERCICIO.- RENACEMENTO: Contextualización
%
\section{A música no <<Renacemento>> } 
%
% -------------------------------
% BANCO DE PREGUNTAS DE HISTORIA
% -------------------------------
%
% Tema 4.- MÚSICA NO RENACEMENTO
% ------------------------------
% Comentario.- 
%
% EXERCICIO.- 
% -------------------------------------
%\newproblem{T4RENA-01}{
%
Adoita situarse a orixe do Renacemento como fenómeno cultural e artístico, nas cidades italianas do norte e en Roma, xunto coas de Flandes e Países Baixos e, por tanto, relacionado con áreas de forte desenvolvemento urbano e comercial. Desde estes focos iniciais, o Renacemento estenderíase a toda Europa paulatinamente. A cronoloxía clásica distingue entre \href{http://es.wikipedia.org/wiki/Quattrocento}{\emph{quattrocento}} e \href{http://es.wikipedia.org/wiki/Cinquecento}{\emph{cinquecento}.}
%
\par
\vspace*{0.25cm}
%
\begin{ejercicio}[Periodización - Renacemento]
Lé con atención o seguinte texto e responde a cuestión.
%\begin{multicols}{2}
    \begin{quote}
    O concepto de \href{http://es.wikipedia.org/wiki/Renacimiento}{Renacemento} aparece para a historiografía da arte e da cultura en época tan distante como o século XIX. Este termo refírese á recuperación da cultura da \href{http://es.wikipedia.org/wiki/Antig\%C3\%BCedad_cl\%C3\%A1sica}{Antigüidade} \href{http://es.wikipedia.org/wiki/Antig\%C3\%BCedad_cl\%C3\%A1sica}{clásica} tras o longo período, supostamente escuro, que suporía a \href{http://es.wikipedia.org/wiki/Edad_Media}{Idade Media.} [\ldots] \\
    Efectivamente, os séculos XV e XVI trouxeron grandes cambios, así como a formación dalgúns dos principios estruturais que estiveron operativos na cultura europea até as revolucións burguesas dos séculos XVIII e XIX. De feito, basta lembrar que a historiografía fai comezar no século (\ldots) unha nova \href{http://es.wikipedia.org/wiki/Edad_Moderna}{Idade Moderna.}
\end{quote}
\begin{flushright}
 J.Jurado: \textit{Apuntamentos para a historia da música}
\end{flushright}
%
Tendo en conta a periodización que coñeces, a que século se está a referir o autor da cita? \ldots
\par
\vspace*{0.15cm}
    \begin{tabular}{c c c c}
        a) XIV & b) XV & c) XIII & d) XVI  \\
    \end{tabular}

%\end{multicols}
\end{ejercicio}
%}
% {a)}
% comentario da resposta:
%    \\ \small{Indica o comentario}
%


% --------------------------------------------
\subsection*{Trazos culturais do Renacemento}
%
% -----------------------------------------
% CUESTIÓN: TRAZOS DA CULTURA RENACENTISTA 
% -----------------------------------------
% Comentario.- 
%
%
% EXERCICIO.- 
% -------------------------------------
%\newproblem{T4RENA-02}{
%
Facendo unha síntese, podemos enunciar os trazos xerais da cultura do
Renacemento: 
\begin{multicols}{2}
Crecemento económico e demográfico: o Renacemento ve nacer os principios da economía capitalista (bancos, letras, crédito...). Paralelamente, obsérvase un crecemento das cidades e das clases que lle son propias: a burguesía. O propio concepto moderno de Estado alcanza a súa formulación nas obras de \href{http://es.wikipedia.org/wiki/Maquiavelo}{Maquiavelo.} A nivel social, a expansión das clases burguesas favoreceu a demanda dunha arte laica, en detrimento do protagonismo da arte relixioso que dominara o período anterior.
\par
Desenvolvemento científico e tecnolóxico, ilustrado pola revolución \href{http://es.wikipedia.org/wiki/Cop\%C3\%A9rnico}{copernicana} e a creatividade de personaxes como \href{http://es.wikipedia.org/wiki/Leonardo_da_Vinci}{Leonardo da Vinci.} sentan as bases do \href{http://es.wikipedia.org/wiki/M\%C3\%A9todo_cient\%C3\%ADfico}{método científico} e experimental que se desenvolve con forza e que abre a porta á ciencia moderna e á sociedade tecnolóxica.
\par
A nivel cultural pode falarse dun xiro fundamental coa invención da \href{http://es.wikipedia.org/wiki/Imprenta}{imprenta} a mediados do XV e as posibilidades de expansión de ideas que iso supón. O vigor das universidades e a circulación de información favoreceron a expansión do \href{http://es.wikipedia.org/wiki/Humanismo}{Humanismo.} O Humanismo comprende toda unha antropoloxía dentro da cal a persoa pasa a ocupar un lugar central como punto desde o que se observa e valora a realidade. O pensamento individual asume a responsabilidade de elaborar unha interpretación correcta do mundo mediante un medio crítico baseado na experimentación. O punto de vista crítico do Humanismo respecto diso dunha interpretación dogmática do mundo é, en boa parte, o desencadenamento da crise relixiosa e dos movementos protestantes do XVI.
\par
A creación artística será un dos aspectos máis rechamantes do período. En primeiro lugar polo novo concepto de arte: o artista xa non é un artesán ao servizo da inspiración divina, senón un creador que aspira ao status de home de ciencia. Poucos períodos da Historia de Occidente coñeceron un ritmo de produción de obras de arte tan intenso en calidade e cantidade como este, porque posta ao servizo da exaltación do poder persoal (príncipes, papas...) a creación artística convértese nun elemento de prestixio privilexiado, facendo do mecenado unha institución obrigada para calquera poderoso.\\
\par
\vspace*{0.25cm}
(J.Jurado \textit{Apuntamentos para a Historia da Música}. Ed. Dos Acordes. Vigo 2017.)
\\
%\hrulefill
%
\end{multicols}

\begin{ejercicio}[Trazos da cultura do Renacemento]
Realiza un resumo das principais características que consideras definen a cultura renacentista, tendo en conta o texto do Profesor Jurado. 
\par
\vspace*{9.5cm}
\end{ejercicio}
%}
% {a)}
% comentario da resposta:
%    \\ \small{Indica o comentario}
%


%\newpage
%
% EXERCICIO 2.- Contexto Escolas Renacemento
% ------------------------------------------
%% EXERCICIOS PARA INCLUÍR DENTRO DO CADERNO DE EXERCICIOS %%
%
% EXERCICIO.- RENACEMENTO: Contextualización
%
% --------------------------------------------
\section*{Escolas e xeracións de compositores do renacemento}
%
\subsection*{A escola franco-flamenca}

\begin{multicols}{2}

A guerra dos Cen Anos conduciu a unha diminución da importancia musical de Francia, que fai desviar a hexemonía musical desta nación e de Italia cara a Inglaterra, Borgoña e, sobre todo, Flandes. Estas cortes serán escenario de festas relixiosas e profanas nas que a música ocupa un
posto relevante. Nelas fúndanse capelas musicais principescas a imitación da papal.
\par
Na evolución da música do continente atopamos dúas tendencias: unha, central, herdeira do \emph{Ars Nova} e outra, periférica, de influencia inglesa. Foi esencial neste sentido a achega de Dunstable, que estaba ao servizo do duque de Bedford e, por tanto, pertencía ás tropas invasoras do continente, e que entrou en contacto con músicos franco-flamencos e borgoñones.\\
A súa música caracterízase por:
%
\begin{itemize}
\item
  \textbf{Ritmos regulares} e \textbf{melodías sinxelas}
\item
  \textbf{Motetes a tres voces} (algún a catro); aínda emprega a isorritmia nalgún deles
\item
  \textbf{Cancións a tres voces} con influencias tanto italianas como da \emph{Chanson}.
\end{itemize}
\end{multicols}
%
% CUESTIÓN:
% ---------
\begin{ejercicio}[Compositores franco-flamencos ou borgoñones]
Dentro da \textbf{escola franco-flamenca}, adóitanse distinguir varias xeracións de compositores. Completa os seguintes apartados:
\begin{multicols}{2}

\begin{itemize}
    \item A \textbf{primeira xeración}, ten como principais representantes a \dotfill \\
    Principais características: 
    \par
    \vspace*{2.5cm}
    \item  A \textbf{segunda xeración} está representada por \dotfill \\
    Principais características: 
    \par
    \vspace*{2.5cm}
    \item A \textbf{terceira xeración} está representada sobre todo por \dotfill \\ 
    Principais características:
    \par
    \vspace*{2.5cm}
    \item  Na \textbf{cuarta xeración}, os compositores máis destacados son \dotfill \\
    Principais características:
    \par
    \vspace*{2.5cm}
    \item  A \textbf{quinta xeración}, ten como principal representante a \dotfill \\
    Principais características:
    \par
    \vspace*{2cm}    
\end{itemize}
\end{multicols}
\end{ejercicio}


%\newpage
%
% EXERCICIO 3.- Pensamento Musical Renacemento
% --------------------------------------------
% TODO: Pendiente de realizar !
% -----------------------------------
% FICHAS DE AUDICIÓNS PARA COMPLETAR:
% -----------------------------------
\subsection*{Exercicio de audición sen partitura}
%
Escoita con atención as audicións propostas para este trimestre, completa as fichas e responde ás cuestións.
%
%
% AUDICIÓN 1.- NUPER ROSARUM FLORES
% ---------------------------------
% 
%
\vspace*{0.25cm}

\begin{ejercicio}[] 
%
Completa a ficha da obra proposta como exercicio de audición.
%
\begin{multicols}{2}
	\begin{enumerate}[1.-]
        \vspace*{0.3cm}
		\item
			Autor: \dotfill
			\vspace*{0.3cm}
		\item
			Obra:
			\begin{enumerate}[a)]
			    \item Título: \dotfill \vspace*{0.3cm}
			    \item Período: \dotfill \vspace*{0.3cm}
			    \item Forma: \dotfill \vspace*{0.3cm}
			    \item Timbre: \dotfill 			\vspace*{0.3cm}
			    \item Textura: \dotfill \vspace*{0.3cm}
			    \item Estilo: \dotfill \vspace*{0.3cm}
			    \item Xénero: \dotfill 
			    \vspace*{0.3cm}
			\end{enumerate}
	\end{enumerate}
\end{multicols}
	\begin{itemize}
%	    \item 
%Identifica a escola e xeración, á que podería pertencer o autor desta obra:
%    \vspace*{0.25cm}
    
%    \begin{tabular}{l l l l}
%        a) Franco-flamenca; $1^a$ xeración \\ 
%        b) Franco-flamenca; $2^a$ xeración \\
%        c) Franco-flamenca; $4^a$ xeración \\
%        d) Franco-flamenca; $3^a$ xeración \\
%    \end{tabular}
    
    \item
    Partindo dos datos dispoñibles na ficha de audición, realiza un breve comentario da obra.
    \vspace*{12cm}
	\end{itemize}

\end{ejercicio}
%
%%
% AUDICIÓN 2.- 
% ----------------------
%
\begin{ejercicio}[]
%
Completa a ficha da obra proposta como exercicio de audición.
%
\begin{multicols}{2}
	\begin{enumerate}[1.-]
        \vspace*{0.3cm}
		\item
			Autor: \dotfill
			\vspace*{0.3cm}
		\item
			Obra:
			\begin{enumerate}[a)]
			    \item Título: \dotfill \vspace*{0.3cm}
			    \item Período: \dotfill \vspace*{0.3cm}
			    \item Forma: \dotfill \vspace*{0.3cm}
			    \item Timbre: \dotfill \vspace*{0.3cm} 		
			    \item Textura: \dotfill \vspace*{0.3cm}
			    \item Estilo: \dotfill \vspace*{0.3cm}
			    \item Xénero: \dotfill \vspace*{0.3cm}
			\end{enumerate}
%		\item 
%		    Resume as principais características que definen a obra:
%			\vspace*{8.0cm}			

	\end{enumerate}
\end{multicols}
	\begin{itemize}
%	    \item 
%Identifica a escola e xeración, á que podería pertencer o autor desta obra:
%    \vspace*{0.25cm}
    
%    \begin{tabular}{l l l l}
%        a) Franco-flamenca; $1^a$ xeración \\ 
%        b) Franco-flamenca; $2^a$ xeración \\
%        c) Franco-flamenca; $4^a$ xeración \\
%        d) Franco-flamenca; $3^a$ xeración \\
%    \end{tabular}
    
    \item
    Partindo dos datos dispoñibles na ficha de audición, realiza un breve comentario da obra.
    \vspace*{14cm}
	\end{itemize}
\end{ejercicio}
%

%\newpage
%
% EXERCICIO 4.- Música e músicos do Renacemento
% ---------------------------------------------
%\input{Exercicio-Renacemento-Italia.tex}
%\newpage
%
% EXERCICIO 5.- Audicións e comentarios
% -------------------------------------
%% -----------------------------------
% FICHAS DE AUDICIÓNS PARA COMPLETAR:
% -----------------------------------
\subsection*{Exercicio de audición sen partitura}
%
Escoita con atención as audicións propostas para este trimestre, completa as fichas e responde ás cuestións.
%
%
% AUDICIÓN 1.- NUPER ROSARUM FLORES
% ---------------------------------
% 
%
\vspace*{0.25cm}

\begin{ejercicio}[] 
%
Completa a ficha da obra proposta como exercicio de audición.
%
\begin{multicols}{2}
	\begin{enumerate}[1.-]
        \vspace*{0.3cm}
		\item
			Autor: \dotfill
			\vspace*{0.3cm}
		\item
			Obra:
			\begin{enumerate}[a)]
			    \item Título: \dotfill \vspace*{0.3cm}
			    \item Período: \dotfill \vspace*{0.3cm}
			    \item Forma: \dotfill \vspace*{0.3cm}
			    \item Timbre: \dotfill 			\vspace*{0.3cm}
			    \item Textura: \dotfill \vspace*{0.3cm}
			    \item Estilo: \dotfill \vspace*{0.3cm}
			    \item Xénero: \dotfill 
			    \vspace*{0.3cm}
			\end{enumerate}
	\end{enumerate}
\end{multicols}
	\begin{itemize}
%	    \item 
%Identifica a escola e xeración, á que podería pertencer o autor desta obra:
%    \vspace*{0.25cm}
    
%    \begin{tabular}{l l l l}
%        a) Franco-flamenca; $1^a$ xeración \\ 
%        b) Franco-flamenca; $2^a$ xeración \\
%        c) Franco-flamenca; $4^a$ xeración \\
%        d) Franco-flamenca; $3^a$ xeración \\
%    \end{tabular}
    
    \item
    Partindo dos datos dispoñibles na ficha de audición, realiza un breve comentario da obra.
    \vspace*{12cm}
	\end{itemize}

\end{ejercicio}
%
%%
% AUDICIÓN 2.- 
% ----------------------
%
\begin{ejercicio}[]
%
Completa a ficha da obra proposta como exercicio de audición.
%
\begin{multicols}{2}
	\begin{enumerate}[1.-]
        \vspace*{0.3cm}
		\item
			Autor: \dotfill
			\vspace*{0.3cm}
		\item
			Obra:
			\begin{enumerate}[a)]
			    \item Título: \dotfill \vspace*{0.3cm}
			    \item Período: \dotfill \vspace*{0.3cm}
			    \item Forma: \dotfill \vspace*{0.3cm}
			    \item Timbre: \dotfill \vspace*{0.3cm} 		
			    \item Textura: \dotfill \vspace*{0.3cm}
			    \item Estilo: \dotfill \vspace*{0.3cm}
			    \item Xénero: \dotfill \vspace*{0.3cm}
			\end{enumerate}
%		\item 
%		    Resume as principais características que definen a obra:
%			\vspace*{8.0cm}			

	\end{enumerate}
\end{multicols}
	\begin{itemize}
%	    \item 
%Identifica a escola e xeración, á que podería pertencer o autor desta obra:
%    \vspace*{0.25cm}
    
%    \begin{tabular}{l l l l}
%        a) Franco-flamenca; $1^a$ xeración \\ 
%        b) Franco-flamenca; $2^a$ xeración \\
%        c) Franco-flamenca; $4^a$ xeración \\
%        d) Franco-flamenca; $3^a$ xeración \\
%    \end{tabular}
    
    \item
    Partindo dos datos dispoñibles na ficha de audición, realiza un breve comentario da obra.
    \vspace*{14cm}
	\end{itemize}
\end{ejercicio}
%

%\newpage
%
% EXERCICIO 6.- 
% -------------
%%
% ----
\hideanswers % oculta respostas
%
% Cargamos os Exercicios:
% ----------------------

\loadallproblems[Tema3-GREGO]{../../Cuestions/Tema3-Canto-gregoriano.tex}

\loadallproblems[Tema3-GREGO1]{../../Cuestions/Tema3-Canto-gregoriano-expansion.tex} %
%\loadallproblems[Tema3-NOTA]{../../Cuestions/Tema3-Notacion-modal.tex} %
%\loadallproblems[Tema1-PENS]{../../Cuestions/Tema1-Pensamento.tex} %
%\loadallproblems[Tema1-RO]{../../Cuestions/Tema1-Roma.tex} %
%\loadrandomproblems[loops]{2}{loops}% aleatorios
%
% FOLLA EXERCICIOS: 
% -----------------
\begin{multicols}{2} % a 2 columnas
    \begin{enumerate}
%\useproblem{input}
    \foreachproblem[Tema3-GREGO]{\item\label{prob:\thisproblemlabel}\thisproblem}
    \foreachproblem[Tema3-GREGO1]{\item\label{prob:\thisproblemlabel}\thisproblem}
 %   \foreachproblem[Tema3-NOTA]{\item\label{prob:\thisproblemlabel}\thisproblem}
%    \foreachproblem[Tema1-PENS]{\item\label{prob:\thisproblemlabel}\thisproblem}
    \end{enumerate}
\end{multicols}
%
% SOLUCIONS:
% ----------
%\newpage
%\begin{multicols}{2}
%\showanswers
%\begin{itemize}
%\foreachdataset{\thisdataset}{%
%\foreachproblem[\thisdataset]{\item[\ref{prob:\thisproblemlabel}]\thisproblem}
%}
%\end{itemize}
%\end{multicols}
%\newpage
%
% EXERCICIO 7.- 
% -------------
%\input{Exercicio-A-Chantar-comentario.tex}
%\newpage
%
% EXERCICIO 8.- 
% -------------
%\input{Exercicio-A-Chantar-comentario.tex}
%\newpage
%
% EXERCICIO 9.- 
% -------------
%\input{Exercicio-A-Chantar-comentario.tex}
%\newpage
%
% EXERCICIO 10.- 
% -------------
%\input{Exercicio-A-Chantar-comentario.tex}
%\newpage
\end{document}
%Fin de Hoja de ejercicios
%