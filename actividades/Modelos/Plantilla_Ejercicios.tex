%
%%%%%%%%%%%%%%%%%%%%%%%%%%%%%%%%%%%%%%%%%%%%%%%%%%
% PLANTILLA EJERCICIOS DE HISTORIA DE LA MÚSICA I
% Este es un modelo para redactar los ejercicios
% 
% Pasos para cubrir la plantilla:
% 1) Realizar una copia de este modelo
% 2) Renombrar el archivo:
%		"HM1_Hoja(número).tex"
% 3) El número de Hoja debe ser correlativo
% 
%%%%%%%%%%%%%%%%%%%%%%%%%%%%%%%%%%%%%%%%%%%%%%%%%%
%
% Archivo externo de configuración
% --------------------------------
\input{../include/config-HM1ejercicios.tex}
% --------------------------------
\usepackage{graphicx}
\begin{document}
%
% TÍTULO y NÚMERO DE HOJA:
%
%
\begin{center}
\Large{
Título de la hoja de ejercicios % título de hoja
} \\
\vspace*{0.5cm}
\normalsize
(Hoja no. 1) % número de hoja
\\
\vspace{1.10cm}
	\begin{flushleft}
	Nombre y Apellidos: \\  % nombre y apellidos
%	\hrulefill \\			% line de nombre
	\vspace*{0.50cm}		% espacio vertical
	Curso/Grupo: \\ 		% curso
	\hrulefill \\			
	\vspace*{0.75cm}
%		\begin{center}
%		\textbf{Instrucciones para realizar los ejercicios} \\
%		\end{center}
%
% Indica aquí comentario e instrucciones de ayuda:		
	Lee con atención y realiza los siguientes ejercicios teniendo en cuenta lo que se indica en cada uno. \\
	\vspace*{0.75cm}		
 	\end{flushleft}
\end{center}
%
%  
\begin{ejercicio}[] 
%% 
Redacta aquí el texto de los ejercicios. Se puede incluír varios tipos de ejercicios, como de selección múltiple. 
%%
Por ejemplo:
%%
\end{ejercicio}
%
\begin{figure}[htp]
\centering
\includegraphics[scale=1.50]{/media/roberto/TOSHIBA/Temas_Historia/Historia1/capitulos/figuras/Fab. 15.jpg}
\caption{Ejemplo de imagen}
\label{Imagen 1}
\end{figure}
%%
%
%
\end{document}