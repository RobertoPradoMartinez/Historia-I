%%%%%%%%%%%%%%%%%%%%%%%%%%%%%%%%%
% MODELO PARA ELABORAR EXÁMENES
% 
% Incluye varios modelos de preguntas:
% - cuestiones tipo test
% - cuestiones de desarrollo
% - cuestiones de opinión
% - cuestiones con respuesta oculta
%
% %%%%%%%%%%%%%%%%%%%%%%%%%%%%%%%
% Nota: algunas cuestiones de examen pueden extraerse de hojas de ejercicios, con lo que es recomendable emplear un formato común en ambas para evitar así errores de compilación.
%
%%%%%%%%%%%%%%%%%%%%%%%%%%%%%%%%%
%
\documentclass[12pt,a4papper,addpoints]{exam}
%
%
%%%%%%%%%%%%%%%%%%%%%%%%%%%%%%%%%%%%%%%%%%%
%% ---------- MODELO EXÁMENES ---------- 
%% MATERIA: HISTORIA
%% CURSO: 
%% AÑO ACADÉMICO: 
%% CENTRO: 
%%%%%%%%%%%%%%%%%%%%%%%%%%%%%%%%%%%%%%%%%%%
%% 
%% MODELO PARA REDACTAR EXÁMENES
%% =============================
%% 
%% Clase de documento
%% ------------------
%\documentclass[12pt,a4papper,notitlepage,addpoints]{exam}
% ------
% Tipografías
%\usepackage{heuristica} % para examenes y ejercicios
\usepackage{libertineRoman} % lectura cómododa
%\usepackage[proportional]{libertine}
\usepackage[utf8]{inputenc}
\usepackage[T1]{fontenc}
%\usepackage[spanish]{babel}
%
% Gráficos
\usepackage{graphicx}
%
\usepackage[spanish,activeacute, es-tabla]{babel}
\usepackage{enumerate} % entornos de listas
\usepackage{multicol}  % varias columnas texto
%\usepackage{fancyhdr}  % encabezado personalizado
\usepackage{fancybox}  % entornos con cajas
\usepackage{pdfpages}  % páginas pdf
% 
\usepackage{lipsum} % generar texto aleatorio "loren ipsum"
\usepackage{environ} 
\usepackage{probsoln} % paquete para soluciones
%\showanswers % para mostrar soluciones
%
% ---
% Lineas de encabezado y pié
% --------------------------
%\renewcommand{\headrulewidth}{0.5pt}
%\renewcommand{\headrulewidth}{1.0pt}
%\renewcommand{\footrulewidth}{0.5pt}
%\renewcommand{\footrulewidth}{1.0pt}
%\pagestyle{fancy} % estilo de página
% ---
% Encabezado y pié de página (textos)
% -----------------------------------
% texto de encabezado izquierda:
%
\lhead{
	\hrule
	\vspace*{0.20cm}
	\normalfont{Historia de la Música I}
	\vspace*{0.20cm}
	\hrule
}
% texto encabezado centro:
\chead{
	%\hrule
	%\vspace*{0.20cm}
	\textbf{Cuestionario de Examen}
	\vspace*{0.42cm}}
% texto de encabezado derecha:
\rhead{
	%\vspace*{0.20cm}
	\normalfont{curso: 2020/2021}
	\vspace*{0.42cm}}

%
% texto pié izquierdo:
%\lfoot{	} % vacío
% texto pié centrado:
\cfoot{
	%\vspace*{0.30cm}
	%\hrule
	\vspace*{1.0cm}
	\small{- Página \thepage -  }\\
	%\small{Conservatorio Profesional de Música de Viveiro}\\
	%\small{avda. da Mariña s/n}
	}
%\vspace*{0.20cm}
% texto pié derecho
%\rfoot{\textit{Pr. $\mathcal{A}$.Kaal}}
%
%
\pointpoints{punto}{puntos}
\marginpointname{ \points} %Texto tras el número
\pointsinmargin %Puntos en el margen izquierdo
\bracketedpoints %Puntos entre corchetes
% ---
%Esto es lo importante. Ponemos la solución al margen.
\NewEnviron{solutionnew}{%
  \leavevmode\marginpar{\raggedright\footnotesize Solución:\\ \BODY}}{}
\renewenvironment{solution}{\solutionnew}{\endsolutionnew}
%
% Fin archivo configuración
% ------
%
% Moi interesante para crear unha folla de respostas:
% ---------------------------------------------------
%Definición del problema de ejemplo. Esto se hace normalmente desde un archivo aparte.
\begin{defproblem}{problem1}
\begin{onlysolution}
\begin{solution}
b)
\end{solution}
\end{onlysolution}
Aquí vai o texto da pregunta do exame ... 
%
%\lipsum[1]
\end{defproblem}
%
% ------------------
%
\begin{document}
%
% ---------
% CABECERA 
% ---------
%
\begin{center}
\Large{
Examen de los temas 1 y 2 \\ (esto es un ejemplo)
} \\
\vspace*{0.5cm}
\normalsize % Trimestre al que corresponde:
(1er / 2º / 3er trimestre indica el que corresponda)
\vspace*{0.5cm}
\\
%\hrule
%\vspace*{0.05cm}
%\hrule
%
\vspace{1.10cm}
	\begin{flushleft}
	Nombre y Apellidos: \hrulefill\\
	\vspace*{0.50cm}
		\begin{center}
		\small{\texttt{Instrucciones para realizar los ejercicios}}\\		
		\end{center}
%	\hrulefill \\
	\vspace*{0.5cm}
%
\small{ % Instrucciones:
\texttt{
Lee con atención y realiza con detenimiento, los siguientes ejercicios teniendo en cuenta lo que se indica en cada uno. \\
}} % fin instrucciones.
%
	\vspace*{0.75cm}		
 	\end{flushleft}
\end{center}
% ----
% Cuestiones basadas en diferentes modelos de preguntas
\begin{questions}
  \question [1] Esta cuestión puede servir de ejemplo para pregunta tipo test de cuatro respuestas, donde el alumno debe (por ejemplo) rodear una o más, según se indique en el enunciado.
  \begin{parts}
      \part Opción a
      \part Opción b
      \part Opción c
      \part Opción d
  \end{parts}
  
  \question [1] Este ejemplo muestra una cuestión con varias subcuestiones agrupadas:
  \begin{parts}
      \part ¿Cuándo ....? 
      \begin{subparts}
          \subpart En el s. XXI (primera opción)
          \subpart segunda opción
      \end{subparts}
  \end{parts}
% Otro modelo:
 \question Referente a las <<novelas cortas>> de Alarcón, responde las siguientes preguntas.
  \begin{parts}
      \part[1] ¿En qué siglo se escribió La Buenaventura?
      \begin{checkboxes}
          \choice XVII
          \choice XVIII
          \CorrectChoice XIX
      \end{checkboxes}
      \part[9] Escribe el primer párrafo de La Buenaventura
      \begin{solution}[4cm]
          No sé qué día de Agosto del año 1816 llegó a las puertas de
          la Capitanía general de Granada cierto haraposo y grotesco
          gitano, de sesenta años de edad, de oficio esquilador y de
          apellido o sobrenombre Heredia, caballero en flaquísimo y
          destartalado burro mohíno, cuyos arneses se reducían a una
          soga atada al pescuezo; y, echado que hubo pie a tierra, dijo
          con la mayor frescura <<que quería ver al Capitán general.>>
      \end{solution}
  \end{parts}
  \question[2] Pedro Antonio de \fillin[Alarcón] nació en \fillin[1833] y murió en \fillin[1891].

  \question[1] Pregunta
    \begin{solutionorbox}[4cm]
        Solución...
    \end{solutionorbox}
    \question[1] Otra pregunta

	\question[1] Pregunta
    \begin{solutionorlines}[4cm]
        Solución...
    \end{solutionorlines}
    \question[1] Otra pregunta    
    
\question[1] Pregunta
    \begin{solutionordottedlines}[4cm]
        Solución...
    \end{solutionordottedlines}
    \question[1] Otra pregunta
    

\end{questions}
%
\useproblem{problem1} % para insertar un ejercicio predefinido

\end{document}