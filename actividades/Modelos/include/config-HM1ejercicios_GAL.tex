%%%%%%%%%%%%%%%%%%%%%%%%%%%%%%%%%%%%%%%%%%%
%% ---------- MODELO EJERCICIOS ---------- 
%% MATERIA: HISTORIA
%% CURSO: 
%% AÑO ACADÉMICO: 
%% CENTRO: 
%%%%%%%%%%%%%%%%%%%%%%%%%%%%%%%%%%%%%%%%%%%
%% 
%% MODELO PARA REDACTAR EJERCICIOS
%% ===============================
%% 
%% Clase de documento
%% ------------------
%\documentclass[letterpaper,12pt,notitlepage,spanish]{article}
%\documentclass[12pt,a4paper,notitlepage]{article}
%
% Márgenes de documento
% ---------------------
\usepackage[left=2.0cm, right=2.0cm, lines=45, top=2.5cm, bottom=2.0cm]{geometry}
%
% Paquetes necesarios
% -------------------
\usepackage[utf8]{inputenc} % acentos en ES
\usepackage[spanish,activeacute, es-tabla]{babel}
\usepackage{enumerate} % entornos de listas
\usepackage{multicol}  % varias columnas texto
\usepackage{fancyhdr}  % encabezado personalizado
\usepackage{fancybox}  % entornos con cajas
\usepackage{pdfpages}  % páginas pdf
%
\usepackage{lipsum} % generar texto aleatorio "loren ipsum"
\usepackage{environ} 
\usepackage{probsoln} % paquete para soluciones
%\showanswers % para mostrar soluciones
%
%Esto es lo importante. Ponemos la solución al margen.
\NewEnviron{solutionnew}{%
%  \leavevmode\marginpar{\raggedright\footnotesize \textbf{Solución:}\\ \BODY}
%  \textbf{Solución:}\\ \BODY} % sol. con salto de liña
  \small{Solución:} \BODY} % sol. na mesma liña
  {}
\renewenvironment{solution}{\solutionnew}{\endsolutionnew}
%
% FIGURAS EN COLUMNAS:
\newenvironment{Figura}
  {\par\medskip\noindent\minipage{\linewidth}}
  {\endminipage\par\medskip}
% ---
%
% Lineas de encabezado y pié
% --------------------------
\renewcommand{\headrulewidth}{0.5pt}
%\renewcommand{\headrulewidth}{1.0pt}
\renewcommand{\footrulewidth}{0.5pt}
%\renewcommand{\footrulewidth}{1.0pt}
\pagestyle{fancy} % estilo de página
%
% Recuadros y figuras
% -------------------
\newcommand\Loadedframemethod{TikZ}
\usepackage[framemethod=\Loadedframemethod]{mdframed}
\usepackage{tikz}
\usetikzlibrary{calc,through,backgrounds}
\usetikzlibrary{matrix,positioning}
%Desssins geometriques
\usetikzlibrary{arrows}
\usetikzlibrary{shapes.geometric}
\usetikzlibrary{datavisualization}
\usetikzlibrary{automata} % LATEX and plain TEX
\usetikzlibrary{shapes.multipart}
\usetikzlibrary{decorations.pathmorphing} 
\usepackage{pgfplots}
\usepackage{physics}
\usepackage{titletoc}
\usepackage{mathpazo} 
\usepackage{algpseudocode}
\usepackage{algorithmicx} 
\usepackage{bohr} 
\usepackage{xlop} 
\usepackage{bbding} 
%\usepackage{minibox} 
% Texto árabe
\usepackage{mathdesign}
\usepackage{bbding} 
% --
% --
%   ---------   Personalización de textos    ---------
%   Opción en Galego.-
\addto\captionsspanish{
\renewcommand{\contentsname}{Índice} 
\renewcommand{\listfigurename}{Índice de ilustracións} 
\renewcommand{\listtablename}{Índice de táboas} 
\renewcommand{\bibname}{Bibliografía} 
\renewcommand{\indexname}{Indice alfabético} 
\renewcommand{\figurename}{Ilustración} 
\renewcommand{\tablename}{Táboa} 
%\renewcommand{\appendixname}{Anexo} 
%\renewcommand{\abstractname}{Resumo}
%\renewcommand{\partname}{BLOQUE} % Cambio Parte BLOQUE
%\renewcommand{\chaptername}{TEMA} % Cambio CAPÍTULO por TEMA en mayúsculas %
}
%   ---------   Fin personalización de textos   ---------

% Tipograía:
% ----------
% Fuente HEURÍSTICA (cómoda de leer)
%\usepackage{heuristica}
% Fuente LIBERTINE (cómoda para apuntes)
\usepackage{libertineRoman}
%\usepackage[proportional]{libertine}
% Fuente ROMANDE (estilo antiguo pero no muy cómoda)
%\usepackage{romande} %
% 
% Encabezado y pié de página (textos)
% -----------------------------------
% Modelo 1:
% ---------
% texto de encabezado izquierda:
%\lhead{\normalfont{Historia de la Música I}}
% texto encabezado centro:
%\chead{\textbf{Ejercicios}}
% texto de encabezado derecha:
%\rhead{\normalfont{curso: 2020/2021}}
% texto pié izquierdo:
%\lfoot{\small{\textit{}}}
% texto pié centrado:
%\cfoot{\textsc{Pág. \thepage }}
% texto pié derecho
%\rfoot{\textit{Pr. $\mathcal{A}$.Kaal}}
% ----------
% Modelo 2:
% ---------
% Encabezado y pié de página (textos)
% -----------------------------------
% texto de encabezado izquierda:
%
%\lhead{
%	\hrule
%	\vspace*{0.20cm}
%	\normalfont{Historia de la Música I}
%	\vspace*{0.10cm}
	%\hrule
%}
% texto encabezado centro:
%\chead{
%	\textbf{Cuestionario de Ejercicios}
%	\vspace*{0.08cm}}
% texto de encabezado derecha:
%\rhead{
%	\normalfont{curso: 2020/2021}
%	\vspace*{0.08cm}}
%
% texto pié izquierdo:
%\lfoot{
	%\begin{center}
		%\vspace*{0.20cm}
		%\hrule
		%\small{
		%Conservatorio Profesional de Música de Viveiro - Avda. da mariña s/n - (27850) Viveiro - Lugo
		%	}
	%\end{center}
%}
% texto pié centrado:
%\cfoot{
	%\vspace*{0.30cm}
	%\hrule
	%\vspace*{0.90cm}
%	\small{- Página \thepage -  }\\
	%\small{Conservatorio Profesional de Música de Viveiro}\\
	%\small{avda. da Mariña s/n}
%}

% ----------
% Modelo 3:
% ---------
% Encabezado y pié de página (textos)
% -----------------------------------
% texto de encabezado izquierda:
%
\lhead{
	\hrule
	\vspace*{0.20cm}
	\normalfont{Historia da Música I}
	\vspace*{0.10cm}
	%\hrule
}
% texto encabezado centro:
\chead{
	\textbf{CADERNO DE EXERCICIOS}
	\vspace*{0.08cm}}
% texto de encabezado derecha:
\rhead{
	\normalfont{curso: 2022/2023}
	\vspace*{0.08cm}}
%
% texto pié izquierdo:
%\lfoot{
%	\begin{center}
%		\vspace*{0.20cm}
%		\hrule
%		\small{
%		Conservatorio Profesional de Música de Viveiro - Avda. da mariña s/n - (27850) Viveiro - Lugo
%			}
%	\end{center}
%}
% texto pié centrado:
\cfoot{
	%\vspace*{0.30cm}
	%\hrule
	%\vspace*{0.90cm}
	\small{- \thepage -  }\\
	%\small{Conservatorio Profesional de Música de Viveiro}\\
	%\small{avda. da Mariña s/n}
}

% --------
%=====================Algo setup
\algblock{If}{EndIf}
\algcblock[If]{If}{ElsIf}{EndIf}
\algcblock{If}{Else}{EndIf}
\algrenewtext{If}{\textbf{si}}
\algrenewtext{Else}{\textbf{sinon}}
\algrenewtext{EndIf}{\textbf{finsi}}
\algrenewtext{Then}{\textbf{alors}}
\algrenewtext{While}{\textbf{tant que}}
\algrenewtext{EndWhile}{\textbf{fin tant que}}
\algrenewtext{Repeat}{\textbf{r\'ep\'eter}}
\algrenewtext{Until}{\textbf{jusqu'\`a}}
\algcblockdefx[Strange]{If}{Eeee}{Oooo}
[1]{\textbf{Eeee} "#1"}
{\textbf{Wuuuups\dots}}

\algrenewcommand\algorithmicwhile{\textbf{TantQue}}
\algrenewcommand\algorithmicdo{\textbf{Faire}}
\algrenewcommand\algorithmicend{\textbf{Fin}}
\algrenewcommand\algorithmicrequire{\textbf{Variables}}
\algrenewcommand\algorithmicensure{\textbf{Constante}}% replace ensure by constante
\algblock[block]{Begin}{End}
\newcommand\algo[1]{\textbf{algorithme} #1;}
\newcommand\vars{\textbf{variables } }
\newcommand\consts{\textbf{constantes}}
\algrenewtext{Begin}{\textbf{debut}}
\algrenewtext{End}{\textbf{fin}}
%================================
%================================

\setlength{\parskip}{1.25cm}
\setlength{\parindent}{1.25cm}
\tikzstyle{titregris} =
[draw=gray,fill=gray, shading = exersicetitle, %
text=gray, rectangle, rounded corners, right,minimum height=.3cm]
\pgfdeclarehorizontalshading{exersicebackground}{100bp}
{color(0bp)=(green!40); color(100bp)=(black!5)}
\pgfdeclarehorizontalshading{exersicetitle}{100bp}
{color(0bp)=(red!40);color(100bp)=(black!5)}
\newcounter{exercise}
%\renewcommand*\theexercise{exercice \textbf{Ejercicio}~n\arabic{exercise}} % CASTELÁN
\renewcommand*\theexercise{exercice \textbf{Exercicio}~n\arabic{exercise}} % GALEGO
\makeatletter
\def\mdf@@exercisepoints{}%new mdframed key:
\define@key{mdf}{exercisepoints}{%
\def\mdf@@exercisepoints{#1}
}
\mdfdefinestyle{exercisestyle}{%
outerlinewidth=1em,outerlinecolor=white,%
leftmargin=-1em,rightmargin=-1em,%
middlelinewidth=0.5pt,roundcorner=3pt,linecolor=black,
apptotikzsetting={\tikzset{mdfbackground/.append style ={%
shading = exersicebackground}}},
innertopmargin=0.1\baselineskip,
skipabove={\dimexpr0.1\baselineskip+0\topskip\relax},
skipbelow={-0.1em},
needspace=0.5\baselineskip,
frametitlefont=\sffamily\bfseries,
settings={\global\stepcounter{exercise}},
singleextra={%
\node[titregris,xshift=0.5cm] at (P-|O) %
{~\mdf@frametitlefont{\theexercise}~};
\ifdefempty{\mdf@@exercisepoints}%
{}%
{\node[titregris,left,xshift=-1cm] at (P)%
{~\mdf@frametitlefont{\mdf@@exercisepoints points}~};}%
},
firstextra={%
\node[titregris,xshift=1cm] at (P-|O) %
{~\mdf@frametitlefont{\theexercise}~};
\ifdefempty{\mdf@@exercisepoints}%
{}%
{\node[titregris,left,xshift=-1cm] at (P)%
{~\mdf@frametitlefont{\mdf@@exercisepoints points}~};}%
},
}
\makeatother

%%%%%%%%%%%%%%%5 Definición modificada %%%%%%%%%%%%%%%%%%%%%%%%%
%
% Modificado para traballar con banco de exercicios
% Elimino as liñas do recadro de exercicios convencional
%
%\mdfdefinestyle{theoremstyle}{%
%outerlinewidth=0.01em,linecolor=white,middlelinewidth=0.5pt,%
%frametitlerule=true,roundcorner=2pt,%
%apptotikzsetting={\tikzset{mfframetitlebackground/.append style={%
%shade,left color=white, right color=blue!20}}},
%frametitlerulecolor=white,innertopmargin=1\baselineskip,%green!60,
%innerbottommargin=0.5\baselineskip,
%frametitlerulewidth=0.1pt,
%innertopmargin=0.7\topskip,skipabove={\dimexpr0.2\baselineskip+0.1\topskip\relax},
%frametitleaboveskip=1pt,
%frametitlebelowskip=1pt
%}
% --------------------------------------

%%%%%%%%%%%%%%% Definición orixinal %%%%%%%%%%%%%%%%%%%%%%%%%
%%       Maquetación con bordes para exercicios            %%
%%
\mdfdefinestyle{theoremstyle}{%
outerlinewidth=0.01em,linecolor=black,middlelinewidth=0.5pt,%
frametitlerule=true,roundcorner=2pt,%
apptotikzsetting={\tikzset{mfframetitlebackground/.append style={%
shade,left color=white, right color=blue!20}}},
frametitlerulecolor=black,innertopmargin=1\baselineskip,%green!60,
innerbottommargin=0.5\baselineskip,
frametitlerulewidth=0.1pt,
innertopmargin=0.7\topskip,skipabove={\dimexpr0.2\baselineskip+0.1\topskip\relax},
frametitleaboveskip=1pt,
frametitlebelowskip=1pt
}
\setlength{\parskip}{0mm}
\setlength{\parindent}{10mm}
%\mdtheorem[style=theoremstyle]{ejercicio}{\textbf{Ejercicio}} % Castelán
\mdtheorem[style=theoremstyle]{ejercicio}{\textbf{Exercicio}} % Galego
%================Liste definition--numList-and alphList=============
\newcounter{alphListCounter}
\newenvironment
{alphList}
{\begin{list}
{\alph{alphListCounter})}
{\usecounter{alphListCounter}
\setlength{\rightmargin}{0cm}
\setlength{\leftmargin}{0.5cm}
\setlength{\itemsep}{0.2cm}
\setlength{\partopsep}{0cm}
\setlength{\parsep}{0cm}}
}
{\end{list}}
\newcounter{numListCounter}
\newenvironment
{numList}
{\begin{list}
{\arabic{numListCounter})}
{\usecounter{numListCounter}
\setlength{\rightmargin}{0cm}
\setlength{\leftmargin}{0.5cm}
\setlength{\itemsep}{0cm}
\setlength{\partopsep}{0cm}
\setlength{\parsep}{0cm}}
}
{\end{list}}
%
%
% --------------- CRONOGRAMAS en LATEX: --------------- 
\usepackage{chronology}

\renewenvironment{chronology}[5][6]{%
    \setcounter{step}{#1}%
    \setcounter{yearstart}{#2}\setcounter{yearstop}{#3}
    \setcounter{deltayears}{\theyearstop-\theyearstart}
    \setlength{\unit}{#4}%
    \setlength{\timelinewidth}{#5}%
    \pgfmathsetcounter{stepstart}%
    {\theyearstart+\thestep-mod(\theyearstart,\thestep)}%
    \pgfmathsetcounter{stepstop}{\theyearstop-mod(\theyearstop,\thestep)}%
    \addtocounter{step}{\thestepstart}%
    \begin{lrbox}{\timelinebox}%
    \begin{tikzpicture}[baseline={(current bounding box.north)}]%
    \draw [|->] (0,0) -- (\thedeltayears*\unit+\unit, 0);%
    \foreach \x in {1,...,\thedeltayears}%
    \draw[xshift=\x*\unit] (0,-.1\unit) -- (0,.1\unit);
   \addtocounter{deltayears}{1}%
    \foreach \x in {\thestepstart,\thestep,...,\thestepstop}{%
        \pgfmathsetlength\xstop{(\x-\theyearstart)*\unit}%
        \draw[xshift=\xstop] (0,-.3\unit) -- (0,.3\unit);%
        \node at (\xstop,0) [below=.2\unit] {\x};}%
    }
{%
\end{tikzpicture}%
\end{lrbox}%
    \raisebox{2ex}{\resizebox{\timelinewidth}{!}{\usebox{\timelinebox}}}}%
%
% Fin do código
%
%% -- Fin del archivo de configuración  --
%%
