%%%%%%%%%%%%%%%%%%%%%%%%%%%%%%%%%%%%%%%%%%%
%% ---------- MODELO EXÁMENES ---------- 
%% MATERIA: HISTORIA
%% CURSO: 
%% AÑO ACADÉMICO: 
%% CENTRO: 
%%%%%%%%%%%%%%%%%%%%%%%%%%%%%%%%%%%%%%%%%%%
%% 
%% MODELO PARA REDACTAR EXÁMENES
%% =============================
%% 
%% Clase de documento
%% ------------------
%\documentclass[12pt,a4papper,notitlepage,addpoints]{exam}
% ------
% Tipografías
%\usepackage{heuristica} % para examenes y ejercicios
\usepackage{libertineRoman} % lectura cómododa
%\usepackage[proportional]{libertine}
\usepackage[utf8]{inputenc}
\usepackage[T1]{fontenc}
%\usepackage[spanish]{babel}
%
% Gráficos
\usepackage{graphicx}
%
\usepackage[spanish,activeacute, es-tabla]{babel}
\usepackage{enumerate} % entornos de listas
\usepackage{multicol}  % varias columnas texto
%\usepackage{fancyhdr}  % encabezado personalizado
\usepackage{fancybox}  % entornos con cajas
\usepackage{pdfpages}  % páginas pdf
% 
\usepackage{lipsum} % generar texto aleatorio "loren ipsum"
\usepackage{environ} 
\usepackage{probsoln} % paquete para soluciones
%\showanswers % para mostrar soluciones
%
% ---
% Lineas de encabezado y pié
% --------------------------
%\renewcommand{\headrulewidth}{0.5pt}
%\renewcommand{\headrulewidth}{1.0pt}
%\renewcommand{\footrulewidth}{0.5pt}
%\renewcommand{\footrulewidth}{1.0pt}
%\pagestyle{fancy} % estilo de página
% ---
% Encabezado y pié de página (textos)
% -----------------------------------
% texto de encabezado izquierda:
%
\lhead{
	\hrule
	\vspace*{0.20cm}
	\normalfont{Historia de la Música I}
	\vspace*{0.20cm}
	\hrule
}
% texto encabezado centro:
\chead{
	%\hrule
	%\vspace*{0.20cm}
	\textbf{Cuestionario de Examen}
	\vspace*{0.42cm}}
% texto de encabezado derecha:
\rhead{
	%\vspace*{0.20cm}
	\normalfont{curso: 2020/2021}
	\vspace*{0.42cm}}

%
% texto pié izquierdo:
%\lfoot{	} % vacío
% texto pié centrado:
\cfoot{
	%\vspace*{0.30cm}
	%\hrule
	\vspace*{1.0cm}
	\small{- Página \thepage -  }\\
	%\small{Conservatorio Profesional de Música de Viveiro}\\
	%\small{avda. da Mariña s/n}
	}
%\vspace*{0.20cm}
% texto pié derecho
%\rfoot{\textit{Pr. $\mathcal{A}$.Kaal}}
%
%
\pointpoints{punto}{puntos}
\marginpointname{ \points} %Texto tras el número
\pointsinmargin %Puntos en el margen izquierdo
\bracketedpoints %Puntos entre corchetes
% ---
%Esto es lo importante. Ponemos la solución al margen.
\NewEnviron{solutionnew}{%
  \leavevmode\marginpar{\raggedright\footnotesize Solución:\\ \BODY}}{}
\renewenvironment{solution}{\solutionnew}{\endsolutionnew}
%
% Fin archivo configuración