%
% TEXTOS PARA COMENTARIOS - ORIXES DA MÚSICA: PATRIMONIO PRIMIXENIO
% -----------------------------------------------------------------
% Fonte: Ulrich Michels, Atlas da música vol. I pp. 158 e 159, 79
% TODO: Revisar a maquetación para incluír nos exercicios do tema.

Sobre os comezos da música e o patrimonio primigenio en materia de instrumentos musicais
Fonte: Ulrich Michels en Atlas de música Vol. 1 p. 158 e 159, 79


 O comezo da música descoñécense. Segundo os mitos dos pobos, a música é de orixe divina. En efecto, na época primitiva, a música pertence ao ámbito do culto, o seu son é o esconxuro do invisible, por parte do mundo circundante e do home. Na procura dos comezos da música é mester incorporar outros fenómenos, chegado o caso, ao ámbito do sonoro, que é o que podemos documentar co termo “música”. A idea occidental de música remóntase á Antigüidade grega, así como ás culturas do Asia Menor e do Afastado Oriente.
Os testemuños máis primitivos da música son:
-   achados de instrumentos: do paleolítico, entre eles o que deu en chamarse o patrimonio primigenio (ver abaixo).
-      rexistros escritos: do terceiro milenio precristiano (escritura ideográfica exipcia, ef. P.170); trátase de excepcións, que de ningún xeito reflicten a práctica da súa época;
-     música sonora: desde o invento do fonógrafo por EDISON, 1877. Con todo, aínda neste caso as falsificacións sonoras e o cambio dos hábitos de pensamento e de audición dificultan unha correcta interpretación do que era a música da súa época;
-      escritos sobre música: na literatura, a crónica, a teoría da música, etc., desde a Antigüedad.
Ao coñecemento da música e á concepción musical de estadios anteriores da humanidade tamén pode contribuír a etnoloxía musical, mediante o estudo da música das civilizacións e pobos primitivos
Desde fins do século XVIII existen, ademais, teorías sobre a orixe da música, que a remontan á linguaxe (HERDER), a voces de animais, sobre todo o canto dos paxaros (
DARWIN), a berros inarticulados (STUMPF), a interxeccións emocionais (SPENCER), etc.

 O patrimonio primigenio en materia  de instrumentos musicais.
Algúns instrumentos parecen existir en todos os tempos, inclusive prehistóricos, porque a súa “invención” é sumamente obvia. Entre este patrimonio primigenio atópanse:
-         percutores: golpeteo rítmico de pés e palmadas con ambas as mans ou sobre as coxas; tamén con paus, varas, etc.;
-         sonajas: de pedras, madeiras, laminillas metálicas, tamén como cadeas;
-         raspas e madeiras vibradoras: de toda sorte de formas e materiais;
-         tambores: troncos de árbores ocas, posiblemente segundo o modelo do golpe da machada;
-         frautas: en forma de canas, ao mesmo tempo, trompeta primitiva;
-         trompas: o corno de animais, p. ex. bovinos, como instrumento de sinais e musical;
-         arcos musicais: como os arcos para lanzar proxectís , áchanse ao comezo de todos os instrumentos de corda.

Achado de instrumentos

Era paleolítica: Os máis antigos son as frautas de falanxes, de ósos de pata de reno, de fins da era paleolítica. Producen un só son, sendo, por certo, máis un instrumento de sinal que de música. Do último período glacial, ou seica antes, proveñen as primeiras frautas de orificio fendido, tamén de ósos de reno, mentres que se demostrou, no Auriñasiense {sebas agrega, segundo Google: [época prehistórica] Que pertence ao Paleolítico superior e precede ao solutrense; caracterízase por unha cultura na que destaca a fabricación de puntas de azagaya de óso.}, a existencia de frautas tubulares, con 3 e logo tamén con 5 orificios de digitación, como instrumentos puramente melódicos (¿pentatónicos?). Pinturas rupestres na que aparecen arcos para o lanzamento de proxectís, permiten deducir o seu emprego como arcos musicais (Magdaleniense). {agrega Sebas tomado de Wikipedia: A cultura Magdaleniense é unha das últimas culturas do Paleolítico Superior en Europa occidental, que foi caracterizada polos trazos da súa industria lítica e ósea. O seu nome foi tomado da Madeleine, cova francesa da Dordoña. Sucede á cultura Solutrense.}
Era neolítica: Só no terceiro milenio antes de Cristo atopámonos en Europa cos primeiros tambores ou tambores de man feitos de arxila (Berbung). O seu corpo está ornamentado (fins de culto) e mostra ojales para anudar neles as membranas (fig). Da mesma época proceden as sonajas de arxila, a miúdo en forma de pequenos animais ou seres humanos.
Idade de bronce: En Europa áchanse ornamentacións metálicas de cornos de animais, estes últimos xa consumidos, pero tamén cornos totalmente confeccionados en metal, segundo o modelo do corno do animal. Unha forma particular destes cornos son os lures nórdicos, especialmente de Dinamarca e do sur de Suecia (fig.). Son de arco amplo, teñen unha embocadura fixa, á maneira dun trombón, un tubo delgado, de varias seccións, cuxa lonxitude oscila entre 1.50 m e 2.40 m, campás planas e ornamentadas e, tal como demostrárono tentativas efectuadas, un son cheo e brando. Os lures atopáronse case sempre por pares de igual afinación. Ese carácter par corresponde ao modelo dos cornos de animais.  O mesmo serve para a intensificación do son, e seica tamén para a execución de acordes (acháronse, xuntos, tres pares de lures, dous dos cales estaban en Do e un no meu bemol). Outros instrumentos da Idade de Bronce son trompetas, láminas sonoras, sonajas de latón, sonajas de arxila, etc.

Situación cronolóxica das civilizacións avanzadas
O desenvolvemento da humanidade leva a cabo en dimensións tan tremendas, que a época do home postglacial (homo sapiens alluvialis) a partir do ano 10000 a.C., aproximadamente, aparece como mínima (cf. A figura que garda fielmente as proporcións). A todo isto, a época das civilizacións avanzadas da antigüidade só comeza despois das catástrofes naturais que se supón ocorran ao redor de 3000 a.C., con inundacións que se lembran como “Diluvio Universal” (Biblia, Epopea de Gilgamesh). Ao principio tamén nas civilizacións avanzadas da Antigüedad a música segue aínda segue vinculada ao culto, e só tardíamente convértese nunha arte de expresión estética. Nalgúns casos, ata o día de hoxe sobreviviron aspectos das tradicións orais (India, China). A improvisación desempañaba un papel significativo.

Resulta notable que, aínda que os conceptos musicais transformáronse constantemente ata o día de hoxe, o instrumentario permaneceu relativamente igual, aínda que algúns tipos de instrumentos haxan experimentado un desenvolvemento diferente. Só os instrumentos electrónicos do século XX achegan novidades fundamentais, aínda que en parte xa non correspondan ás posibilidades de execución e audición humanas. 