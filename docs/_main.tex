%%%%%%%%%%%%%%%%%%%%%%%%%%%%%%%%%%%%%%%%%%%%%%%%%%%%%%%%%%%%%%%
%% OXFORD THESIS TEMPLATE

% Use this template to produce a standard thesis that meets the Oxford University requirements for DPhil submission
%
% Originally by Keith A. Gillow (gillow@maths.ox.ac.uk), 1997
% Modified by Sam Evans (sam@samuelevansresearch.org), 2007
% Modified by John McManigle (john@oxfordechoes.com), 2015
% Modified by Ulrik Lyngs (ulrik.lyngs@cs.ox.ac.uk), 2018-, for use with R Markdown
%
% Ulrik Lyngs, 25 Nov 2018: Following John McManigle, broad permissions are granted to use, modify, and distribute this software
% as specified in the MIT License included in this distribution's LICENSE file.
%
% John commented this file extensively, so read through to see how to use the various options.  Remember that in LaTeX,
% any line starting with a % is NOT executed.  Several places below, you have a choice of which line to use
% out of multiple options (eg draft vs final, for PDF vs for binding, etc.)  When you pick one, add a % to the beginning of
% the lines you don't want.


%%%%% PAGE LAYOUT
% The most common choices should be below.  You can also do other things, like replacing "a4paper" with "letterpaper", etc.

% This one formats for two-sided binding (ie left and right pages have mirror margins; blank pages inserted where needed):
%\documentclass[a4paper,twoside]{templates/ociamthesis}
% This one formats for one-sided binding (ie left margin > right margin; no extra blank pages):
%\documentclass[a4paper]{ociamthesis}
% This one formats for PDF output (ie equal margins, no extra blank pages):
%\documentclass[a4paper,nobind]{templates/ociamthesis}

% As you can see from the uncommented line below, oxforddown template uses the a4paper size, 
% and passes in the binding option from the YAML header in index.Rmd:
\documentclass[a4paper, twoside]{templates/ociamthesis}


%%%%% ADDING LATEX PACKAGES
% add hyperref package with options from YAML %

%%%%%%%%%%%%%
% TIPOGRAFÍA 
%%%%%%%%%%%%%
\usepackage{libertinus} % Fuente LIBERTINE Linux
%%
\usepackage[pdfpagelabels]{hyperref}
% change the default coloring of links to something sensible
\usepackage{xcolor}
\definecolor{myurlcolor}{RGB}{0,0,139}
\definecolor{mycitecolor}{RGB}{0,33,71}

\hypersetup{
  hidelinks,
  colorlinks,
  linkcolor=.,
  urlcolor=myurlcolor,
  citecolor=mycitecolor
}



% add float package to allow manual control of figure positioning %
\usepackage{float}

% enable strikethrough
\usepackage[normalem]{ulem}

% use soul package for correction highlighting
\usepackage{color, soul}
\definecolor{correctioncolor}{HTML}{CCCCFF}
\sethlcolor{correctioncolor}
\newcommand{\ctext}[3][RGB]{%
  \begingroup
  \definecolor{hlcolor}{#1}{#2}\sethlcolor{hlcolor}%
  \hl{#3}%
  \endgroup
}
\soulregister\ref7
\soulregister\cite7
\soulregister\autocite7
\soulregister\textcite7
\soulregister\pageref7

%%%%% FIXING / ADDING THINGS THAT'S SPECIAL TO R MARKDOWN'S USE OF LATEX TEMPLATES
% pandoc puts lists in 'tightlist' command when no space between bullet points in Rmd file,
% so we add this command to the template
\providecommand{\tightlist}{%
  \setlength{\itemsep}{0pt}\setlength{\parskip}{0pt}}
 
% UL 1 Dec 2018, fix to include code in shaded environments
\usepackage{color}
\usepackage{fancyvrb}
\newcommand{\VerbBar}{|}
\newcommand{\VERB}{\Verb[commandchars=\\\{\}]}
\DefineVerbatimEnvironment{Highlighting}{Verbatim}{commandchars=\\\{\}}
% Add ',fontsize=\small' for more characters per line
\usepackage{framed}
\definecolor{shadecolor}{RGB}{248,248,248}
\newenvironment{Shaded}{\begin{snugshade}}{\end{snugshade}}
\newcommand{\AlertTok}[1]{\textcolor[rgb]{0.94,0.16,0.16}{#1}}
\newcommand{\AnnotationTok}[1]{\textcolor[rgb]{0.56,0.35,0.01}{\textbf{\textit{#1}}}}
\newcommand{\AttributeTok}[1]{\textcolor[rgb]{0.77,0.63,0.00}{#1}}
\newcommand{\BaseNTok}[1]{\textcolor[rgb]{0.00,0.00,0.81}{#1}}
\newcommand{\BuiltInTok}[1]{#1}
\newcommand{\CharTok}[1]{\textcolor[rgb]{0.31,0.60,0.02}{#1}}
\newcommand{\CommentTok}[1]{\textcolor[rgb]{0.56,0.35,0.01}{\textit{#1}}}
\newcommand{\CommentVarTok}[1]{\textcolor[rgb]{0.56,0.35,0.01}{\textbf{\textit{#1}}}}
\newcommand{\ConstantTok}[1]{\textcolor[rgb]{0.00,0.00,0.00}{#1}}
\newcommand{\ControlFlowTok}[1]{\textcolor[rgb]{0.13,0.29,0.53}{\textbf{#1}}}
\newcommand{\DataTypeTok}[1]{\textcolor[rgb]{0.13,0.29,0.53}{#1}}
\newcommand{\DecValTok}[1]{\textcolor[rgb]{0.00,0.00,0.81}{#1}}
\newcommand{\DocumentationTok}[1]{\textcolor[rgb]{0.56,0.35,0.01}{\textbf{\textit{#1}}}}
\newcommand{\ErrorTok}[1]{\textcolor[rgb]{0.64,0.00,0.00}{\textbf{#1}}}
\newcommand{\ExtensionTok}[1]{#1}
\newcommand{\FloatTok}[1]{\textcolor[rgb]{0.00,0.00,0.81}{#1}}
\newcommand{\FunctionTok}[1]{\textcolor[rgb]{0.00,0.00,0.00}{#1}}
\newcommand{\ImportTok}[1]{#1}
\newcommand{\InformationTok}[1]{\textcolor[rgb]{0.56,0.35,0.01}{\textbf{\textit{#1}}}}
\newcommand{\KeywordTok}[1]{\textcolor[rgb]{0.13,0.29,0.53}{\textbf{#1}}}
\newcommand{\NormalTok}[1]{#1}
\newcommand{\OperatorTok}[1]{\textcolor[rgb]{0.81,0.36,0.00}{\textbf{#1}}}
\newcommand{\OtherTok}[1]{\textcolor[rgb]{0.56,0.35,0.01}{#1}}
\newcommand{\PreprocessorTok}[1]{\textcolor[rgb]{0.56,0.35,0.01}{\textit{#1}}}
\newcommand{\RegionMarkerTok}[1]{#1}
\newcommand{\SpecialCharTok}[1]{\textcolor[rgb]{0.00,0.00,0.00}{#1}}
\newcommand{\SpecialStringTok}[1]{\textcolor[rgb]{0.31,0.60,0.02}{#1}}
\newcommand{\StringTok}[1]{\textcolor[rgb]{0.31,0.60,0.02}{#1}}
\newcommand{\VariableTok}[1]{\textcolor[rgb]{0.00,0.00,0.00}{#1}}
\newcommand{\VerbatimStringTok}[1]{\textcolor[rgb]{0.31,0.60,0.02}{#1}}
\newcommand{\WarningTok}[1]{\textcolor[rgb]{0.56,0.35,0.01}{\textbf{\textit{#1}}}}

%UL set white space before and after code blocks
\renewenvironment{Shaded}
{
  \vspace{10pt}%
  \begin{snugshade}%
}{%
  \end{snugshade}%
  \vspace{8pt}%
}

% User-included things with header_includes or in_header will appear here
% kableExtra packages will appear here if you use library(kableExtra)
\usepackage{booktabs}
\usepackage{longtable}
\usepackage{array}
\usepackage{multirow}
\usepackage{wrapfig}
\usepackage{float}
\usepackage{colortbl}
\usepackage{pdflscape}
\usepackage{tabu}
\usepackage{threeparttable}
\usepackage{threeparttablex}
\usepackage[normalem]{ulem}
\usepackage{makecell}
\usepackage{xcolor}


%UL set section header spacing
\usepackage{titlesec}
% 
\titlespacing\subsubsection{0pt}{24pt plus 4pt minus 2pt}{0pt plus 2pt minus 2pt}


%UL set whitespace around verbatim environments
\usepackage{etoolbox}
\makeatletter
\preto{\@verbatim}{\topsep=0pt \partopsep=0pt }
\makeatother



%%%%%%% PAGE HEADERS AND FOOTERS %%%%%%%%%
\usepackage{fancyhdr}
\setlength{\headheight}{15pt}
\fancyhf{} % clear the header and footers
\pagestyle{fancy}
\renewcommand{\chaptermark}[1]{\markboth{\thechapter. #1}{\thechapter. #1}}
\renewcommand{\sectionmark}[1]{\markright{\thesection. #1}} 
\renewcommand{\headrulewidth}{0pt}

\fancyhead[LO]{\emph{\leftmark}} 
\fancyhead[RE]{\emph{\rightmark}} 

% UL page number position 
\fancyfoot[RO, LE]{\emph{\thepage}} %regular pages
\fancypagestyle{plain}{\fancyhf{}\fancyfoot[C]{\emph{\thepage}}} %chapter pages

% JEM fix header on cleared pages for openright
\def\cleardoublepage{\clearpage\if@twoside \ifodd\c@page\else
   \hbox{}
   \fancyfoot[RO, LE]{}
   \newpage
   \if@twocolumn\hbox{}\newpage
   \fi
   \fancyhead[LO]{\emph{\leftmark}} 
   \fancyhead[RE]{\emph{\rightmark}} 
   \fi\fi}


%%%%% SELECT YOUR DRAFT OPTIONS
% This adds a "DRAFT" footer to every normal page.  (The first page of each chapter is not a "normal" page.)

% IP feb 2021: option to include line numbers in PDF

% This highlights (in blue) corrections marked with (for words) \mccorrect{blah} or (for whole
% paragraphs) \begin{mccorrection} . . . \end{mccorrection}.  This can be useful for sending a PDF of
% your corrected thesis to your examiners for review.  Turn it off, and the blue disappears.
\correctionstrue


%%%%% BIBLIOGRAPHY SETUP
% Note that your bibliography will require some tweaking depending on your department, preferred format, etc.
% If you've not used LaTeX before, I recommend reading a little about biblatex/biber and getting started with it.
% If you're already a LaTeX pro and are used to natbib or something, modify as necessary.
% Either way, you'll have to choose and configure an appropriate bibliography format...


\usepackage[style=authoryear, sorting=nyt, backend=biber, maxcitenames=2, useprefix, doi=true, isbn=false, uniquename=false]{biblatex}
\newcommand*{\bibtitle}{Bibliografía}

\addbibresource{bibliography/references.bib}
\addbibresource{bibliography/additional-references.bib}


% This makes the bibliography left-aligned (not 'justified') and slightly smaller font.
\renewcommand*{\bibfont}{\raggedright\small}


% Uncomment this if you want equation numbers per section (2.3.12), instead of per chapter (2.18):
%\numberwithin{equation}{subsection}


%%%%% THESIS / TITLE PAGE INFORMATION
% Everybody needs to complete the following:
\title{Historia I\\
Apuntes e materiais didácticos}
\author{Roberto Prado Martínez}
\college{\url{https://aulademusica.netlify.app}}

% Master's candidates who require the alternate title page (with candidate number and word count)
% must also un-comment and complete the following three lines:

% Uncomment the following line if your degree also includes exams (eg most masters):
%\renewcommand{\submittedtext}{Submitted in partial completion of the}
% Your full degree name.  (But remember that DPhils aren't "in" anything.  They're just DPhils.)
\degree{Ensinanzas Profesionais de Música}
% Term and year of submission, or date if your board requires (eg most masters)
\degreedate{2021 - 2022}


%%%%% YOUR OWN PERSONAL MACROS
% This is a good place to dump your own LaTeX macros as they come up.

% To make text superscripts shortcuts
	\renewcommand{\th}{\textsuperscript{th}} % ex: I won 4\th place
	\newcommand{\nd}{\textsuperscript{nd}}
	\renewcommand{\st}{\textsuperscript{st}}
	\newcommand{\rd}{\textsuperscript{rd}}

%%%%%%%%%%%%%%%%%%%%%%
% COMEZA O DOCUMENTO
%%%%%%%%%%%%%%%%%%%%%%

\begin{document}

%\selectlanguage{spanish}
%
%%%%%%%%%%%%%%%%%%%%%%%
% DOCUMENTO EN GALEGO
%%%%%%%%%%%%%%%%%%%%%%%
%
% Tradución ao galego:
\renewcommand{\contentsname}{Índice} 
\renewcommand{\mtctitle}{Índice} % se empregamos minitoc; comentar se non.
\renewcommand{\listfigurename}{Índice de Figuras} 
\renewcommand{\listtablename}{Índice de Táboas} 
\renewcommand{\bibname}{Bibliografía} 
\renewcommand{\indexname}{Indice alfabético} 
\renewcommand{\figurename}{Figura} 
\renewcommand{\tablename}{Táboa} 
\renewcommand{\partname}{Parte} 
\renewcommand{\chaptername}{} 
\renewcommand{\appendixname}{Apéndice} 
\renewcommand{\abstractname}{Resumo}
%    
%%%%% CHOOSE YOUR LINE SPACING HERE
% This is the official option.  Use it for your submission copy and library copy:
\setlength{\textbaselineskip}{14pt plus3pt}
% This is closer spacing (about 1.5-spaced) that you might prefer for your personal copies:
%\setlength{\textbaselineskip}{18pt plus2pt minus1pt}

% You can set the spacing here for the roman-numbered pages (acknowledgements, table of contents, etc.)
\setlength{\frontmatterbaselineskip}{17pt plus1pt minus1pt}

% UL: You can set the line and paragraph spacing here for the separate abstract page to be handed in to Examination schools
\setlength{\abstractseparatelineskip}{13pt plus1pt minus1pt}
\setlength{\abstractseparateparskip}{0pt plus 1pt}

% UL: You can set the general paragraph spacing here - I've set it to 2pt (was 0) so
% it's less claustrophobic
\setlength{\parskip}{2pt plus 1pt}

%
% Oxford University logo on title page
%
\def\crest{{\includegraphics[width=5cm]{templates/beltcrest.pdf}}}
\renewcommand{\university}{Conservatorios Profesionais de Música}
\renewcommand{\submittedtext}{Apuntes e materiais didácticos}


% Leave this line alone; it gets things started for the real document.
\setlength{\baselineskip}{\textbaselineskip}


%%%%% CHOOSE YOUR SECTION NUMBERING DEPTH HERE
% You have two choices.  First, how far down are sections numbered?  (Below that, they're named but
% don't get numbers.)  Second, what level of section appears in the table of contents?  These don't have
% to match: you can have numbered sections that don't show up in the ToC, or unnumbered sections that
% do.  Throughout, 0 = chapter; 1 = section; 2 = subsection; 3 = subsubsection, 4 = paragraph...

% The level that gets a number:
\setcounter{secnumdepth}{2}
% The level that shows up in the ToC:
\setcounter{tocdepth}{1}


%%%%% ABSTRACT SEPARATE
% This is used to create the separate, one-page abstract that you are required to hand into the Exam
% Schools.  You can comment it out to generate a PDF for printing or whatnot.

% JEM: Pages are roman numbered from here, though page numbers are invisible until ToC.  This is in
% keeping with most typesetting conventions.
\begin{romanpages}

% Title page is created here
\maketitle

%%%%% DEDICATION -- If you'd like one, un-comment the following.
\begin{dedication}
  A todas aquelas persoas que colaboraron neste traballo
\end{dedication}

%%%%% ACKNOWLEDGEMENTS -- Nothing to do here except comment out if you don't want it.
\begin{acknowledgements}
 	Este proxecto sae adiante partindo do esforzo de anos de incansable traballo pola miña parte e dende logo, non sería posible sen a axuda de toda aquela xente que durante este tempo se mantén ao meu carón, apoiando a miña labor docente no Conservatorio Profesional de Música de Viveiro (Lugo).

  Debo agradecer a John Gruber por ofrecer e compartir de xeito desinteresado o \texttt{Markdown}; a John MacFarlane por crear o \texttt{Pandoc} (\url{http://pandoc.org}) indispensable na conversión de Markdown a outros formatos; a Yihui Xie por crear \texttt{knitr} e \texttt{bookdown} sen os cales todo este traballo non sería posible de realizar.

  Un agradecemento especial a Ulrik Lyngs por crear e desenvolver o modelo \texttt{oxfordown} que sirve de base na elaboración, maquetación e deseño deste traballo, sen o cal sería impensable dada a súa magnitude, e como non a J.J Allaire, fundador e CEO de \href{http://rstudio.com}{RStudio} software empregado para a elaboración deste proxecto.

  \begin{flushright}
  Roberto Prado \\
  Fene, A Coruña \\
  2021
  \end{flushright}
\end{acknowledgements}


%%%%% ABSTRACT -- Nothing to do here except comment out if you don't want it.
\begin{abstract}
	En construcción \ldots{}
\end{abstract}

%%%%% MINI TABLES
% This lays the groundwork for per-chapter, mini tables of contents.  Comment the following line
% (and remove \minitoc from the chapter files) if you don't want this.  Un-comment either of the
% next two lines if you want a per-chapter list of figures or tables.
  \dominitoc % include a mini table of contents

% This aligns the bottom of the text of each page.  It generally makes things look better.
\flushbottom

% This is where the whole-document ToC appears:
\tableofcontents

\listoffigures
	\mtcaddchapter
  	% \mtcaddchapter is needed when adding a non-chapter (but chapter-like) entity to avoid confusing minitoc

% Uncomment to generate a list of tables:
\listoftables
  \mtcaddchapter
%%%%% LIST OF ABBREVIATIONS
% This example includes a list of abbreviations.  Look at text/abbreviations.tex to see how that file is
% formatted.  The template can handle any kind of list though, so this might be a good place for a
% glossary, etc.
% First parameter can be changed eg to "Glossary" or something.
% Second parameter is the max length of bold terms.
\begin{mclistof}{Glosario}{3.2cm}

\item[1-D, 2-D]

One- or two-dimensional, referring \textbf{in this thesis} to spatial dimensions in an image.

\item[Otter]

One of the finest of water mammals.

\item[Hedgehog]

Quite a nice prickly friend.

\end{mclistof} 


% The Roman pages, like the Roman Empire, must come to its inevitable close.
\end{romanpages}

%%%%% CHAPTERS
% Add or remove any chapters you'd like here, by file name (excluding '.tex'):
\flushbottom

% all your chapters and appendices will appear here
\hypertarget{introduciuxf3n}{%
\chapter*{Introdución}\label{introduciuxf3n}}
\addcontentsline{toc}{chapter}{Introdución}

\adjustmtc
\markboth{Introdución}{}

\begin{savequote}
O obxetivo de toda obra artística é axudar a cantos viven neste mundo a
abandonar as súas miserias e conducilos á verdadeira felicidade\ldots{}
\qauthor{--- Dante Alighieri. \emph{Carta al Gran Can de la Scala de Verona}, no preámbulo ao Paraíso.}\end{savequote}



O estudo da historia da música que interpretamos e transmitimos, así como da que escoitamos supón un aspecto de gran importancia dentro da formación de todo músico.

A finalidade da Historia da Música é escoitar música, captar as características das distintas correntes estéticas de cada época, comprender a música e relacionala coas correntes estéticas, comprender e coñecer os feitos históricos e movementos socioculturais máis destacados así como o contexto no que se orixinaron, permite valorar a importancia que a música ten na sociedade e igualmente a relación entre a música e o resto de artes.

Comezamos este curso facendo un percorrido histórico, artístico e musical polas épocas anteriores á actual, coa finalidade de coñecer e comprender mellor a música e os elementos que forman parte dunha obra de arte musical. Faremos un percorrido pola música de diferentes épocas e civilizacións, centrándonos na música occidental e a súa evolución ata os nosos días, tendo en conta a importanacia da cultura musical na Península Ibérica e Galicia.

Nos primeiros capítulos, trataremos a orixe da música e a música na prehistoria, prestando especial atención ás primeiras evidencias conservadas de música escrita que foron descifradas e comprendidas (desde a idade da memoria); veremos as teorías sobre a música da Antigüidade e finalmente trataremos en profundidade a evolución da música escrita desde a Idade Media (idade da notación) ata o Renacemento.

\hypertarget{definiciuxf3ns-e-conceptos}{%
\section*{Definicións e conceptos}\label{definiciuxf3ns-e-conceptos}}
\addcontentsline{toc}{section}{Definicións e conceptos}

\hypertarget{concepto-de-muxfasica}{%
\subsection*{Concepto de música}\label{concepto-de-muxfasica}}
\addcontentsline{toc}{subsection}{Concepto de música}

A todos resulta en certo modo familiar este concepto dado que vivimos rodeados de música; en certo modo é un medio de expresión dos sentimentos humanos; unha manifestación artística e cultural dos pobos que adquire diferentes formas, valores estéticos e funcións segundo o seu contexto.

Técnicamente podemos entender por música calquera combinación ordenada de ritmo, melodía e harmonía, agradable no oído humano.

\hypertarget{concepto-de-historia-da-muxfasica}{%
\subsection*{Concepto de Historia da Música}\label{concepto-de-historia-da-muxfasica}}
\addcontentsline{toc}{subsection}{Concepto de Historia da Música}

Comecemos o estudo da Historia da Música, tentando comprender o significado de «historia da música», e a que nos estamos a referir.\\
Vexamos algunhas definicións das moitas que atoparemos sobre o significado de «historia da música» que nos aproximan á temática que vai tratar esta materia.

\begin{quote}
\emph{La \textbf{Historia de la música} es el estudio de las diferentes tradiciones en la música y su orden en el planeta}.

\[...\] \emph{aquella disciplina que trata el estudio de la evolución de las diferentes tradiciones musicales a lo largo del tiempo}.
\end{quote}

A \href{https://dle.rae.es/}{Real Academia Española da lingua} (RAE) define \href{https://dle.rae.es/historia\#otras}{«historia»} como:

\begin{quote}
1.- \emph{Narración y exposición de los acontecimientos pasados y dignos de memoria, sean públicos o privados}.

2.- \emph{Disciplina que estudia y narra cronológicamente los acontecimientos pasados}

3.- \emph{Conjunto de los sucesos o hechos políticos, sociales, económicos, culturales, etc., de un pueblo o de una nación}.

5.- \emph{Conjunto de los acontecimientos ocurridos a alguien a lo largo de su vida o en un período de ella} \footnote{\emph{Definición de historia}, RAE consultado en \url{https://www.rae.es} , (Setembro, 2020).}
\end{quote}

A \href{https://academia.gal/dicionario}{Real Adacemia Galega da lingua} (RAG), define \href{https://academia.gal/dicionario/-/termo/busca/Historia}{«historia»}:

\begin{quote}
\begin{enumerate}
\def\labelenumi{\arabic{enumi}.}
\item
  Conxunto de feitos ocorridos no pasado, que afectan a toda a humanidade, a un grupo, unha persoa, unha institución, a unha faceta concreta dese pasado etc.
\item
  Ciencia que estuda eses feitos. \footnote{\emph{Definición de historia}, RAG consultado en \url{https://academia.gal/diccionario} , (Setembro 2020)}
\end{enumerate}
\end{quote}

Vexamos agora, a definición da RAG sobre \href{https://digalego.xunta.gal/es/termo/44501/m\%C3\%BAsica}{«música»}:
\textgreater{} Arte de combinar harmoniosamente os sons, segundo unhas regras preestablecidas.

Unha definición tradicional de «música» \footnote{Os gregos definen a música como «a arte das musas»} máis ou menos aceptada podería ser:
\textgreater{} es, {[}\ldots{]}, el \href{https://es.wikipedia.org/wiki/Arte}{arte} de organizar sensible y lógicamente una combinación coherente de \href{https://es.wikipedia.org/wiki/Sonido}{sonidos} y \href{https://es.wikipedia.org/wiki/Silencio_(sonido)}{silencios} respetando los principios fundamentales de la \href{https://es.wikipedia.org/wiki/Melodía}{melodía}, la \href{https://es.wikipedia.org/wiki/Armonía}{armonía} y el \href{https://es.wikipedia.org/wiki/Ritmo}{ritmo}, {[}\ldots{]}. \footnote{Definición de \href{https://es.wikipedia.org/wiki/M\%C3\%BAsica\#Definici\%C3\%B3n}{música} consultada na wikipedia.}

\begin{correction}
Polo tanto, tendo en conta o indicado nos parágrafos anteriores, a
Historia da música occidental, céntrase principalmente no estudo da
evolución das diferentes manifestacións musicais (tradición musical) das
culturas de occidente (neste caso as culturas e sociedades musicais
europeas) ao longo do tempo.
\end{correction}

\hypertarget{obxectivos-e-problemuxe1tica-da-materia}{%
\section*{Obxectivos e problemática da materia}\label{obxectivos-e-problemuxe1tica-da-materia}}
\addcontentsline{toc}{section}{Obxectivos e problemática da materia}

O principal obxectivo da Historia da Música é o \textbf{estudo da evolución da música ao longo da historia da humanidade}. Agora ben, que imos estudar e como o faremos?

\begin{itemize}
\tightlist
\item
  centrámonos nas obras musicais ou en como se usaron?
\item
  baseamos o estudo nas persoas que as desenvolveron(crearon) ou no ambiente social da época?
\item
  Que criterios empregamos para seleccionar a música que estudaremos?
\end{itemize}

O principal problema é atopar unha definición máis ou menos aceptada e consensuada do que se entende por «música» dado que non significa e non se refire ao mesmo en tódalas culturas. Algunhas, inclúen dentro do concepto de «música» aspectos da danza, poesía, etc. e outras culturas, pola contra, non empregan ningún termo para referírense á música en sí.

Son frecuentes as discusións entre musicólogos, historiadores e grades entendidos sobre a música para lograr unha definición universal da música, pero é complicado universalizar este concepto dada a diversidade de culturas e pobos. O profesor Francisco Callejo, expón esta problemática:

\begin{quote}
{[}\ldots{]} el primer problema que nos encontramos es acordar una definición universal de \emph{música}: el concepto de música varía de una cultura a otra; por ejemplo, la mayor parte de los musulmanes no considerarían música la llamada a la oración del almuédano, que a los oídos occidentales suena similar a muchas salmodias medievales; los toques de campanas de las iglesias cristianas, por el contrario, no son considerados como manifestaciones musicales en occidente, aunque a muchos africanos les recordarían melodías suyas. \footnote{Callejo, F.: \emph{Historia de la Música}, Conservatorio Profesional de Música Francisco Guerreo (2017).}
\end{quote}

Como vemos no exemplo anterior, o que se considera música para unha cultura, pode non selo para outra e viceversa. Pensemos nestas dúas preguntas, relacionadas co exemplo de Callejo:

\begin{itemize}
\tightlist
\item
  considerdamos como música todo aquelo que alguén considera como tal? (caso do \emph{almuédano} \footnote{{[}\ldots{]} o muecín ou almuédano (``gritador'') era o musulmán que realizaba tradicionalmente a chamada á oración (\emph{salat}) mediante a voz. Na actualidade, o almuédano foi substituído con frecuencia por un megáfono.\\
    (Fonte:
    \href{https://www.wikiwand.com/gl/Minarete?wprov=srpw1_0}{wikiland})} e as campás)
\item
  consideramos como música, pola contra, só o que todos consideramos música?
\end{itemize}

Por outra parte, a «historia da música occidental» exclúe moitas manifestacións musicais como a música popular actual, a música tradicional europea e non europea. Exclúe tamén do seu ámbito de estudo, a música clásica oriental chinesa, xaponesa ou india. Así o seu campo de estudo redúcese, exclusivamente á música culta europea, a pesares de si estudar algunha música non europea pero que segue certos cánones europeos.

Outra cuestión que influirá no concepto é a «orixe da cultura occidental»: cando comeza a cultura occidental? ou mellor dito, desde cando consideramos que comeza a cultura occidental?

\hypertarget{a-muxfasica-como-actividade-vs-a-muxfasica-como-produto}{%
\section{\texorpdfstring{A música como actividade \emph{vs} a música como produto}{A música como actividade vs a música como produto}}\label{a-muxfasica-como-actividade-vs-a-muxfasica-como-produto}}

En primeiro lugar, diferenciaremos a música como \textbf{actividade}: unha ou máis persoas participan creando, interpretando ou escoitando música, en comparación coa música como \textbf{produto}: o resultado desta actividade é algo sólido, coa posibilidade de ser escrito con sistemas de notación dando como resultado, por exemplo unha obra musical. Neste caso, obtemos un produto (obra musical) resultante dunha actividade (composición).
O enfoque máis común adoita ser o segundo, estudando exclusivamente obras musicais e non a actividade xerada ao seu redor.

\hypertarget{transmisiuxf3n-oral-e-transmisiuxf3n-escrita}{%
\section{Transmisión oral e transmisión escrita}\label{transmisiuxf3n-oral-e-transmisiuxf3n-escrita}}

A posibilidade de estudar música historicamente, baséase na existencia dunha transmisión dela ao longo do tempo.
En case todas as culturas e tempos, a música transmitiuse por medio da escoita e posterior repetición, isto é: escoitando e observando por exemplo aos profesores e profesoras. Isto é o que se chama \textbf{transmisión oral}(propio da idade da memoria)

Tamén existe a posibilidade de transmitir - e almacenar - música con varios métodos de escritura musical, dando lugar a transmisión escrita (idade de notación).

\hypertarget{muxfasica-culta-e-muxfasica-popular}{%
\section{Música culta e música popular}\label{muxfasica-culta-e-muxfasica-popular}}

A actividade musical prodúcese en todos os grupos sociais e nun gran número de situacións diferentes. Algunhas manifestacións musicais adquiriron un maior prestixio social, ben pola súa relación con altos estratos da sociedade, ben polas súas características de formación e profesionalización. Estamos a diferencar música académica, tamén coñecida como ``clásica'' ou ``culta'', fronte a unha enorme variedade de música popular, normalmente con menos prestixio. O estudo da música debería abarcar todos os estilos pero, normalmente atende só ao estudo dos estilos académicos.

\hypertarget{muxfasica-europea-e-non-europea}{%
\section{Música europea e non europea}\label{muxfasica-europea-e-non-europea}}

O obxectivo principal da historia da música foi sempre o estudo da música europea, especialmente a dos últimos séculos; en parte, porque foi no continente europeo onde se crearon os principais tratados e estudos sobre música. Este enfoque ``eurocéntrico'' deixa fóra numerosas manifestacións musicais, tanto académicas como populares de fóra de Europa, que nalgúns casos tiveron unha forte influencia no propio desenvolvemento da música europea. Hoxe en día é común centrarse só na música europea e a súa influencia en músicos doutros continentes.

Cando estudamos a historia da música, adoitamos centrarnos en produtos musicais escritos da tradición académica europea, polo que acurtamos drasticamente o obxecto de estudo. O resto - actividade musical, transmisión oral, música popular ou non europea - son obxecto de estudo da etnomusicoloxía, que normalmente non aplica o enfoque histórico.

\hypertarget{a-actividade-musical-e-os-seus-produtos}{%
\section{A actividade musical e os seus produtos}\label{a-actividade-musical-e-os-seus-produtos}}

A actividade musical pode considerarse como un proceso bastante complexo, que abarca varias fases. Así falaremos de: \textbf{produción}, \textbf{difusión} e \textbf{consumo}.

\begin{longtable}[]{@{}ll@{}}
\toprule
ACTIVIDADE & TERMINO MUSICAL\tabularnewline
\midrule
\endhead
Produción & Composición\tabularnewline
Difusión & Interpretación\tabularnewline
Consumo & Audición\tabularnewline
\bottomrule
\end{longtable}

Para estudar a actividade musical historicamente (o ``proceso musical''), imos centrarnos igualmente nas tres fases do proceso. Non obstante, en moitos casos, as diferentes historias musicais céntranse só na primeira (produción) sen referirse aos intérpretes, ás técnicas de interpretación, aos contextos de escoita (audición), etc.

Ao longo do século XIX desenvolvéronse dúas ideas ou conceptos importantes: \emph{o canon} e o \emph{repertorio}. O primeiro refírese ao conxunto de compositores e obras obxecto de estudo; o segundo é o conxunto de obras que, por unha ou outra razón, seguimos interpretando e escoitando. Ámbolos dous conceptos derivan de certos criterios de ``calidade musical'' malia que é certo que son, á súa vez, produtos culturais europeos creados en contextos políticos, sociais e ideolóxicos específicos.

O feito de que se exclúa a música non europea ou popular, fainos pensar na discriminación étnica e de clase, que mantiveron certos musicólogos, intérpretes, críticos, etc. do século XIX. A exclusión do canon da muller como compositora, é outro exemplo destes prexuízos e discriminación {[}\^{}cita:exclusión\_muller{]}, así como o silencio ao que foron sometidos aqueles compositores {[}\^{}cita:exclusión\_compo{]} que non se axustaban ao modelo ou idea de evolución da música occidental da época. Sen dúbida, outra das ideas que marcaron este concepto de canon foi a valoración dos nacionalismos, \footnote{A idea do nacional ou nacionalista tamén influíu na creación do canon. O feito de que as universidades máis importantes de finais do século XIX e principios do XX fosen as de Alemaña e que a escola historiográfica alemá dominase un período decisivo na historiografía musical, explica a abundancia de compositores xermanos no canon.} que explica así que predominase certa música sobre outra.

A modo de conclusión, o concepto de \emph{historia da música} redúcese ao estudo dunha serie de compositores e obras musicais da música culta (académica) occidental, que foron seleccionados seguindo certos criterios impostos en certas ocasións polas ``modas musicais'' da época.

\begin{Shaded}
\begin{Highlighting}[]
\NormalTok{knitr}\OperatorTok{::}\KeywordTok{include\_graphics}\NormalTok{(}\StringTok{\textquotesingle{}figures/ud{-}00/rmarkdown.png\textquotesingle{}}\NormalTok{)}
\end{Highlighting}
\end{Shaded}

\begin{figure}

{\centering \includegraphics[width=0.3\linewidth]{figures/ud-00/rmarkdown} 

}

\caption{Logo de rmarkdown (desde archivo PNG).}\label{fig:rmarkdown}
\end{figure}

\hypertarget{conceptos-previos}{%
\section*{Conceptos previos}\label{conceptos-previos}}
\addcontentsline{toc}{section}{Conceptos previos}

Chegados a este punto faremos unha introdución dos coñecementos que debemos ter en conta

\hypertarget{a-muxfasica-como-feito-social-e-cultural}{%
\section*{A música como feito social e cultural}\label{a-muxfasica-como-feito-social-e-cultural}}
\addcontentsline{toc}{section}{A música como feito social e cultural}

Aquí falaremos da función social da música e a súa importancia como modo de introducir á historia da música do presente curso.

\hypertarget{as-fontes-de-informaciuxf3n-histuxf3rica}{%
\section*{As fontes de información histórica}\label{as-fontes-de-informaciuxf3n-histuxf3rica}}
\addcontentsline{toc}{section}{As fontes de información histórica}

A actividade musical é tan antiga como a especie humana. Salvo a época prehistórica, da que só se teñen vagas nocións por restos de posibles instrumentos atopados en xacementos e por pinturas rupestres, o coñecemento da música das culturas antigas ven dado polo que denominamos «fontes de información».

\hypertarget{fontes-para-o-estudo-da-muxfasica-na-prehistoria-e-antiguxfcidade}{%
\subsection*{Fontes para o estudo da Música na Prehistoria e Antigüidade}\label{fontes-para-o-estudo-da-muxfasica-na-prehistoria-e-antiguxfcidade}}
\addcontentsline{toc}{subsection}{Fontes para o estudo da Música na Prehistoria e Antigüidade}

En \textbf{historiografía}, denomínanse «fontes» a todo o que aporta información para o estudo dunha determinada cultura.\\
No caso da Historia da Música das Civilizacións da Prehistoria e a Antigüidade, as fontes son moi variadas. Así, falaremos de fontes de tipo iconográfico, como pinturas e esculturas; documentos escritos, como xeroglíficos e inscripcións en tumbas ou templos; literarios como a Biblia, (entre outros); restos arqueolóxicos, como é o caso de fragmentos de instrumentos desa época atopados en sarcófagos.

Dentro do noso ámbito de estudo, consideramos como principais fontes de información as seguintes:

\begin{enumerate}
\def\labelenumi{\arabic{enumi}.}
\tightlist
\item
  \textbf{Arqueoloxía}. Os restos arqueolóxicos proporcionan importante información sobre a música de épocas antigas. Os máis importantes son os instrumentos musicais ---ou partes deles--- que non se destruíron co paso do tempo; pero tamén se atopan restos de edificios e lugares onde se interpretaba música e danza. Entre os restos arqueolóxicos atópanse tamén as mostras máis antigas de notación musical.
\item
  \textbf{Iconografía}. A pintura, a escultura e outras obras das artes visuais proporcionan información sobre instrumentos musicais, contextos e prácticas de interpretación, danzas, etc.
\item
  \textbf{Literatura}. A literatura, entendida como o conxunto de todo o escrito, ofrece abundante información musical: algunhas fontes literarias describen escenas ou pensamentos musicais e tamén ideas sobre música; os textos da música vocal indican a estrutura rítmica, malia que non se conserven as melodías. Dentro da literatura hai que incluír tamén as obras técnicas sobre música como tratados, métodos, etc.
\item
  \textbf{Etnomusicoloxía}. A etnomusicología, o estudo das músicas de tradición oral actuais, pode axudar á comprensión da actividade musical antiga. Aínda que non é correcto supoñer que en condicións de vida iguais desenvólvense culturas musicais iguais, ás veces o coñecemento das músicas tradicionais actuais pode proporcionar detalles sobre técnicas de interpretación de instrumentos antigos ou sobre movementos de danza, por exemplo.
\end{enumerate}

\begin{verbatim}
mermaid
graph TB;
    Aa(Fontes de Información);
    B(Arqueoloxía);
    C(Iconografía);
    D(Literatura);
    E(Etnomusicoloxía);
    
    A-->B
    A-->C
    A-->D
    A-->E
\end{verbatim}

Case todos os libros sobre Historia da Música, comezan narrando as circunstancias da Música na Idade Media. Este feito, transmite a idea de que a orixe da música na cultura occidental está relacionado co canto gregoriano. Ata hai ben pouco, eran contados os manuais que trataban a importancia da cultura musical da Antigüidade Grega. Que pasa entón coa música anterior? Que sabemos sobre as danzas e os ``concertos cortesáns'' da época dos faraóns? Que instrumentos empregaban nas celebracións funerarias e nas ofrendas aos deuses?

\hypertarget{orixes-da-historia-da-muxfasica-occidental}{%
\chapter{Orixes da Historia da Música Occidental}\label{orixes-da-historia-da-muxfasica-occidental}}

\minitoc 

\hypertarget{introduciuxf3n-1}{%
\section*{Introdución}\label{introduciuxf3n-1}}
\addcontentsline{toc}{section}{Introdución}

\hypertarget{as-fontes-de-informaciuxf3n-histuxf3rica-1}{%
\section*{As fontes de información histórica}\label{as-fontes-de-informaciuxf3n-histuxf3rica-1}}
\addcontentsline{toc}{section}{As fontes de información histórica}

A actividade musical é tan antiga como a especie humana. Salvo a época prehistórica, da que só se teñen vagas nocións por restos de posibles instrumentos atopados en xacementos e por pinturas rupestres, o coñecemento da música das culturas antigas ven dado polo que denominamos «fontes de información».

\hypertarget{fontes-para-o-estudo-da-muxfasica-na-prehistoria-e-antiguxfcidade-1}{%
\subsection*{Fontes para o estudo da Música na Prehistoria e Antigüidade}\label{fontes-para-o-estudo-da-muxfasica-na-prehistoria-e-antiguxfcidade-1}}
\addcontentsline{toc}{subsection}{Fontes para o estudo da Música na Prehistoria e Antigüidade}

En \textbf{historiografía}, denomínanse «fontes» a todo o que aporta información para o estudo dunha determinada cultura.\\
No caso da Historia da Música das Civilizacións da Prehistoria e a Antigüidade, as fontes son moi variadas. Así, falaremos de fontes de tipo iconográfico, como pinturas e esculturas; documentos escritos, como xeroglíficos e inscripcións en tumbas ou templos; literarios como a Biblia, (entre outros); restos arqueolóxicos, como é o caso de fragmentos de instrumentos desa época atopados en sarcófagos.

Dentro do noso ámbito de estudo, consideramos como principais fontes de información as seguintes:

\begin{enumerate}
\def\labelenumi{\arabic{enumi}.}
\tightlist
\item
  \textbf{Arqueoloxía}. Os restos arqueolóxicos proporcionan importante información sobre a música de épocas antigas. Os máis importantes son os instrumentos musicais ---ou partes deles--- que non se destruíron co paso do tempo; pero tamén se atopan restos de edificios e lugares onde se interpretaba música e danza. Entre os restos arqueolóxicos atópanse tamén as mostras máis antigas de notación musical.
\item
  \textbf{Iconografía}. A pintura, a escultura e outras obras das artes visuais proporcionan información sobre instrumentos musicais, contextos e prácticas de interpretación, danzas, etc.
\item
  \textbf{Literatura}. A literatura, entendida como o conxunto de todo o escrito, ofrece abundante información musical: algunhas fontes literarias describen escenas ou pensamentos musicais e tamén ideas sobre música; os textos da música vocal indican a estrutura rítmica, malia que non se conserven as melodías. Dentro da literatura hai que incluír tamén as obras técnicas sobre música como tratados, métodos, etc.
\item
  \textbf{Etnomusicoloxía}. A etnomusicología, o estudo das músicas de tradición oral actuais, pode axudar á comprensión da actividade musical antiga. Aínda que non é correcto supoñer que en condicións de vida iguais desenvólvense culturas musicais iguais, ás veces o coñecemento das músicas tradicionais actuais pode proporcionar detalles sobre técnicas de interpretación de instrumentos antigos ou sobre movementos de danza, por exemplo.
\end{enumerate}

\begin{verbatim}
mermaid
graph TB;
    Aa(Fontes de Información);
    B(Arqueoloxía);
    C(Iconografía);
    D(Literatura);
    E(Etnomusicoloxía);
    
    A-->B
    A-->C
    A-->D
    A-->E
\end{verbatim}

Case todos os libros sobre Historia da Música, comezan narrando as circunstancias da Música na Idade Media. Este feito, transmite a idea de que a orixe da música na cultura occidental está relacionado co canto gregoriano. Ata hai ben pouco, eran contados os manuais que trataban a importancia da cultura musical da Antigüidade Grega. Que pasa entón coa música anterior? Que sabemos sobre as danzas e os ``concertos cortesáns'' da época dos faraóns? Que instrumentos empregaban nas celebracións funerarias e nas ofrendas aos deuses?

\hypertarget{a-orixe-da-muxfasica}{%
\section{A orixe da música}\label{a-orixe-da-muxfasica}}

\hypertarget{as-fontes-de-informaciuxf3n-histuxf3rica-2}{%
\section*{As fontes de información histórica}\label{as-fontes-de-informaciuxf3n-histuxf3rica-2}}
\addcontentsline{toc}{section}{As fontes de información histórica}

A actividade musical é tan antiga como a especie humana. Salvo a época prehistórica, da que só se teñen vagas nocións por restos de posibles instrumentos atopados en xacementos e por pinturas rupestres, o coñecemento da música das culturas antigas ven dado polo que denominamos «fontes de información».

\hypertarget{fontes-para-o-estudo-da-muxfasica-na-prehistoria-e-antiguxfcidade-2}{%
\subsection*{Fontes para o estudo da Música na Prehistoria e Antigüidade}\label{fontes-para-o-estudo-da-muxfasica-na-prehistoria-e-antiguxfcidade-2}}
\addcontentsline{toc}{subsection}{Fontes para o estudo da Música na Prehistoria e Antigüidade}

En \textbf{historiografía}, denomínanse «fontes» a todo o que aporta información para o estudo dunha determinada cultura.\\
No caso da Historia da Música das Civilizacións da Prehistoria e a Antigüidade, as fontes son moi variadas. Así, falaremos de fontes de tipo iconográfico, como pinturas e esculturas; documentos escritos, como xeroglíficos e inscripcións en tumbas ou templos; literarios como a Biblia, (entre outros); restos arqueolóxicos, como é o caso de fragmentos de instrumentos desa época atopados en sarcófagos.

Dentro do noso ámbito de estudo, consideramos como principais fontes de información as seguintes:

\begin{enumerate}
\def\labelenumi{\arabic{enumi}.}
\tightlist
\item
  \textbf{Arqueoloxía}. Os restos arqueolóxicos proporcionan importante información sobre a música de épocas antigas. Os máis importantes son os instrumentos musicais ---ou partes deles--- que non se destruíron co paso do tempo; pero tamén se atopan restos de edificios e lugares onde se interpretaba música e danza. Entre os restos arqueolóxicos atópanse tamén as mostras máis antigas de notación musical.
\item
  \textbf{Iconografía}. A pintura, a escultura e outras obras das artes visuais proporcionan información sobre instrumentos musicais, contextos e prácticas de interpretación, danzas, etc.
\item
  \textbf{Literatura}. A literatura, entendida como o conxunto de todo o escrito, ofrece abundante información musical: algunhas fontes literarias describen escenas ou pensamentos musicais e tamén ideas sobre música; os textos da música vocal indican a estrutura rítmica, malia que non se conserven as melodías. Dentro da literatura hai que incluír tamén as obras técnicas sobre música como tratados, métodos, etc.
\item
  \textbf{Etnomusicoloxía}. A etnomusicología, o estudo das músicas de tradición oral actuais, pode axudar á comprensión da actividade musical antiga. Aínda que non é correcto supoñer que en condicións de vida iguais desenvólvense culturas musicais iguais, ás veces o coñecemento das músicas tradicionais actuais pode proporcionar detalles sobre técnicas de interpretación de instrumentos antigos ou sobre movementos de danza, por exemplo.
\end{enumerate}

\begin{verbatim}
mermaid
graph TB;
    Aa(Fontes de Información);
    B(Arqueoloxía);
    C(Iconografía);
    D(Literatura);
    E(Etnomusicoloxía);
    
    A-->B
    A-->C
    A-->D
    A-->E
\end{verbatim}

Case todos os libros sobre Historia da Música, comezan narrando as circunstancias da Música na Idade Media. Este feito, transmite a idea de que a orixe da música na cultura occidental está relacionado co canto gregoriano. Ata hai ben pouco, eran contados os manuais que trataban a importancia da cultura musical da Antigüidade Grega. Que pasa entón coa música anterior? Que sabemos sobre as danzas e os ``concertos cortesáns'' da época dos faraóns? Que instrumentos empregaban nas celebracións funerarias e nas ofrendas aos deuses?

\hypertarget{a-muxfasica-durante-a-prehistoria}{%
\section{A música durante a Prehistoria}\label{a-muxfasica-durante-a-prehistoria}}

\hypertarget{a-muxfasica-na-prehistoria}{%
\section{A música na prehistoria}\label{a-muxfasica-na-prehistoria}}

Este tema está redactado en modo texto sinxelo \texttt{txt} pero empregando sintase \texttt{markdown} para integralo no RStudio.

\hypertarget{a-muxfasica-nas-primeiras-civilizaciuxf3ns}{%
\section{A música nas primeiras civilizacións}\label{a-muxfasica-nas-primeiras-civilizaciuxf3ns}}

\hypertarget{exipto}{%
\subsection{Exipto}\label{exipto}}

\hypertarget{mesopotamia}{%
\subsection{Mesopotamia}\label{mesopotamia}}

\hypertarget{o-antigo-oriente}{%
\subsection{O antigo Oriente}\label{o-antigo-oriente}}

\hypertarget{o-pobo-hebreo}{%
\subsection{O pobo Hebreo}\label{o-pobo-hebreo}}

\hypertarget{a-muxfasica-no-mundo-cluxe1sico}{%
\section{A música no mundo clásico}\label{a-muxfasica-no-mundo-cluxe1sico}}

\hypertarget{grecia}{%
\subsection{Grecia}\label{grecia}}

\hypertarget{roma}{%
\subsection{Roma}\label{roma}}

\hypertarget{actividades}{%
\section{Actividades}\label{actividades}}

\hypertarget{resumo}{%
\section{Resumo}\label{resumo}}

\startappendices

\hypertarget{the-first-appendix}{%
\chapter{The First Appendix}\label{the-first-appendix}}

This first appendix includes an R chunk that was hidden in the document (using \texttt{echo\ =\ FALSE}) to help with readibility:

\textbf{In 02-rmd-basics-code.Rmd}

\textbf{And here's another one from the same chapter, i.e.~Chapter \ref{code}:}

\hypertarget{the-second-appendix-for-fun}{%
\chapter{The Second Appendix, for Fun}\label{the-second-appendix-for-fun}}


%%%%% REFERENCES
\setlength{\baselineskip}{0pt} % JEM: Single-space References

{\renewcommand*\MakeUppercase[1]{#1}%
\printbibliography[heading=bibintoc,title={\bibtitle}]}


\end{document}
