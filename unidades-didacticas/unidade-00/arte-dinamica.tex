%
% ARTE DIN�MICA VS. ARTE EST�TICA
% -------------------------------
Tr�tase dunha arte con un alto grao de abstracci�n, superior ao das dem�is; � unha arte din�mica que se desenvolve no tempo e pervive con el e polo tanto � unha arte viva.

% TODO: Crear esquema e incluir;
\bigskip
\begin{center}
\begin{tikzpicture}[
	scale=1,
	start chain=1 going below, 
	start chain=2 going right,
	node distance=1mm,
	desc/.style={
		scale=1,
		on chain=2,
		rectangle,
		rounded corners,
		draw=black, 
		very thick,
		text centered,
		text width=8cm,
		minimum height=12mm,
		fill=azulF
		},
	it/.style={
		fill=grisamarillo
	},
	level/.style={
		scale=1,
		on chain=1,
		minimum height=10mm,
		text width=4cm,
		text centered
	},
	every node/.style={font=\sffamily}
]

% Levels
\node [level] (Level 5) {Artes est�ticas};
\node [level] (Level 4) {Artes din�micas};
%\node [level] (Level 3) {Level 3};
%\node [level] (Level 2) {Level 2};
%\node [level] (Level 1.5) { };
%\node [level] (Level 1) {Level 1};
%\node [level] (Level 0) {Level 0};

% Descriptions
\chainin (Level 5); % Start right of Level 5
% IT levels
\node [desc, it] (Archives) {Pintura, Escultura e Arquitectura};
\node [desc, it, continue chain=going below] (ERP) {Poes�a, Danza, M�sica e Cine};
% ICS levels
%\node [desc] (Operations) {Operations Management/Historians};
%\node [desc] (Supervisory) {Supervisory Controls};
%\node [desc, text width=3.5cm, xshift=2.25cm] (PLC) {PLC/RTU IP Communication};
%\node [desc, text width=3.5cm, xshift=-4.5cm] (SIS) {Safety Instrumented Systems};
%\node [desc, xshift=2.25cm] (IO) {I/O from Sensors};
\end{tikzpicture}
\end{center}
\bigskip


% --- Esquema artes est�ticas vs. din�micas
%\bigskip
%
% \schema % Esquema artes est�ticas 
%  {
%  \schemabox{Artes est�ticas}
%  }
%  {
%  \schemabox{Pintura \\ Escultura \\ Arquitectura}
%  }
%  
%
%\schema % Esquema artes est�ticas 
%  {
%  \schemabox{Artes din�micas}
%  }
%  {
%  \schemabox{Poes�a \\ Danza \\ M�sica \\ Cine}
%  }
%
%\bigskip
% ---

A diferencia das artes est�ticas, que ocupan un espazo f�sico, a m�sica existe co devir do tempo. Para chegar a apreciala, � preciso ter certos co�ecementos espec�ficos complexos que permitan comprender plenamente a s�a realidade cient�fica.

O xeito en que nos chega a <<obra de arte>> difire entre as artes est�ticas e as din�micas. A obra de arte est�tica, permanece no tempo ao igual que a din�mica; para que a obra de arte musical chegue a n�s � preciso que sexa interpretada e a idea coa que se crea, revive con cada interpretaci�n. 

\begin{quote}
    {\small
    Entre a obra dun pintor ...
    }
\end{quote}


