% 
% PERSPECTIVAS E REFLEXI�NS SOBRE A M�SICA:
% -----------------------------------------

\subsection*{PERSPECTIVAS e REFLEXI�NS} \label{reflexions}

\begin{multicols}{2}

Unha primeira aproximaci�n ao concepto de m�sica\footnote{
A palabra, prodece do grego \emph{musike} en referencia �s nove musas que presid�an as ciencias e artes. (Jurado, 2008)} 
l�vanos �s teor�as da linguaxe musical, que a definen como "a arte de combinar os sons no tempo". 

Outras teor�as tratan tam�n de explicar que � a m�sica desde diferentes puntos de vista, o que da lugar a diferentes \textbf{perspectivas}.
\begin{caja}[Perspectivas sobre a m�sica:]
    \begin{description}
    \item [M�sica como ciencia.-] Pit�goras, no s.{\scriptsize V} a.{\scriptsize C}, afirmaba: \\
    \textit{
    "Os n�meros son as cousas; agora ben, a m�sica � n�mero.\\
    O mundo � m�sica; o cosmos � unha lira sublime de sete cordas.
    "
    }
    \item [M�sica como arte.-] Richard Wagner, no s.{\scriptsize XIX} afirmaba sobre a m�sica: \\
    \textit{
    "O son v�n do coraz�n e a s�a linguaxe art�stica natural � a m�sica. \\
    A melod�a � a lingua absoluta, a trav�s da que o m�sico fala a todos os coraz�ns."
    }
    \item [M�sica como feito musical.-] Descartes, no s.{\scriptsize XVII} fac�a a seguinte reflexi�n: \\
    \textit{
    "A mesma cousa que a uns invita a bailar a outros pode facer chorar. Pois isto non prov�n sen�n da asociaci�n de ideas na nosa mente; como aqueles que algunha vez se divertiron bailando con certa peza, tan pronto como a volvan a escoltar volver�n �s ganas de bailar; pola contra, se alg�n s� o�u gallardas cando lle aconteceu algo malo, volver�  a entristecerse cando as escoite de novo."
    }
    \end{description}
\end{caja}

\end{multicols}

