%
% ORIXES DA M�SICA: PREHISTORIA MUSICAL
% -------------------------------------
\subsection{AS ORIXES DA M�SICA: PREHISTORIA MUSICAL} \label{sub:prehistoria}

Despois de achegarnos ao significado da m�sica, �s fontes de informaci�n e aos per�odos, comezaremos o estudo da s�a historia desde un punto de vista cronol�xico.

Ad�itase situar o comezo da historia no momento da aparici�n da escritura, hai uns 6000 anos aproximadamente; o per�odo anterior denom�nase Prehistoria. A�nda que est� constatada a existencia de actividade musical durante a prehistoria, desco��cese que m�sica se fac�a ent�n.

O primeiro per�odo do que se pode dicir algo � o da Prehistoria1, a�nda que non existen
fontes musicais directas. Por iso, a an�lise directa da m�sica � imposible e s� podemos basear as
nosas afirmaci�ns en hip�teses baseadas en diversas fontes secundarias. No caso da Prehistoria estas
fontes non incl�en fontes escritas, s� organol�xicas (restos de instrumentos), iconogr�ficas
(representaci�ns pict�ricas) e etnol�xicas (comparaci�ns con tribos actuais que te�en unha forma de
vida similar � Prehistoria).
